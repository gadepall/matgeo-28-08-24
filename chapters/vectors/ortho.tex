\begin{enumerate}[label=\thesubsection.\arabic*, ref=\thesubsection.\theenumi]
\item Name the type of quadrilateral formed,  if any,  by the following points, and give reasons for your answer
\begin{enumerate}
\item $\vec{A}(-1, -2),  \vec{B}(1, 0),  \vec{C}(-1, 2),  \vec{D}(-3, 0)$
\item $\vec{A}(-3, 5),  \vec{B}(3, 1),  \vec{C}(0, 3),  \vec{D}(-1, -4)$
\item $\vec{A}(4, 5),  \vec{B}(7, 6),  \vec{C}(4, 3),  \vec{D}(1, 2)$
\end{enumerate}
\solution
			See \tabref{tab:10/7/1/6/inner},
	\figref{fig:10/7/1/6/Fig1}, \figref{fig:10/7/1/6/Fig2}.
	and 
	\figref{fig:10/7/1/6/Fig3}. 
In b), forming the collinearity matrix
\begin{align}
\myvec{\vec{B}-\vec{A} & \vec{C}-\vec{B}} 
=
		\myvec{6&-3\\-4&2} \xleftrightarrow{R_{2}\rightarrow R_{2}+\frac{2}{3}R_{1}}= \myvec{6&-3\\0&0}
\end{align}
which is a rank 1 matrix.  Hence, $\vec{A}, \vec{B}, \vec{C}$  are collinear.
\begin{figure}[H]
	\begin{center} 
	    \includegraphics[width=0.75\columnwidth]{chapters/10/7/1/6/figs/fig1.pdf}
	\end{center}
\caption{}
\label{fig:10/7/1/6/Fig1}
\end{figure}
%
\begin{figure}[H]
	\begin{center} 
	    \includegraphics[width=0.75\columnwidth]{chapters/10/7/1/6/figs/fig2.pdf}
	\end{center}
\caption{}
\label{fig:10/7/1/6/Fig2}
\end{figure}
%	
\begin{figure}[H]
	\begin{center} 
	    \includegraphics[width=0.75\columnwidth]{chapters/10/7/1/6/figs/fig3.pdf}
	\end{center}
\caption{}
\label{fig:10/7/1/6/Fig3}
\end{figure}
%
\begin{table}[H]
    \centering
%    \begin{tabular}{|c|c|c|c|c|}
	    \begin{tabularx}{\columnwidth}{|c|X|X|X|c|}
        \hline
		    &{\scriptsize $\vec{B}-\vec{A} = \vec{C}-\vec{D}$?} & {\tiny $(\vec{B}-\vec{A})^\top (\vec{C}-\vec{B}) =  0$?} & {\tiny $(\vec{C}-\vec{A})^\top (\vec{D}-\vec{B}) = 0$}& \textbf{Geometry}\\
        \hline
	    a)& Yes & Yes & Yes& Square \\
        \hline
	    b)& No & -&- & Triangle\\
        \hline
	    c)&Yes & No & No & Parallelogram\\
        \hline
	\end{tabularx}
%    \end{tabular}
	\caption{}
	\label{tab:10/7/1/6/inner}
\end{table}

\item Find the projection of the vector $\hat{i}+3\hat{j}+7\hat{k}$ on the vector $7\hat{i}-\hat{j}+8\hat{k}$.
	\\
	\solution
				Let 
\begin{align}
 \vec{A} =\myvec{1\\3\\7}, \vec{B} =\myvec{7\\-1\\8}
\end{align}
The projection of $\vec{A}$ on $\vec{B}$ is defined as
the foot of the perpendicular from 
$\vec{A}$ to $\vec{B}$ and obtained in 
	\eqref{eq:12/10/3/4/proj}.
Substituting numerical values,
\begin{align}
	\vec{C}
		=\frac{10}{19}\myvec{7\\-1\\8}
 \end{align}

\item Find the projection of the vector $\hat{i}-\hat{j}$ on the vector $\hat{i}+\hat{j}$.
	\\
\solution
		The given points are
\begin{align}
 \vec{A}=\myvec{1\\ -1},
 \vec{B}=\myvec{1\\ 1}
\end{align}
Since
\begin{align}
	\vec{A}^\top \vec{B} =0,
\end{align}
	from \eqref{eq:12/10/3/4/proj},
the projection vector is the origin.
		See \figref{fig:12/10/3/3fig}.
\begin{figure}[H]
	\centering
\includegraphics[width=0.75\columnwidth]{chapters/12/10/3/3/figs/fig.pdf}
\caption{}
		\label{fig:12/10/3/3fig}
\end{figure}

\item Show that each of the given three vectors is a unit vector: 
 $\frac{1}{7}(2\hat{i}+3\hat{j}+6\hat{k}), \frac{1}{7}(3\hat{i}-6\hat{j}+2\hat{k}), \frac{1}{7}(6\hat{i}+2\hat{j}-3\hat{k}$).
Also, show that they are mutually perpendicular to each other.
	\\
	\solution
		\begin{align}
\vec{A} = 	\myvec{
	\frac{2}{7} & \frac{3}{7} & \frac{6}{7} \\[1ex]
    \frac{3}{7} & -\frac{6}{7} & \frac{2}{7} \\[1ex]
    \frac{6}{7} & \frac{2}{7} & -\frac{3}{7}
}
\end{align}
is an orthogonal matrix satisfying
\eqref{eq:12/10/3/5/inner},
which verifies the given conditions.

\item Show that the vectors $2\hat{i}-\hat{j}+\hat{k}, \hat{i}-3\hat{j}-5\hat{k}$ and  $3\hat{i}-4\hat{j}-4\hat{k}$ from the vertices of a right angled triangle.
	\\
	\solution
		\begin{align}
\vec{A} = \myvec{2\\-1\\1}, \, \vec{B} = \myvec{1\\-3\\-5}, \, \vec{C}=\myvec{3\\-4\\-4},
\\
\implies \vec{B}-\vec{C} = \myvec{-2\\1\\-1} ,\, 
\vec{C}-\vec{A} = \myvec{1\\-3\\-5} ,\, 
\\
	\text{or, }
\brak{\vec{B}-\vec{C}}^{\top}\brak{\vec{C}-\vec{A}} = 0
\end{align}

\item Show that the points $\vec{A},  \vec{B}$ and $\vec{C}$ with position vectors,  $3\hat{i}-4\hat{j}-4\hat{k},  2\hat{i}-\hat{j}+\hat{k}$ and $\hat{i}-3\hat{j}-5\hat{k}$,  respectively,  form the vertices of a right angled
triangle.
\\
\solution
		    \begin{align}
         \vec{B} - \vec{A} = \myvec{-1\\3\\5},\, 
         \vec{C} - \vec{B} = \myvec{-1\\-2\\-6},\,
         \vec{C} - \vec{A} = \myvec{-2\\1\\-1},
        \label{eq:chapters/12/10/2/17/dir-vec}
	\\
	    \implies 
	    \brak{\vec{B} - \vec{A}}^\top
	    \brak{\vec{C} - \vec{A}} = 0
    \end{align}
Hence, $\triangle ABC$ is right angled at $\vec{A}$. 

\item Let $\vec{a}=\hat{i}+4\hat{j}+2\hat{k},  \vec{b}=3\hat{i}-2\hat{j}+7\hat{k}$ and $\vec{c}=2\hat{i}-\hat{j}+4\hat{k}$. Find a vector $\vec{d}$ which is perpendicular to both $\vec{a}$ and $\vec{b}$,  and $\vec{c}\cdot \vec{d}$=15.\\
	\solution
		From the given information, 
\begin{align}
\vec{a}^{\top}\vec{d} &= 0\\
\vec{b}^{\top}\vec{d} &= 0\\
\vec{c}^{\top}\vec{d} &= 15
\end{align}
yielding
\begin{align}
\myvec{\vec{a}^{\top} \\\vec{b}^{\top}\\\vec{c}^{\top}}\vec{d} &= \myvec{0\\0\\15}\\
\implies \myvec{1&4&2 \\3&-2&7 \\2&-1&4}\vec{d} &= \myvec{0\\0\\15}
\label{eq:chapters/12/10/5/12/1}
\end{align}
%
Forming the augmented matrix, 
\begin{align}
	\myvec{1&4&2&\vrule&0\\ 3&-2&7&\vrule&0 \\ 2&-1&4&\vrule&15} 
	\xleftrightarrow[R_3\leftarrow R_3-2R_1]{R_2\leftarrow R_2-3R_1}
	\myvec{1&4&2&\vrule&0\\ 0&-14&1&\vrule&0 \\ 0&-9&0&\vrule&15}
\nonumber	\\
	\xleftrightarrow[]{R_3\leftarrow R_3-\frac{9}{14}R_2}
	\myvec{1&4&2&\vrule&0\\ 0&-14&1&\vrule&0 \\ 0&0&-\frac{9}{14}&\vrule&15}
	\label{eq:chapters/12/10/5/12/2}
\end{align}
yielding
%
\begin{align}
	\vec{d} &= \myvec{\frac{160}{3}\\[1ex]-\frac{5}{3}\\[1ex]-\frac{70}{3}}
\end{align}
upon back substitution.


\item $ABCD$ is a rectangle formed by the points $\vec{A}(–1,  –1),  \vec{B}(– 1,  4),  \vec{C}(5,  4)$  and  $\vec{D}(5,  – 1)$. $\vec{P},  \vec{Q},  \vec{R}$ and $\vec{S}$ are the mid-points of $AB,  BC,  CD$ and $DA$ respectively. Is the quadrilateral $PQRS$ a square? a rectangle? or a rhombus? Justify your answer.
	\\
	\solution 
See Fig. \ref{fig:10/7/4/8Fig3}. From 
  \eqref{eq:10/7/4/8det2f}, $PQRS$ is a parallelogram.
\begin{align}
  %\label{eq:10/7/4/8det2f}
  \vec{P}  = 
 \frac{3}{2},\, 
 \vec{Q}  = \myvec{
 2 \\
 4 \\
 } ,\,
 \vec{R}  = \myvec{
 5 \\
 \frac{3}{2}
 }   
  ,\,
 \vec{S}  = \myvec{
 2\\
 -1 \\
 }   
 \\
	\implies 
 \brak{\vec{Q}-\vec{P}}^\top\brak{\vec{R}-\vec{Q}}  \neq 0
 \\
 \brak{\vec{R}-\vec{P}}^\top\brak{\vec{S}-\vec{Q}}  = 0
\end{align}
Therefore $PQRS$ is a rhombus.
\begin{figure}[H]
	\begin{center}
		\includegraphics[width=0.75\columnwidth]{chapters/10/7/4/8/figs/fig.pdf}
	\end{center}
\caption{}
\label{fig:10/7/4/8Fig3}
\end{figure}


\item Without using the Baudhayana theorem,  show that the points $\vec{A}(4, 4),  \vec{B}(3, 5)$ and $\vec{C}(-1, -1)$ are the vertices of a right angled triangle.
\label{chapters/11/10/1/6}
\\
\solution
		See \figref{fig:11/10/1/6}.
\begin{align}
	\vec{C}-\vec{A}=\myvec{
-5 \\
	-5},\,
	\vec{A}-\vec{B}&=\myvec{
1 \\
-1 
}
\\
	\implies \brak{\vec{C}-\vec{A}}^{\top}
	\brak{\vec{A}-\vec{B}}&=0
\end{align}
Thus, $AB \perp AC$.
	\begin{figure}[H]
		\centering
 \includegraphics[width=0.75\columnwidth]{chapters/11/10/1/6/figs/fig.pdf}
		\caption{}
		\label{fig:11/10/1/6}
  	\end{figure}

\item In the following cases,  determine whether the given planes are parallel or perpendicular,  and in case they are neither,  find the angles between them.
\begin{enumerate}
\item $7x + 5y + 6z + 30 = 0$ and $3x – y – 10z + 4 = 0$
\item $2x + y + 3z – 2 = 0$ and $x – 2y + 5 = 0$
\item $2x – 2y + 4z + 5 = 0$ and $3x – 3y + 6z – 1 = 0$
\item $2x – y + 3z – 1 = 0$ and $2x – y + 3z + 3 = 0$
\item $4x + 8y + z – 8 = 0$ and $y + z – 4 = 0$
\end{enumerate}
    \solution
		    See \tabref{tab:12/11/3/13}.
\begin{table}[H]
    \centering
    \caption{}
    \label{tab:12/11/3/13}
    \begin{tabular}{|c|c|c|c|c|c|}
        \hline
        $\vec{n}_1$ & $\vec{n}_1$ & $\vec{n}_1^{\top}\vec{n}_2$ & $\norm{\vec{n}_1}$ & $\norm{\vec{n}_2}$ & Angle\\
        \hline
        $\myvec{7\\5\\6}$ & $\myvec{3\\-1\\-10}$ & $-44$ & $\sqrt{110}$ & $\sqrt{110}$ & $\cos^{-1}-\frac{2}{5}$ \\
        \hline
        $\myvec{2\\1\\3}$ & $\myvec{1\\-2\\0}$ & $0$ & & & perpendicular \\
        \hline
        $\myvec{2\\-2\\4}$ & $\myvec{3\\-3\\6}$ & $36$ & $\sqrt{24}$ & $\sqrt{54}$ & parallel \\
        \hline
        $\myvec{2\\-1\\3}$ & $\myvec{2\\-1\\3}$ & $14$ & $\sqrt{14}$ & $\sqrt{14}$ & parallel \\
        \hline
        $\myvec{4\\8\\1}$ & $\myvec{0\\1\\1}$ & $9$ & $9$ & $\sqrt{2}$ & $45\degree$ \\
        \hline
    \end{tabular}
\end{table}

\iffalse
\begin{table}[H]
    \centering
    \caption{}
    \label{}
    \begin{tabular}{|c|c|c|c|c|c|}
        \hline
	    $\vec{n}_1$ & $\vec{n}_1$ &  $\vec{n}_1^{\top}\vec{n}_2$& $\norm{\vec{n}_1}$ &$\norm{\vec{n}_2}$  &  Angle\\
        \hline
	    \myvec{7\\5\\6} & \myvec{3\\-1\\-10} & -44 & \sqrt{110} &\sqrt{110}  & \cos^{-1}-\frac{2}{5} \\
        \hline
\myvec{2\\1\\3}  & \myvec{1\\-2\\0}& 0 &  &  & perpendicular\\
        \hline
 \myvec{2\\-2\\4} & \myvec{3\\-3\\6} & 36  & \sqrt{24} & \sqrt{54} &  parallel \\
        \hline
 \myvec{2\\-1\\3} & \myvec{2\\-1\\3} & 14 & \sqrt{14} & \sqrt{14} & parallel \\
        \hline
 \myvec{4\\8\\1} & \myvec{0\\1\\1} & 9 & 9 & \sqrt{2} &  45\degree  \\
        \hline
    \end{tabular}
\end{table}
\fi

		\item 
 Show that the line joining the origin to the point $\vec{P}(2,  1,  1)$ is perpendicular to the
line determined by the points $\vec{A}(3,  5,  – 1),  \vec{B}(4,  3,  – 1)$.
\\
    \solution
				\begin{align}
			\brak{\vec{A}-\vec{B}}^\top\vec{P}=
			\myvec{-1&2&0}\myvec{2\\1\\1}=0 \qed
		\end{align}

\item Find a unit vector perpendicular to each of the vectors $\overrightarrow{a}+\overrightarrow{b}$ and $\overrightarrow{a}-\overrightarrow{b}, \text{ where } \overrightarrow{a}=3\hat{i}+2\hat{j}+2\hat{k}$ and $ \overrightarrow{b}=\hat{i}+2\hat{j}-2\hat{k}$. 
	\\
		\solution
		Let the desired vector be $\vec{x}$.  Then, 
\begin{align} 
\myvec{\vec{a}+\vec{b}& \vec{a}-\vec{b}}^\top\vec{x}=0\\
	\label{eq:12/10/4/2/given}
\end{align}
\begin{align}
	\because 
	\vec{a}+\vec{b}= \myvec{\vec{a} & \vec{b}}\myvec{1 \\ 1}
	\\
	\vec{a} - \vec{b}= \myvec{\vec{a} & \vec{b}}\myvec{1 \\ -1}, 
\end{align}
	\eqref{eq:12/10/4/2/given}
	can be expressed as 
\begin{align}
	\cbrak{\myvec{\vec{a} & \vec{b}}\myvec{1 & 1\\ 1 & -1}}^\top\vec{x}=0\\
\implies 	\myvec{1 & 1\\ 1 & -1}^\top\myvec{\vec{a} & \vec{b}}^\top\vec{x}=0
\\
\implies 	\myvec{1 & 1\\ 1 & -1}\myvec{1 & 1\\ 1 & -1}^\top\myvec{\vec{a} & \vec{b}}^\top\vec{x}=0
\\
	\text{or, }
\myvec{\vec{a} & \vec{b}}^\top\vec{x}=0
\end{align}
which can be expressed as 
\begin{align} 
\myvec{
3&2&2\\
1&2&-2
}
	\xleftrightarrow[R_2 = \frac{R_2}{4}]{R_2=3R_2-R_1}
\myvec{
3&2&2\\
0&1&-2
	}
	\\
	\xleftrightarrow[R_1 = \frac{R_1}{3}]{R_1=R_1 - 2R_2}
\myvec{
1&0&2\\
0&1&-2
}
\end{align}
yielding
\begin{align}
\begin{split}
x_1+2x_3=0\\
x_2-2x_3=0
\end{split}
\implies 
\vec{x}
=x_3\myvec{-2\\2\\1}
\end{align}
Thus, the desired unit vector  is 
\begin{align}
\vec{x}
	=\frac{1}{3}\myvec{-2\\2\\1}
\end{align}

\item If $\overrightarrow {a}=2\hat{i}+2\hat{j}3\hat{k}, \overrightarrow {b}=\hat{-i}+2\hat{j}+\hat{k}$ and $\overrightarrow {c}=3\hat{i}+\hat{j}$ are such that $\overrightarrow {a}+\lambda\overrightarrow {b}$ is perpendicular to $\overrightarrow {c}$, then find the value of $\lambda$.
	\\
		\solution
\begin{align}
\because	(\vec{a}+\lambda \vec{b})^{\top} \vec{c} = 0,
	\\
	\lambda=-\frac{\vec{a}^{\top}\vec{c}}{\vec{b}^{\top}\vec{c}}
	=8,
\end{align}
upon substituting numerical values.



\item Check whether $(5, -2),  (6, 4)$ and $(7, -2)$ are the vertices of an isosceles triangle.
\item The perpendicular bisector of the line segment joining the points $\vec{A} (1,  5) $ and $
\vec{B} (4,  6)$ cuts the y-axis at
\item The point which lies on the perpendicular bisector of the line segment joining the
	points $\vec{A} (–2,  –5)\text { and } \vec{B} (2,  5) $ is
\begin{enumerate}
\item  	$(0,  0)$
\item  $(0,  2)$ 
\item  $(2,  0)$ 
\item  $(–2,  0)$
\end{enumerate}
\item The points $ (–4,  0),  (4,  0),  (0,  3) $ are the vertices of
	\begin{enumerate}
\item right triangle 
\item isosceles triangle
\item  equilateral triangle
\item  scalene triangle 
\end{enumerate}
\item The point $\vec{A}(2, 7)$ lies on the perpendicular bisector of line segment joining the points $\vec{P}(6, 5)$ and $ \vec{Q}(0, -4)$.
\item The points $\vec{A}(-1, -2),  \vec{B}(4, 3),  \vec{C}(2, 5) $ and $ \vec{D}(-3, 0)$ in that order form a rectangle.
\item Name the type of triangle formed by the points $\vec{A}(-5, 6), \vec{B}(-4, -2), $ and $\vec{C}(7, 5)$.
\item What type of a quadrilateral do the points $\vec{A}(2, -2), \vec{B}(7, 3), \vec{C}(11, -1), $ and $\vec{D}(6, -6)$ taken in that order,  form?
\item Find the coordinates of the point $\vec{Q}$ on the $x$-axis which lies on the perpendicular bisector of the line segment joining the points $\vec{A}(-5, -2) $ and $ \vec{B}(4, -2)$. Name the type of triangle formed by points $\vec{Q}, \vec{A}$ and $\vec{B}$.
\item The points $\vec{A}(2, 9), \vec{B}(a, 5) $ and $\vec{C}(5, 5)$ are the vertices of a triangle $\vec{ABC}$ right angled at $\vec{B}$. Find the values of a and hence the area of $\triangle \vec{ABC}$.
\item Find a vector of magnitude 6,  which is perpendicular to both the vectors $2\hat{i}-\hat{j}$+$2\hat{k}$ and $4\hat{i}-\hat{j}+3\hat{k}$.
\item If $\vec{A}, \vec{B}, \vec{C}, \vec{D}$  are the points with position vectors $\hat{i}+\hat{j}-\hat{k}$,  $2\hat{i}-\hat{j}+3\hat{k}$,  $2\hat{i}-3\hat{k}$,  $3\hat{i}$-$2\hat{j}$+$\hat{k}$,  respectively,  find the projection of $\overline{AB}$ $\text{ along }$ $\overline{CD}$.
\item Find the value of $\lambda$ such that the vectors $\vec{a}=2\hat{i}+\lambda\hat{j}+\hat{k}$ $\text{and}$ $\vec{b}=\hat{i}+2\hat{j}+3\hat{k}$ are orthogonal.
\item The number of vectors of unit length perpendicular to the vectors $\vec{a}=2\hat{i}+\hat{j}+2\hat{k}$  and  $\vec{b}=\hat{j}+\hat{k}$ is
\item Find the equation of a plane which  bisects perpendicularly the line joining the points $\vec{A}(2, 3, 4)$ and $\vec{B}(4, 5, 8)$ at right angles.
\item $\overrightarrow{AB}=3\hat{i}-\hat{j}+\hat{k}$ and $\overrightarrow{CD}=-3\hat{i}+2\hat{j}+4\hat{k}$ are two vectors. The position vectors of the points $\vec{A}$ and $\vec{C}$ are $6\hat{i}+7\hat{j}+4\hat{k}$ and $-9\hat{j}+2\hat{k}, $ respectively. Find the position vector of a point $\vec{P}$ on the line AB and a point $\vec{Q}$ on the line CD such that $\overrightarrow{PQ}$ is perpendicular to $\overrightarrow{AB}$ and $\overrightarrow{CD}$ both.
\item Line joining the points (3, -4) and (-2, 6) is perpendicular to the line joining the points (-3, 6) and (9, -18).
\item Verify the following:
\begin{enumerate}
\item $(0, 7, -10),  (1, 6, -6)$ and $(4, 9, -6)$ are the vertices of an isoceles triangle.
\item $(0, 7, 10),  (-1, 6, 6)$ and $(-4, 9, 6)$ are the vertices of a right angled triangle.
\end{enumerate}
\item  Show that the line through the points $(1, -1, 2), (3, 4, -2 )$ is perpendicular to the line through the points $(0, 3, 2)$ and $(3, 5, 6)$.
\item Find the values of $p$ so that the lines $ \frac{1-x}{3}=\frac{7y-14}{2p}=\frac{z-3}{2}$ and $ \frac{7-7x}{3p}=\frac{y-5}{1}=\frac{6-z}{5}$ are at right angles.
\item Show that the lines $ \frac{x-5}{7}=\frac{y+2}{-5}=\frac{z}{1}$ and $ \frac{x}{1}=\frac{y}{2}=\frac{z}{3}$ are perpendicular to each other.
\item Do the points $(3,2), (-2,-3)$ and $(2,3)$ form a triangle? If so, name the type of triangle formed.
\item Show that the points $(1,7),(4,2),(-1,-1)$ and $(-4,4)$ are the vertices of a square.
\item Line through the points $(-2,6)$ and $(4,8)$ is perpendicular to the line through the points $(8,12)$ and $(x,24)$. Find the value of $x$.
\item Find the angle between the lines $y-\sqrt 3x-5=0$ and $\sqrt 3y-x+6=0$.
\item Are the points $\vec{A}(3,6,9), \vec{B}(10,20,30)$ and $\vec{C}(24,-41,5)$ the vertices of a right angled triangle?
\item Show that the points $\vec{A}(1,2,3), \vec{B}(-1,-2,-1), \vec{C}(2,3,2)$ and $\vec{D}(4,7,6)$ are the vertices of a parallelogram $ABCD$, but it is not a rectangle.
\item Show that the points $\vec{A}(2\hat{i} -\hat{j} +\hat{k}), \vec{B}(\hat{i} -3\hat{j}-5\hat{k}),\vec{C}(3\hat{i} -4\hat{j} -4\hat{k})$ are the vertices of a right angled triangle.
\item If $\overrightarrow{a} = 5\hat{i} -\hat{j} -3{k}$ and $\overrightarrow{b} = \hat{i} +3\hat{j} -5\hat{k}$, then show that the vectors $\overrightarrow{a}+\overrightarrow{b}$ and $\overrightarrow{a}-\overrightarrow{b}$ are perpendicular.
\item Find the projection of the vector $\overrightarrow{a} = 2\hat{i} +3\hat{j} +2\hat{k}$ on the vector $\overrightarrow{b} = \hat{i} +2\hat{j} +\hat{k}$.
\item Find a unit vector perpendicular to each of the vectors $(\overrightarrow{a} +\overrightarrow{b})$ and $(\overrightarrow{a} -\overrightarrow{b})$, where $\overrightarrow{a}=\hat{i} +\hat{j} +\hat{k}, \overrightarrow{b} = \hat{i} +2\hat{j} +3\hat{k}$.
\end{enumerate}
