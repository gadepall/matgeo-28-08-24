\begin{enumerate}[label=\thesubsection.\arabic*,ref=\thesubsection.\theenumi]
\item Find the sum of the vectors $\vec{a}=\hat{i}-2\hat{j}+\hat{k}$, $\vec{b}=-2\hat{i}+4\hat{j}+5\hat{k}$ and $\vec{c}=\hat{i}-6\hat{j}-7\hat{k}$.
\item 

	In triangle ABC 
		(\figref{fig:chapters/12/10/2/18/}),
		which of the following is not true:
 \begin{enumerate}
         \item $\overrightarrow{AB}+\overrightarrow{BC}+\overrightarrow{CA}$=$\vec{0}$
         \item $\overrightarrow{AB}+\overrightarrow{BC}-\overrightarrow{CA}$=$\vec{0}$
         \item $\overrightarrow{AB}+\overrightarrow{BC}-\overrightarrow{CA}$=$\vec{0}$
         \item $\overrightarrow{AB}-\overrightarrow{BC}+\overrightarrow{CA}$=$\vec{0}$
\end{enumerate}
\begin{figure}[!ht]
\centering
\includegraphics[width = \columnwidth]{./chapters/12/10/2/18/figs/triangle.png}
\caption{}
	\label{fig:chapters/12/10/2/18/}
\end{figure}
\solution
		\begin{align}
	\overrightarrow{AB}+\overrightarrow{BC}+\overrightarrow{CA} &=
\vec{B}-\vec{A} + \vec{C} - \vec{B} + \vec{A} - \vec{C}
= 0
\\
	\overrightarrow{AB}+\overrightarrow{BC}-\overrightarrow{AC} &=
\vec{B}-\vec{A} + \vec{C} - \vec{B} - (\vec{C} - \vec{A})
= 0
\\
	\overrightarrow{AB}+\overrightarrow{BC}+\overrightarrow{AC} &=
\vec{B}-\vec{A} + \vec{C} - \vec{B} + \vec{C} - \vec{A}
= 2(\vec{C}-\vec{A})
\\
	\overrightarrow{AB}-\overrightarrow{CB}+\overrightarrow{CA} &=
\vec{B}-\vec{A} - (\vec{B} - \vec{C}) + \vec{A} - \vec{C}
= 0
\end{align}

\item A girl walks 4 km towards west, then she walks 3 km in a direction 30$^{\circ}$ east of north and stops. Determine the girl's displacement from her initial point of departure.\\
	\solution
		See  
\figref{fig:chapters/12/10/5/3Fig1}.
Let the initial position
be
\begin{align}
	\vec{A}=\myvec{0\\0}
\end{align}
After going west, the position becomes
\begin{align}
			\vec{B}=\myvec{-4\\0}
\end{align}
If the final position be $\vec{C}$, from the given information,
\begin{align}
	 \vec{C}-\vec{B}=3\myvec{\cos{60\degree}\\\sin{60\degree}}
	 \implies 
	\vec{C}  
=\myvec{-\frac{5}{2}\\[2pt] \frac{3\sqrt{3}}{2}}
\end{align}
which is the desired displacement. 
\begin{figure}[H] 
 \begin{center} 
 \includegraphics[width=0.75\columnwidth]{chapters/12/10/5/3/figs/fig.pdf} 
 \end{center} 
\caption{} 
\label{fig:chapters/12/10/5/3Fig1} 
\end{figure}

\item Without using distance formula, show that points A(– 2, – 1), B(4, 0), C(3, 3) and D(–3, 2) are the vertices of a parallelogram.
\label{chapters/11/10/1/9}
\\
\solution
	  From \eqref{eq:two-pgm},
\begin{align}
\vec{A}-\vec{B} = 
\vec{D}-\vec{C} =  \myvec{-6\\-1}
\end{align}
Hence, $ABCD$ is a parallelogram.
See \figref{fig:chapters/11/10/1/91}.
\begin{figure}[H]
  \centering
   %\includegraphics[width=0.75\columnwidth]{chapters/11/10/1/9/figs/paralellogram.png}
   \includegraphics[width=0.75\columnwidth]{chapters/11/10/1/9/figs/fig.pdf}
    \caption{}
     \label{fig:chapters/11/10/1/91}  
\end{figure}




\item The fourth vertex $\vec{D}$ of a parallelogram $\vec{ABCD}$ whose three vertices are
	$\vec{A} (–2, 3), \vec{B} (6, 7)\text { and } \vec{C} (8, 3)$ is
\begin{enumerate}
	\item $(0, 1)$
	\item $(0, –1)$
	\item $ (–1,0)$
	\item$(1, 0)$
\end{enumerate}
\item Points $\vec{A}(4,3), \vec{B}(6,4),\vec{C}(5,-6)$  and  $\vec{D}(-3,5)$ are the vertices of a parallelogram.
\end{enumerate}
