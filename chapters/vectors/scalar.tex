\begin{enumerate}[label=\thesubsection.\arabic*,ref=\thesubsection.\theenumi]
\item Find the angle between two vectors $\overrightarrow{a}$ and $\overrightarrow {b} $ with magnitudes $\sqrt{3}$ and 2 respectively having $\overrightarrow {a}\cdot\overrightarrow {b}=\sqrt{6}$.
		\label{prob:12/10/3/1/inner}
	\\
	\solution
		From the given information,
\begin{align}
\norm{\vec{a}}=\sqrt{3},
\norm{\vec{b}}= 2,
{\vec{a}^{\top}}{\vec{b}}=\sqrt{6}  
\\
\implies \cos\theta=\frac{{\vec{a}^{\top}}{\vec{b}}}{\norm{\vec{a}}\norm{\vec{b}}}
=\frac{1}{\sqrt{2}}\\
	\text{or, }\theta={45}\degree
\end{align}

\item Find the angle between the the vectors $\hat{i}-2\hat{j}+3\hat{k}$ and $3\hat{i}-2\hat{j}+\hat{k}$.
	\\
	\solution
		Let 
\begin{align}
	\vec{a} = \myvec{1\\-2\\3} , \vec{b} = \myvec{3\\ -2 \\ 1},
\end{align}
		From problem \ref{prob:12/10/3/1/inner},
\begin{align}
\cos\theta=\frac{\vec{a}^{\top}\vec{b}}{\norm{\vec{a}}\norm{\vec{b}}}
	= \frac{10}{\sqrt{14}\times \sqrt{14}}= \frac{5}{7}
\end{align}

\item Evaluate the product $(3\overrightarrow {a}-5\overrightarrow {b})\cdot (2\overrightarrow {a}+7\overrightarrow {b})$.
	\\
	\solution
		\begin{multline}
    \brak{3\vec{a}-5\vec{b}}^{\top}\brak{2\vec{a}+7\vec{b}}= 3\vec{a}^{\top}\brak{2\vec{a}+7\vec{b}} - 5\vec{b}^{\top}\brak{2\vec{a}+7\vec{b}}
    \\
     =6\norm{\vec{a}}^2-35\norm{\vec{b}}^2+11\vec{a}^{\top}\vec{b}
\end{multline}

\item If the vertices $A,B,C$ of a triangle $ABC$ are (1,2,3), (-1,0,0), (0,1,2), respectively, then find  $\angle{ABC}$.
	\\
	\solution
		From the given information, 
\begin{align}
\vec{A} - \vec{B} &= \myvec{2\\2\\3},
\vec{C} - \vec{B} = \myvec{1\\1\\2}\\
	\implies \angle{ABC} &= \cos^{-1}{\frac{\brak{\vec{A} -\vec{B}}^{\top}\brak{\vec{C}-\vec{B}}}{\norm{\vec{A} -\vec{B}}  \norm{\vec{C}-\vec{B}}}}\\
&= \cos^{-1}{\frac{10}{\sqrt{102}}}\\
\end{align}




	\item The slope of a line is double of the slope of another line. If tangent of the angle between them is 1/3, find the slopes of the lines.
\label{chapters/11/10/1/11}
\\
\solution 
The direction vectors of the lines can be expressed as
\begin{align}
\vec{m}_1=\myvec{1\\m},
\vec{m}_2=\myvec{1\\2m}
\end{align}
If the angle between the lines be $\theta$,
\begin{align}
\tan \theta = \frac{1}{3}
\implies \cos \theta=\frac{3}{\sqrt{10}}
\end{align}
Thus,
\begin{align}
	\frac{3}{\sqrt{10}} = \frac{\vec{m}_1^\top \vec{m}_2}{\norm{\vec{m}_1}\norm{\vec{m}_2}}
	\\
	= \frac{2m^2 +1}{\sqrt{m^2 + 1}\sqrt{4m^2 + 1}}
	\\
	\implies \frac{9}{10}=\frac{4m^4 + 4m^2 +1}{4m^4 + 5m^2 +1}
\\
	\text{or, } 4m^4 - 5m^2 +1 = 0
\end{align}
yielding
\begin{align}
m=\pm \frac{1}{2}, 
\pm 1
\end{align}

\item    Find angle between the lines, $\sqrt{3}x+y=1$ and $x+\sqrt{3}y$=1.
\label{chapters/11/10/3/9}
\\
   \solution 
From    the given equations, the normal vectors can be expressed as
   \begin{align}
	   \vec{n}_1=\myvec{\sqrt{3}\\1},\,
	   \vec{n}_2=\myvec{1\\\sqrt{3}}
   \end{align}
The angle between the lines can then be expressed as
\begin{align}
	\cos\theta=\frac{\vec{n}_1^T\vec{n}_2}{\norm{\vec{n}_1}\norm{\vec{n}_2}}
	=\frac{\sqrt{3}}{2} 
	\\
	\text{or, }
\theta=30\degree
\end{align}

\item Find the angle between the vectors $2\hat{i}-\hat{j}+\hat{k}$ and $3\hat{i}+4\hat{j}-\hat{k}$.
\item The angles between two vectors $\vec{a}, \vec{b}$ with magnitude $\sqrt{3}, 4$ respectively, and $\vec{a} \cdot \vec{b}= 2\sqrt{3}$ is
	\begin{enumerate}
\item $\frac{\pi}{6}$
\item $\frac{\pi}{3}$
\item $\frac{\pi}{2}$ 
\item $\frac{5\pi}{2}$
\end{enumerate}
\item Find the angle between the lines 
\begin{align}
	\overrightarrow{r}&=3\hat{i}-2\hat{j}+6\hat{k}+\lambda(2\hat{i}+\hat{j}+2\hat{k})
	\text{ and}
	\\
	\overrightarrow{r}&=(2\hat{j}-5\hat{k})+\mu(6\hat{i}+3\hat{j}+2\hat{k})
\end{align}
%
\solution  The given lines can be expressed  in the form 
of 
	\eqref{eq:param-form}
	as
\begin{align}
	\vec{x} = \myvec{3 \\ -2 \\ 6} + \kappa_1 \myvec{2 \\ 1 \\ 2}
	\\
	\vec{x} = \myvec{0 \\ 2 \\ -5 } + \kappa_2 \myvec{6 \\ 3 \\ 2}
\end{align}
From the above, it is obvious that the direction vectors of the two lines are
\begin{align}
\vec{m}_1 =\myvec{2 \\ 1 \\ 2},\
	\vec{m}_2=\myvec{6 \\ 3 \\ 2}
\end{align}
	From \eqref{eq:angle-inner}, the angle between the two lines is  obtained as
\begin{align}
	\cos \theta = \frac{19}{21}
\end{align}
\item The vectors $\vec{a}=3\hat{i}-2\hat{j}+2\hat{k}$ $\text{ and }$ $\vec{b}=\hat{i}-2\hat{k}$ are the adjancent sides of a parallelogram. The acute angle between its diagonals is \rule{1cm}{0.15mm}.
\item The sine of the angle between the straight line 
\begin{align}
	\frac{x-2}{3}=\frac{y-3}{4}=\frac{z-4}{5} 
\end{align}
and the plane  
\begin{align}
2x-2y+z=5
\end{align}
is
\begin{enumerate}
	\item $\frac{10}{6\sqrt{5}}$ 
	\item $\frac{4}{5\sqrt{2}}$
	\item $\frac{2\sqrt{3}}{5}$
	\item $\frac{\sqrt{2}}{10}$
\end{enumerate}
\solution The given line can be expressed in the form 
	\eqref{eq:param-form}
	as
\begin{align}
	\vec{x} = \myvec{2 \\ 3 \\ 4} + \kappa_1 \myvec{3 \\ 4 \\ 5}
\end{align}
Hence the direction vector of this line is 
\begin{align}
\myvec{3 \\ 4 \\ 5}
\end{align}
	From \eqref{eq:normal-form}, the normal vector of the given plane is 
\begin{align}
\myvec{2 \\ -2 \\ 1}
\end{align}
Thus, the cosine of the angle between the two is 
obtained from \eqref{eq:angle-inner} as
\begin{align}
	\frac{\sqrt{2}}{10},
\end{align}
which is sine of the angle between the plane and the line.
\item The plane $2x-3y+6z-11=0$ makes an angle $\sin^{-1}(\alpha)$ with x-axis. The value of $\alpha$ is equal to 
\begin{enumerate}
	\item  $\frac{\sqrt{3}}{2}$
	\item  $\frac{\sqrt{2}}{3}$
	\item  $\frac{2}{7}$
	\item  $\frac{3}{7}$
\end{enumerate}
\item Find the angle between the vectors $2\hat{i}-\hat{j}+\hat{k}$ $\text{and}$ $3\hat{i}+4\hat{j}-\hat{k}$.
\item The angles between two vectors $\vec{a}$ $\text{and}$ $\vec{b}$ with magnitude $\sqrt{3}$ $\text{ and }$ 4, respectively, and $\vec{a}$, $\vec{b}$= $2\sqrt{3}$ is
	\begin{enumerate}
\item $\frac{\pi}{6}$
\item $\frac{\pi}{3}$
\item $\frac{\pi}{2}$ 
\item $\frac{5\pi}{2}$
\end{enumerate}

\item The angle between the line 
\begin{align}
	\overrightarrow{r}=(5\hat{i}-\hat{j}-4\hat{k})+\lambda(2\hat{i}-\hat{j}+\hat{k})
\end{align}
	and the plane 
\begin{align}
	\overrightarrow{r} \cdot (3\hat{i}-4\hat{j}-\hat{k})+5=0
\end{align}
	is $\sin^{-1}\brak{\frac{5}{2\sqrt{91}}}$.
\item The angle between the planes 
\begin{align}
	\overrightarrow{r} \cdot (2\hat{i}-3\hat{j}+\hat{k})&=1 
	\text{ and }
	\\
	\overrightarrow{r} \cdot (\hat{i}-\hat{j})&=4  
\end{align}
is
	$\cos^{-1} \brak{\frac{-5}{\sqrt{58}}}$.
\item Find the angle between the lines 
\begin{align}
	y&=(2-\sqrt{3})(x+5)\text{ and }
	\\
	y&=(2+\sqrt{3})(x-7).
\end{align}
\item The unit vector normal to the plane $x+2y+3z-6=0$ is $\frac{1}{\sqrt{14}}\hat{i} + \frac{2}{\sqrt{14}}\hat{j} + \frac{3}{\sqrt{14}}\hat{k}$.
\item The scalar product of the vector $\hat{i}+\hat{j}+\hat{k}$ with a unit vector along the sum of vectors $2\hat{i}+4\hat{j}-5\hat{k}$ and $\lambda\hat{i}+2\hat{j}+3\hat{k}$ is equal to one. Find the value of $\lambda$.
\end{enumerate}
