Find the equation of line 
\begin{enumerate}[label=\thesubsection.\arabic*,ref=\thesubsection.\theenumi]
	\item passing through the point $\vec{P}(– 4, 3)$ with slope $\frac{1}{2}$.
\label{chapters/11/10/2/2}
\\
\solution
			From \eqref{eq:line-school-normal},
\begin{align}
\vec{n}\equiv \myvec{\frac{1}{2}\\ -1}
\implies \myvec{\frac{1}{2}&-1}{\vec{x}}&=-5
\end{align}
using \eqref{eq:geo-normal}.
See 
		\figref{fig:chapters/11/10/2/2/Figure}.
\begin{figure}[H]
\centering
\includegraphics[width=0.75\columnwidth]{chapters/11/10/2/2/figs/fig.pdf}
\caption{}
		\label{fig:chapters/11/10/2/2/Figure}
\end{figure}

	\item passing through $\myvec{0\\0}$ with slope $m$.\\
\label{chapters/11/10/2/3}
\solution
\begin{align}
\because			\vec{n} =\myvec{m \\ -1},
		\end{align}
		the desired equation is 
		\begin{align}
			\myvec{m & -1}\brak{\vec{x}-\myvec{0\\0}} &=0\\
\implies			\myvec{m & -1}\vec{x} &= 0
		\end{align}

    \item passing through 
    $\vec{A} = \myvec{2\\2\sqrt{3}}$ and inclined with the x-axis at an angle 
    of 75\textdegree.
\label{chapters/11/10/2/4}
\\
    \solution 
    \begin{align}
	    \vec{n} &= \myvec{-1\\2+\sqrt{3}}
        \label{eq:11/10/2/4normal-vec}
	\\
	    \implies
        \implies \vec{n}^\top\vec{x} = \vec{n}^\top\vec{A} &= 4\brak{\sqrt{3}+1} \\
        \implies \myvec{-1&2+\sqrt{3}}\vec{x} &=\myvec{-1&2+\sqrt{3}}\myvec{2\\2\sqrt{3}}  
	    \\
	    &= 4\brak{\sqrt{3}+1}
        \label{eq:11/10/2/4line}
    \end{align}
is the desired equation.  See \figref{fig:11/10/2/4line}.
    \begin{figure}[!ht]
        \centering
        \includegraphics[width=\columnwidth]{chapters/11/10/2/4/figs/line.png}
        \caption{}
        \label{fig:11/10/2/4line}
    \end{figure}

\item intersecting the x-axis at a distance of 3 units to the left of origin with slope of -2.
\label{chapters/11/10/2/5}
\\
\solution 
		From the given information,
\begin{align}		
	\vec{A}=\myvec{-3\\0},\,
\vec{n} = \myvec{2 \\1}.
\end{align}
The desired equation of the line is
\begin{align}
\implies	\myvec { 2 & 1 } \brak{ \vec{x} - \myvec{ -3 \\ 0}} &= 0  \\
	\text{or, }	\myvec{ 2 & 1} \vec{x}  &= -6
        \label{eq:chapters/11/10/2/5/1}
\end{align}
See \figref{fig:chapters/11/10/2/5/Fig1}.
\begin{figure}[H]
	\begin{center}
		\includegraphics[width=0.75\columnwidth]{chapters/11/10/2/5/figs/fig.pdf}
	\end{center}
\caption{}
\label{fig:chapters/11/10/2/5/Fig1}
\end{figure}


\item intersecting the y-axis at a distance of 2 units above the origin and making an
angle of $30\degree$ with positive direction of the x-axis.
\\
\solution 
\begin{align}
    \vec{n} =  \myvec{-\frac{1}{\sqrt{3}} \\ 1},
    \vec{A} = \myvec{0 \\ 2}.
\end{align}
Hence, 
the equation of the line is given by
\begin{align}
\myvec{-\frac{1}{\sqrt{3}}&1}\brak{ \vec{x} - \myvec{0 \\ 2}} &= 0  \\
    \text{or, }	\myvec{-\frac{1}{\sqrt{3}}&1} \vec{x}  &= 2
\end{align}
%
\iffalse
See
    \figref{fig:chapters/11/10/2/6/line}.
\begin{figure}[H]
    \centering
    \includegraphics[width=0.75\columnwidth]{chapters/11/10/2/6/figs/fig.pdf}
    \caption{}
    \label{fig:chapters/11/10/2/6/line}
\end{figure}
\fi


\item Find the equation of the line passing through the points $\vec{A}\myvec{-1\\1}$ and $\vec{B}\myvec{2\\-4}$.
\label{chapters/11/10/2/7}
\\
\solution 
\begin{align}
	\vec{m} &= \vec{A} - \vec{B}
= \myvec{-3\\5}
\implies
\vec{n} &= \myvec{5\\3}
\end{align}
Thus, the equation of line is
\begin{align}
 \myvec{ 5 & 3}\vec{x}  &= -2
\end{align}
See 
   \figref{fig:chapters/11/10/2/7/Line_AB}.
\begin{figure}[h!]
  \centering
   \includegraphics[width=\linewidth]{chapters/11/10/2/7/figs/Figure_1.png}
   \caption{}
   \label{fig:chapters/11/10/2/7/Line_AB}
\end{figure}





\item 
The vertices of triangle $PQR$ are $\vec{P}(2,1), \vec{Q}(-2,3), \vec{R}(4,5)$. Find the equation of the median through $\vec{R}$.
\label{chapters/11/10/2/9}
\\
\solution
	\begin{figure}[!ht]
		\centering
 \includegraphics[width=\columnwidth]{chapters/11/10/2/9/figs/line.png}
		\caption{}
		\label{fig:11/10/2/9}
  	\end{figure}
	See Fig. 
		\ref{fig:11/10/2/9}.
Using section formula, the mid point of $PQ$ is
\begin{align}
\vec{A} = \frac{\vec{P} +\vec{Q} }{2}
	= {\myvec{0\\2}}
\end{align} 
So, the direction vector of $AR$ is 
\begin{align}
	\vec{m} ={\vec{R} - \vec{A}}
= \myvec{4 \\ 3}
\\
	\implies \vec{n} = 
 \myvec{3 \\ -4}
\end{align}
which is the normal vector.  Thus,
the equation of the line is 
\begin{align}
	\myvec{3 & -4}\brak{\vec{x} - \vec{R}} = 0
	\\
	\implies 
	\myvec{3 & -4}\vec{x} =8 
\end{align}

\item Find the equation of line  drawn perpendicular to the line $\frac{x}{4}+\frac{y}{6}=1$ through the point where it meets the y-axis \\
\solution
				The given line
parameters are
\begin{align}
		\vec{n} = \myvec{3\\2},\, c=12 ,\,
	\vec{m} =\myvec{-2 \\ 3}.
\end{align}
and the point on the y-axis is
\begin{align}
	\vec{A} =\myvec{0\\6}.
\end{align}
Thus, the equation of the desired line is 
\begin{align}
	\vec{m}^\top\brak{\vec{x}-\vec{A}}&=0\label{eq:chapters/11/10/4/7/5}
	\\
\implies
			\myvec{-2 & 3}\vec{x} &=-18
		\end{align}
		\iffalse
		See 
  \figref{fig:chapters/11/10/4/7/Figure}.
\begin{figure}[H]
\includegraphics[width=0.75\columnwidth]{chapters/11/10/4/7/figs/fig.png}
\caption{}
  \label{fig:chapters/11/10/4/7/Figure}
\end{figure}
\fi

\item Find the equation of line whose perpendicular distance from the origin is 5 units and the angle made by the perpendicular with the positive $x$-axis is $30\degree$.
\label{chapters/11/10/2/8}
\\
\solution
			From 
\eqref{eq:chapters/11/10/2/8-final},
		Thus, the equation of lines are
\begin{align}
	\myvec{\frac{\sqrt{3}}{2}& \frac{1}{2}}\vec{x}=\pm5
\end{align}
\iffalse
See 
\figref{fig:chapters/11/10/2/8/Fig1}.
\begin{figure}[H]
\begin{center}
\includegraphics[width=0.75\columnwidth]{chapters/11/10/2/8/figs/fig.pdf}
\end{center}
\caption{}
\label{fig:chapters/11/10/2/8/Fig1}
\end{figure}
\fi

\item 
	Find the equation of the line passing through  (-3,5) and perpendicular to the line through the points (2,5) and (-3,6).
	\\
	\solution 
\label{chapters/11/10/2/10}
See 
		\figref{fig:11/10/2/10}.
	\begin{figure}[H]
		\centering
 \includegraphics[width=0.75\columnwidth]{chapters/11/10/2/10/figs/fig.pdf}
		\caption{}
		\label{fig:11/10/2/10}
  	\end{figure}
The normal vector is
\begin{align}
\vec{n} =\myvec{2 \\5} -  \myvec{-3 \\ 6} 
=\myvec{
    5\\
    -1
}
\end{align}
Thus, the equation of the line is 
\begin{align}
\myvec{
    5 &-1
	}\brak{\vec{x} - \myvec{-3 \\5}}
= 0
\\
\implies 
\myvec{
    5 &-1
	}\vec{x} 
= -20
\end{align}

\item 
A line perpendicular to the line segment joining the points $\vec{P}(1,0)$ and $\vec{Q}(2,3)$ divides it in the ratio $1:n$. Find the equation of the line.
	\\
	\solution 
\label{chapters/11/10/2/11}
The direction vector of 
$PQ$ is 
\begin{align}
     \vec{Q
 }-  \vec{P
 }
=
     \myvec{
  1\\
  3
 }
\end{align}
 Using section formula, 
 \begin{align}
	 \vec{R}=\frac{\vec{Q}+n\vec{P}}{1+n}
\end{align}
is the point of intersection.
The 
equation of the desired line  is
\begin{align}
	\vec{m}^{\top}\brak{\vec{x}-\vec{R}}=0
\\
\implies 
	   \myvec{
		   1 &  3}\vec{x}
	   &= \myvec{
  1\ 3}\myvec{
  \frac{2+n}{1+n}\\
  \frac{3}{1+n}} 
  \\
	=	  \frac{11+n}{1+n} 
\end{align}
\iffalse
See
		\figref{fig:11/10/2/11}.
	\begin{figure}[H]
		\centering
 \includegraphics[width=0.75\columnwidth]{chapters/11/10/2/11/figs/linefig.pdf}
		\caption{}
		\label{fig:11/10/2/11}
  	\end{figure}
	\fi

\item Find the equation of a line that cuts off equal intercepts on the coordinate axes and passes through the point $(2,3)$.  
	\\
\solution 
\label{chapters/11/10/2/12}
Let $(a,0)$  and  $(0,a)$ be the intercept points. 
\begin{align}
\vec{m} 
        &=   \myvec{
		a \\
		0 
		} - \myvec{
		   0 \\
		   a
		}  
        		  \equiv \myvec{
                           1 \\
			   -1 
		         } 
			 \\
			 \implies
\vec{n} &=  \myvec{
		     1 \\
		     1
	     } 
\end{align}
and 
the equation of the  line is
\begin{align}
	\myvec { 1 & 1 } \brak{ \vec{ x  - \myvec{ 2 \\
                                   3
			     }
		}}  &= 0  \\
\implies		\myvec{ 1 & 1} \vec{x}  &= 5 
        \label{eq:11/10/2/12/1}
\end{align}
See  \figref{fig:11/10/2/12/Fig1}.
\begin{figure}[H]
	\begin{center}
		\includegraphics[width=0.75\columnwidth]{chapters/11/10/2/12/figs/fig.pdf}
	\end{center}
\caption{}
\label{fig:11/10/2/12/Fig1}
\end{figure}


\item 
Find the equation of a line passing trough a point (2,2) and cutting off intercepts on the axes whose sum is 9.
	\\
	\solution 
\label{chapters/11/10/2/13}
Let  the intercept points be
\begin{align}
{\vec{P}}=\myvec{
  a\\
  0}
 , {\vec{Q}}=\myvec{
  0\\
  b}
  \text{ and }
   {\vec{R}}=\myvec{
  2\\
  2}
\end{align}
be the given point.  
Forming the collinearity matrix from 
		\eqref{prop:lin-dep-rank},
\begin{align}
	\myvec{ \vec{P}-\vec{Q} &\vec{P}-\vec{R}} 
	=
	 \myvec{
  a & a-2\\
  -b & -2
 }
\end{align}
which is singular if 
\begin{align}
 ab -2\brak{a+b} = 0
 \implies ab = 18
		\label{eq:11/10/2/13-a+b}
		\\
\because  a + b = 9.
\end{align}
$\therefore a,b$
are the roots of
\begin{align}
	x^2 -9x +18 = 0.
\end{align}
yielding
\begin{align}
	\myvec{a \\ b} = \myvec{6 \\ 3}, \myvec{3\\6}
\end{align}
Since 
\begin{align}
	\vec{m} = \myvec{a \\ -b},
	\vec{n} = \myvec{b \\ a} \equiv \myvec{1 \\ 2}, \myvec{2\\1}
\end{align}
Thus, the possible equations of the line are 
\begin{align}
\myvec{1 & 2}\vec{x} = 6
	\\
	\myvec{2&1}\vec{x} = 6
\end{align}
		See \figref{fig:11/10/2/13}.
	\begin{figure}[H]
		\centering
 \includegraphics[width=0.75\columnwidth]{chapters/11/10/2/13/figs/fig.pdf}
		\caption{}
		\label{fig:11/10/2/13}
  	\end{figure}

\item 
	Find the equation of the line through the point (0,2) making an angle $\frac{2\pi}{3}$ with the positive X-axis. Also find the equation of the line parallel to it and crossing the Y-axis at a distance of 2 units below the origin.
	\\
	\solution
\label{chapters/11/10/2/14}
The equation of the first line is 
\begin{align}
	\myvec{\sqrt{3} &1}\myvec{\vec{x}-\myvec{0\\2}}&=0
	\\
	\implies 
	\myvec{\sqrt{3}&1}
	\vec{x}&=2
\end{align}
The equation of the second line is 
\begin{align}
	\myvec{\sqrt{3} &1}\myvec{\vec{x}-\myvec{0\\-2}}&=0
	\\
	\implies 
	\myvec{\sqrt{3}&1}
\vec{x}=-2
\end{align}
See
		\figref{fig:11/10/2/14}.
	\begin{figure}[H]
		\centering
 \includegraphics[width=0.75\columnwidth]{chapters/11/10/2/14/figs/fig.pdf}
		\caption{}
		\label{fig:11/10/2/14}
  	\end{figure}

\item 
	The perpendicular from the origin to a line meets it at the point $(-2,9)$. Find the equation of the line.
\label{chapters/11/10/2/15}
	\\
	\solution
It is obvious that the normal vector to the line is 
\begin{align}
\vec{n} =\myvec{2 \\ -9} -\vec{0} 
=\myvec{2 \\ -9}
\end{align}
Hence, the equation of the line is 
\begin{align}
	\myvec{2 & -9}\brak{\vec{x} - \myvec{2 \\ -9}}&= 0
	\\
	\implies 
	\myvec{2 & -9}\vec{x} &= 85
\end{align}
See 
		\figref{fig:11/10/2/15}.
	\begin{figure}[H]
		\centering
 \includegraphics[width=0.75\columnwidth]{chapters/11/10/2/15/figs/fig.pdf}
		\caption{}
		\label{fig:11/10/2/15}
  	\end{figure}

\item 
$P(a,b)$ is the mid-point of the line segment between axes. Show that the equation of the line is $\frac{x}{a}+\frac{y}{b}=2$
\label{chapters/11/10/2/18}
\\
\solution
From \probref{chapters/11/10/2/13},
\begin{align}
	\vec{n} = \myvec{b \\ a}
\\	
\implies	\myvec{b & a} \brak{\vec{x}-\myvec{a\\b}} &= 0\\
	\text{or, }	\myvec{b & a}\vec{x} &= 2ab.
\end{align}
is the desired line equation.



\item Point $\vec{R}\brak{h, k}$ divides a line segment between the axes in the ratio 1: 2. Find the equation of the line.
\label{chapters/11/10/2/19}
	\\
	\solution 
Choosing the intercept points in \probref{chapters/11/10/2/13},
\begin{align}
\vec{R} &= \frac{2\vec{A} + \vec{B}}{3} 
\implies
\myvec{h\\k} = \frac{1}{3}\myvec{2a\\b} \\
	\text{or, }
\myvec{b\\a} 
	&= \vec{n}  \equiv \myvec{2k\\h}
\end{align}
%
Thus, the equation of the line is given by,
\begin{align}
\myvec{2k&h}\vec{x} = \myvec{2k&h}\myvec{h\\k}= 3hk
\end{align}





\item Find the equation of the line  parallel to the line 3x-4y+2=0 and passing through the point (-2,3).
\label{chapters/11/10/3/7}
\\
\solution 
\begin{align}
	\myvec{3&-4}\vec{x}=\myvec{3&-4}\myvec{-2\\3}
	=-18 
\end{align}
is the required equation of the line.

\item Find the equation of line perpendicular to the line $x-7y+5=0$ and having $x$ intercept $3$\\
\label{chapters/11/10/3/8}
\solution
The desired equation is
		\begin{align}
			\myvec{7 & 1}\brak{\vec{x}-\myvec{3\\0}} &=0\\
		\implies 	\myvec{7 & 1}\vec{x} &= 21
		\end{align}
		\iffalse
		See 
\figref{fig:chapters/11/10/3/8/Fig1}.
		\begin{figure}[H]
\begin{center}
\includegraphics[width=0.75\columnwidth]{chapters/11/10/3/8/figs/fig.pdf}
\end{center}
\caption{}
\label{fig:chapters/11/10/3/8/Fig1}
\end{figure}
\fi

\item Prove that the line through the point $(x_1,y_1)$ and parallel to the line $Ax+By+C=0$ is $A(x-x_1)+B(y-y_1)=0$.
\label{chapters/11/10/3/11}
\\
\solution
The equation of the desired line is
\begin{align}
	\myvec{A &B}\brak{\vec{x}-\myvec{x_1\\y_1}}&=0\\
	\implies 
	\myvec{A &B}\vec{x} &= Ax_1+By_1
\end{align}

	\item Find the equation of the line passing through the point $\brak{1,2,-4}$ and perpendicular to the two lines
\begin{align}
	\frac{x-8}{3}=\frac{y+19}{-16}=\frac{z-10}{7} \text{ and }\\ \frac{x-15}{3}=\frac{y-29}{8}=\frac{z-5}{-5} 
\end{align}
    \solution
		The direction vector of the desired line 
is given by 
\begin{align*}
	\myvec{3 & -16 & 7\\3 & 8 & -5}\vec{m} = 0
	\xleftrightarrow[]{R_2\leftarrow R_2-R_1}
 	\myvec{3 & -16 & 7\\0 & 24 & -12}
	\\
	\xleftrightarrow[]{R_1\leftarrow R_1+\frac{2}{3}R_2}
	\myvec{3 & 0 & -1\\0 & 24 & -12}
	\xleftrightarrow[]{R_2\leftarrow R_2/12}
	\myvec{3 & 0 & -1\\0 & 2 & -1}
\end{align*}
yielding
\begin{align}
	\vec{m} = \myvec{2\\3\\6}
\end{align}
Hence the vector equation of the line passing through $\brak{1,2,-4}$ is,
\begin{align}
	\vec{x} = \myvec{1\\2\\-4} + \kappa \myvec{2\\3\\6}
\end{align}



	\item  Find the vector equation of the line passing through $\myvec{1&2&3}^{\top}$ and parallel to the planes $\myvec{1&-1&2}\vec{x} = 5$ and $\myvec{3&1&1}\vec{x} = 6$.  
		\\
    \solution
		The direction vector of the line  is given by 
\begin{align*}
 \myvec{1&-1&2 \\ 3&1&1}\vec{m} = 0
 \xleftrightarrow[]{R_2\rightarrow -\frac{3}{4}{R_1} + \frac{1}{4}{R_2}} \myvec{1&-1&2 \\ 0&1&-\frac{5}{4}}\\
   \myvec{1&-1&2 \\ 0&1&-\frac{5}{4}} \xleftrightarrow[]{R_1\rightarrow {R_1} + {R_2}} \myvec{1&0&\frac{3}{4} \\ 0&1&-\frac{5}{4}}
   \\
 \implies \vec{m} =\myvec{-3\\5\\4}
\end{align*}
$\therefore$ the equation of the line is
\begin{align}
    \vec{x} = \myvec{1\\2\\3} + \lambda\myvec{-3\\5\\4} 
\end{align}


	\item
 Two lines passing through the point (2,3) intersect each other at an angle of $60\degree$. If slope of one line is 2, find the equation of the other line.
\label{chapters/11/10/3/12}
 \\
 \solution
		Using the scalar product
\begin{align}
  \cos{60\degree}=
\frac{1}{2}=\frac{\myvec{
        1&2
    }\myvec{
        1\\m
    }}{\sqrt{5}\sqrt{m^2+1}}\\
\implies 11m^2+16m-1=0\\
   or, m=\frac{-8\pm5\sqrt{3}}{11}    
\end{align}
So, the desired equation of the line is
\begin{align}
\myvec{
    \frac{-8\pm5\sqrt{3}}{11}&-1
}\vec{x} 
	&=
\myvec{
    \frac{-8\pm5\sqrt{3}}{11}&-1
}
\myvec{
    2\\3}
    \\
	&=\frac{-49\pm16\sqrt{3}}{11}
\end{align}
\iffalse
See 
    \figref{fig:11.10.3.12}.
\begin{figure}[H]
    \centering
    \includegraphics[width=0.75\columnwidth]{chapters/11/10/3/12/fig/asgnt1.png}
    \caption{}
    \label{fig:11.10.3.12}
\end{figure}
\fi


\item
Find the value of $p$ so that the three lines $3x+y-2=0,px+2y-3=0$ and $2x-y-3=0$ may intersect at one point.
\label{11.10.4.9}
\\
\solution
Performing row operations
on the matrix
\begin{align*}  
\myvec{
    3 &1&-2 \\
     p&2&-3\\
     2&-1&-3
}
\xrightarrow[R_3=3R_3-2R_1]{R_2=3R_2-pR_1}&\myvec{
    3&1&-2\\
     $0$&6-p&-9+2p\\
     0&-5&-5}\\
 \xrightarrow{R_3=R_3(6-p)+5R_2}&\myvec{
    3&1&-2\\
     0&6-p&-9+2p\\
     0&0&-75+15p}
     \\
  \implies 
    p=5
\end{align*}
    See \figref{fig:11.10.4.9}.
\begin{figure}[H]
    \centering
    \includegraphics[width=0.75\columnwidth]{chapters/11/10/4/9/fig/11.10.4.9.png}
    \caption{}
    \label{fig:11.10.4.9}
\end{figure}


 \item The perpendicular from the origin to the line $y=mx+c$ meets it at the point $(-1,2)$. Find the values of m and c.
 \label{11.10.3.15}
	 \\
 \solution
 From \probref{chapters/11/10/2/15},
\begin{align}
	\vec{n} = \myvec{-1\\2} \implies m = \frac{1}{2}
\end{align}
Also, from the given equation of the line and the given point, 
\begin{align}
	c = \myvec{-m & 1}\myvec{-1\\2} = 
\frac{5}{2}  
\end{align}
\iffalse
 See \figref{fig:pic}.
\begin{figure}[H]
 \centering
\includegraphics[width=0.75\columnwidth]{chapters/11/10/3/15/figs/graph.jpg}
 \caption{Graph}
 \label{fig:pic}
\end{figure}
\fi

\item Find the equation of the lines through the point (3, 2) which make an angle of $45\degree$  with the line $x – 2y = 3$.
\label{chapters/11/10/4/11}\\
\solution
Following the approach in \probref{chapters/11/10/3/12},
\begin{align}
\cos45\degree =
\frac{1}{\sqrt{2}} = \frac{\myvec{2 & 1} \myvec{1\\m}}{\norm{\myvec{2\\1}}\norm{\myvec{1\\m}}}
\\
\implies 
 3m^2 - 8m -3 = 0
 \\
\text{or, }
m= - \frac{1}{3}, 3
\end{align} 
Thus, the desired equations are 
\begin{align}
	\myvec{1&3}\cbrak{\vec{x}-\myvec{3\\2}}&=0\\
 \implies 	\myvec{1 & 3}\vec{x} &= 9
\end{align}
and 
\begin{align}
	\myvec{3&-1}\cbrak{\vec{x}-\myvec{3\\2}}&=0\\
		\implies 	\myvec{3 & -1}\vec{x} &= 7
\end{align}
See
\figref{fig:chapters/11/10/4/11/figs/strline.jpg}.
\begin{figure}[H]
\centering
\includegraphics[width=0.75\columnwidth]{chapters/11/10/4/11/figs/fig.pdf}
\caption{}
\label{fig:chapters/11/10/4/11/figs/strline.jpg}
\end{figure}

\item Consider the following population and year graph. Find the slope of the line AB and using it, find what will be the population in the year 2010.
\\
\begin{figure}[!ht]
\centering
\includegraphics[width = \columnwidth]{chapters/11/10/1/14/figs/fig.png}
\caption{}
\label{fig:chapters/11/10/1/14/1}
\end{figure}
\solution
The direction vector of the line in \figref{fig:chapters/11/10/1/14/1} is
\begin{align}
\vec{m} = \vec{B} - \vec{A}
= \myvec{2 \\ 1}
\\
\implies \vec{n}
= \myvec{1 \\ -2}
\end{align}
 The equation of the line is then given by 
\begin{align}
\vec{n}^{\top} (\vec{x} -\vec{A}) &= 0 \\
\implies 
\myvec{1& -2} \vec{x} &= 1801
\\
\implies  \myvec{1&-2} \myvec{2010\\y} &= 1801 \\
\implies y &= \frac{209}{2}
\end{align}





\item  Find the vector equation of the line which is parallel to the vector $3\hat{i}-2\hat{j}+6\hat{k}$ and which passes through the point $(1,-2,3)$.
\item Find the equations of the two lines through the origin which intersect the line $ \dfrac{x-3}{2}=\dfrac{y-3}{1}=\dfrac{z}{1}$ at angles of  $\dfrac{\pi}{3}$each.
\item Find the equations of the line passing through the point $(3,0,1)$ and parallel to the planes $x+2y=0$ and $3y-z=0.$
\item The vector equation of the line $\dfrac{x-5}{3}=\dfrac{y+4}{7}=\dfrac{z-6}{2}$ is \noindent\rule{2cm}{0.4pt}. 
\item The vector equation of the line through the points $(3,4,-7)$ and $(1,-1,6)$ is \noindent\rule{2cm}{0.4pt}.
\item the unit vector normal to the plane $x+2y+3z-6=0$ is $\dfrac{1}{\sqrt{14}}\hat{i} + \dfrac{2}{\sqrt{14}}\hat{j} + \dfrac{3}{\sqrt{14}}\hat{k}$.
\item The vector equation of the line $\dfrac{x-5}{3}=\dfrac{y+4}{7}=\dfrac{z-6}{2}$ is
$$\overrightarrow{r}=5\hat{i}-4\hat{j}+6\hat{k}+\lambda(3\hat{i}+7\hat{j}+2\hat{k}).$$
\item The equation of a line, which is parallel to $2\hat{i}+\hat{j}+3\hat{k}$ and which passes through the point $(5,-2,4)$ is $\dfrac{x-5}{2}=\dfrac{y+2}{-1}=\dfrac{z-4}{3}$.
\item  Point $\vec{P}(0,2)$ is the point of intersection of $y$-axis and perpendicular bisector of line segment joining the points $\vec{A}(-1,1) \text{ and } \vec{B}(3,3)$

\item Prove that the line through A$(0,-1,-1)$ and B$(4,5,1)$ intersects the line through C$(3,9,4)$ and D$(-4,4,4)$.
\item Show the lines
$$\frac{x-1}{2}=\frac{y-2}{3}=\frac{z-3}{4}$$
$$\text{ and } \frac{x-4}{5}=\frac{y-1}{2}=z  \text{ intersect }.$$
 Also, find their point of intersection.
\item The area of the region bounded by the curve $y = x + 1$ and the lines $x = 2\text{ and }x = 3$ is
\begin{enumerate}
\item $\frac{7}{2}$ sq units
\item $\frac{9}{2}$ sq units
\item $\frac{11}{2}$ sq units
\item $\frac{13}{2}$ sq units
\end{enumerate}   
\item The area of the region bounded by the curve $x = 2 + 3$ and the $y$ lines $y = 1\text{ and }y = - 1$ is
\begin{enumerate}
\item 4 sq units 
\item $\frac{3}{2}$ sq units
\item 6 sq units
\item 8 sq units
\end{enumerate}
\item Compute the area bounded by the line $x + 2y = 2$, $y - x = 1\text{ and }2x + y = 7$.
\item Find the area bounded by the lines $y = 4x + 5$, $y = 5 - x\text{ and }4y = x + 5$.
\end{enumerate}
