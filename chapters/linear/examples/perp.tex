\begin{enumerate}[label=\thesubsection.\arabic*,ref=\thesubsection.\theenumi]
\item  Reduce the following equations into normal form. Find their perpendicular distances from the origin and angle between perpendicular and the positive $x$-axis.
\label{chapters/11/10/3/3}
\begin{enumerate}
	\item $x-\sqrt{3}y+8=0$ 
	\item $y-2=0$
	\item $x-y=4$
\end{enumerate}
\solution
  See \tabref{tab:11/10/3/3}.
			\eqref{eq:PQ-final} was used for computing the distance from the origin.
			\begin{table}[H]
  \centering
  \begin{tabular}{|c|c|c|c|c|}
    \hline
    & $\vec{n}$ & Angle & $c$& Distance \\
    \hline
    a) & $\myvec{1 \\ -\sqrt{3}}$ & $\tan^{-1}(-\sqrt{3}) = \frac{2\pi}{3}$ &-8 & 4 \\
    \hline
    b) & $\myvec{0 \\ 1}$ & $\tan^{-1}\infty = \frac{\pi}{2}$ &2 & 2 \\
    \hline
    c) & $\myvec{1 \\ -1}$ & $\tan^{-1}(-1) = \frac{3\pi}{4}$ &4 & $2\sqrt{2}$ \\
    \hline
  \end{tabular}
  \caption{}
  \label{tab:11/10/3/3}
\end{table}


 \item  In each of the following cases, determine the direction cosines of the normal to
the plane and the distance from the origin.
\begin{enumerate}
	\item $z=2$ 
	\item $x + y + z = 1$
	\item $2x + 3y – z = 5$
	\item $5y + 8 = 0$
\end{enumerate}
    \solution
		  See 
  \tabref{tab:12/11/3/1}.
			\eqref{eq:PQ-final} was used for computing the distance from the origin.
			\begin{table}[H]
  \centering
  \begin{tabular}{|c|c|c|c|}
    \hline
    & $\vec{n}$ & $c$ & Distance \\
    \hline
    a) &		\myvec{0\\0\\1}  &2  & 2 \\
    \hline
    b) & $\myvec{1\\1\\1}$ & 1 & $\frac{1}{\sqrt{3}}$ \\
    \hline
    c) & $\myvec{2\\3\\-1}$ & 5 & $\frac{5}{\sqrt{14}}$ \\
    \hline
    d) & $\myvec{0\\-5\\0}$ & 8 & $\frac{8}{5}$ \\
    \hline
  \end{tabular}
  \caption{}
  \label{tab:12/11/3/1}
\end{table}
 


\item Find the distance of the point $(-1,1)$ from the line $12\brak{x+6} = 5\brak{y-2}$. 
\label{chapters/11/10/3/4}
	\\
\solution 
\begin{align}
		\vec{n} = \myvec{
	  12 \\
	  -5 
	  } ,   c = -82 
	  \\
	  \implies 
	d 
	= \frac{\abs{  \myvec{12 & -5 }\myvec{-1 \\ 1}-\brak{-82} }}{\sqrt{12^2+\brak{-5}^2}} 	
	= 5
\end{align}
\iffalse
See \figref{fig:11/10/3/4/Fig1}.
\begin{figure}[H]
	\begin{center}
		\includegraphics[width=0.75\columnwidth]{chapters/11/10/3/4/figs/problem4.pdf}
	\end{center}
\caption{}
\label{fig:11/10/3/4/Fig1}
\end{figure}
\fi

\item Find the coordinates of the foot of the perpendicular from $(-1, 3)$ to the line $3x-4y-16=0$.  
\label{chapters/11/10/3/14}
\\
\solution
Substituting
\begin{align}
 \vec{P}=\myvec{
-1\\
3
},
\vec{n}=\myvec{
3\\
-4
}, c=16
\end{align}
in 
	\eqref{eq:11/10/3/4/foot_of_perpendicular},
the desired foot of the perpendicular is then given by 
\begin{align}
\myvec{4&3\\3&-4}\vec{Q}=\myvec{\myvec{4&3}\myvec{-1\\3}\\16}
=\myvec{5\\16}  
\\
\implies
  \myvec{
   4 &  3  & 5\\
   3 & -4  & 16} 
  \xleftrightarrow[]{R_2=R_2-\frac{3}{4}R_1}
  \myvec{
  4 & 3 & 5\\
  0 & \frac{-25}{4} & \frac{49}{4}} 
\\
  \xleftrightarrow{R_2=\frac{-4}{25}}
  \myvec{
  4 & 3 & 5\\
  0 & 1 & \frac{-49}{25}}
  \xleftrightarrow{R_1=\frac{1}{4}R_1}
  \myvec{
  1 & \frac{3}{4} & \frac{5}{4}\\[1ex]
  0 & 1 & \frac{-49}{25}}
\\
  \xleftrightarrow{R_1=R_1-\frac{3}{4}R_2}
  \myvec{
  1 & 0 & \frac{68}{25}\\[1ex]
  0 & 1 & \frac{-49}{25}}          
\implies \vec{Q}=\myvec{
\frac{68}{25}\\[1ex]
\frac{-49}{25}
}
\end{align}
See 
\figref{fig:chapters/11/10/3/14/Fig}.
\begin{figure}[H]
	\begin{center} 
	    \includegraphics[width=0.75\columnwidth]{chapters/11/10/3/14/figs/fig.pdf}
	\end{center}
\caption{}
\label{fig:chapters/11/10/3/14/Fig}
\end{figure}

\item  If ${p}$ and ${q}$ are the lengths of perpendiculars from the origin to the lines ${x}\cos\theta - {y}\sin\theta =  {k}\cos2\theta$ and ${x}\sec\theta + {y}\cosec\theta = {k}$, respectively, prove that ${p}^2 + 4{q}^2 = {k}^2$
\label{chapters/11/10/3/16}
\\
\solution
The line parameters are
\begin{align}
    \vec{n}_1 = \myvec{\cos\theta \\ -\sin\theta},  {c}_1 &= {k}\cos2\theta\\
    \vec{n}_2 = \myvec{\sin\theta \\ \cos\theta},  {c}_2 &= \frac{1}{2}{k}\sin2\theta
\end{align}
			From \eqref{eq:PQ-final},
\begin{align}
    {p} &= \frac{\abs{  \vec{n}_1^{\top}\vec{x}-{c}_1 }}{\norm{\vec{n}_1}} 
    = \abs{{k}\cos2\theta} \\
     {q} &= \frac{\abs{  \vec{n}_2^{\top}\vec{x}-{c}_2 }}{\norm{\vec{n}_2}} 
    = \abs{ \frac{1}{2}{k}\sin2\theta}
    \\
	\implies
	{p}^2 + 4{q}^2 & 
= {k}^2
\end{align}

\item In the triangle $ABC$ with vertices $\vec{A} \brak{2, 3}$, $\vec{B} \brak{4, –1}$ and $\vec{C} \brak{1, 2}$, find the equation and length of altitude from the vertex $\vec{A}$.
\label{chapters/11/10/3/17}
\\
\solution
\begin{enumerate}
\item The normal vector of the altitude from $\vec{A}$ is,
\begin{align}
\vec{m}_{BC}
= \myvec{1\\-1},
\because \vec{n}_{BC} &= \myvec{1\\1}.
\end{align}
The equation of the desired altitude  is given by
\begin{align}
\vec{m}_{BC}^{\top}\vec{x} &=\vec{m}_{BC}^{\top}\vec{A}\\
\implies \myvec{1&-1}\vec{x} &= -1
\end{align}
	\item
The equation of line $BC$ is given by,
\begin{align}
{\vec{n}^{\top}_{BC}}\vec{x} &= {\vec{n}^{\top}_{BC}}\vec{B}\\
\implies \myvec{1&1}\vec{x}  &= 3
\end{align}
			From \eqref{eq:PQ-final},
the length of the desired altitude is 
\begin{align}
d =  \sqrt{2}
\end{align}

\end{enumerate}
\iffalse
See 
\figref{fig:chapters/11/10/3/17/1}.
\begin{figure}[H]
\centering
\includegraphics[width=0.75\columnwidth]{chapters/11/10/3/17/figs/fig.png}
\caption{}
\label{fig:chapters/11/10/3/17/1}
\end{figure}
\fi

\item If $p$ is the length of perpendicular from origin to the line whose intercepts on the axes are $a$ and $b$, then show that 
\begin{align}
	\frac{1}{p^2} = \frac{1}{a^2}+ \frac{1}{b^2}
\label{eq:11/10/3/18}
\end{align}
\label{chapters/11/10/3/18}
\\
\solution
Let the 
intercept points be
\begin{align}
\myvec{a\\0},\myvec{0\\b},
\because	\vec{n} = \myvec{b\\a},
\end{align}
The line equation is,
\begin{align}
\myvec{b & a}\brak{\vec{x} - \myvec{a\\0}} &= 0\\
\implies	\myvec{ b & a}\vec{x} &= ab
\end{align}
			From \eqref{eq:PQ-final},
the perpendicular distance from the origin  to the line is
\begin{align}
	p  
	&= \frac{ab}{\sqrt{a^2+b^2}}
	\implies 
\eqref{eq:11/10/3/18}
\end{align}

\item Find the points on the x-axis, whose distances from the line $\frac{x}{3}+\frac{y}{4}=1$ are 4 units.
\label{chapters/11/10/3/5}
	\\
	\solution
Let the desired point be
\begin{align}
	\vec{P} = x\vec{e}_{1} = \myvec{x\\0}
\end{align}
From the distance formula, 
\begin{align}
	d &= \frac{\abs{\vec{n}^\top\vec{P}-c}}{\norm{\vec{n}}}
	  = \frac{\abs{x\vec{n}^\top\vec{e}_{1}-c}}{\norm{\vec{n}}}
	  \\
	  \implies 
		x &= \frac{\pm d\norm{\vec{n}}+c}{\vec{n}^\top\vec{e}_{1}}
\end{align}
Substituting
\begin{align}
		\vec{n} = \myvec{4\\3} , c = 12,
	d = 4,
	\\
	x = 8,
	 -2
\end{align}
See \figref{fig:11/10/3/5/Fig1}.	
\begin{figure}[H]
	\begin{center} 
	    \includegraphics[width=0.75\columnwidth]{chapters/11/10/3/5/figs/fig.pdf}
	\end{center}
\caption{}
\label{fig:11/10/3/5/Fig1}
\end{figure}



\item What are the points on the y-axis whose distance from the line $\frac{x}{3}+\frac{y}{4}=1$ is 4 units.
\\
\solution
		Following the approach in \probref{chapters/11/10/3/5},
\begin{align}
		y = \frac{\pm d\norm{\vec{n}}+c}{\vec{n}^\top\vec{e}_{2}}
= \frac{32}{3}, \frac{-8}{3}.
\end{align}
\iffalse
See 
		\figref{fig:chapters/11/10/4/4/Figure}.
\begin{figure}[H]
\centering
\includegraphics[width=0.75\columnwidth]{chapters/11/10/4/4/figs/fig.png}
\caption{}
		\label{fig:chapters/11/10/4/4/Figure}
\end{figure}
\fi

\item Find perpendicular distance from the origin to the line joining the points $(\cos\theta,\sin\theta)$ and $(\cos\phi,\sin\phi)$.
\\
\solution
		The equation of the line is
\begin{align}
\myvec{\sin\phi-\sin\theta&\cos\theta-\cos\phi}\vec{x}&=\sin\brak{\phi-\theta}
\label{eq:chapters/11/10/4/5/1}
\end{align}
and from 
			\eqref{eq:PQ-final},
the distance is
\begin{align}
d
=\frac{\sin\brak{\phi-\theta}}{2\sin\brak{\frac{\phi-\theta}{2}}} = \cos\brak{\frac{\phi-\theta}{2}}
\label{eq:chapters/11/10/4/5/2}
\end{align}

\item Find the distance between parallel lines
\label{chapters/11/10/3/6}
\begin{enumerate}
	\item $15x+8y-34=0$ and  $15x+8y+31=0$ \\
	\item  $l(x+y)+p=0$ and  $l(x+y)-r=0$
\end{enumerate}
	\solution
	From \eqref{eq:parallel_lines}, the desired values are available in
  \tabref{tab:11/10/3/6}.
\begin{table}[H]
  \centering
  \begin{tabular}{|c|c|c|c|c|}
    \hline
    & $\vec{n}$ & $c_1$ & $c_2$ & $d$ \\
    \hline
    a) & $\myvec{15 \\ 8}$ & 34 & -31 & $\frac{65}{17}$ \\
    \hline
    b) & $\myvec{1 \\ 1}$ & $\frac{-p}{l}$ & $\frac{r}{l}$ & $\frac{\lvert p-r \rvert}{l\sqrt{2}}$ \\
    \hline
  \end{tabular}
  \caption{}
  \label{tab:11/10/3/6}
\end{table}

\item Find the equation of line which is equidistant from parallel lines $9x+6y-7=0$ and $3x+2y+6=0$.
\\
\solution
		Given
\begin{align}
	c_1 = \frac{7}{3},\,
c_2 = -6.
\end{align}
	From \eqref{eq:parallel_lines},
we need to find $c$ such that,
\begin{align}
	\abs{c-c_1} = \abs{c-c_2} \implies c = \frac{c_1+c_2}{2}
	 = -\frac{11}{6}.
\end{align}
Hence, the desired equation is
\begin{align}
	\myvec{3 & 2}\vec{x} &= -\frac{11}{6}
\end{align}
	See \figref{fig:chapters/11/10/4/21/1}.
\begin{figure}[H]
	\centering
	\includegraphics[width=0.75\columnwidth]{chapters/11/10/4/21/figs/fig.pdf}
	\caption{}
	\label{fig:chapters/11/10/4/21/1}
\end{figure}

	\item Prove that the products of the lengths of the perpendiculars drawn from the points $\myvec{\sqrt{a^2-b^2}& 0}^{\top}$ and $\myvec{-\sqrt{a^2-b^2} &0}^{\top}$ to the line $\frac{x}{a} \cos{\theta} + \frac{y}{b}\sin{\theta} =1 $ is $ b^2 $.
\\
    \solution 
		The input parameters for 
			\eqref{eq:PQ-final}
			are
\begin{align}
	\vec{n}=\myvec{\frac{\cos{\theta}}{a}  \\ \frac{\sin{\theta}}{b}},\,
  c = 1,\,
	\vec{P} =\pm \myvec{\sqrt{a^2-b^2}\\0} 
\end{align} 
The product of the distances is
\begin{align}
	d_1d_2 &=\frac{\abs{ \brak{\vec{n}^{\top} \vec{P}}^2 -  c^2 } }{\norm{\vec{n}}}
	=\frac{\abs{ \frac{\cos^2{\theta}\brak{a^2-b^2}}{a^2}- 1 }}{\frac{\cos^2{\theta}}{a^2} +\frac{\sin^2{\theta}}{b^2} }\\ 
	&= \frac{\brak{b^2 \cos^2{\theta} + a^2 \sin^2{\theta}}a^2 b^2}{\brak{b^2 \cos^2{\theta} + a^2 \sin^2{\theta}}a^2}
	= b^2
\end{align}

	\item The distance of the point $\vec{P}(2, 3)$ from the x-axis is

\begin{enumerate}
\item 2
\item 3
\item 1
\item 5 
\end{enumerate}

\item Find the foot of perpendicular from the point $(2,3,-8)$ to the line  $\dfrac{4-x}{2}=\dfrac{y}{6}=\dfrac{1-z}{3}$.Also, find the perpendicular distance from the given point to the line.
\item Find the distance of a point $(2,4,-1)$ from the line $$\frac{x+5}{1}=\frac{y+3}{4}=\frac{z-6}{-9}$$
\item Find the length and the foot of perpendicular from the point $ \brak{1,\dfrac{3}{2} ,2 }$ to the plane $2x-2y+4z+5=0.$
\item Show that the points $(\hat{i}-\hat{j}+3\hat{k})$ and $3(\hat{i}+\hat{j}+\hat{k})$ are equidistant from the plane $\overrightarrow{r} \cdot (5\hat{i}+2\hat{j}-7\hat{k})+9=0$ and lies on opposite side of it.
\item The distance of the plane $\overrightarrow{r} \cdot \brak{ \dfrac{2}{7}\hat{i}+\dfrac{3}{7}\hat{j}-\dfrac{6}{7}\hat{k}}=1$ from the origin is 
\begin{enumerate}
	\item 1
	\item 7
	\item $\dfrac{1}{7}$
	\item None of these	
\end{enumerate}
\item If the foot of perpendicular drawn from the origin to a plane is $(5,-3,-2)$, then the equation of plane is $\overrightarrow{r} \cdot (5\hat{i}-3\hat{j}-2\hat{k})=38.$
\item Find the equation of line  drawn perpendicular to the line $\frac{x}{4}+\frac{y}{6}=1$ through the point where it meets the y-axis \\
\solution
				The given line
parameters are
\begin{align}
		\vec{n} = \myvec{3\\2},\, c=12 ,\,
	\vec{m} =\myvec{-2 \\ 3}.
\end{align}
and the point on the y-axis is
\begin{align}
	\vec{A} =\myvec{0\\6}.
\end{align}
Thus, the equation of the desired line is 
\begin{align}
	\vec{m}^\top\brak{\vec{x}-\vec{A}}&=0\label{eq:chapters/11/10/4/7/5}
	\\
\implies
			\myvec{-2 & 3}\vec{x} &=-18
		\end{align}
		\iffalse
		See 
  \figref{fig:chapters/11/10/4/7/Figure}.
\begin{figure}[H]
\includegraphics[width=0.75\columnwidth]{chapters/11/10/4/7/figs/fig.png}
\caption{}
  \label{fig:chapters/11/10/4/7/Figure}
\end{figure}
\fi

\item Find the equation of line whose perpendicular distance from the origin is 5 units and the angle made by the perpendicular with the positive $x$-axis is $30\degree$.
\label{chapters/11/10/2/8}
\\
\solution
			From 
\eqref{eq:chapters/11/10/2/8-final},
		Thus, the equation of lines are
\begin{align}
	\myvec{\frac{\sqrt{3}}{2}& \frac{1}{2}}\vec{x}=\pm5
\end{align}
\iffalse
See 
\figref{fig:chapters/11/10/2/8/Fig1}.
\begin{figure}[H]
\begin{center}
\includegraphics[width=0.75\columnwidth]{chapters/11/10/2/8/figs/fig.pdf}
\end{center}
\caption{}
\label{fig:chapters/11/10/2/8/Fig1}
\end{figure}
\fi

\item 
	Find the equation of the line passing through  (-3,5) and perpendicular to the line through the points (2,5) and (-3,6).
	\\
	\solution 
\label{chapters/11/10/2/10}
See 
		\figref{fig:11/10/2/10}.
	\begin{figure}[H]
		\centering
 \includegraphics[width=0.75\columnwidth]{chapters/11/10/2/10/figs/fig.pdf}
		\caption{}
		\label{fig:11/10/2/10}
  	\end{figure}
The normal vector is
\begin{align}
\vec{n} =\myvec{2 \\5} -  \myvec{-3 \\ 6} 
=\myvec{
    5\\
    -1
}
\end{align}
Thus, the equation of the line is 
\begin{align}
\myvec{
    5 &-1
	}\brak{\vec{x} - \myvec{-3 \\5}}
= 0
\\
\implies 
\myvec{
    5 &-1
	}\vec{x} 
= -20
\end{align}

\item 
	The perpendicular from the origin to a line meets it at the point $(-2,9)$. Find the equation of the line.
\label{chapters/11/10/2/15}
	\\
	\solution
It is obvious that the normal vector to the line is 
\begin{align}
\vec{n} =\myvec{2 \\ -9} -\vec{0} 
=\myvec{2 \\ -9}
\end{align}
Hence, the equation of the line is 
\begin{align}
	\myvec{2 & -9}\brak{\vec{x} - \myvec{2 \\ -9}}&= 0
	\\
	\implies 
	\myvec{2 & -9}\vec{x} &= 85
\end{align}
See 
		\figref{fig:11/10/2/15}.
	\begin{figure}[H]
		\centering
 \includegraphics[width=0.75\columnwidth]{chapters/11/10/2/15/figs/fig.pdf}
		\caption{}
		\label{fig:11/10/2/15}
  	\end{figure}

\item Find the equation of line perpendicular to the line $x-7y+5=0$ and having $x$ intercept $3$\\
\label{chapters/11/10/3/8}
\solution
The desired equation is
		\begin{align}
			\myvec{7 & 1}\brak{\vec{x}-\myvec{3\\0}} &=0\\
		\implies 	\myvec{7 & 1}\vec{x} &= 21
		\end{align}
		\iffalse
		See 
\figref{fig:chapters/11/10/3/8/Fig1}.
		\begin{figure}[H]
\begin{center}
\includegraphics[width=0.75\columnwidth]{chapters/11/10/3/8/figs/fig.pdf}
\end{center}
\caption{}
\label{fig:chapters/11/10/3/8/Fig1}
\end{figure}
\fi

	\item Find the equation of the line passing through the point $\brak{1,2,-4}$ and perpendicular to the two lines
\begin{align}
	\frac{x-8}{3}=\frac{y+19}{-16}=\frac{z-10}{7} \text{ and }\\ \frac{x-15}{3}=\frac{y-29}{8}=\frac{z-5}{-5} 
\end{align}
    \solution
		The direction vector of the desired line 
is given by 
\begin{align*}
	\myvec{3 & -16 & 7\\3 & 8 & -5}\vec{m} = 0
	\xleftrightarrow[]{R_2\leftarrow R_2-R_1}
 	\myvec{3 & -16 & 7\\0 & 24 & -12}
	\\
	\xleftrightarrow[]{R_1\leftarrow R_1+\frac{2}{3}R_2}
	\myvec{3 & 0 & -1\\0 & 24 & -12}
	\xleftrightarrow[]{R_2\leftarrow R_2/12}
	\myvec{3 & 0 & -1\\0 & 2 & -1}
\end{align*}
yielding
\begin{align}
	\vec{m} = \myvec{2\\3\\6}
\end{align}
Hence the vector equation of the line passing through $\brak{1,2,-4}$ is,
\begin{align}
	\vec{x} = \myvec{1\\2\\-4} + \kappa \myvec{2\\3\\6}
\end{align}



 \item The perpendicular from the origin to the line $y=mx+c$ meets it at the point $(-1,2)$. Find the values of m and c.
 \label{11.10.3.15}
	 \\
 \solution
 From \probref{chapters/11/10/2/15},
\begin{align}
	\vec{n} = \myvec{-1\\2} \implies m = \frac{1}{2}
\end{align}
Also, from the given equation of the line and the given point, 
\begin{align}
	c = \myvec{-m & 1}\myvec{-1\\2} = 
\frac{5}{2}  
\end{align}
\iffalse
 See \figref{fig:pic}.
\begin{figure}[H]
 \centering
\includegraphics[width=0.75\columnwidth]{chapters/11/10/3/15/figs/graph.jpg}
 \caption{Graph}
 \label{fig:pic}
\end{figure}
\fi

\item  Point $\vec{P}(0,2)$ is the point of intersection of $y$-axis and perpendicular bisector of line segment joining the points $\vec{A}(-1,1) \text{ and } \vec{B}(3,3)$
\item 
A line perpendicular to the line segment joining the points $\vec{P}(1,0)$ and $\vec{Q}(2,3)$ divides it in the ratio $1:n$. Find the equation of the line.
	\\
	\solution 
\label{chapters/11/10/2/11}
The direction vector of 
$PQ$ is 
\begin{align}
     \vec{Q
 }-  \vec{P
 }
=
     \myvec{
  1\\
  3
 }
\end{align}
 Using section formula, 
 \begin{align}
	 \vec{R}=\frac{\vec{Q}+n\vec{P}}{1+n}
\end{align}
is the point of intersection.
The 
equation of the desired line  is
\begin{align}
	\vec{m}^{\top}\brak{\vec{x}-\vec{R}}=0
\\
\implies 
	   \myvec{
		   1 &  3}\vec{x}
	   &= \myvec{
  1\ 3}\myvec{
  \frac{2+n}{1+n}\\
  \frac{3}{1+n}} 
  \\
	=	  \frac{11+n}{1+n} 
\end{align}
\iffalse
See
		\figref{fig:11/10/2/11}.
	\begin{figure}[H]
		\centering
 \includegraphics[width=0.75\columnwidth]{chapters/11/10/2/11/figs/linefig.pdf}
		\caption{}
		\label{fig:11/10/2/11}
  	\end{figure}
	\fi

\item Find the vector equation of a plane which is at a distance of 7 units from the origin and normal to the vector $3\hat{i}+5\hat{j}-6\hat{k}$.
	\\
    \solution
		From the given information, 
\begin{align} 
\vec{n}=\myvec{3\\5\\-6},\,
	d=\frac{\abs{c}}{\norm{\vec{n}}} = 7
	\\
	\implies
c =\pm7\sqrt{70}
\end{align}	  

\item Find the equation of a plane which is at a distance 3$\sqrt{3}$ units from origin and the normal to which is equally inclined to coordinate axis.
\item If the line drawn from the point $(-2,-1,-3)$ meets a plane at right angle at the point $(1,-3,3)$, find the equation of the plane.
\item O is the origin and A is $(a,b,c)$.Find the direction cosines of the line OA and the equation of plane through A at right angle at OA.
\item Two systems of rectangular axis have the same origin. If a plane cuts them at distances $a,b,c$ and $a^{\prime},b^{\prime},c^{\prime}$, respectively, from the origin, prove that $$\frac{1}{a^2}+\frac{1}{b^2}+\frac{1}{c^2}=\frac{1}{{a^{\prime}}^2}+\frac{1}{{b^{\prime}}^2}+\frac{1}{{c^{\prime}}^2}$$.
\item Find the equation of the plane through the points $(2,1,-1)$ and $(-1,3,4),$ and 
perpendicular to the plane $x-2y+4z=10.$
	\item Find the values of $\theta \text{ and } p$, if the equation $x\cos\theta+y\sin\theta=p$ is the normal form
of the line $\sqrt{3}x+y+2=0$.
\\
\solution
				\begin{align}
	\vec{n}=\myvec{\sqrt{3}\\1},
			c=-2
			\\
			\implies
			\theta=\tan^{-1}\brak{\frac{1}{\sqrt{3}}}
			=\frac{\pi}{6},
			p=\frac{\abs{c}}{\norm{\vec{n}}}=1
		\end{align}
		\iffalse
See \figref{fig:chapters/11/10/4/2/Fig1}.
\begin{figure}[H]
	\begin{center} 
	    \includegraphics[width=0.75\columnwidth]{chapters/11/10/4/2/figs/line.png}
	\end{center}
\caption{}
\label{fig:chapters/11/10/4/2/Fig1}
\end{figure}
\fi

\end{enumerate}
