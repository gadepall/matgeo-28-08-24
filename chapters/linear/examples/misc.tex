\begin{enumerate}[label=\thesubsection.\arabic*,ref=\thesubsection.\theenumi]


\item Find the values of $k$ for which the line 
\begin{align}
(k-3)x-(4-k^2)y+k^2-7k+6=0 \label{eq:chapters/11/10/4/1/1}
\end{align}
is
\begin{enumerate}
\item Parallel to the $x$-axis
\item Parallel to the $y$-axis
\item Passing through the origin
\end{enumerate}
    \solution 
		\begin{align}
\vec{n} = \myvec{k-3\\-4+k^2}, c  = -k^2+7k-6
\label{eq:chapters/11/10/4/1/6}
\end{align}
\begin{enumerate}
    \item 
\begin{align}
\myvec{k-3\\-4+k^2} =\alpha\myvec{0\\1}
\implies
k =3,
\\
\implies        \myvec{0 & 5}\vec{x} =6
\end{align}
upon substituting from 
\eqref{eq:chapters/11/10/4/1/6}.

\item In this case, 
\begin{align}
\myvec{k-3\\-4+k^2} =\beta\myvec{1\\0}
	\implies k =\pm2
	\\
	\implies
        \myvec{-1 & 0}\vec{x} =4, \quad  k =2\\
        \myvec{-5 & 0}\vec{x} =-24, \quad  k =-2
\end{align}
\item 
	In this case, 
\begin{align}
	-k^2+7k-6 = 0
	\implies k =1,  k=6
	\\
	\implies
        \myvec{-2 & -3}\vec{x} =0, \quad  k =1\\
       \myvec{3 & 32}\vec{x} =0, \quad  k =6
\end{align}
\end{enumerate}

	\item Find the values of $\theta \text{ and } p$, if the equation $x\cos\theta+y\sin\theta=p$ is the normal form
of the line $\sqrt{3}x+y+2=0$.
\\
\solution
				\begin{align}
	\vec{n}=\myvec{\sqrt{3}\\1},
			c=-2
			\\
			\implies
			\theta=\tan^{-1}\brak{\frac{1}{\sqrt{3}}}
			=\frac{\pi}{6},
			p=\frac{\abs{c}}{\norm{\vec{n}}}=1
		\end{align}
		\iffalse
See \figref{fig:chapters/11/10/4/2/Fig1}.
\begin{figure}[H]
	\begin{center} 
	    \includegraphics[width=0.75\columnwidth]{chapters/11/10/4/2/figs/line.png}
	\end{center}
\caption{}
\label{fig:chapters/11/10/4/2/Fig1}
\end{figure}
\fi

	\item Find the  equations of the lines, which cutoff intercepts on the axes  whose sum and product are 1 and -6 respectively.
\\
\solution
		Let the intercepts be $a$ and  $b$. Then
\begin{align}
a+b=1,
ab=-6 \label{eq:11/10/4/32a}
\\
\implies  a = 3, b = -2
\end{align}
Thus, the possible 
intercepts are
\begin{align}
\myvec{3\\0}, \myvec{0\\-2},
\myvec{-2\\0}, \myvec{0\\3}
\end{align}
From
		\eqref{prop:lin-eq-unit-mat},
\begin{align}
	\myvec{3 & 0 \\ 0 &-2}\vec{n} = \myvec{1 \\ 1}
	\\
	\implies \vec{n} = \myvec{\frac{1}{3} \\ -\frac{1}{2}}
	\\
	\text{or, } \myvec{2 & -3}\vec{x} = 6
\end{align}
using		\eqref{prop:lin-eq-unit}.
Similarly, the other line can be obtained
as
\begin{align}
	\myvec { 3 & -2 }  \vec{x}  = -6        
\end{align}
\iffalse
See  
\figref{fig:11/10/4/3line segmenta}.
\begin{figure}[H]
\centering
\includegraphics[width=0.75\columnwidth]{chapters/11/10/4/3/figs/inter.png}
\caption{}
\label{fig:11/10/4/3line segmenta}
\end{figure}
\fi

	\item  Find the equation of the line parallel to y-axis and drawn through the point of
intersection of the lines x – 7y + 5 = 0 and 3x + y = 0.
\\
\solution
				Following the approach in \probref{prob:12/11/3/9/plane},
		the desired equation is 
\begin{align}
\myvec{	1&-7}\vec{x} -5
+
	k\myvec{3&1} \vec{x} = 0
	\\
	\implies 
	\myvec{	1 + 3k&-7+k} 
	 \vec{x} =5 
	 \\
	 \implies 
	\myvec{	1 + 3k \\ -7+k}  = \alpha \myvec{1 \\ 0}
	\text{or, } k = 7, \alpha =  22.
\end{align}
The desired equation is then given by 
\begin{align}
	\myvec{1&0}\vec{x}=\frac{5}{22}
\end{align}
The intersection of the lines is obtained using the augemented matrix as
\begin{align}
	\augvec{2}{1}{
		1 &-7 &-5
		\\
		3 & 1 & 0
	}
	\xleftrightarrow[R_1 = 22R_1+7R_2]{R_2 = R_2 - 3R_1}
	\augvec{2}{1}{
		22 &0 &-5
		\\
		0 & 22 & 15
	}
	\\
	\implies \vec{x} = \frac{5}{22}\myvec{-1 \\ 3}
\end{align}
See  
\figref{fig:chapters/11/10/4/6/Fig3}.
\begin{figure}[H]
  \begin{center} 
      \includegraphics[width=0.75\columnwidth]{chapters/11/10/4/6/figs/fig.pdf}
  \end{center}
\caption{}
\label{fig:chapters/11/10/4/6/Fig3}
\end{figure}

	\item Find the area of triangle formed by the lines $y-x=0, x+y=0, \text{ and } x-k=0$.
		\\
\solution
		The vertices of the triangle can be expressed using the equations
\begin{align}
	\myvec{1&1\\-1&1} \vec{A} &= \vec{0}
	\\
	\myvec{1&1\\1&0} \vec{B} &= \myvec{0\\k}
	\\
	\myvec{1&0\\-1&1} \vec{C} &= \myvec{k\\0}
\end{align}
from which
\begin{align}
\vec{A} = \myvec{0\\0},
	\vec{B}=\myvec{k\\-k},
	\vec{C}=\myvec{k\\k}
\end{align}
are trivially obtained.
Thus, 
\begin{align}
ar(ABC) &=\frac{1}{2}\norm{(\vec{A}-\vec{B})\times(\vec{A}-\vec{C})}\\
	&=\frac{1}{2}\norm{\myvec{-k\\k}\times\myvec{-k\\-k}}
=k^2
\end{align}

\item A ray of light passing through the point $\vec{P} = \brak{1, 2}$ reflects on the x-axis at point $\vec{A}$ and the reflected ray passes through the point $\vec{Q} =\brak{5, 3}$. Find the coordinates of $\vec{A}$.
\\
    \solution 
			From \eqref{eq:11/10/4/22},
the reflection of $\vec{Q}$ is 
\begin{align}
\vec{R}  
= \myvec{5\\-3}
\end{align}
Letting
\begin{align}
\vec{A} = \myvec{x\\0},
\end{align}
since 
$\vec{P},
\vec{A},  
\vec{R}  
$
are collinear, 
		from \eqref{prop:lin-dep-rank},
\begin{align}
	\myvec{
		1 & 1 & 2 
		\\ 
		1 & 5 & -3 
		\\
		1 & x & 0 }
	\xleftrightarrow[R_3=R_3 - R_1]{R_2 = R_2 - R_1}
	\myvec{
		1 & 1 & 2 
		\\ 
		0 & 4 & -5 
		\\
		0 & x-1 & -2 }
	\\
	\xleftrightarrow[]{R_3 = 4R_3 - \brak{x-1}R_2}
	\myvec{
		1 & 1 & 2 
		\\ 
		0 & 4 & -5 
		\\
		0 & 0 & 5x-13 }
	\implies x = \frac{13}{5}
\end{align}
See  
\figref{fig:chapters/11/10/4/22/1}.
\begin{figure}[H]
\centering
\includegraphics[width=0.75\columnwidth]{chapters/11/10/4/22/figs/fig.pdf}
\caption{}
\label{fig:chapters/11/10/4/22/1}
\end{figure}




%\item Find the equations of the lines, which cut-off intercepts on the axes whose sum and product are 1 and -6, resspectively.
%	\\
%    \solution 
%		%Let the intercepts be $a$ and  $b$. Then
\begin{align}
a+b=1,
ab=-6 \label{eq:11/10/4/32a}
\\
\implies  a = 3, b = -2
\end{align}
Thus, the possible 
intercepts are
\begin{align}
\myvec{3\\0}, \myvec{0\\-2},
\myvec{-2\\0}, \myvec{0\\3}
\end{align}
From
		\eqref{prop:lin-eq-unit-mat},
\begin{align}
	\myvec{3 & 0 \\ 0 &-2}\vec{n} = \myvec{1 \\ 1}
	\\
	\implies \vec{n} = \myvec{\frac{1}{3} \\ -\frac{1}{2}}
	\\
	\text{or, } \myvec{2 & -3}\vec{x} = 6
\end{align}
using		\eqref{prop:lin-eq-unit}.
Similarly, the other line can be obtained
as
\begin{align}
	\myvec { 3 & -2 }  \vec{x}  = -6        
\end{align}
\iffalse
See  
\figref{fig:11/10/4/3line segmenta}.
\begin{figure}[H]
\centering
\includegraphics[width=0.75\columnwidth]{chapters/11/10/4/3/figs/inter.png}
\caption{}
\label{fig:11/10/4/3line segmenta}
\end{figure}
\fi

    \item A person standing at the junction (crossing) of two straight paths 
    represented by the equations 
    \begin{align}
        \myvec{2&-3}\vec{x} = -4 
        \label{eq:chapters/11/10/4/24/L1}
    \end{align}
    and
    \begin{align}
        \myvec{3&4}\vec{x} = 5
        \label{eq:chapters/11/10/4/24/L2}
    \end{align} 
    wants to reach the path whose equation is 
    \begin{align}
        \myvec{6&-7}\vec{x} = -8
        \label{eq:chapters/11/10/4/24/L3}
    \end{align}
    Find equation of the path that he should follow.
\\
    \solution 
		The junction of \eqref{eq:chapters/11/10/4/24/L1}
    and \eqref{eq:chapters/11/10/4/24/L2} is obtained as
    \begin{align*}
	    \augvec{2}{1}{2&-3&-4\\3&4&5} \xleftrightarrow[]{R_2\rightarrow2R_2-3R_1} 
        \augvec{2}{1}{2&-3&-4\\0&17&22} \\
		      \xleftrightarrow[]{R_1\rightarrow17R_1+3R_2} \augvec{2}{1}{17&0&-1\\0&17&22} 
		      \implies
        \vec{A} = \frac{1}{17}\myvec{-1\\22}
    \end{align*}
    Clearly, the man should follow the path perpendicular to \eqref{eq:chapters/11/10/4/24/L3} from
    $\vec{A}$ to reach it in the shortest time. The normal vector 
    of \eqref{eq:chapters/11/10/4/24/L3} is 
    \begin{align}
         \myvec{6\\-7}
	 \implies
        \vec{n} = \myvec{7\\6}
        \label{eq:chapters/11/10/4/24/L4-norm}
    \end{align}
    and the equation of the desired line is
   \begin{align}
        \myvec{7&6}\vec{x} &= \frac{1}{17}\myvec{7&6}\myvec{-1\\22} = \frac{125}{17}
        \label{eq:chapters/11/10/4/24/L4}
    \end{align}
		See Fig. \ref{fig:chapters/11/10/4/24/crossing}.
		\begin{figure}[H]
        \centering
        \includegraphics[width=0.75\columnwidth]{chapters/11/10/4/24/figs/fig.pdf}
        \caption{}
        \label{fig:chapters/11/10/4/24/crossing}
    \end{figure}

	\item Find the equation of the line passing through the point of intersection of the lines $4x + 7y - 3 = 0$ and $2x - 3y + 1 = 0$ that has equal intercepts on the axes.\\
	\solution 
	  		From \probref{prob:12/11/3/9/plane},
the intersection of the lines is given by 
		\begin{align}
       \myvec{4 + 2k &7-3k}\vec{x}=3-k
       \label{eq:11/10.4/12/3}
       \\
       \implies \myvec{4 + 2k \\7-3k} = \alpha\myvec{1 \\ 1} 
		\end{align}
			from  \probref{chapters/11/10/2/12}, yielding,
		\begin{align}
	\augvec{2}{1}{
				1 & -2 & 4
				\\
				1 & 3 & 7
			}
			\xleftrightarrow[]{R_2 = R_2 -R_1}
	\augvec{2}{1}{
				1 & -2 & 4
				\\
				0 & 5 & 3 
			}
			\\
			\text{or, } k = \frac{3}{5}
       \label{eq:11/10.4/12/4}
   \end{align}
 Substituting the above  
in       \eqref{eq:11/10.4/12/3}, the desired equation is 
    \begin{align}
        \myvec{1&1}\vec{x}=\frac{6}{13}
    \end{align}
    See
    \figref{fig:enter-label}.
\begin{figure}[H]
    \centering
    \includegraphics[width=0.75\columnwidth]{chapters/11/10/4/12/figs/fig.pdf}
    \caption{}
    \label{fig:enter-label}
\end{figure}

\item  Point $\vec{P}(0,2)$ is the point of intersection of $y$-axis and perpendicular bisector of line segment joining the points $\vec{A}(-1,1) \text{ and } \vec{B}(3,3)$
\item Prove that in any $\triangle{ABC}$, cos A=$\frac{b^2+c^2-a^2}{2bc}$, where a,b,c are the magnitudes of the sides opposite to the vertices A,B,C respectively.
\item The vector having intial and terminal points as (2,5,0)and (-3,7,4),respectively is
	\begin{enumerate}
\item -$\hat{i}+12\hat{j}+4\hat{k}$
\item $5\hat{i}+2\hat{j}-4\hat{k}$
\item $5\hat{i}+2\hat{j}+4\hat{k}$
\item $\hat{i}+\hat{j}+\hat{k}$
\end{enumerate}
\item The value of $\lambda$ for which the vectors $3\hat{i}-6\hat{j}+\hat{k}$ $\text{and}$,  $2\hat{i}-4\hat{j}$+$\lambda\hat{k}$ are parallel is
	\begin{enumerate}
\item $\frac{2}{3}$
\item $\frac{3}{2}$
\item $\frac{5}{2}$
\item $\frac{2}{5}$
	\end{enumerate}	
\item The value of the expression $\abs{\vec{a}\times\vec{b}}$+ $({\vec{a}.\vec{b}})$ is \rule{1cm}{0.15mm}.
\item If $\abs{\vec{a}\times\vec{b}}^2$ + $\abs{\vec{a}.\vec{b}}^2$=144 $\text{and}$  $\abs{\vec{a}}$=4, then $\abs{\vec{b}}$ is equal to \rule{1cm}{0.15mm}.
\item  Find the position vector of a point A in space such that $\overrightarrow{OA}$ is inclined at 60 $\degree$ to OX and at 45 $\degree$ to OY and $\abs{\overrightarrow{OA}} =10$ units.
\item Distance of the point $(\alpha \beta \gamma)$ from y-axis is
\begin{enumerate}
	\item $\beta$ 
	\item $\abs{\beta}$
	\item $\abs{\beta+\gamma}$
	\item $\sqrt{\alpha^2+\gamma^2}$
\end{enumerate}
\item If the directions cosines of a line are $k,k,k,$ then
\begin{enumerate}
	\item $k>0$
	\item $0<k<1$
	\item $k=1$ 
	\item $k=\dfrac{1}{\sqrt{3}}$ or $-\dfrac{1}{\sqrt{3}}$
\end{enumerate}
\item The reflection of the point $(\alpha \beta \gamma )$ in the xy-plane is 
\begin{enumerate}
	\item $\alpha,\beta,0)$
	\item $(0,0,\gamma)$
	\item $(-\alpha,-\beta,\gamma)$
	\item $(\alpha,\beta,-\gamma)$
\end{enumerate}
\end{enumerate}
