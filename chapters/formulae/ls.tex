%\begin{enumerate}[label=\arabic*.,ref=\theenumi]
\begin{enumerate}[label=\thesubsection.\arabic*.,ref=\thesubsection.\theenumi]
	\item The lines
\begin{align}
\begin{split}
	L_1: \quad   \vec{x} &=\vec{A}+ \kappa_1\vec{m_1}
	\\
L_2: \quad        
	\vec{x} &= \vec{B}  + \kappa_2\vec{m_2} 
\end{split}
	    \label{eq:chapters/12/11/2/16/svd/L1L2}
\end{align}
will intersect if 
\begin{align}
\vec{A}+ \kappa_1\vec{m_1}
= \vec{B}  + \kappa_2\vec{m_2} 
\\
\implies 
	 \vec{B}-\vec{A}=
 \myvec{\vec{m_1} & \vec{m_2}}\myvec{\kappa_1\\-\kappa_2}
 \\
	\implies \rank\cbrak{\vec{M} = 
	\myvec{\vec{m_1} & \vec{m_2}}} = 2
\end{align}
\item If $L_1, L_2$, do not intersect, let 
\begin{align}
\begin{split}
	\vec{x}_1 &=\vec{A}+ \kappa_1\vec{m_1}
	\\
	\vec{x}_2 &= \vec{B}  + \kappa_2\vec{m_2} 
\end{split}
	    \label{eq:chapters/12/11/2/16/svd/x1x2}
\end{align}
be points on 
$L_1, L_2$ respectively, that are closest to each other.
Then, 
	    from \eqref{eq:chapters/12/11/2/16/svd/x1x2}
\begin{align}
\vec{x_1} - \vec{x_2} =
	 \vec{A}-\vec{B}+
 \myvec{\vec{m_1} & \vec{m_2}}\myvec{\kappa_1\\-\kappa_2}
	\label{eq:chapters/12/11/2/16/svd/x-diff}
\end{align}
Also, 
    \begin{align}
	    \brak{\vec{x}_1 -\vec{x}_2}^\top\vec{m}_1
	    =
	    \brak{\vec{x}_1 -\vec{x}_2}^\top\vec{m}_2
	    =0
	    \\
	    \implies 
	    \brak{\vec{x}_1 -\vec{x}_2}^\top\myvec{\vec{m_1} & \vec{m_2}} = \vec{0}
	    \\
	    \text{or, }	    \vec{M}^\top\brak{\vec{x}_1 -\vec{x}_2} = \vec{0}
	    \\
	    \implies \vec{M}^\top
	    \brak{\vec{A}-\vec{B}}+
 \vec{M}^\top\vec{M}\myvec{\kappa_1\\-\kappa_2} = \vec{0}
	    \label{eq:chapters/12/11/2/16/svd/m-orth}
    \end{align}
	    from 
	\eqref{eq:chapters/12/11/2/16/svd/x-diff},
	yielding
    \begin{align}
	    \vec{M}^\top\vec{M}\myvec{\kappa_1\\-\kappa_2} = \vec{M}^\top\brak{\vec{B}-\vec{A}}
        \label{eq:chapters/12/11/2/16/svd/vec-eqn}
    \end{align}
    This is known as the {\em least squares solution}.
\end{enumerate}
