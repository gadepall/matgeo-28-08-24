%\begin{enumerate}[label=\arabic*.,ref=\theenumi]
\begin{enumerate}[label=\thesubsection.\arabic*.,ref=\thesubsection.\theenumi]
	\item 
The affine transformation is given by 
\begin{align}
	\label{eq:conic_affine}
	\vec{x} = \vec{P}\vec{y}+\vec{c}
\end{align}
where $\vec{c}$ is the translation vector.
\item The matrix
\begin{align}
\vec{P} =
\myvec{
\cos\theta & -\sin\theta \\
\sin\theta & \cos\theta 
}
\end{align}
is defined to be the rotation matrix. 
\item 
\begin{align}
	\vec{P}^{\top} \vec{P} = \vec{I}
\end{align}
		$\vec{P}$ is known as as {\em orthogonal} matrix.
\item Given vertices $\vec{A}, \vec{C}$ of a square, the other two vertices are given by
\begin{align}
\begin{split}
	\vec{B} = \norm{\vec{C}-\vec{A}}\cos \frac{\pi}{4}\vec{P}\vec{e}_1+\vec{A}
	\\
	\vec{D} = \norm{\vec{C}-\vec{A}}\cos \frac{\pi}{4}\vec{P}\vec{e}_2+\vec{A}
\end{split}
	\label{eq:affine-square-bd}
\end{align}
%-c}}{\norm{\vec{n}}}\vec{n}
	\\
		\solution Shifting $\vec{A}$ to the origin and rotating the square clockwise by an angle $\phi$ made by $CA$ with the $x$-axis,
	from \eqref{eq:conic_affine},
\begin{align}
\vec{A} = \vec{P}\vec{0}+\vec{c}
\\
\implies 
\vec{c} = \vec{A}
\\
	\theta =  \phi -\frac{\pi}{4} 
\end{align}
and we obtain a square with the other vertices as
\begin{align}
\begin{split}
	\vec{B}_1 = \norm{\vec{C}-\vec{A}}\cos \frac{\pi}{4}\vec{e}_1
	\\
	\vec{D}_1 = \norm{\vec{C}-\vec{A}}\cos \frac{\pi}{4}\vec{e}_2
\end{split}
	\label{eq:affine-bd}
\end{align}
	From \eqref{eq:conic_affine}
	and 
	\eqref{eq:affine-bd},
	we obtain \eqref{eq:affine-square-bd}.
\item Show that $$(\overrightarrow{a}-\overrightarrow{b})\times (\overrightarrow{a}+\overrightarrow{b})=2(\overrightarrow{a}\times \overrightarrow{b})$$
	\\
		\solution
		  \begin{align}
	  \brak{\vec{a}-\vec{b}}\times\brak{\vec{a}+\vec{b}}
	  &=
\vec{a}\times\vec{a}
-\vec{b}\times\vec{b}
%  \\
	 % &\quad
	  +\vec{a}\times\vec{b}
-\vec{b}\times\vec{a}
\nonumber \\
	  &=
  2\myvec{\vec{a}\times\vec{b}}
  \end{align}
  from 
  \eqref{eq:cross3d-commute}.
  and
  \eqref{eq:cross3d-same}

\item If either $\overrightarrow{a} = \overrightarrow{0}$ or $\overrightarrow{b} = \overrightarrow{0}$, then $\overrightarrow{a} \times \overrightarrow{b} = \overrightarrow{0}$. Is the converse true? Justify your answer with an example.
	\\
		\solution
		For
\begin{align}
\vec{a} = \myvec{1 \\ 0 \\ 0},\, \vec{b} = \myvec{2 \\ 0 \\ 0}
\\
   \vec{a} \times \vec{b} = \vec{0}. 
\end{align}


\item Given that $\overrightarrow{a} \cdot \overrightarrow{b} = 0$ and $\overrightarrow{a} \times \overrightarrow{b} = \overrightarrow{0}$. What can you conclude about the vectors $\overrightarrow{a} \text{ and }\overrightarrow{b}$?
\item The area of a triangle with vertices $(a,b+c), (b,c+a)\text{ and }(c,a+b)$ is
\begin{enumerate}
\item $(a+b+c)^2$
\item 0
\item a+b+c
\item abc 
\end{enumerate}
\item If $\vec{a}+\vec{b}+\vec{c}$=0, show that $\vec{a}\times\vec{b}=\vec{b}\times\vec{c}=\vec{c}\times\vec{a}$. Interpret the result geometrically.
\item If $\vec{a}, \vec{b}, \vec{c}$, determine the vertices of a triangle, show that $\frac{1}{2}$ $\left[\vec{b} \times\vec{c}+\vec{c} \times\vec{a}+\vec{a}\times\vec{b} \right]$ gives the vector area of the triangle. Hence deduce the condition that the three points $\vec{a},\vec{b},\vec{c},$ are collinear. Also find the unit vector normal to the plane of the triangle.
\item For any vector $\vec{a}$, the value of $(\vec{a}\times\hat{i})^2+(\vec{a}\times\hat{j})^2 + (\vec{a}\times\hat{k})^2$ is equal to 
	\begin{enumerate}
\item a 
\item 3a
\item 4a
\item 2a
\end{enumerate}
\item If $\theta$ is the angle between any two vectors $\vec{a}$ and $\vec{b}$,then $|\vec{a}\cdot\vec{b}|=|\vec{a}\times\vec{b}|$ when $\theta$ is equal to
\begin{enumerate}
\item 0
\item $\frac{\pi}{4}$
\item $\frac{\pi}{2}$
\item $\pi$
\end{enumerate}
\end{enumerate}
