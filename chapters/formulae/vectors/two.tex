%\begin{enumerate}[label=\arabic*.,ref=\theenumi]
\begin{enumerate}[label=\thesubsection.\arabic*.,ref=\thesubsection.\theenumi]
\numberwithin{equation}{enumi}
  \item If $ABCD$ be a parallelogram,
	  \label{prop:two-pgm}
  \begin{align}
	  \label{eq:two-pgm}
 \vec{B}-\vec{A} = \vec{C} -\vec{D}
  \end{align}
  \item 
If $PQRS$ is formed by joining the mid points of $ABCD$, 
\begin{align}
  \vec{P} = \frac{1}{2}\brak{\vec{A}+\vec{B}} 
  ,\,
 \vec{Q} = \frac{1}{2}\brak{\vec{B}+\vec{C}} 
 \\
 \vec{R} = \frac{1}{2}\brak{\vec{C}+\vec{D}}   
  ,\,
 \vec{S} = \frac{1}{2}\brak{\vec{D}+\vec{A}}  
 \\
	\implies 
 \vec{P}-\vec{Q} = \vec{S} -\vec{R}.
  \label{eq:10/7/4/8det2f}
\end{align}
Hence, $PQRS$ is a parallelogram
	  from \eqref{eq:two-pgm}.
\end{enumerate}
