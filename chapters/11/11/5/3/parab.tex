The parameters are then listed in  
    \tabref{tab:chapters/11/11/5/3/points}.
\begin{table}[H]
	\centering
    \input{chapters/11/11/5/3/tables/table.tex}
    \caption{points}
    \label{tab:chapters/11/11/5/3/points}
\end{table}
For the conic,
\begin{align}
    \vec{V} = \myvec{1&0\\0&0}.
\end{align}
Points $\vec{O}, \vec{A}$, and $\vec{B}$ are on conic, so we have
\begin{align}
	\vec{O}^{\top}\vec{V}\vec{O} + 2\vec{u}^{\top}\vec{O} + f &= 0\\
	\vec{A}^{\top}\vec{V}\vec{A} + 2\vec{u}^{\top}\vec{A} + f &= 0\\
	\vec{B}^{\top}\vec{V}\vec{B} + 2\vec{u}^{\top}\vec{B} + f &= 0	 
\end{align}
which can be expressed as
\begin{align}
	2\vec{O}^{\top}\vec{u} + f &= - \vec{O}^{\top}\vec{V}\vec{O}\\
	2\vec{A}^{\top}\vec{u} + f &= - \vec{A}^{\top}\vec{V}\vec{A}\\
	2\vec{B}^{\top}\vec{u} + f &= - \vec{B}^{\top}\vec{V}\vec{B}	
\end{align}
leading to the matrix equation
\begin{align}
	\myvec{2\vec{O}^{\top} & 1\\ 2\vec{A}^{\top} & 1\\ 2\vec{B}^{\top} & 1}\myvec{\vec{u} \\ f} = -\myvec{\vec{O}^{\top}\vec{V}\vec{O}\\ \vec{A}^{\top}\vec{V}\vec{A}\\ \vec{B}^{\top}\vec{V}\vec{B}}
\end{align}
Substituting numerical values in the above equation,
\begin{align}
    \myvec{0&0&1\\ 100&48&1\\ -100&48&1}\myvec{\vec{u} \\ f} = -\myvec{0\\-2500\\-2500}\\
    \implies f = 0 \text{ and } \vec{u} = \myvec{0\\-\frac{625}{12}}
\end{align}
So, the equation of the parabola is
\begin{align}
    \label{eq:chapters/11/11/5/3/parab1}  \vec{x}^{\top}\myvec{1&0\\0&0}\vec{x} + 2\myvec{0&-\frac{625}{12}}\vec{x} = 0 
\end{align}
The desired point can be expressed as
\begin{align}
	\vec{D} = \myvec{18 \\ x_2}
\end{align}
Substituting this in the parabola equation,
\begin{align}
    18^2 - \frac{6}{625}\lambda_2 = 0
    \\
\implies \lambda_2 = \frac{1944}{625}
\end{align}
Thus, the length of a supporting wire attached to the roadway $18 m$ from the middle is 
\begin{align}
     \lambda_2 + d_2 = \frac{5694}{625} m   
\end{align}
See  
    \figref{fig:chapters/11/11/5/3/parabola}.
\begin{figure}[H]
    \centering
    \includegraphics[width=0.75\columnwidth]{chapters/11/11/5/3/figs/parabola.png}
    \caption{}
    \label{fig:chapters/11/11/5/3/parabola}
\end{figure}

