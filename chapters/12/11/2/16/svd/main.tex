                From \eqref{eq:chapters/12/11/2/16/given},
    \begin{align}
        \vec{M}^\top\vec{M} = \myvec{1&-3&2\\2&3&1}\myvec{1&2\\-3&3\\2&1} = \myvec{14&-5\\-5&14} \\
        \vec{MM}^\top = \myvec{1&2\\-3&3\\2&1}\myvec{1&-3&2\\2&3&1} = \myvec{5&3&4\\3&18&-3\\4&-3&5}
    \end{align}
    \begin{enumerate}
        \item For $\vec{MM}^\top$, the characteristic polynomial is
        \begin{align}
		\text{char}{\vec{MM}^\top} &= \mydet{\lambda-5&-3&-4\\-3&\lambda-18&3\\-4&3&\lambda-5} \\
                                      &= \lambda\brak{\lambda-9}\brak{\lambda-19}
                                      \label{eq:chapters/12/11/2/16/svd/char-2}
        \end{align}
        Thus, the eigenvalues are given by
        \begin{align}
            \lambda_1 = 19,\ \lambda_2 = 9,\ \lambda_3 = 0
        \end{align}
        For $\lambda_1$, the augmented matrix formed from the 
        eigenvalue-eigenvector equation is
        \begin{align*}
            \myvec{-14&3&4\\3&-1&-3\\4&-3&-14} 
            \xleftrightarrow[]{R_1 \leftarrow \frac{R_1+R_3}{-10}} \myvec{1&0&1\\3&-1&-3\\4&-3&-14} \\
%            \xleftrightarrow[]{} \myvec{1&0&1\\0&1&6\\4&-3&-14} 
            \xleftrightarrow[R_2 \leftarrow -R_2+3R_1]{R_3 \leftarrow R_3-4R_1} \myvec{1&0&1\\0&-1&-6\\0&-3&-18} 
            \xleftrightarrow[]{R_3 \leftarrow R_3-3R_2} \myvec{1&0&1\\0&-1&-6\\0&0&0}
        \end{align*}
        Hence, the normalized eigenvector is
        \begin{align}
		\vec{u}_1 = \frac{1}{\sqrt{38}}\myvec{-1\\-6\\1}
        \end{align}
        For $\lambda_2$, the augmented matrix formed from the 
        eigenvalue-eigenvector equation is
        \begin{align*}
            \myvec{-4&3&4\\3&9&-3\\4&3&-4} 
            \xleftrightarrow[]{R_3 \leftarrow R_1+R_3} \myvec{-4&3&4\\3&9&-3\\0&0&0} \\
            \xleftrightarrow[R_2 \leftarrow \frac{4R_2+3R_1}{45}]{R_1 \leftarrow \frac{R_1-3R_2}{-4}} \myvec{1&0&-1\\0&1&0\\0&0&0}
        \end{align*}
        Hence, the normalized eigenvector is
        \begin{align}
		\vec{u}_2 = \frac{1}{\sqrt{2}}\myvec{1\\0\\1}
        \end{align}
        For $\lambda_3$, the augmented matrix formed from the 
        eigenvalue-eigenvector equation is
        \begin{align*}
            \myvec{5&3&4\\3&18&-3\\4&-3&5} 
            \xleftrightarrow[]{R_1 \leftarrow \frac{R_1+R_3}{9}} \myvec{1&0&1\\3&18&-3\\4&-3&5} \\
            \xleftrightarrow[R_2 \leftarrow R_2-3R_1]{R_3 \leftarrow R_3-4R_1} \myvec{1&0&1\\0&18&-6\\0&-3&1} 
            \xleftrightarrow[R_2 \leftarrow \frac{R_2}{6}]{R_3 \leftarrow R_3+R_2} \myvec{1&0&1\\0&3&-1\\0&0&0}
        \end{align*}
	yielding
        \begin{align}
		\vec{u}_3 = \frac{1}{\sqrt{19}}\myvec{-3\\1\\3}
        \end{align}
        Using \eqref{eq:chapters/12/11/2/16/svd/decomp-1}, we see that
        \begin{align}
            \vec{U} = \myvec{-\frac{1}{\sqrt{38}}&\frac{1}{\sqrt{2}}&-\frac{3}{\sqrt{19}}\\-\frac{6}{\sqrt{38}}&0&\frac{1}{\sqrt{19}}\\\frac{1}{\sqrt{38}}&-\frac{1}{\sqrt{2}}&\frac{3}{\sqrt{19}}} \\
            \vec{D_1} = \myvec{19&0&0\\0&9&0\\0&0&0}
            \label{eq:chapters/12/11/2/16/svd/eig-params-1}
        \end{align}
        \item For $\vec{M}^\top\vec{M}$, the characteristic polynomial is
        \begin{align}
		\text{char}{\vec{M}^\top\vec{M}} &= \mydet{\lambda-14&5\\5&\lambda-14} \\
                                      &= \brak{\lambda-9}\brak{\lambda-19}
%                                      \label{eq:chapters/12/11/2/16/svd/char-1}
        \end{align}
        Thus, the eigenvalues are given by
        \begin{align}
            \lambda_1 = 19,\ \lambda_2 = 9
            \label{eq:chapters/12/11/2/16/svd/eig-vals}
        \end{align}
        For $\lambda_1$, the augmented matrix formed from the 
        eigenvalue-eigenvector equation is
        \begin{align}
            \myvec{-5&-5\\-5&-5} &\xleftrightarrow[]{R_1 \leftarrow R_1-R_2} \myvec{0&0\\-5&-5}
        \end{align}
        Hence, the normalized eigenvector is
        \begin{align}
		\vec{v}_1 = \frac{1}{\sqrt{2}}\myvec{1\\-1}
        \end{align}
        For $\lambda_2$, the augmented matrix formed from the 
        eigenvalue-eigenvector equation is
        \begin{align}
            \myvec{5&-5\\-5&5} &\xleftrightarrow[]{R_1 \leftarrow R_1+R_2} \myvec{0&0\\5&-5}
        \end{align}
        Hence, the normalized eigenvector is
        \begin{align}
		\vec{v}_2 = \frac{1}{\sqrt{2}}\myvec{1\\1}
        \end{align}
        Thus, from \eqref{eq:chapters/12/11/2/16/svd/decomp-2},
        \begin{align}
            \vec{V} = \myvec{\frac{1}{\sqrt{2}}&-\frac{1}{\sqrt{2}}\\\frac{1}{\sqrt{2}}&\frac{1}{\sqrt{2}}},\ \vec{D_2} = \myvec{9&0\\0&19}
            \label{eq:chapters/12/11/2/16/svd/eig-params-2}
        \end{align}
    \end{enumerate}
            Using \eqref{eq:chapters/12/11/2/16/svd/eig-vals},
    \begin{align}
        \vec{\Sigma} \triangleq \myvec{\sqrt{\lambda_1}&0\\0&\sqrt{\lambda_2}\\0&0}
         = \myvec{\sqrt{19}&0\\0&3\\0&0}
        \label{eq:chapters/12/11/2/16/svd/svd-params}
    \end{align}
    and substituting into 
	    \eqref{eq:chapters/12/11/2/16/svd/kappa-sol},
    \begin{align}
        \vec{\kappa} = \frac{1}{19}\myvec{10\\28}
    \end{align}
    which agrees with 
\eqref{eq:chapters/12/11/2/16/L2/svd/kappa}.
