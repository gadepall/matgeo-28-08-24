	\begin{figure}[H]
		\centering
 \includegraphics[width=0.75\columnwidth]{chapters/12/6/3/10/figs/conic1.pdf}
		\caption{}
		\label{fig:12/6/3/10}
  	\end{figure}
From the given information, 
\begin{align}
	\vec{V}
	=\myvec{
		0 & \frac{1}{2}\\\frac{1}{2}& 0\\
	},
\vec{u} = \myvec{
0 \\-\frac{1}{2}\\
},  f = -1, m=-1
	\label{eq:matrix-10-13-param}
\end{align}
From the above, the  normal vector is
\begin{align}
\vec{n}=\myvec{
-m \\ 1
	} = \myvec{1 \\ 1}
	\label{eq:matrix-10-13-param-n}
\end{align}
From 
\eqref{eq:conic_tangent_qk},
	the point(s) of contact are given by
\begin{align}
	\vec{q}&=\vec{V}^{-1}(k_i\vec{n}-\vec{u}) 
	\text{ where},\\
	k_i&=\pm \sqrt{\frac{f_0}{\vec{n}^{\top}\vec{V}^{-1}\vec{n}}}\\
	f_0&=f+\vec{u}^{\top}\vec{V}^{-1}\vec{u}
\end{align}
Substituting from 
	\eqref{eq:matrix-10-13-param-n}
	and
	\eqref{eq:matrix-10-13-param}
	in the above,
\begin{align}
\vec{q}=\myvec{0 \\-1}, \myvec{2 \\ 1}.
\end{align}
From 
  \eqref{eq:conic_tangent_final},
the equations of tangents are given by
\begin{align}
(\vec{V}\vec{q}+\vec{u})^{\top}\vec{x}+\vec{u}^{\top}\vec{q}+f=0
\end{align}
yielding
\begin{align}
	\myvec{1 & 1}\vec{x}+1&=0\\
\myvec{1 & 1}\vec{x}-3&=0\\
\end{align}
See 
		\figref{fig:12/6/3/10}.
