\documentclass{beamer}
\mode<presentation>
\usepackage{amsmath}
\usepackage{amssymb}
%\usepackage{advdate}
\usepackage{adjustbox}
\usepackage{subcaption}
%\usepackage{enumitem}
\usepackage{enumerate}
\usepackage{multicol}
\usepackage{mathtools}
\usepackage{listings}
\usepackage{url}
\def\UrlBreaks{\do\/\do-}
\usetheme{Boadilla}
\usecolortheme{lily}
\setbeamertemplate{footline}
{
  \leavevmode%
  \hbox{%
  \begin{beamercolorbox}[wd=\paperwidth,ht=2.25ex,dp=1ex,right]{author in head/foot}%
    \insertframenumber{} / \inserttotalframenumber\hspace*{2ex} 
  \end{beamercolorbox}}%
  \vskip0pt%
}
\setbeamertemplate{navigation symbols}{}

\providecommand{\nCr}[2]{\,^{#1}C_{#2}} % nCr
\providecommand{\nPr}[2]{\,^{#1}P_{#2}} % nPr
\providecommand{\mbf}{\mathbf}
\providecommand{\pr}[1]{\ensuremath{\Pr\left(#1\right)}}
\providecommand{\qfunc}[1]{\ensuremath{Q\left(#1\right)}}
\providecommand{\sbrak}[1]{\ensuremath{{}\left[#1\right]}}
\providecommand{\lsbrak}[1]{\ensuremath{{}\left[#1\right.}}
\providecommand{\rsbrak}[1]{\ensuremath{{}\left.#1\right]}}
\providecommand{\brak}[1]{\ensuremath{\left(#1\right)}}
\providecommand{\lbrak}[1]{\ensuremath{\left(#1\right.}}
\providecommand{\rbrak}[1]{\ensuremath{\left.#1\right)}}
\providecommand{\cbrak}[1]{\ensuremath{\left\{#1\right\}}}
\providecommand{\lcbrak}[1]{\ensuremath{\left\{#1\right.}}
\providecommand{\rcbrak}[1]{\ensuremath{\left.#1\right\}}}
\theoremstyle{remark}
\newtheorem{rem}{Remark}
\newcommand{\sgn}{\mathop{\mathrm{sgn}}}
\providecommand{\abs}[1]{\left\vert#1\right\vert}
\providecommand{\res}[1]{\Res\displaylimits_{#1}} 
\providecommand{\norm}[1]{\lVert#1\rVert}
\providecommand{\mtx}[1]{\mathbf{#1}}
\providecommand{\mean}[1]{E\left[ #1 \right]}
\providecommand{\fourier}{\overset{\mathcal{F}}{ \rightleftharpoons}}
%\providecommand{\hilbert}{\overset{\mathcal{H}}{ \rightleftharpoons}}
\providecommand{\system}{\overset{\mathcal{H}}{ \longleftrightarrow}}
	%\newcommand{\solution}[2]{\vec{Solution:}{#1}}
%\newcommand{\solution}{\noindent \vec{Solution: }}
\providecommand{\dec}[2]{\ensuremath{\overset{#1}{\underset{#2}{\gtrless}}}}
\newcommand{\myvec}[1]{\ensuremath{\begin{pmatrix}#1\end{pmatrix}}}
\newenvironment{amatrix}[1]{%
  \left(\begin{array}{@{}*{#1}{c}|c@{}}
}{%
  \end{array}\right)
}
\let\vec\mathbf

\lstset{
%language=C,
frame=single, 
breaklines=true,
columns=fullflexible
}

%\numberwithin{equation}{section}

\title{Fractal in Action}
\author{G. V. V. Sharma \\ Dept. of Electrical Engg.,\\IIT Hyderabad.}

\date{\today} 
\begin{document}

\begin{frame}
\titlepage
\end{frame}

\section*{Outline}
\begin{frame}
\tableofcontents
\end{frame}
\section{Class 10}
\begin{frame}
\frametitle{Question}
The centre of a circle is at (2,0). If one end of a diameter is at (6,0), then the other end is at :
\begin{enumerate}
\item $\brak{0,0}$
\item $\brak{4,0}$
\item $\brak{-2,0}$
\item $\brak{-6,0}$
\end{enumerate}
\end{frame}
%
\begin{frame}
\frametitle{Solution}
Let
\begin{align}
	\vec{O} =
    \myvec{
2 \\
0 
},\
\vec{A} =
    \myvec{
6 \\
0 
}
\end{align}
Then,
\begin{align}
\vec{O} &= \frac{\brak{\vec{A} + \vec{B}}}{2}\\
\implies \vec{B} &= 2\vec{O} - \vec{A}\\
    &=  2\myvec{
2 \\
0 
} -  \myvec{
6 \\
0 
}
=\myvec{
   -2\\
   0
}
\end{align}
\end{frame}
\begin{frame}
\frametitle{Question }
\begin{enumerate}
\item [2)]
$AD$ is a median of $\Delta ABC$ with vertices $A\brak{5,-6}, B\brak{6,4}$ and $C\brak{0,0}$. Length $AD$ is equal to:
\begin{enumerate}
\item  $\sqrt{68}$
\item  $2\sqrt{15}$
\item  $\sqrt{101}$
\item  $10$
\end{enumerate}
\end{enumerate}
\end{frame}
\begin{frame}
\frametitle{Solution}
The midpoint of $\vec{BC}$ is 
\begin{align}
    \vec{D} &= \frac{\vec{B} + \vec{C}}{2}\\
    &=\frac{1}{2}\myvec{
        6\\
        4
    }
    +
   \frac{1}{2} \myvec{
        0\\
        0
    } = \myvec{
        3\\
        2
    },
\end{align}
Since
\begin{align}
\vec{A}-\vec{D} &= \myvec{
        5\\
        -6
    }-\myvec{
        3\\
        2
    }=
    \myvec{
        2\\
        -8
    }
\\    \implies \lvert\lvert \vec{A}-\vec{D} \rvert\rvert &\triangleq \sqrt{\brak{\vec{A}-\vec{D}}^\top\brak{\vec{A}-\vec{D}}} \label{9}\\
    &= \sqrt{\myvec{
        2 & -8 
    }
    \myvec{
        2\\
        -8
    }}
    = \sqrt{2^2+8^2} =\sqrt{68} 
\end{align}
\end{frame}

\begin{frame}
\frametitle{Question }
If the distance between the points $\brak{3,-5}$ and $\brak{x,-5}$ is $15$ units, then the values of $x$ are
\begin{enumerate}
\item  $12,-18$
\item  $-12,18$
\item  $18,5$
\item  $-9,-12$
\end{enumerate}
\end{frame}
\begin{frame}
\frametitle{Solution}
\begin{align}
	\vec{A} &=
    \myvec{
3 \\
-5 
},\
\vec{B} =
    \myvec{
x \\
-5 
}\\
\implies \vec{A}-\vec{B} &= \myvec{
        3-x\\
        -5 -\brak{-5}
    } = \myvec{
        3-x\\
        0
    }
    \\
\implies      \lvert\lvert \vec{A}-\vec{B} \rvert\rvert 
&= \sqrt{\myvec{
        3-x & 0
    }
\myvec{
        3-x\\
        0
    }}
     = \sqrt{\brak{3-x}^2}\\
\implies  15 &= \pm\brak{3-x}\\
 \implies  x &= -12,18
\end{align}
\end{frame}
%
\begin{frame}
\frametitle{Question }
Solve the following system of linear equations algebraically:\\
$2x+5y=-4$;$4x-3y$=$5$
\end{frame}

\begin{frame}
\frametitle{Solution}
The above system of equations can be written as:
\begin{align}
    \myvec{
        2 & 5\\ 
        4 & -3
    }
    \myvec{
        x\\
        y
    } &= \myvec{
        -4\\
        5
    } \label{8}
    \end{align}
    Writing the augmented matrix for using Gauss elimination
    \begin{align}
        &\begin{amatrix}{2}
   2 & 5 & -4 \\  4 & -3 & 5
 \end{amatrix} \xleftarrow{R_2 \to R_2-2R_1}
  \myvec{
   2 & 5 & -4 \\  0 & -13 & 13
 } 
 \\
&\myvec{
   2 & 5 & -4 \\  0 & -13 & 13
 } \xleftarrow{R_1 \to \frac{13}{5}R_1+R_2}
 \myvec{
   \frac{26}{5} & 0 & \frac{13}{5} \\  0 & -13 & 13
 }  \label{14}
 \\
    & \implies \myvec{
        x\\
        y
    } =
    \myvec{
        \frac{1}{2}\\
        -1
    }
    \end{align}
\end{frame}






\begin{frame}
\frametitle{Question }
\begin{enumerate}
    \item[5)]
The sum of the digits of a $2$-digit number is 14. The number obtained by interchanging     its digits exceeds the given number by 18. Find the number.
\end{enumerate}
\end{frame}






\begin{frame}
\frametitle{Solution}
Let the digits of the number be $x_1$(tens) and $x_2$(units).Given
\begin{align}
    x_1+x_2 &= 14 \label{1}\\
    10x_2+x_1 &= 18 + 10x_1+x_2\\
    \implies x_1-x_2 &= -2\label{2}
\end{align}
Solving the equations $\eqref{1}$,$\eqref{2}$ in their matrix forms
\begin{align}
    \myvec{
        1 & 1\\
        -1 & 1
    }
    \myvec{
        x_1\\
        x_2
    } &= \myvec{
        14\\
        -2
    }   \\
    \end{align}
\end{frame}






\begin{frame}
\frametitle{Solution}
For the matrix
    \begin{align}
	    \vec{A}&=\frac{1}{\sqrt{2}}\myvec{
        1 & 1\\
        -1 & 1
    }\\
	    \vec{A}^\top \vec{A}&= \vec{I}    
    \end{align}
$\vec{A}$ is an orthogonal matrix.
    \begin{align}
            \myvec{
        1 & 1\\
        -1 & 1
    }
    \myvec{
        1 & 1\\
        -1 & 1
    }
    \myvec{
        x_1\\
        x_2
    } &= 
    \myvec{
        1 & 1\\
        -1 & 1
    }
    \myvec{
        14\\
        -2
    } \\
\implies    2\vec{I}\vec{x} &= \myvec{
        12\\
        16
    }\\
    \implies \vec{x} &= \myvec{
        6\\
        8
    }
\end{align}
\end{frame}
%
\begin{frame}
\frametitle{Question }
Find the ratio in which the point $C\brak{\frac{8}{5},y}$ divides the line segment joining the points $A\brak{1,2}$ and $B\brak{2,3}$. Also, find the value of $y$. 
\end{frame}
%
\begin{frame}
\frametitle{Solution}
    For collinearity,
    \begin{align}
          \text{rank}\myvec{
        1 & 1 & 1\\
        A & B & C
    } =2 \label{10}\\
    \myvec{
        1 & 1 & 1\\
        1 & 2 & 8/5\\
        2 & 3 & y
    } \xleftrightarrow{}
    \myvec{
        1 & 1 & 1\\
        0 & \brak{2-1} & \brak{\frac{8}{5}-1}\\
        0 & \brak{3-2} & \brak{y-3}
    }\xleftrightarrow{}
    \myvec{
        1 & 1 & 1\\
        0 & 1 & \frac{3}{5}\\
        0 & 1 & y-3
    }
    \\
    \xleftrightarrow{R_3\rightarrow R_3-R_2}
    \myvec{
        1 & 1 & 1\\
        0 & 1 & \frac{3}{5}\\
        0 & 0 & y-\frac{18}{5}
    }
    \implies y = \frac{18}{5}
\end{align}
\end{frame}
%
\begin{frame}
\frametitle{Question }
$ABCD$ is a rectangle formed by the points $A\brak{-1,-1}$,$B\brak{-1,6}$,$C\brak{3,6}$ and $D\brak{3,-1}$. $P$,$Q$,$R$ and $S$ are mid-points of sides $AB$,$BC$,$CD$ and $DA$ respectively. Show that the diagonal of the quadrilateral $PQRS$ bisect each other. 
\end{frame}
%
\begin{frame}
\frametitle{Solution}
\begin{align}
    \vec{P}&=\frac{\vec{A}+\vec{B}}{2},\
        \vec{Q}=\frac{\vec{B}+\vec{C}}{2}\\
    \vec{R}&=\frac{\vec{C}+\vec{D}}{2},\
    \vec{S}=\frac{\vec{D}+\vec{A}}{2}
    \end{align}
Let $\vec{O}_1$ and $\vec{O}_2$ be the midpoints of $PR$ and $QS$ respectively
\begin{align}
    \vec{O}_1 = \frac{\vec{P}+\vec{R}}{2}=\frac{\vec{A}+\vec{B}+\vec{C}+\vec{D}}{4}\\
      \vec{O}_2 = \frac{\vec{Q}+\vec{S}}{2}=\frac{\vec{A}+\vec{B}+\vec{C}+\vec{D}}{4}
\end{align}
Since the midpoints of the diagonals coincide, the diagonals bisect each other. 
\end{frame}
%
\begin{frame}{Topics covered so far}
    \begin{enumerate}
        \item {Vectors}
        \item {Section Formula}
        \item {Norm}
        \item {Gauss Elimination}
        \item {Orthogonal matrix}
        \item {Rank}
    \end{enumerate}
\end{frame}
%
\section{Class 12}
%
\begin{frame}
\frametitle{Question }
If $\overrightarrow{a} = 2\hat{i} - \hat{j} + \hat{k}$ and  $\overrightarrow{b} = \hat{i} + \hat{j} - \hat{k}$, then $\overrightarrow{a}$ and $\overrightarrow{b}$ are:
    \begin{enumerate}
\item  Collinear vectors which are not parallel
\item  Parallel vectors
\item  Perpendicular vectors
\item  Unit vectors
\end{enumerate}
\end{frame}
%
\begin{frame}
\frametitle{Solution}
Let
\begin{align}
    \vec{a} &= \myvec{
        2\\
        -1\\
        1
    } , 
    \vec{b}=\myvec{
        1\\
        1\\
        -1
    }
    \end{align}
    Applying concept of rank from $\eqref{10}$
    \begin{align}
                  \text{rank}\myvec{
        1 & 1 & -1\\
        2 & -1 & 1
    } &=2 \neq 1 ,
    \text{Not parallel}
    \end{align}
    Applying condition for perpendicularity:
    \begin{align}
            \vec{a}^{\top}\vec{b} = \myvec{
        2 &-1 &1
    }\myvec{
        1\\
        1\\
        -1
    } &= 0
    \implies \vec{a} \perp \vec{b}
\end{align}
\end{frame}
%
\begin{frame}
\frametitle{Question }
\begin{enumerate}
    \item [2)]
If $\alpha$,$\beta$ and $\gamma$ are the angles which a line makes with positive directions of $x$,$y$ and $z$ axes respectively, then which of the following are {not} true?
    \begin{enumerate}
\item  $\cos^2{\alpha} + \cos^2{\beta} + \cos^2{\gamma} = 1$
\item  $\sin^2{\alpha} + \sin^2{\beta} + \sin^2{\gamma} = 2$
\item  $\cos{2\alpha} + \cos{2\beta} + \cos{2\gamma} =-1$
\item  $\cos{\alpha} + \cos{\beta} + \cos{\gamma} = 1$
\end{enumerate}
\end{enumerate}
\end{frame}
%
\begin{frame}
\frametitle{Solution}
Let $\vec{m}$ represent the unit direction vector of the line.  Then,
\begin{align}
    \vec{m} = \myvec{
        \cos{\alpha} \\
        \cos{\beta}\\
        \cos{\gamma}
    }
\end{align}
with 
\begin{align}
	\norm{\vec{m}} =  1
\end{align}
From $\eqref{8}$
\begin{align}
    2x + 5y = -4
    \end{align}
    can be expressed as
\begin{align}
    \vec{n}^{\top}\vec{x} &= c
\end{align}
where
\begin{align}
	\vec{n} = \myvec{2\\5},\ c = -4
\end{align}
\end{frame}
%
\begin{frame}
\frametitle{Solution}
This is defined to be the normal equation of a line, where $\vec{n}$ represents the normal vector.
Also, 
\begin{align}
    2x + 5y &= -4\\
    \implies 2x &= -4 -5y
    \\
    \implies
    \myvec{
        x\\
        y
    } &= \myvec{
        -2\\
        0
    } + y\myvec{
        -\frac{5}{2}\\
        1
    }\\
    \vec{x} &= \myvec{
        -2\\
        0
    } -\frac{5y}{2}\myvec{
        1\\
        -\frac{2}{5}
    }\\
    &= \vec{A} + k\vec{m} \label{11}
\end{align}
The general equation written above is of the parametric form.
\end{frame}
%
\begin{frame}
\frametitle{Question }
$\overrightarrow{a}$,$\overrightarrow{b}$ and $\overrightarrow{c}$ are three mutually perpendicular unit vectors. If $\theta$ is the angle between $\overrightarrow{a}$ and $\brak{\overrightarrow{2a}+\overrightarrow{3b}+\overrightarrow{6c}}$, find the value of $\cos{\theta}$. 
\end{frame}
%
\begin{frame}
\frametitle{Solution}
Given:
\begin{align}
    \vec{a}^{\top}\vec{b} &=  \vec{b}^{\top}\vec{c} =  \vec{c}^{\top}\vec{a} = 0\\
    \lvert \lvert \vec{a} \rvert \rvert 
&= \lvert \lvert \vec{b} \rvert \rvert = \lvert \lvert \vec{c} \rvert \rvert = 1\\
\cos{\theta} &= \frac{\vec{a}^{\top}\brak{2\vec{a}+3\vec{b}+6\vec{c}}}{\lvert \lvert \vec{a} \rvert \rvert \lvert \lvert 2\vec{a} + 3\vec{b} + 6\vec{c} \rvert \rvert}
\end{align}
Now,
\begin{align}
    \vec{a}^{\top}\brak{2\vec{a}+3\vec{b}+6\vec{c}} &= 2\vec{a}^{\top}\vec{a} + 3\vec{a}^{\top}\vec{b} + 6\vec{a}^{\top}\vec{c} = 2 + 0 + 0 = 2\\
    \lvert \lvert \vec{a} \rvert \rvert \lvert \lvert 2\vec{a} + 3\vec{b} + 6\vec{c} \rvert \rvert &= \lvert \lvert 2\vec{a} + 3\vec{b} + 6\vec{c} \rvert \rvert
    \end{align}
\end{frame}









\begin{frame}
\frametitle{Solution}
From $\eqref{9}$ norm definition:
    \begin{align}
        \brak{\lvert \lvert 2\vec{a} + 3\vec{b} + 6\vec{c} \rvert \rvert}^2 &= \lvert \lvert 4\vec{a}^2 \rvert \rvert + \lvert \lvert 9\vec{b}^2 \rvert \rvert + \lvert \lvert 36\vec{c}^2  \rvert \rvert = 49\\
    \implies \lvert \lvert 2\vec{a} + 3\vec{b} + 6\vec{c} \rvert \rvert &= +7\\
    \implies \cos{\theta} &= \frac{2}{7}
\end{align}
    
\end{frame}







\begin{frame}
\frametitle{Question }
Find the position vector of point $\vec{C}$ which divides the line segment joining points $\vec{A}$ and $\vec{B}$ having position vectors $\hat{i} + 2\hat{j} - \hat{k}$ and $-\hat{i} + \hat{j} + \hat{k}$ respectively in the ratio $4:1$ externally. Further, find $\lvert \overrightarrow{AB}\rvert : \lvert \overrightarrow{BC} \rvert$. 
\end{frame}





\begin{frame}
\frametitle{Solution}
We know that
\begin{align}
\vec{C} = \frac{4\vec{B}-\vec{A}}{4-1}
\end{align}
Simplify the above for $\vec{C}$. 
    
\end{frame}
%
\begin{frame}
\frametitle{Question }
\begin{enumerate}
    \item [5)]
Find the equation of the line passing through the point of intersection of the lines $\frac{x}{1} = \frac{y-1}{2} = \frac{z-2}{3}$ and $\frac{x-1}{0} = \frac{y}{-3} = \frac{z-7}{2}$ and perpendicular to these given lines. 
\end{enumerate}
\end{frame}
%
\begin{frame}
\frametitle{Solution}
Let the given lines be denoted by $\vec{x}_1$ and $\vec{x}_2$ respectively. From $\eqref{11}$:
\begin{align}
    \vec{x}_1 &= \myvec{
        0\\
        1\\
        2
    } + k_1\myvec{
        1\\
        2\\
        3
    } = \vec{A} + k_1\vec{m}_1 \label{12} \\
    \vec{x}_2 &= \myvec{
        1\\
        0\\
        7
    } + k_2\myvec{
        0\\
        -3\\
        2
    } = \vec{B} + k_2\vec{m}_2 \label{13}
\end{align}
\end{frame}





\begin{frame}
\frametitle{Solution}
The two equations required to solve for the direction of line are 
\begin{align}
\vec{m}^\top\vec{m}_1 &= 0\\
\vec{m}^\top\vec{m}_2 &= 0\\
	\implies \myvec{\vec{m}_1 &\vec{m}_2}^{\top}\vec{m} &= 0
\end{align}
\begin{align}
    \myvec{
        1 & 2 & 3\\
        0 & -3 & 2   \\
    }\vec{m} &\xleftarrow{R_1 \to 2R_2 + 3R_1} \myvec{
        3 & 0 & 13\\
        0 & -3 & 2   \\
    }\vec{m} = 0\\
  &\implies  \myvec{
        3 & 0 & 13\\
        0 & -3 & 2   \\
    }\myvec{
        m_x\\
        m_y\\
        m_z
    } = 0\\
    &\implies \vec{m} = \myvec{
        \frac{-13}{3}\\
        \frac{2}{3}\\
        1
    }
\end{align}
\end{frame}









\begin{frame}
\frametitle{Solution}
Let the unknown line in its parametric form be denoted as follows from $\eqref{11}$.
\begin{align}
    \vec{x}_3 = \vec{C} + k_3\vec{m}
\end{align}
For finding the point of intersection of the lines($\vec{C}$), equating $\eqref{12}$ and $\eqref{13}$
\begin{align}
    \vec{A} + k_1\vec{m}_1 = \vec{B} + k_2\vec{m}_2\\
    \myvec{
        \vec{m}_1 & \vec{m}_2
    }\myvec{
        k_1 \\
        -k_2
    } = \vec{B}-\vec{A}
\end{align}
From the above $k_1$ and $k_2$ can be found by gauss elimination given in $\eqref{14}$ and thus $\vec{C}$.

\end{frame}









\begin{frame}
\frametitle{Question }
Two vertices of the parallelogram $\vec{ABCD}$ are given as $\vec{A}\brak{-1,2,1}$ and $\vec{B}\brak{1,-2,5}$. If the equation if the line passing through $\vec{C}$ and $\vec{D}$ is $\frac{x-4}{1} = \frac{y+7}{-2} = \frac{z-8}{2}$, then find the distance between the sides $\vec{AB}$ and $\vec{CD}$. Hence, find the area of parallelogram $\vec{ABCD}$
\end{frame}
%
\begin{frame}
\frametitle{Solution}
Let  the two parallel lines be
\begin{align}
    \vec{x} & = \vec{A} + k_1\vec{m}  \\
    \vec{x} &=  \vec{B} + k_2\vec{m} 
\end{align}
If $\vec{P}$ be a point on the second line, 
\begin{align}
    \vec{P} &=  \vec{B} + k_2\vec{m} 
    \\
	\brak{\vec{A}-\vec{P}}^{\top}\vec{m} &= 0
\end{align}
From the above, 
\begin{align}
	\brak{\vec{A}-\vec{B}}^{\top}\vec{m} -k_2\norm{\vec{m}}^2&= 0
	\\
	\implies k_2 = \frac{
		\brak{\vec{A}-\vec{B}}^{\top}\vec{m}}{ \norm{\vec{m}}^2}
\end{align}
    
\end{frame}








    \section{CLASS 10}






\begin{frame}
\frametitle{Question }
\begin{enumerate}
    \item [8)]
The sum of first and eight terms of an A.P is $32$ and their product is $60$. Find the first term and common difference of the A.P. Hence, also find the sum of its first $20$ terms. 
\end{enumerate}
\end{frame}





\begin{frame}
\frametitle{Solution}
Let the first and eighth terms be $x\brak{0}$ and $x\brak{7}$ respectively,given:
\begin{align}
x\brak{0}+x\brak{7} &= 32\\
x\brak{0} &= 32 - x\brak{7}\label{3} \\
x\brak{0}x\brak{7} &= 60 \label{4}
\end{align}
From $\eqref{3}$ and $\eqref{4}$
\begin{align}
    x\brak{7}\brak{32-x\brak{7}} &= 60
\end{align}
The roots are $\brak{30,2}$, therefore ,if $x\brak{7}=30$ then $x\brak{0}=2$ and if $x\brak{7}=2$ then $x_0=30$\\
Now
\begin{align}
    x\brak{n} &= \brak{x\brak{0} + nd}u\brak{n}\label{5}
    \end{align}
    Where $d$ is the common difference of the A.P and $u(n)$ is the unit step function.\\($u\brak{n}=0 \forall n<0$,$u\brak{n}=1 \forall n\geq0$)
\end{frame}






\begin{frame}
\frametitle{Solution}
\begin{align}
        \implies     x\brak{7} &= \brak{x\brak{0} + 7d}\\
    \implies 7d &= \pm 28 \implies d = \pm 4
\end{align}
Therefore the A.P is $2,6,10...$ or $30,26,22...$.\\
Considering the former for calculations and taking Z-Transform of $\eqref{5}$ for sum.\\
Since
\begin{align}
X(z) &= \sum_{n=-\infty}^{\infty} x(n)z^{-n} \label{6}
\end{align}
Let $y\brak{n}$ denote the sum, let:
\begin{align}
    y\brak{n} &= x\brak{n} * h\brak{n}\\
    &= \sum_{k=-\infty}^{\infty} x(k)h(n-k)
\end{align}    
\end{frame}





\begin{frame}
\frametitle{Solution}
Replace $h\brak{n}$ with $u\brak{n}$.
\begin{align}
    y\brak{n} &= \sum_{k=0}^{n} x(k)u_{\brak{n-k}}\\
    &=x(0)u\brak{n} + x(1)u\brak{n-1} + ..... x(n)u_{\brak{0}}
\end{align}
    This denotes the sum of terms $x(0),x(1)....x(n)$ i.e. first $n+1$ terms. From \eqref{6}\\
\begin{align}
    u\brak{n} &\xrightarrow{\mathcal{Z}} \frac{1}{\brak{1-z^{-1}}} \label{7}\\
nu\brak{n} &\xrightarrow{\mathcal{Z}} \frac{z^{-1}}{\brak{1-z^{-1}}^{2}}\\
  \implies  X(z)&= \frac{2}{\brak{1-z^{-1}}} + \frac{4z^{-1}}{\brak{1-z^{-1}}^{2}} \label{eq:ee25-4}
,\quad \abs {z}>\abs{1} 
\end{align}
\end{frame}





\begin{frame}
\frametitle{Solution}
Now as convolution in the time domain corresponds to multiplication in the frequency domain and \eqref{eq:ee25-4} and \eqref{7}.
\begin{align}
    Y\brak{z} &= X\brak{z} * H\brak{z}\\
 &= \brak{\frac{2}{\brak{1-z^{-1}}} +
\frac{4z^{-1}}{\brak{1-z^{-1}}^{2}}}\brak{\frac{1}{\brak{1-z^{-1}}}}
,\quad \abs {z}>\abs{1}     
\end{align}
Using normal inversion for inverse Z-transform:
\begin{align}
 Y(z) &= \frac{2}{\brak{1-z^{-1}}^2} +
\frac{4z^{-1}}{\brak{1-z^{-1}}^{3}}
,\quad \abs {z}>\abs{1}\\
   &= \frac{8z^{-1}}{1-z^{-1}} + \frac{10z^{-2}}{\brak{1-z^{-1}}^2} + \frac{4z^{-3}}{\brak{1-z^{-1}}^{3}} + 2\label{final}
\end{align}
    
\end{frame}




\begin{frame}
\frametitle{Solution}
For proceeding forwards here are some important generalizations.\\
Shifting property
\begin{align}
x(n-k) \leftrightarrow z^{-k} X(z) \label{shift}
\end{align}
Differentiation property
\begin{align}
nx(n) \leftrightarrow -zX^{\prime}(z) \label{diff}
\end{align}
From \eqref{7} and \eqref{shift}
\begin{align}
u\brak{n-1} &\xrightarrow{\mathcal{Z}} \frac{z^{-1}}{1-z^{-1}} \label{21}
\end{align}
From $\eqref{7}$ and $\eqref{diff}$
\begin{align}
nu\brak{n} &\xrightarrow{\mathcal{Z}} -z\frac{d}{dz}\brak{\frac{1}{1-z^{-1}}}\\
nu\brak{n}    &\xrightarrow{\mathcal{Z}} \frac{z^{-1}}{\brak{1-z^{-1}}^{2}}
\end{align}
    
\end{frame}












\begin{frame}
\frametitle{Solution}
From $\eqref{shift}$
\begin{align}
(n-1)u\brak{n-1}    &\xrightarrow{\mathcal{Z}} z^{-1}\frac{z^{-1}}{\brak{1-z^{-1}}^{2}}\\
(n-1)u\brak{n-1}    &\xrightarrow{\mathcal{Z}} \frac{z^{-2}}{\brak{1-z^{-1}}^{2}} \label{22}
\end{align}
Now, using $\eqref{diff}$ and writing the corresponding L.H.S
\begin{align}   
(n)(n-1)u\brak{n-1}    &\xrightarrow{\mathcal{Z}} \frac{2z^{-2}}{\brak{1-z^{-1}}^3} 
\end{align}
Using $\eqref{shift}$
\begin{align}
    \frac{(n-1)(n-2)u\brak{n-2}}{2}    &\xrightarrow{\mathcal{Z}} \frac{z^{-3}}{\brak{1-z^{-1}}^3} \label{23} 
\end{align}
\end{frame}















\begin{frame}
\frametitle{Solution}
The inverse-Z of a constant will be $\delta(n)$, so it is ruled out.
Plugging these values in $\eqref{final}$ we get
\begin{align}
   y(n) = 8u\brak{n-1} + 10(n-1)u\brak{n-1} &+ 4\frac{(n-1)(n-2)u\brak{n-2}}{2} \notag \\
   &+ 2\delta(n)
\end{align}
Putting $n=19$
\begin{align}
      y(19) &= 2(19+1)^2 = 800
\end{align}
    
\end{frame}







\begin{frame}
\frametitle{Solution}
Using contour integration for inverse Z-transform
\begin{align}
    y(19)&=\frac{1}{2\pi j}\oint_{C}Y(z) \;z^{18} \;dz\\  
 &=\frac{1}{2\pi j}\oint_{C}\brak{2z^{20}\brak{z-1}^{-2}+
       4z^{20}\brak{z-1}^{-3}} \;dz\\
       R&=\frac{1}{\brak {m-1}!}\lim\limits_{z\to a}\frac{d^{m-1}}{dz^{m-1}}\brak {{(z-a)}^{m}f\brak z}\label{eq:6}  
\end{align}
For $R_1$ , $m=2$ , where $m$ corresponds to number of repeated poles .
\begin{align}
    R_1 &=\frac{1}{\brak {1}!}\lim\limits_{z\to 1}\frac{d}{dz}\brak {{(z-1)}^{2}2z^{20}\brak{z-1}^{-2}}   \\
    &=2\lim\limits_{z\to 1}\frac{d}{dz}(z^{20})   \\
    &= 40
    \end{align}
    
\end{frame}





\begin{frame}
\frametitle{Solution}
\begin{align}
    R_2 &=\frac{1}{\brak {2}!}\lim\limits_{z\to 1}\frac{d^{2}}{dz^{2}}\brak {{(z-1)}^{3}4z^{20}\brak{z-1}^{-3}}   \\
    &=\brak2\lim\limits_{z\to 1}\frac{d^2}{dz^2}(z^{20})   \\
    &= 760\\
    R_1 + R_2 &= 800\\
    \implies  y{(19)} &= 800
\end{align}
Similarly, the sum for the A.P. $30,26,22...$ can be found by the same procedure.
    
\end{frame}









\begin{frame}
\frametitle{Question }
\begin{enumerate}
    \item [9)]
In an A.P. of $40$ terms, the sum of first $9$ terms is $153$ and the sum of last $6$ terms is $687$. Determine the first term and the common difference of the A.P. Also find the sum of all the terms of the A.P. 
\end{enumerate}
\end{frame}
Given:
\begin{align}
y(8) &= 153 \label{17}\\
y(39)-y(34) &= 687 \label{18}
\end{align}
Now, let the first term be $x(0)$ and common difference be $d$. From $\eqref{6}$ and $\eqref{5}$
\begin{align}
    X(z) &= \frac{x(0)}{1-z^{-1}} + \frac{dz^{-1}}{\brak{1-z^{-1}}^2} ,\quad \abs {z}>\abs{1} 
\end{align}
For finding the sum (Assuming $h(n)=u\brak{n}$)
\begin{align}
    y\brak{n} &= x\brak{n} * h\brak{n}\\
Y\brak{z} &= X\brak{z} * H\brak{z}\\
&= \brak{\frac{x(0)}{\brak{1-z^{-1}}} +
\frac{dz^{-1}}{\brak{1-z^{-1}}^{2}}}\brak{\frac{1}{\brak{1-z^{-1}}}}
,\quad \abs {z}>\abs{1}     
\end{align}







\begin{frame}
\frametitle{Solution}
\begin{align}
 Y\brak{z}   = \frac{\brak{2x(0)+d}z^{-1}}{1-z^{-1}} + \frac{\brak{x(0)+2d}z^{-2}}{\brak{1-z^{-1}}^2} + \frac{dz^{-3}}{\brak{1-z^{-1}}^{3}} + x(0)
\end{align}
Using normal inversion for inverse Z-transform:\\
Using the results $\eqref{21}$,$\eqref{22}$ and $\eqref{23}$
\begin{align}
   y(n) = (2x_0+d)u\brak{n-1} &+ (n-1)u\brak{n-1}(x_0+2d) + \notag\\
    &\frac{d(n-1)(n-2)u\brak{n-2}}{2} + x(0)\delta(n)
\end{align}
Now use $\eqref{17}$ and $\eqref{18}$ to solve for $x(0)$ and $d$ and put in $\eqref{19}$ for the sum of $40$ terms.
\end{frame}








\begin{frame}
\frametitle{Solution}
Using contour integration for inverse Z-transform
\begin{align}
    y(n)&=\frac{1}{2\pi j}\oint_{C}Y(z) \;z^{n-1} \;dz\\  
 &=\frac{1}{2\pi j}\oint_{C}\brak{x(0)z^{n+1}\brak{z-1}^{-2}+
       dz^{20}\brak{z-1}^{-3}} \;dz\\
       R&=\frac{1}{\brak {m-1}!}\lim\limits_{z\to a}\frac{d^{m-1}}{dz^{m-1}}\brak {{(z-a)}^{m}f\brak z}
\end{align}
For $R_1$ , $m=2$ , where $m$ corresponds to number of repeated poles .
\begin{align}
    R_1 &=\frac{1}{\brak {1}!}\lim\limits_{z\to 1}\frac{d}{dz}\brak {{(z-1)}^{2}x(0)z^{n+1}\brak{z-1}^{-2}}   \\
    &=x(0)\lim\limits_{z\to 1}\frac{d}{dz}(z^{n+1})   \\
    &= \brak{n+1}x(0)
    \end{align}
\end{frame}












\begin{frame}
\frametitle{Solution}
   For $R_2$ , $m=3$ 
    \begin{align}
    R_2 &=\frac{1}{\brak {2}!}\lim\limits_{z\to 1}\frac{d^{2}}{dz^{2}}\brak {{(z-1)}^{3}dz^{n+1}\brak{z-1}^{-3}}   \\
        &=\brak{\frac{d}{2}}\lim\limits_{z\to 1}\frac{d^2}{dz^2}(z^{n+1})   \\
    &= \brak{\frac{d}{2}}\brak{n}\brak{n+1}\\
   y(n)= R_1 + R_2 &= \brak{\frac{n+1}{2}}\brak{2x(0) + nd} \label{19}
\end{align}
Now use $\eqref{17}$ and $\eqref{18}$ to solve for $x(0)$ and $d$ and put in $\eqref{19}$ for the sum of $40$ terms.
\end{frame} 

\begin{frame}
\frametitle{Solution}
\begin{itemize}

\item Hardware products to be available at low cost.
\item Related skill sets to be provided to the public through simple projects. 
\end{itemize}
\end{frame}
\section{Market}
\begin{frame}
\frametitle{Market}
\begin{itemize}

\item Students 
\item Working professionals
\item Industry and academia
\end{itemize}
\end{frame}
\section{Product}
\begin{frame}
\frametitle{Product}
    \begin{figure}[t]
        \centering
        \includegraphics[width=0.3\columnwidth]{figs/kit}
        \caption{Decimal Game}
\label{fig3:subfig1}        
    \end{figure}%
\end{frame}
\section{Financials}
\begin{frame}
\frametitle{Financials}
\begin{itemize}
\item Investment - negligible
\item Expected profit - $> 50 \%$
\end{itemize}

\end{frame}
\section{Team}
\begin{frame}
\frametitle{Team}
\begin{itemize}
\item G V V   Sharma
\item Faculty in the EE Dept at IITH
\item 22 years exp. 
\item Currently running the FWC certificiate program at IITH.
\item Leading the SatCom 6G team at IITH.  
\end{itemize}
%\item I want to take the technologies developed at IITH to the masses.
\end{frame}
%\begin{frame}
%\frametitle{Introduction}
%\framesubtitle{Literature}
%%\begin{figure}[t!]
%%    \centering
%%    \begin{subfigure}[t]{0.4\columnwidth}
%%        \centering
%%        \includegraphics[width=\columnwidth]{point_source}
%%        \caption{Single point source}
%%\label{fig3:subfig1}        
%%    \end{subfigure}%
%%    ~ 
%%    \begin{subfigure}[t]{0.4\columnwidth}
%%        \centering
%%        \includegraphics[width=\columnwidth]{pointNoPowerDist_new}
%%        \caption{SNR profile}
%%\label{fig3:subfig2}
%%    \end{subfigure}
%%  %  \caption{Average SNR for a BPP. $N=16$}
%%    \label{fig3}
%%  \end{figure}
%
%\end{frame}
%  
%
%
%%

\end{document}
