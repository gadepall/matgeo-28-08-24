\documentclass[journal]{IEEEtran}
\usepackage{gvv-book}
\usepackage{gvv}

\makeindex

\begin{document}
\bibliographystyle{IEEEtran}
\onecolumn


\title{
	\begin{flushleft}
	MATRICES \\ In Geometry
	\\
\rule{0.4\columnwidth}{0.4pt}
\end{flushleft}
}
\author{
\vspace{7cm}
	\begin{flushleft}
\includegraphics[width=0.2\columnwidth]{figs/logo.jpg}
\\
		{	\huge G. V. V. Sharma}
		\\
\vspace{1cm}
https://github.com/gadepall/matgeo
		\\
\vspace{1cm}
https://creativecommons.org/licenses/by-sa/3.0/
\\
and
\\
https://www.gnu.org/licenses/fdl-1.3.en.html
	\end{flushleft}
%\IEEEpubid{\makebox[\columnwidth]{978-1-7281-5966-1/20/\$31.00 ©2020 IEEE \hfill} \hspace{\columnsep}\makebox[\columnwidth]{ }}
}
\maketitle

\newpage


\tableofcontents

\newpage
\twocolumn


%\renewcommand{\theequation}{\theenumi}
\numberwithin{equation}{enumi}
\renewcommand{\thefigure}{\theenumi}
\renewcommand{\thetable}{\theenumi}


\section{Vectors}
\subsection{Addition and Subtraction}
\begin{enumerate}[label=\thesubsection.\arabic*,ref=\thesubsection.\theenumi]
\item A girl walks 4 km towards west, then she walks 3 km in a direction 30$^{\circ}$ east of north and stops. Determine the girl's displacement from her initial point of departure.\\
	\solution
		See  
\figref{fig:chapters/12/10/5/3Fig1}.
Let the initial position
be
\begin{align}
	\vec{A}=\myvec{0\\0}
\end{align}
After going west, the position becomes
\begin{align}
			\vec{B}=\myvec{-4\\0}
\end{align}
If the final position be $\vec{C}$, from the given information,
\begin{align}
	 \vec{C}-\vec{B}=3\myvec{\cos{60\degree}\\\sin{60\degree}}
	 \implies 
	\vec{C}  
=\myvec{-\frac{5}{2}\\[2pt] \frac{3\sqrt{3}}{2}}
\end{align}
which is the desired displacement. 
\begin{figure}[H] 
 \begin{center} 
 \includegraphics[width=0.75\columnwidth]{chapters/12/10/5/3/figs/fig.pdf} 
 \end{center} 
\caption{} 
\label{fig:chapters/12/10/5/3Fig1} 
\end{figure}

\item Without using distance formula, show that points A(– 2, – 1), B(4, 0), C(3, 3) and D(–3, 2) are the vertices of a parallelogram.
\label{chapters/11/10/1/9}
\\
\solution
	  From \eqref{eq:two-pgm},
\begin{align}
\vec{A}-\vec{B} = 
\vec{D}-\vec{C} =  \myvec{-6\\-1}
\end{align}
Hence, $ABCD$ is a parallelogram.
See \figref{fig:chapters/11/10/1/91}.
\begin{figure}[H]
  \centering
   %\includegraphics[width=0.75\columnwidth]{chapters/11/10/1/9/figs/paralellogram.png}
   \includegraphics[width=0.75\columnwidth]{chapters/11/10/1/9/figs/fig.pdf}
    \caption{}
     \label{fig:chapters/11/10/1/91}  
\end{figure}




\item The fourth vertex $\vec{D}$ of a parallelogram $\vec{ABCD}$ whose three vertices are
	$\vec{A} (–2, 3), \vec{B} (6, 7)\text { and } \vec{C} (8, 3)$ is
\begin{enumerate}
	\item $(0, 1)$
	\item $(0, –1)$
	\item $ (–1,0)$
	\item$(1, 0)$
\end{enumerate}
\item Points $\vec{A}(4,3), \vec{B}(6,4),\vec{C}(5,-6)$  and  $\vec{D}(-3,5)$ are the vertices of a parallelogram.
\item The vector having intial and terminal points as (2,5,0)and (-3,7,4),respectively is
	\begin{enumerate}
\item -$\hat{i}+12\hat{j}+4\hat{k}$
\item $5\hat{i}+2\hat{j}-4\hat{k}$
\item $5\hat{i}+2\hat{j}+4\hat{k}$
\item $\hat{i}+\hat{j}+\hat{k}$
\end{enumerate}
\item Find the sum of the vectors $\vec{a}=\hat{i}-2\hat{j}+\hat{k}$, $\vec{b}=-2\hat{i}+4\hat{j}+5\hat{k}$ and $\vec{c}=\hat{i}-6\hat{j}-7\hat{k}$.
\end{enumerate}

\subsection{Section Formula}
\begin{enumerate}[label=\thesubsection.\arabic*,ref=\thesubsection.\theenumi]
\item Find the coordinates of the point which divides the join of $(-1,7) $ and $ (4,-3)$ in the ratio 2:3.
	\\
		\solution
	Using section formula \eqref{eq:section_formula}, the desired point is
\begin{align}
\frac{1}{1+\frac{3}{2}}  \myvec{\myvec{
4\\
-3
}
  +
   \frac{3}{2}\myvec{
-1\\
7
}}
=\myvec{
1\\
3
}
\end{align}
See 
\figref{fig:chapters/10/7/2/1/Fig}
\begin{figure}[H]
\begin{center}
   \includegraphics[width=0.75\columnwidth]{chapters/10/7/2/1/figs/fig.pdf}
\end{center}
\caption{}
\label{fig:chapters/10/7/2/1/Fig}
\end{figure}


\item Find the coordinates of the point $\vec{R}$ on the line segment joining the points $\vec{P}(-1,3)$ and $\vec{Q}(2,5)$ such that $PR=\frac{3}{5}PQ$.
\item Find the ratio in which the point $\vec{P}\brak{\frac{3}{4},\frac{5}{12}}$ divides the line segment joining the points $\vec{A}\brak{\frac{1}{2},\frac{3}{2}}$ and $ \vec{B}(2,-5)$.
\item Find the coordinates of the point which divides the line segment joining the points $(4,-3)$ and $(8,5)$ in the ratio $3:1$ internally.
\item Find the coordinates of the point $\vec{P}$ on $AD$ such that $AP : PD = 2 : 1$.
\item If the point $\vec{P} (2, 1)$ lies on the line segment joining points $\vec{A} (4, 2)$  and $ \vec{B} (8, 4)$,
then
\begin{enumerate}
	\item $AP =\frac{1}{3}{AB}$ 
\item ${AP}={PE}$
\item ${PB}=\frac{1}{3}{AB}$
\item${AP}=\frac{1}{2}{AB}$
 \end{enumerate}
\item Find the ratio in which the line segment joining the points $(-3,10)$  and  $(6,-8)$  is divided by $ (-1,6)$.
	\\
		\solution
	Using section formula,
\begin{align}
         \myvec{-1\\6} &=\frac{{\myvec{-3\\10}+k\myvec{6\\-8}}}{1+k}\\
	 \implies 7k\myvec{1 \\ -2} &= 2\myvec{1 \\ -2}
	 \\
	 \text{or, } k &= \frac{2}{7}.
\end{align}
\iffalse
See \figref{fig:10/7/2/4Fig1}.
\begin{figure}[H]
 \begin{center}
  \includegraphics[width=0.75\columnwidth]{chapters/10/7/2/4/figs/fig.png}
 \end{center}
\caption{}
\label{fig:10/7/2/4Fig1}
\end{figure}
\fi

\item Find the position vector of the mid point of the vector joining the points $\vec{P}$(2, 3, 4)
and $\vec{Q}$(4, 1, –2).
\\
\solution
		The desired vector is
\begin{align}
\frac{1}{2}\myvec{2\\3\\4} +  \frac{1}{2}\myvec{4\\1\\-2} =\myvec{3\\2\\1} 
\end{align}




\item Let $\vec{A}(4, 2), \vec{B}(6, 5)$  and $ \vec{C}(1, 4)$ be the vertices of $\triangle ABC$.
\begin{enumerate}
\item If $\vec{A}$ and  $\vec{B}$ are $(-2,-2)$ and  $(2,-4)$, respectively, find the coordinates of $\vec{P}$ such that $AP= \frac {3}{7}AB$  and $ \vec{P}$ lies on the line segment $AB$.
	\\
		\solution
	Using section formula, 
\begin{align}
\vec{P}&=\frac{1}{1+\frac{3}{4}}\brak{\myvec{-2\\-2}+\frac{3}{4}\myvec{2\\-4}}
=\myvec{\frac{-2}{7}\\[1pt] \frac{-20}{7}}
\end{align}
\iffalse
See 
   \figref{fig:chapters/10/7/2/8/vec.png}.
\begin{figure}
   \centering 
 \includegraphics[width=0.75\columnwidth]{chapters/10/7/2/8/figs/vec.png}
   \caption{}
   \label{fig:chapters/10/7/2/8/vec.png}
   \end{figure}
   \fi

\item Find the coordinates of the points which divide the line segment joining $A(-2,2)$  and  $\vec{B}(2,8)$ into four equal parts.
	\\
		\solution
	Using section formula,
\begin{align}
\vec{R}_k=\frac{\vec{B}+k\vec{A}}{1+k}
\end{align}
See 
\tabref{tab:10/7/2/9}
and 
\figref{fig:chapters/10/7/2/9/Fig}
\begin{table}[H]
\centering
\caption{}
\label{tab:10/7/2/9}
\begin{tabular}{|c|c|}
\hline
	$k$ & $\vec{R}_k$ \\
\hline
3 & 
\myvec{
-1\\
\\
\frac{7}{2}
}\\
\hline
1 & \myvec{
0\\
5
}
\\
\hline
	$\frac{1}{3}$ &\myvec{
1
\\
\frac{13}{2}
}
 \\
\hline
\end{tabular}
\end{table}
\begin{figure}[H]
\begin{center}
   \includegraphics[width=0.75\columnwidth]{chapters/10/7/2/9/figs/10.7.2.9.png}
\end{center}
\caption{}
\label{fig:chapters/10/7/2/9/Fig}
\end{figure}


\item In what ratio does the point $(-4,6)$ divide the line segment joining the points $\vec{A}(-6,0)$ and $\vec{B}(3,-8)$?
\item Given that $\vec{P}(3,2,-4), \vec{Q}(5,4,-6)$ and $\vec{R}(9,8,-10)$ are collinear. Find the ratio in which $\vec{Q}$ divides $PR$.
\item Points $\vec{A}(-6,10),\vec{B}(-4,6)$  and  $\vec{C}(3,-8)$ are collinear such that $AB=  \frac{2}{9}AC$.
\item The point which divides the line segment joining the points $\vec{P} (7, –6) $  and  $\vec{Q}(3, 4)$ in the 
ratio 1 : 2 internally lies in  which quadrant?
\item Find the coordinates of the points of trisection of the line segment joining $(4,-1)$  and  $(-2,3)$.
	\\
		\solution
	Using section formula,
\begin{align}
\vec{R}=\frac{1}{1+\frac{1}{2}}\brak{\myvec{4\\-1}+\frac{1}{2}\myvec{-2\\-3}}
=\myvec{2\\ \frac{-5}{3}}\\
\vec{S}=\frac{1}{1+\frac{2}{1}}\brak{\myvec{4\\-1}+\frac{2}{1}\myvec{-2\\-3}}
=\myvec{0\\ \frac{-7}{3}}
\end{align}
which are the desired points of trisection.
\iffalse
See
		\figref{fig:chapters/10/7/2/2/Figure}
\begin{figure}[H]
\centering
\includegraphics[width=0.75\columnwidth]{chapters/10/7/2/2/figs/dj.pdf}
\caption{}
		\label{fig:chapters/10/7/2/2/Figure}
\end{figure}
\fi

\item Find the coordinates of the points which trisect the line segment joining the points $\vec{P}(4,2,-6)$ and $\vec{Q}(10,-16,6)$.
\item Find the coordinates of the points of trisection (i.e. points dividing to three equal parts) of the line segment joining the points $\vec{A}(2,-2)$ and $\vec{B}(-7,4)$.
\item Point $\vec{P}(5,-3)$ is one of the two points of trisection of line segment joining the points $\vec{A}(7,-2)$ and $\vec{B}(1,-5)$
\item Find the position vector of a point $\vec{R}$ which divides the line joining two points $\vec{P}$
and $\vec{Q}$ whose position vectors are $\hat{i}+2\hat{j}-\hat{k}$ and $-\hat{i}+\hat{j}+\hat{k}$ respectively, in the
ratio 2 : 1
\begin{enumerate}
    \item  internally
    \item  externally
\end{enumerate}
%\solution
%		See 
\tabref{tab:chapters/12/10/2/15/}.
\begin{table}[H]
\centering
\caption{}
\label{tab:chapters/12/10/2/15/}
\begin{tabular}{|c|c|}
\hline
$k$ & $R_k$ \\
\hline
2 & $\frac{1}{3}\myvec{-1 \\ 4 \\ 1}$ \\
\hline
-2 & $\myvec{-3 \\ 0 \\ 3}$ \\
\hline
\end{tabular}
\end{table}

\item Find the coordinates of the point which divides the line segment joining the points which divides the line segment joining  the points $(-2,3,5)$ and $(1,-4,6)$ in the ratio 
\begin{enumerate}
\item $2:3$ internally,
\item $2:3$ externally
\end{enumerate}
\item Find the coordinates of the point which divides the line segment joining the points $(1,-2,3)$ and $(3,4,-5)$ in the ratio $2:3$
\begin{enumerate}
\item internally, and
\item externally
\end{enumerate}
\item Consider two points $\vec{P}$ and $\vec{Q}$ with position vectors $\overrightarrow{OP} = 3\overrightarrow{a}-2\overrightarrow{b}$ and $\overrightarrow{OQ}=\overrightarrow{a}+\overrightarrow{b}$. Find the position vector of a point $\vec{R}$ which divides the line joining $\vec{P}$ and $\vec{Q}$ in the ratio $2:1$, 
\begin{enumerate}
\item internally, and 
\item externally.
\end{enumerate}
\item The median from $\vec{A}$ meets $BC$ at $\vec{D}$. Find the coordinates of the point $\vec{D}$.
\item Find the coordinates of points $\vec{Q}$ and $\vec{R}$ on medians $BE$ and $CF$ respectively such that $BQ : QE = 2 : 1$  and  $CR : RF = 2 : 1$.
\item What do you observe?
\item If $\vec{A}, \vec{B}$ and $\vec{C}$  are the vertices of $\triangle ABC$, find the coordinates of the centroid of the triangle.
\end{enumerate}
\solution
	\begin{align}
\vec{D}&=\frac{\vec{B}+\vec{C}}{2}
=\myvec{\frac{7}{2}\\[2pt] \frac{9}{2}},\
\vec{E}=\frac{\vec{A}+\vec{C}}{2}
=\myvec{\frac{5}{2}\\ 3}\\
\vec{F}&=\frac{\vec{A}+\vec{B}}{2}
=\myvec{5\\ \frac{7}{2}}
,\
\vec{G}
	=\vec{Q}
=\vec{R}
=\frac{1}{3}\myvec{11\\11}
\end{align} 
is the centroid.
See 
  \figref{fig:chapters/10/7/4/7/Figure}.
\begin{figure}[H]
\centering
\includegraphics[width=0.75\columnwidth]{chapters/10/7/4/7/figs/fig.pdf}
\caption{}
  \label{fig:chapters/10/7/4/7/Figure}
\end{figure}

\item If $\vec{P}(9a-2,-b)$ divides line segment joining $\vec{A}(3a+1,-3)$ and $\vec{B}(8a,5)$ in the ratio 3:1, find the values of $a$ and $b$.
\item Find the position vector of a point $\vec{R}$ which divides the line joining two points $\vec{P}$ and $\vec{Q}$ whose position vectors are $2\vec{a}+\vec{b}$ and $\vec{a}-3\vec{b}$ externally in the ratio $1:2$.
\item The position vector of the point which divides the join of points 2$\vec{a}$-3$\vec{b}$ $\text{and}$ $\vec{a}+\vec{b}$ in the ratio 3:1 is
\item If $\vec{a}$ and $\vec{b}$ are the postion vectors of $\vec{A}$ and $\vec{B}$, respectively, find the position vector of a point $\vec{C}$ in $BA$ produced such that $BC=1.5BA$.
\item Find the position vector of a point $\vec{R}$ which divides the line joining two points $\vec{P}$ and $\vec{Q}$ whose position vectors are $(2\vec{a}+\vec{b})$ and $(\vec{a}-3\vec{b})$
externally in the ratio 1 : 2. Also, show that $\vec{P}$ is the mid point of the line segment $RQ$.
\end{enumerate}

\subsection{Formulae}
%\begin{enumerate}[label=\arabic*.,ref=\theenumi]
\begin{enumerate}[label=\thesubsection.\arabic*.,ref=\thesubsection.\theenumi]
\numberwithin{equation}{enumi}
  \item If $ABCD$ be a parallelogram,
	  \label{prop:two-pgm}
  \begin{align}
	  \label{eq:two-pgm}
 \vec{B}-\vec{A} = \vec{C} -\vec{D}
  \end{align}
  \item 
If $PQRS$ is formed by joining the mid points of $ABCD$, 
\begin{align}
  \vec{P} = \frac{1}{2}\brak{\vec{A}+\vec{B}} 
  ,\,
 \vec{Q} = \frac{1}{2}\brak{\vec{B}+\vec{C}} 
 \\
 \vec{R} = \frac{1}{2}\brak{\vec{C}+\vec{D}}   
  ,\,
 \vec{S} = \frac{1}{2}\brak{\vec{D}+\vec{A}}  
 \\
	\implies 
 \vec{P}-\vec{Q} = \vec{S} -\vec{R}.
  \label{eq:10/7/4/8det2f}
\end{align}
Hence, $PQRS$ is a parallelogram
	  from \eqref{eq:two-pgm}.
\end{enumerate}

\subsection{Rank}
\begin{enumerate}[label=\thesubsection.\arabic*, ref=\thesubsection.\theenumi]
\item 
Prove that the three points (3,  0),  (– 2,  – 2) and (8,  2) are collinear.
\label{chapters/11/10/2/20}
	\\
	\solution 
			From \eqref{eq:line-rank-2},
the collinearity matrix can be expressed as
 \begin{align}
			    \myvec{-5 & -2
			    \\
			    5 & 2 }  
			    \xleftrightarrow[]{R_2 \leftarrow {R_1 + R_2}}
			    \myvec{	    -5 & -2  
			    \\
			    0 & 0}  
\end{align}
which is a rank 1 matrix. The above process is known as row reduction, where we try to obtain zero rows in the matrix using arithmetic operations.  The number of nonzero rows in the row reduced matrix (also known as {\em echelon form})
is defined as the rank.
		\figref{fig:11/10/2/20}.
	\begin{figure}[H]
		\centering
 \includegraphics[width=0.75\columnwidth]{chapters/11/10/2/20/figs/fig.pdf}
		\caption{}
		\label{fig:11/10/2/20}
  	\end{figure}

\item Show that the points $\vec{A}(1, 2, 7),  \vec{B}(2, 6, 3)$ and $\vec{C}(3, 10, -1)$ are collinear.
	\\
		\solution The matrix
\begin{align}
	\myvec{\vec{B}-\vec{A}& \vec{C}-\vec{A}}^\top 
	= \myvec{1 & 4 & -4 \\ 2 & 8 & -8}
	\\
	\xleftrightarrow[]{R_2 = R_2 - 2R_1}
	 \myvec{1 & 4 & -4 \\ 0 & 0 & 0}
\end{align}
which has rank 1.  Using 
			\eqref{eq:mat-rank-t}, 
			 we conclude that the given points are collinear.
\item Determine if the points $(1, 5), (2, 3)$ and $(-2, -11)$ are collinear.
\item Show that the vectors $2\hat{i}-3\hat{j}+4\hat{k}$ and $-4\hat{i}+6\hat{j}-8\hat{k}$ are collinear.
\item Show that the points (2,  3,  4),  (–1,  –2,  1),  (5,  8,  7) are collinear.
\item In each of the following,  find the value of $k$,  for which the points are collinear.
\begin{enumerate}
\item $(7,  –2),  (5,  1),  (3,  k)$
\item $(8,  1),  (k,  – 4),  (2,  –5)$
\end{enumerate}
		\label{10/7/3/2}
\item Find a relation between $x$ and $y$ if the points $(x,  y),  (1,  2)$  and  $(7,  0)$ are collinear.
\item If three points $(x,  -1),  (2,  1)$ and $(4,  5)$ are collinear,  find the value of $x$.
\label{chapters/11/10/1/8}
\item If three points $(h,  0),  (a,  b)$ and $(0,  k)$ lie on a line,  
show that 
\begin{align}
\frac{a}{h}+\frac{b}{k}=1
\end{align}
\label{chapters/11/10/1/13}
\item Show that the points $\vec{A} (1,  -2,  -8),  \vec{B} (5,  0,  -2)$ and $\vec{C} (11,  3,  7)$ are collinear,  and find the ratio in which $\vec{B}$ divides $AC$.
\item lf the points $\vec{A}(1, 2), \vec{0}(0, 0)$ and $\vec{C}(a, b)$ are collinear, then find the relation between $a$ and $b$.
	\item Point $ (-4, 2)$ lies on the line segment joining the points $ \vec{A}(-4, 6)$  and  $\vec{B}(-4, -6)$.
 \item The points $(0, 5), (0, -9)$ and $(3, 6)$ are collinear.
\item Points $\vec{A}(3, 1),  \vec{B}(12, -2)$  and  $\vec {C}(0, 2)$ cannot be the vertices of a triangle.
\item Find the value of $m$ if the points $(5, 1), (-2, -3)$  and $(8, 2m)$ are collinear.
\item Find the values of $k$ if the points $\vec{A}(k+1, 2k), \vec{B}(3k, 2k+3)$ and $\vec{C}(5k-1, 5k)$ are collinear.
\item Using vectors,  find the value of $k$ such that the points $(k, -10, 3)$,  $(1, -1, 3)$  and  $(3, 5, 3)$ are collinear.
\item The points $\vec{A}(2, 1)$,  $\vec{B}(0, 5)$,  $\vec{C}(-1, 2)$ are collinear.
\item The vectors $\lambda\hat{i}+\lambda\hat{j}+2\hat{k}$,  $\hat{i}+\lambda\hat{j}-\hat{k}$ $\text{ and }$ $2\hat{i}-\hat{j}+\lambda\hat{k}$ are coplanar if
	$\lambda=$
\item Show that the points $(-2, 3, 5),  (1, 2, 3)$ and $(7, 0, -1)$ are collinear.
\item Show that points $\vec{A}(a,  b+c),  \vec{B}(b,  c+a),  \vec{C}(c,  a+b)$ are collinear.
\item Show that the points $\vec{A}(2, -3, 4),  \vec{B}(-1, 2, 1)$ and $\vec{C}(0, \frac{1}{3}, 2)$ are collinear.
\item Are $\vec{A}(3, 1), \vec{B}(6, 4)$ and $\vec{C}(8, 6)$ collinear?
\item Find the values of $k$ if the points $\vec{A}(2, 3),  \vec{B}(4, k)$ and $\vec{C}(6, -3)$ are collinear.
\item Three points $\vec{P}(h, k),  \vec{Q}(x_1, y_1)$ and $\vec{R}(x_2, y_2)$ lie on a line. Show that $(h-x_1)(y_2-y_1)=(k-y_1)(x_2-x_1)$.
\item Show that the points $\vec{P}(-2, 3, 5),  \vec{Q}(1, 2, 3)$ and $\vec{R}(7, 0, -1)$ are collinear. 
\item Prove that the three points $(-4, 6, 10),  (2, 4, 6)$ and $(14, 0, -2)$ are collinear.
\item Show that the points $\vec{A}(-2\hat{i} +3\hat{j} +5\hat{k}),  \vec{B}(\hat{i}+2\hat{j} +3\hat{k}$ and $\vec{C}(7\hat{i} -\hat{k})$ are collinear.
\item Show that the points $\vec{A}(2,  3,  -4),  \vec{B}(1,  -2,  3)$ and $\vec{C}(3,  8,  -11)$ are collinear.
\end{enumerate}

\subsection{Length}
\begin{enumerate}[label=\thesubsection.\arabic*, ref=\thesubsection.\theenumi]
\item Compute the magnitude of the following vectors:
\begin{align}
	\vec{a}&=\hat{i}+\hat{j}+\hat{k}
	\\
	\vec{b}&=2\hat{i}-7\hat{j}-3\hat{k}
	\\
	\vec{c}&=\frac{1}{\sqrt{3}}\hat{i}+\frac{1}{\sqrt{3}}\hat{j}-\frac{1}{3}\hat{k}
\end{align}
    \solution 
		Let 
\begin{align}
	\vec{a} = \myvec{1\\1\\1} , \vec{b} = \myvec{2\\ -7 \\ 3}, 
\vec{c} = \myvec{\dfrac{1}{\sqrt{3}}\\[2ex] \dfrac{1}{\sqrt{3}} \\[2ex] -\dfrac{1}{\sqrt{3}}} 
\label{eq:chapters/12/10/2/1/1}
\end{align}
Then
\begin{align}
	{\vec{a}^{\top}\vec{a}} &= \myvec{1  &  1  &  1}\myvec{1\\1\\1} = 3
\\
\implies 
	\norm{\vec{a}}&=\sqrt{3}, 
	\label{eq:chapters/12/10/2/1/3}
\end{align}
		from \eqref{eq:side-length}. Similarly,
\begin{align}
	\norm{\vec{b}}&=\sqrt{\vec{b}^{\top}\vec{b}}= \sqrt{62}, 
	\label{eq:chapters/12/10/2/1/4}
	\\ \norm{\vec{c}}&=\sqrt{\vec{c}^{\top}\vec{c}}	
=1
	\label{eq:chapters/12/10/2/1/5}
\end{align}




\item Find the distance between the following pairs of points:
\begin{enumerate}[label=(\roman*)]
\item $(2, 3, 5)$ and $(4, 3, 1)$
\item $(-3, 7, 2)$ and $(2, 4, -1)$
\item $(-1, 3, -4)$ and $(1, -3, 4)$
\item $(2, -1, 3)$ and $(-2, 1, 3)$
\end{enumerate}
\item Find the lengths of the medians of the triangle with vertices $A(0, 0, 6),  B(0, 4, 0)$ and $(6, 0, 0)$.
\item Find the coordinates of a point on y-axis which is at a distance of $5\sqrt2$ from the point $P(3, -2, 5)$.
\item If $A$ and $B$ be the points $(3, 4, 5)$ and $(-1, 3, -7)$ respectively,  find the equation of the set of the points $P$ such that $PA^2+PB^2=K^2$ where $K$ is a constant.
\item Find the distances between the following pairs of points:
\begin{enumerate}
\item $(2, 3), (4, 1)$
\item $(-5, 7), (-1, 3)$
\item $(a, b), (-a, -b)$
\end{enumerate}
\solution
		\begin{enumerate}
\item 
	\begin{align}
\because
		\vec{A} - \vec{B} = \myvec{2\\3} - \myvec{4\\1} &= \myvec{-2\\2},		
\\
(\vec{A}-\vec{B})^\top (\vec{A}-\vec{B}) &= 8
	\end{align}
	Thus, the desired distance is 
	\begin{align}
		d=\norm{\vec{A}-\vec{B}} =\sqrt{8}
	\end{align}
\item 
	\begin{align}
		\vec{C} - \vec{D} = \myvec{-5\\7} - \myvec{-1\\3} &= \myvec{-4\\4}		
		\\
		\implies		(\vec{C}-\vec{D})^\top (\vec{C}-\vec{D}) &= 32
	\end{align}
Thus,	
	\begin{align}
		d=\norm{\vec{C}-\vec{D}}
 =4\sqrt{2}
\end{align}	
%	
\item 
	\begin{align}
\vec{E} - \vec{F} = \myvec{a\\b} - \myvec{-a\\-b} &= \myvec{2a\\2b}		
		\\
		\implies
		(\vec{E}-\vec{F})^\top (\vec{E}-\vec{F}) = 4a^2+4b^2 
	\end{align}
Thus,	
	\begin{align}
		d=\norm{\vec{E}-\vec{F}} =
2\sqrt{a^2+b^2}
\end{align}	
\iffalse
\begin{figure}[H]
	\begin{center} 
	    %\includegraphics[width=0.75\columnwidth]{chapters/10/7/1/1/figs/graph.png}
	    \includegraphics[width=0.75\columnwidth]{chapters/10/7/1/1/figs/fig.pdf}
	\end{center}
\caption{}
\label{fig:10/7/1/1Fig}
\end{figure}
\fi
\end{enumerate}

\item Find the distance between the points $(0, 0)$ and $ (36, 15)$.
	\\
		\solution
		\begin{align}
\vec{A}&=\myvec{0 \\ 0},\  
\Vec{B}=\myvec{36 \\ 15} \\ 
\implies 
\vec{d}&=\norm{\vec{A}-\vec{B}}
=39
\end{align}
\iffalse
See 
\figref{fig:10/7/1/2vec}.
\begin{figure}[H]
\centering
\includegraphics[width=0.75\columnwidth]{chapters/10/7/1/2/figs/vec.pdf}
\caption{}
\label{fig:10/7/1/2vec}
\end{figure}
\fi

\item Find the point on the x-axis which is equidistant from $(2, -5)$ and $(-2, 9)$.
	\label{it:10/7/1/7}
	\\
\solution
				The input parameters for this problem are available in Table \ref{tab:10/7/1/7Table-1}
\begin{table}[H]
\input{chapters/10/7/1/7/tables/table.tex}
\caption{}
\label{tab:10/7/1/7Table-1}	
\end{table}
%
  If $\vec{O}$ lies on the  $x$-axis and is  equidistant from the points $\vec{A}$ and $\vec{B}$, 
\begin{align}
 \norm{\vec{O}-\vec{A}} &=
\norm{\vec{A}-\vec{B}} 
\\
 \implies \norm{\vec{O}-\vec{A}}^2 &=
\norm{\vec{O}-\vec{B}}^2 
\\
 \implies \norm{\vec{O}}^2-2{\vec{O}}^{\top}\vec{A} + \norm{\vec{A}}^2
	&= \norm{\vec{O}}^2-2{\vec{O}}^{\top}\vec{B} + \norm{\vec{B}}^2,
\end{align}
which can be simplified to obtain
  \begin{align}
	  \brak{\vec{A}-\vec{B}}^\top   \vec{O}&=\frac{\norm{\vec{A}}^2 -\norm{\vec{B}}^2 }{2}.
  \end{align}
  \begin{align}
  \because
   \vec{O} &=
    x\vec{e}_1,
  \end{align}
  \begin{align}
   x &=\frac{\norm{\vec{A}}^2 -\norm{\vec{B}}^2 }{2\brak{\vec{A}-\vec{B}}^{\top }\vec{e}_1.
}\label{eq:10/7/1/75}  
  \end{align}
  Substituting from \tabref{tab:10/7/1/7Table-1} in \eqref{eq:10/7/1/75},
 $x =  -7$.  Thus, 
		\begin{align}
\vec{O} = \myvec{ -7 \\ 0}.
		\end{align}
		See  
\figref{fig:10/7/1/7Fig1}.
\begin{figure}[H]
 \begin{center}
  \includegraphics[width=0.75\columnwidth]{chapters/10/7/1/7/figs/fig.pdf}
 \end{center}
\caption{}
\label{fig:10/7/1/7Fig1}
\end{figure}


\item Find the values of $y$ for which the distance between the points                  $\vec{P}(2, -3)$ and $\vec{Q}(10, y)$ is 10 units.
\item  If $\vec{Q}(0,  1)$ is equidistant from $\vec{P}(5,  -3)$ and $\vec{R}(x,  6)$,  find the values of $x$. Also find the
distances $QR$ and $PR$.
\item  Find a relation between $x$ and $y$ such that the point $(x, y)$ is equidistant from the point
$(3,  6)$ and $(– 3,  4)$.
	\item Find a point on the x-axis, which is equidistant from the points $\myvec{
  7 \\
  6 \\
 }$ and $\myvec{
  3 \\
  4 \\
 }$
.
\label{chapters/11/10/1/4}
	\\
	\solution 
	Using \probref{it:10/7/1/7},
\begin{align}
	\vec{x} &=x\vec{e}_1
	\implies x = \frac{15}{2}
\end{align}
\iffalse
		See \figref{fig:11/10/1/4}.
	\begin{figure}[H]
		\centering
 \includegraphics[width=0.75\columnwidth]{chapters/11/10/1/4/figs/line.png}
		\caption{}
		\label{fig:11/10/1/4}
  	\end{figure}
	\fi

\item The distance between the points $\vec{A}(0,  6) \text{ and } \vec{B}(0,  –2)$ is
	\begin{enumerate}
\item 6
\item 8
\item 4	
\item 2
	\end{enumerate}
\item The distance of the point $\vec{P} (–6,  8)$ from the origin is
\begin{enumerate}

\item 8
\item 2$\sqrt{7}$ 
\item 10
\item 6
\end{enumerate}
\item The distance between the points $\vec(0,  5)\text{ and }(–5,  0)$ is
\begin{enumerate}

\item 5
\item 5
\item 5
\item 10
\end{enumerate}
\item $\vec{AOBC}$ is a rectangle whose three vertices are vertices $\vec{A} (0,  3),  \vec{O}(0,  0)\text{ and }
	\vec{B} (5,  0)$. The length of its diagonal is
\begin{enumerate}
\item 5
\item3
\item 34
\item 4
\end{enumerate}
\item The perimeter of a triangle with vertices $\vec(0,  4),  (0,  0) \text{ and } (3,  0)$ is
\begin{enumerate}

\item 5
\item 12
\item 11
\item 7
\end{enumerate}
\item If the distance between the points $(4, P)$  and $ (1, 0)$ is 5, then the value of ${P}$ is
\begin{enumerate}                       
\item4 only
\item+4 only
\item-4 only
\item0
\end{enumerate}
\item Find the points on the $x$-axis which are at a distance on $2\sqrt{5}$ from the point $ (7, -4).$ How many such points are there?
\item Find the value of $a$,  if the if the distance between the points $\vec{A}(-3, -14)$  and $\vec{B}(a, -5)$ is 9 units.
\item Find a point which is equidistant from the points $\vec{A}(-5, 4)$  and $(-1, 6)$.  How many such points are there ?
\item If the point $\vec{A}(2, -4)$ is equidistant from $\vec{P}(3, 8)$  and $\vec{Q}(-10, y)$,  find the values of $y$.  Also find distance $PQ$.
\item If $(a, b)$ is the mid-point of the line segment joining the point $\vec{A}(10, -6)\text{ and }\vec{B}(k, 4)$ and $a-2b=18$,  find the value of $a, b$ and the distance $AB$.
\item Find a relation between $x$ and $y$ such that the point $(x,y)$ is equidistant from the points $(7,1)$ and $(3,5)$.
\item Find a point on the Y-axis which is equidistant from the points $A(6,5)$ and $B(-4,3)$.
\item Find the equation of set of points $P$ such that $PA^2+PB^2=2k^2$, where $A$ and $B$ are the points $(3,4,5)$ and $(-1,3,-7)$, respectively.
\item Find the equation of the set of the points $P$ such that its distances from the points $A(3,4,-5)$ and $B(-2,1,4)$ are equal.
\end{enumerate}

%
\subsection{Direction}
\begin{enumerate}[label=\thesubsection.\arabic*, ref=\thesubsection.\theenumi]
	\item 		Find the values of $x$ and $y$ so that the vectors
$2\hat{i}+3\hat{j}$
and 
$x\hat{i}+y\hat{j}$
are equal.
\\
\solution
From the given informatin, 
\begin{align}
	\myvec{2\\3} = \myvec{x \\ y} 
	\\
	\implies x = 2, y = 3
\end{align}

\item Find the values of $x, y, z$ so that the vectors 
$x\hat{i}+2\hat{j}+z\hat{k}$
and 
$2\hat{i}+y\hat{j}+\hat{k}$
are equal.
\item Find the sum of the vectors $\vec{a}=\hat{i}-2\hat{j}+\hat{k}$,  $\vec{b}=-2\hat{i}+4\hat{j}+5\hat{k}$ and $\vec{c}=\hat{i}-6\hat{j}-7\hat{k}$.
\item Find the slope of a line,  which passes through the origin and the mid point of the line segment joining the points $\vec{P}$(0, -4) and $\vec{B}$(8, 0).
\label{chapters/11/10/1/5}
	\\
	\solution
The mid point of $PB$ is
\begin{align}
\vec{M} =\frac{1}{2}(\vec{P}+\vec{B})
	= \myvec{4 \\ -2}  
\end{align}
which is equal to the direction vector of $OM$.
\begin{align}
\because \vec{M} \equiv
	 \myvec{1 \\ -\frac{1}{2}},
	m = -\frac{1}{2}
\end{align}
which is the desired slope.
See 
		\figref{fig:11/10/1/5}.
	\begin{figure}[!ht]
		\centering
 \includegraphics[width=\columnwidth]{chapters/11/10/1/5/figs/line.png}
		\caption{}
		\label{fig:11/10/1/5}
  	\end{figure}

\item Find the angle between x-axis and the line joining points (3, -1) and (4, -2).
\label{chapters/11/10/1/10}
\\
\solution 
The direction vector of the given line is 
\begin{align}
	\vec{C}
=\myvec{ -1\\ 1 }
\end{align}
Hence, the desired angle is given by
\begin{align}
	\cos\theta=\frac{\vec{C}^{\top}\vec{e}_1}{\norm{\vec{C}}\norm{\vec{e}_1}}
	&= -\frac{1}{\sqrt{2}}
	\\
	\implies 
	\theta&=135\degree
 \end{align}

\item A line passes through $\vec{A}(x_1, y_1)$ and $\vec{B}(h, k)$. If slope of the line is m,  show that $(k-y_1)=m(h-x_1)$.
\label{chapters/11/10/1/12}
\\
\solution 
The direction vector
\begin{align}
	\vec{B}-\vec{A}
	=
	\myvec{
  h-x_1\\
  k-y_1
  }
   \equiv
	\myvec{
1\\
	\frac{ k-y_1}{h-x_1}
  }
  \\
	\implies m = 
	\frac{ k-y_1}{h-x_1},
\end{align}
yielding the desired result.

\item
Show that the line through the points \brak{4, 7, 8}, \brak{2, 3, 4} is parallel to the line through the points \brak{-1, -2, 1}, \brak{1, 2, 5}.
	\label{12.11.2.3}
\\
\solution
	\begin{align}
\myvec{4 \\ 7 \\ 8}- \myvec{2 \\ 3 \\ 4}= \myvec{-1 \\ -2 \\ 1}- \myvec{1 \\ 2 \\ 5}
\equiv \myvec{2\\4\\4}
\end{align}
which means that the given lines have the same direction vector and are hence parallel.

\item The vector having intial and terminal points as (-2, 5, 0) and (3, 7, 4), respectively is
\solution
The desired vector is
\begin{align}
	\myvec{3 \\ 7 \\ 4}
	-\myvec{-2 \\ 5 \\ 0} = 
	\myvec{5 \\ 2 \\ 4}  
\end{align}
\item Find the vector joining the points $\vec{P}\brak{2, 3, 0}$ and $\vec{Q}\brak{-1, -2, -4}$ directed from $\vec{P}$ to $\vec{Q}$.
\item Without using distance formula,  show that points $\vec{A}(– 2,  – 1),  \vec{B}(4,  0),  \vec{C}(3,  3)$ and $\vec{D}(–3,  2)$ are the vertices of a parallelogram.
\label{chapters/11/10/1/9}
\\
\solution
	  From \eqref{eq:two-pgm},
\begin{align}
\vec{A}-\vec{B} = 
\vec{D}-\vec{C} =  \myvec{-6\\-1}
\end{align}
Hence, $ABCD$ is a parallelogram.
See \figref{fig:chapters/11/10/1/91}.
\begin{figure}[H]
  \centering
   %\includegraphics[width=0.75\columnwidth]{chapters/11/10/1/9/figs/paralellogram.png}
   \includegraphics[width=0.75\columnwidth]{chapters/11/10/1/9/figs/fig.pdf}
    \caption{}
     \label{fig:chapters/11/10/1/91}  
\end{figure}




\item If the points $\vec{A}(6,  1),  \vec{B}(8,  2),  \vec{C}(9,  4)$ and $\vec{D}(p,  3)$ are the vertices of a parallelogram,  taken in order,  find the value of $p$.
\label{10/7/0/10}
\item 
If $(1,  2),  (4,  y),  (x,  6)$ and $(3,  5)$ are the vertices of a parallelogram taken in order,  find $x$ and $y$.
\label{10/7/2/6}
\item The fourth vertex $\vec{D}$ of a parallelogram $ABCD$ whose three vertices are
	$\vec{A} (–2,  3),  \vec{B} (6,  7)$ and  $\vec{C} (8,  3)$ is
\item Verify if the points $\vec{A}(4, 3),  \vec{B}(6, 4), \vec{C}(5, -6)$  and  $\vec{D}(-3, 5)$ are the vertices of a parallelogram.
\item A girl walks 4 km towards west,  then she walks 3 km in a direction 30$^{\circ}$ east of north and stops. Determine the girl's displacement from her initial point of departure.\\
	\solution
		See  
\figref{fig:chapters/12/10/5/3Fig1}.
Let the initial position
be
\begin{align}
	\vec{A}=\myvec{0\\0}
\end{align}
After going west, the position becomes
\begin{align}
			\vec{B}=\myvec{-4\\0}
\end{align}
If the final position be $\vec{C}$, from the given information,
\begin{align}
	 \vec{C}-\vec{B}=3\myvec{\cos{60\degree}\\\sin{60\degree}}
	 \implies 
	\vec{C}  
=\myvec{-\frac{5}{2}\\[2pt] \frac{3\sqrt{3}}{2}}
\end{align}
which is the desired displacement. 
\begin{figure}[H] 
 \begin{center} 
 \includegraphics[width=0.75\columnwidth]{chapters/12/10/5/3/figs/fig.pdf} 
 \end{center} 
\caption{} 
\label{fig:chapters/12/10/5/3Fig1} 
\end{figure}

\item $(-1, 2, 1),  (1, -2, 5),  (4, -7, 8)$ and $(2, -3, 4)$ are the vertices of a parallelogram.
\item Three vertices of a parallelogram $ABCD$ are $\vec{A}(3, -1, 2),  \vec{B}(1, -2, 4)$ and $\vec{C}(-1, 1, 2)$. Find the coordinates of the fourth vertex.
\item If the origin is the centroid of the triangle $PQR$ with vertices $\vec{P}(2a, 2, 6),  \vec{Q}(-4, 3b, -10)$ and $R(8, 14, 2c)$,  then find the values of $a,  b$ and $c$.
\item Find the slope of lines
\begin{enumerate}
\item  Passing through the points $(3, -2)$ and $(-1, 4)$
\item  Passing through the points $(3, -2)$ and $(7, -2)$
\item  passing through the points $(3, -2)$ and $(3, 4)$	
\item  Making inclination of $60\degree$ with the positive direction of x-axis.
\end{enumerate}
\item The centroid of a triangle $ABC$ is at the point $(1, 1, 1)$. If the coordinates of $\vec{A}$ and $\vec{B}$ are $(3, -5, 7)$ and $(-1, 7, -6)$,  respectively find the coordinates of the point $\vec{C}$.
\item Represent graphically a displacement of $40$ km,  $30\degree$ west of south.
	\item Rain is falling vertically with a speed of 35 $m s^{-1}$
. Winds starts blowing after sometime with a speed of 12 $m s^{-1}$ in
east to west direction. In which direction should a boy waiting at a bus stop hold his umbrella ?
%
\item A motorboat is racing towards north at 25 km/h and the water current in that region is 10 km/h in the direction of 60$\degree$ east of south. Find the resultant velocity of the boat.
\item Rain is falling vertically with a speed of 35 $m s^{-1}$
. A woman rides a bicycle with a speed of 12 $ms^{-1}$ in east to west
direction. What is the direction in which she should hold her umbrella ?
\item Rain is falling vertically with a speed of 30 $m s^{-1}$. A woman rides a bicycle with a speed  of 10 $m s^{-1}$ in the north to south direction. What is the direction in which she should
hold her umbrella?
\item A man can swim with a speed of 4.0 km/h in still water. How long does he take to cross a river 1.0 km wide if the river flows steadily at 3.0 km/h and he makes his strokes normal to the river current? How far down the river does he go when he reaches the other bank ?
\item In a harbour,  wind is blowing at the speed of 72 km/h and the flag on the mast of a boat anchored in the harbour flutters along the N-E direction. If the boat starts moving at a speed of 51 km/h to the north,  what is the direction of the flag on the mast of the boat ?
\item In which quadrant or on which axis do each of the points (-2, 4), (3, -1), (-1, 0), (1, 2) and (-3, -5) lie? Verify your answer by locating them on the Cartesian plane.
\item Plot the points $(x, y)$ given in 
\tabref{table:Table of values}.
\begin{table}[H]
	\centering
\begin{tabular}{|c|c|c|c|c|c|}
\hline	
x & -2 & -1 & 0 & 1 & 3\\
\hline
y & 8 & 7 & -1.25 & 3 & -1\\
\hline
\end{tabular}
\caption{}
\label{table:Table of values}
\end{table}
\end{enumerate}

\subsection{Scalar Product}
\begin{enumerate}[label=\thesubsection.\arabic*, ref=\thesubsection.\theenumi]
\item Find the angle between two vectors $\overrightarrow{a}$ and $\overrightarrow {b} $ with magnitudes $\sqrt{3}$ and 2 respectively having $\overrightarrow {a}\cdot\overrightarrow {b}=\sqrt{6}$.
		\label{prob:12/10/3/1/inner}
	\\
	\solution
		From the given information,
\begin{align}
\norm{\vec{a}}=\sqrt{3},
\norm{\vec{b}}= 2,
{\vec{a}^{\top}}{\vec{b}}=\sqrt{6}  
\\
\implies \cos\theta=\frac{{\vec{a}^{\top}}{\vec{b}}}{\norm{\vec{a}}\norm{\vec{b}}}
=\frac{1}{\sqrt{2}}\\
	\text{or, }\theta={45}\degree
\end{align}

\item Find the angle between the the vectors $\hat{i}-2\hat{j}+3\hat{k}$ and $3\hat{i}-2\hat{j}+\hat{k}$.
	\\
	\solution
		Let 
\begin{align}
	\vec{a} = \myvec{1\\-2\\3} , \vec{b} = \myvec{3\\ -2 \\ 1},
\end{align}
		From problem \ref{prob:12/10/3/1/inner},
\begin{align}
\cos\theta=\frac{\vec{a}^{\top}\vec{b}}{\norm{\vec{a}}\norm{\vec{b}}}
	= \frac{10}{\sqrt{14}\times \sqrt{14}}= \frac{5}{7}
\end{align}

\item Evaluate the product $(3\overrightarrow {a}-5\overrightarrow {b})\cdot (2\overrightarrow {a}+7\overrightarrow {b})$.
	\\
	\solution
		\begin{multline}
    \brak{3\vec{a}-5\vec{b}}^{\top}\brak{2\vec{a}+7\vec{b}}= 3\vec{a}^{\top}\brak{2\vec{a}+7\vec{b}} - 5\vec{b}^{\top}\brak{2\vec{a}+7\vec{b}}
    \\
     =6\norm{\vec{a}}^2-35\norm{\vec{b}}^2+11\vec{a}^{\top}\vec{b}
\end{multline}

\item If the vertices $\vec{A}, \vec{B}, \vec{C}$ of a triangle $ABC$ are (1, 2, 3),  (-1, 0, 0),  (0, 1, 2),  respectively,  then find  $\angle{ABC}$.
	\\
	\solution
		From the given information, 
\begin{align}
\vec{A} - \vec{B} &= \myvec{2\\2\\3},
\vec{C} - \vec{B} = \myvec{1\\1\\2}\\
	\implies \angle{ABC} &= \cos^{-1}{\frac{\brak{\vec{A} -\vec{B}}^{\top}\brak{\vec{C}-\vec{B}}}{\norm{\vec{A} -\vec{B}}  \norm{\vec{C}-\vec{B}}}}\\
&= \cos^{-1}{\frac{10}{\sqrt{102}}}\\
\end{align}




	\item The slope of a line is double of the slope of another line. If tangent of the angle between them is 1/3,  find the slopes of the lines.
\label{chapters/11/10/1/11}
\\
\solution 
The direction vectors of the lines can be expressed as
\begin{align}
\vec{m}_1=\myvec{1\\m},
\vec{m}_2=\myvec{1\\2m}
\end{align}
If the angle between the lines be $\theta$,
\begin{align}
\tan \theta = \frac{1}{3}
\implies \cos \theta=\frac{3}{\sqrt{10}}
\end{align}
Thus,
\begin{align}
	\frac{3}{\sqrt{10}} = \frac{\vec{m}_1^\top \vec{m}_2}{\norm{\vec{m}_1}\norm{\vec{m}_2}}
	\\
	= \frac{2m^2 +1}{\sqrt{m^2 + 1}\sqrt{4m^2 + 1}}
	\\
	\implies \frac{9}{10}=\frac{4m^4 + 4m^2 +1}{4m^4 + 5m^2 +1}
\\
	\text{or, } 4m^4 - 5m^2 +1 = 0
\end{align}
yielding
\begin{align}
m=\pm \frac{1}{2}, 
\pm 1
\end{align}

\item    Find angle between the lines,  $\sqrt{3}x+y=1$ and $x+\sqrt{3}y$=1.
\label{chapters/11/10/3/9}
\\
   \solution 
From    the given equations, the normal vectors can be expressed as
   \begin{align}
	   \vec{n}_1=\myvec{\sqrt{3}\\1},\,
	   \vec{n}_2=\myvec{1\\\sqrt{3}}
   \end{align}
The angle between the lines can then be expressed as
\begin{align}
	\cos\theta=\frac{\vec{n}_1^T\vec{n}_2}{\norm{\vec{n}_1}\norm{\vec{n}_2}}
	=\frac{\sqrt{3}}{2} 
	\\
	\text{or, }
\theta=30\degree
\end{align}

\item Find the angle between the vectors $2\hat{i}-\hat{j}+\hat{k}$ and $3\hat{i}+4\hat{j}-\hat{k}$.
\item The angles between two vectors $\vec{a},  \vec{b}$ with magnitude $\sqrt{3},  4$ respectively,  and $\vec{a} \cdot \vec{b}= 2\sqrt{3}$ is
\item Find the angle between the lines 
\begin{align}
	\overrightarrow{r}&=3\hat{i}-2\hat{j}+6\hat{k}+\lambda(2\hat{i}+\hat{j}+2\hat{k})
	\text{ and}
	\\
	\overrightarrow{r}&=(2\hat{j}-5\hat{k})+\mu(6\hat{i}+3\hat{j}+2\hat{k})
\end{align}
%
\solution  The given lines can be expressed  in the form 
of 
	\eqref{eq:param-form}
	as
\begin{align}
	\vec{x} = \myvec{3 \\ -2 \\ 6} + \kappa_1 \myvec{2 \\ 1 \\ 2}
	\\
	\vec{x} = \myvec{0 \\ 2 \\ -5 } + \kappa_2 \myvec{6 \\ 3 \\ 2}
\end{align}
From the above,  it is obvious that the direction vectors of the two lines are
\begin{align}
\vec{m}_1 =\myvec{2 \\ 1 \\ 2}, \
	\vec{m}_2=\myvec{6 \\ 3 \\ 2}
\end{align}
	From \eqref{eq:angle-inner},  the angle between the two lines is  obtained as
\begin{align}
	\cos \theta = \frac{19}{21}
\end{align}
\item The vectors $\vec{a}=3\hat{i}-2\hat{j}+2\hat{k}$ $\text{ and }$ $\vec{b}=\hat{i}-2\hat{k}$ are the adjancent sides of a parallelogram. The acute angle between its diagonals is \rule{1cm}{0.15mm}.
\item The sine of the angle between the straight line 
\begin{align}
	\frac{x-2}{3}=\frac{y-3}{4}=\frac{z-4}{5} 
\end{align}
and the plane  
\begin{align}
2x-2y+z=5
\end{align}
is
\solution The given line can be expressed in the form 
	\eqref{eq:param-form}
	as
\begin{align}
	\vec{x} = \myvec{2 \\ 3 \\ 4} + \kappa_1 \myvec{3 \\ 4 \\ 5}
\end{align}
Hence the direction vector of this line is 
\begin{align}
\myvec{3 \\ 4 \\ 5}
\end{align}
	From \eqref{eq:normal-form},  the normal vector of the given plane is 
\begin{align}
\myvec{2 \\ -2 \\ 1}
\end{align}
Thus,  the cosine of the angle between the two is 
obtained from \eqref{eq:angle-inner} as
\begin{align}
	\frac{\sqrt{2}}{10}, 
\end{align}
which is sine of the angle between the plane and the line.
\item The plane $2x-3y+6z-11=0$ makes an angle $\sin^{-1}(\alpha)$ with x-axis. The value of $\alpha$ is equal to 
\item Find the angle between the vectors $2\hat{i}-\hat{j}+\hat{k}$ $\text{and}$ $3\hat{i}+4\hat{j}-\hat{k}$.
\item The angles between two vectors $\vec{a}$ $\text{and}$ $\vec{b}$ with magnitude $\sqrt{3}$ $\text{ and }$ 4,  respectively,  and $\vec{a}$,  $\vec{b}$= $2\sqrt{3}$ is
\item The angle between the line 
\begin{align}
	\overrightarrow{r}=(5\hat{i}-\hat{j}-4\hat{k})+\lambda(2\hat{i}-\hat{j}+\hat{k})
\end{align}
	and the plane 
\begin{align}
	\overrightarrow{r} \cdot (3\hat{i}-4\hat{j}-\hat{k})+5=0
\end{align}
	is $\sin^{-1}\brak{\frac{5}{2\sqrt{91}}}$.
\item The angle between the planes 
\begin{align}
	\overrightarrow{r} \cdot (2\hat{i}-3\hat{j}+\hat{k})&=1 
	\text{ and }
	\\
	\overrightarrow{r} \cdot (\hat{i}-\hat{j})&=4  
\end{align}
is
	$\cos^{-1} \brak{\frac{-5}{\sqrt{58}}}$.
\item Find the angle between the lines 
\begin{align}
	y&=(2-\sqrt{3})(x+5)\text{ and }
	\\
	y&=(2+\sqrt{3})(x-7).
\end{align}
\item The unit vector normal to the plane $x+2y+3z-6=0$ is $\frac{1}{\sqrt{14}}\hat{i} + \frac{2}{\sqrt{14}}\hat{j} + \frac{3}{\sqrt{14}}\hat{k}$.
\item The scalar product of the vector $\hat{i}+\hat{j}+\hat{k}$ with a unit vector along the sum of vectors $2\hat{i}+4\hat{j}-5\hat{k}$ and $\lambda\hat{i}+2\hat{j}+3\hat{k}$ is equal to one. Find the value of $\lambda$.
\item  Find the angle between the following pairs of lines.
\begin{enumerate}	
\item  
\begin{align}
\overrightarrow{r}=2\hat{i}-5\hat{j}+\hat{k}+\lambda(3\hat{i}+2\hat{j}+6\hat{k}) \text{ and }\\ \overrightarrow{r}=7\hat{i}-6\hat{k}+\mu(\hat{i}+2\hat{j}+2\hat{k}) 
\end{align} 
\item 
\begin{align}
\overrightarrow{r}=3\hat{i}+\hat{j}-2\hat{k}+\lambda(\hat{i}-\hat{j}-2\hat{k}) \text{ and }\\ \overrightarrow{r}=2\hat{i}-\hat{j}-56\hat{k}+\mu(3\hat{i}-5\hat{j}-4\hat{k})
\end{align}
\item 
\begin{align} \frac{x-2}{2}=\frac{y-1}{5}=\frac{z+3}{-3}\text{ and } \frac{x+2}{-1}=\frac{y-4}{8}=\frac{z-5}{4}.
\end{align}
\item
\begin{align} \frac{x}{2}=\frac{y}{2}=\frac{z}{1}\text{ and } \frac{x-5}{4}=\frac{y-2}{1}=\frac{z-3}{8}.
\end{align}
\end{enumerate}
\item If the co-ordinates of the points $\vec{A}, \vec{B}, \vec{C}, \vec{D}$ be $(1, 2, 3)$,  $(4, 5, 7)$,  $(-4, 3, -6)$ and $(2, 9, 2)$ respectively,  then find the angle between the lines $AB$ and $CD$.
\item If the angle between two lines is $\frac{\pi}{4}$ and slope of one of the lines is $\frac{1}{2}$, find the slope of the other line.
\item Find the angle between two vectors $\overrightarrow{a}$ and $\overrightarrow{b}$ with magnitudes $1$ and $2$ respectively and when $\overrightarrow{a}\cdot \overrightarrow{b} = 1$.
\item Find angle $\theta$ between the vectors $\overrightarrow{a} = \hat{i} +\hat{j} -\hat{k}$ and $\overrightarrow{b} = \hat{i} -\hat{j}+\hat{k}$.
\item If $\hat{i}+\hat{j}+\hat{k}, 2\hat{i}+5\hat{j}, 3\hat{i}+2\hat{j}-3\hat{k}$ and $\hat{i}-6\hat{j}-\hat{k}$ are the position vectors of points $\vec{A}, \vec{B}, \vec{C}$ and $\vec{D}$ respectively, then find the angle between $\overrightarrow{AB}$ and $\overrightarrow{CD}$. Deduce that $\overrightarrow{AB}$ and $\overrightarrow{CD}$ are collinear.
\item Find the angle between the pair of lines given by
\begin{align}
\overrightarrow{r}= 3 \hat{i}+ 2 \hat{j}- 4 \hat{k}+ \lambda(\hat{i}+ 2 \hat{j}+ 2 \hat{k}) \\
\text{ and } \overrightarrow{r}= 5 \hat{i}+ 2 \hat{j}+ \mu(3 \hat{i}+ 2 \hat{j}+ 6 \hat{k}) 
\end{align}
\item Find the angle between the pair of lines:
\begin{align}
\frac{x+3}{3}= \frac{y-1}{5}= \frac{z+3}{4}\\
\text{ and }\frac{x+1}{1}= \frac{y-4}{1}= \frac{z+5}{2}
\end{align}
\item Find the angle between the two planes $2x +y -2z =5$ and $3x- 6y- 2z= 7$ using vector method.
\item Find the angle between the two planes $3x -6y +2z =7$ and $2x +2y -2z =5$.
\item Find the angle between the line $\frac{x+1}{2} =\frac{y}{3} =\frac{z-3}{6}$ and the plane $10x +2y -11z =3$.
\end{enumerate}

\subsection{Formulae}
\begin{enumerate}[label=\thesubsection.\arabic*.,ref=\thesubsection.\theenumi]
	\item Mathematically, 
the projection of $\vec{A}$ on $\vec{B}$ is defined as
		\begin{align}
	\vec{C} = k \vec{B},\, \text{such that}
	\brak{\vec{A}-\vec{C}}^{\top}\vec{C} = 0
\end{align}
yielding
\begin{align}
	\brak{\vec{A}-k\vec{B}}^{\top}\vec{B} = 0
	\\
	\text{or, } k = 
	\frac{\vec{A}^{\top}\vec{B}}{\norm{\vec{B}}^2}
	\implies 
	\vec{C} = 
	\frac{\vec{A}^{\top}\vec{B}}{\norm{\vec{B}}^2}
 \vec{B}
	\label{eq:12/10/3/4/proj}
\end{align}
\item If $\vec{A}, \vec{B}$ are unit vectors, 
\begin{multline}
	\brak{\vec{A}-\vec{B}}^{\top} 
	\brak{\vec{A}+\vec{B}} 
	\\
\norm{\vec{A}}^2 - \norm{\vec{B}}^2
	= 0
	\label{eq:12/10/3/11/unit}
\end{multline}
  \item If 
\begin{align}
	\vec{A}^{\top}\vec{A} =\vec{I},
\label{eq:12/10/3/5/inner}
\end{align}
		then $	\vec{A}$ is an {\em orthogonal} matrix.
\end{enumerate}

\subsection{Orthogonality}
\begin{enumerate}[label=\thesubsection.\arabic*,ref=\thesubsection.\theenumi]
\item
Find the angle between the lines whose direction ratios are $a,b,c$ and $b-c,c-a,a-b$.
\\
\solution
    \begin{align}
\because \myvec{a&b&c}\myvec{b-c\\c-a\\a-b} = 0,
   \theta=\frac{\pi}{2}
    \end{align}

\item Name the type of quadrilateral formed, if any, by the following points,and give reasons for your answer
\begin{enumerate}
\item $A(-1,-2), B(1,0), (C-1,2), D(-3,0)$
\item $A(-3,5), B(-3,1), C(0,3), D(-1,-4)$
\item $A(4,5), B(7,6), C(4,3), D(1,2)$
\end{enumerate}
\solution
			See \tabref{tab:10/7/1/6/inner},
	\figref{fig:10/7/1/6/Fig1}, \figref{fig:10/7/1/6/Fig2}.
	and 
	\figref{fig:10/7/1/6/Fig3}. 
In b), forming the collinearity matrix
\begin{align}
\myvec{\vec{B}-\vec{A} & \vec{C}-\vec{B}} 
=
		\myvec{6&-3\\-4&2} \xleftrightarrow{R_{2}\rightarrow R_{2}+\frac{2}{3}R_{1}}= \myvec{6&-3\\0&0}
\end{align}
which is a rank 1 matrix.  Hence, $\vec{A}, \vec{B}, \vec{C}$  are collinear.
\begin{figure}[H]
	\begin{center} 
	    \includegraphics[width=0.75\columnwidth]{chapters/10/7/1/6/figs/fig1.pdf}
	\end{center}
\caption{}
\label{fig:10/7/1/6/Fig1}
\end{figure}
%
\begin{figure}[H]
	\begin{center} 
	    \includegraphics[width=0.75\columnwidth]{chapters/10/7/1/6/figs/fig2.pdf}
	\end{center}
\caption{}
\label{fig:10/7/1/6/Fig2}
\end{figure}
%	
\begin{figure}[H]
	\begin{center} 
	    \includegraphics[width=0.75\columnwidth]{chapters/10/7/1/6/figs/fig3.pdf}
	\end{center}
\caption{}
\label{fig:10/7/1/6/Fig3}
\end{figure}
%
\begin{table}[H]
    \centering
%    \begin{tabular}{|c|c|c|c|c|}
	    \begin{tabularx}{\columnwidth}{|c|X|X|X|c|}
        \hline
		    &{\scriptsize $\vec{B}-\vec{A} = \vec{C}-\vec{D}$?} & {\tiny $(\vec{B}-\vec{A})^\top (\vec{C}-\vec{B}) =  0$?} & {\tiny $(\vec{C}-\vec{A})^\top (\vec{D}-\vec{B}) = 0$}& \textbf{Geometry}\\
        \hline
	    a)& Yes & Yes & Yes& Square \\
        \hline
	    b)& No & -&- & Triangle\\
        \hline
	    c)&Yes & No & No & Parallelogram\\
        \hline
	\end{tabularx}
%    \end{tabular}
	\caption{}
	\label{tab:10/7/1/6/inner}
\end{table}

\item Find the projection of the vector $\hat{i}+3\hat{j}+7\hat{k}$ on the vector $7\hat{i}-\hat{j}+8\hat{k}$.
	\\
	\solution
				Let 
\begin{align}
 \vec{A} =\myvec{1\\3\\7}, \vec{B} =\myvec{7\\-1\\8}
\end{align}
The projection of $\vec{A}$ on $\vec{B}$ is defined as
the foot of the perpendicular from 
$\vec{A}$ to $\vec{B}$ and obtained in 
	\eqref{eq:12/10/3/4/proj}.
Substituting numerical values,
\begin{align}
	\vec{C}
		=\frac{10}{19}\myvec{7\\-1\\8}
 \end{align}

\item Find the projection of the vector $\hat{i}-\hat{j}$ on the vector $\hat{i}+\hat{j}$.
	\\
\solution
		The given points are
\begin{align}
 \vec{A}=\myvec{1\\ -1},
 \vec{B}=\myvec{1\\ 1}
\end{align}
Since
\begin{align}
	\vec{A}^\top \vec{B} =0,
\end{align}
	from \eqref{eq:12/10/3/4/proj},
the projection vector is the origin.
		See \figref{fig:12/10/3/3fig}.
\begin{figure}[H]
	\centering
\includegraphics[width=0.75\columnwidth]{chapters/12/10/3/3/figs/fig.pdf}
\caption{}
		\label{fig:12/10/3/3fig}
\end{figure}

\item Show that each of the given three vectors is a unit vector: 
 $\frac{1}{7}(2\hat{i}+3\hat{j}+6\hat{k}),\frac{1}{7}(3\hat{i}-6\hat{j}+2\hat{k}),\frac{1}{7}(6\hat{i}+2\hat{j}-3\hat{k}$).
Also,show that they are mutually perpendicular to each other.
	\\
	\solution
		\begin{align}
\vec{A} = 	\myvec{
	\frac{2}{7} & \frac{3}{7} & \frac{6}{7} \\[1ex]
    \frac{3}{7} & -\frac{6}{7} & \frac{2}{7} \\[1ex]
    \frac{6}{7} & \frac{2}{7} & -\frac{3}{7}
}
\end{align}
is an orthogonal matrix satisfying
\eqref{eq:12/10/3/5/inner},
which verifies the given conditions.

\item If $\overrightarrow {a}=2\hat{i}+2\hat{j}3\hat{k},\overrightarrow {b}=\hat{-i}+2\hat{j}+\hat{k}$ and $\overrightarrow {c}=3\hat{i}+\hat{j}$ are such that $\overrightarrow {a}+\lambda\overrightarrow {b}$ is perpendicular to $\overrightarrow {c}$,then find the value of $\lambda$.
	\\
		\solution
\begin{align}
\because	(\vec{a}+\lambda \vec{b})^{\top} \vec{c} = 0,
	\\
	\lambda=-\frac{\vec{a}^{\top}\vec{c}}{\vec{b}^{\top}\vec{c}}
	=8,
\end{align}
upon substituting numerical values.



\item Show that $\abs {\overrightarrow {a}}\overrightarrow {b}+\abs{\overrightarrow {b}}\overrightarrow {a}$ is perpendicular to $\abs{\overrightarrow {a}} \overrightarrow {b}-\abs{\overrightarrow {b}} \overrightarrow {a}$, for any two nonzero vectors $\overrightarrow {a}$ and $\overrightarrow {b}$.
	\\
	\solution
		\begin{align}
\norm{\vec{a}}\vec{b}+\norm{\vec{b}}\vec{a}
=
	\norm{\vec{a}}\norm{\vec{b}}\brak{\frac{\vec{b}}{\norm{\vec{b}}}+\frac{\vec{a}}{\norm{\vec{a}}}}
	\\
\norm{\vec{a}}\vec{b}-\norm{\vec{b}}\vec{a}
=
	\norm{\vec{a}}\norm{\vec{b}}\brak{\frac{\vec{b}}{\norm{\vec{b}}}-\frac{\vec{a}}{\norm{\vec{a}}}}
	\\
	\implies 
	\brak{\norm{\vec{a}}\vec{b}+\norm{\vec{b}}\vec{a}}^{\top} \brak{\norm{\vec{a}}\vec{b}-\norm{\vec{b}}\vec{a}} = 0
\end{align}
	from \eqref{eq:12/10/3/11/unit}.

\item If $\overrightarrow {a},\overrightarrow {b},\overrightarrow {c}$ are unit vectors such that $\overrightarrow {a}+\overrightarrow {b}+\overrightarrow {c}=\overrightarrow {0}$, find the value of $\overrightarrow {a}.\overrightarrow {b}+\overrightarrow {b}.\overrightarrow {c}+\overrightarrow {c}.\overrightarrow {a}$.
	\\
	\solution
		\begin{align}
	\norm{{\vec{a}}+{\vec{b}}+{\vec{c}}}^2=0
	\nonumber \\
	\implies{\norm{\vec{a}}}^2+{\norm{\vec{b}}}^2+{\norm{\vec{c}}}^2+2({{\vec{a}^\top}{\vec{b}}+{\vec{b}^\top}{\vec{c}}+{\vec{c}^\top}{\vec{a}}})=0
	\nonumber \\
	\implies3+2({{\vec{a}^\top}{\vec{b}}+{\vec{b}^\top}{\vec{c}}+{\vec{c}^\top}{\vec{a}}})=0\nonumber \\
	\implies{\vec{a}^\top}{\vec{b}}+{\vec{b}^\top}{\vec{c}}+{\vec{c}^\top}\vec{a}=-\frac{3}{2}
\end{align}

\item If either vector $\overrightarrow {a}=0$ or $\overrightarrow {b}=0$, then $\overrightarrow {a}.\overrightarrow {b}$=0. But the converse need not be true. Justify your answer with an example.
	\\
	\solution
		\begin{align}
	\vec{a}=\myvec{1\\1},\,
\vec{b}=\myvec{1\\-1}\\
\implies \vec{a} ^\top \vec{b} =  0 
\end{align}



\item Show that the vectors $2\hat{i}-\hat{j}+\hat{k},\hat{i}-3\hat{j}-5\hat{k}$ and  $3\hat{i}-4\hat{j}-4\hat{k}$ from the vertices of a right angled triangle.
	\\
	\solution
		\begin{align}
\vec{A} = \myvec{2\\-1\\1}, \, \vec{B} = \myvec{1\\-3\\-5}, \, \vec{C}=\myvec{3\\-4\\-4},
\\
\implies \vec{B}-\vec{C} = \myvec{-2\\1\\-1} ,\, 
\vec{C}-\vec{A} = \myvec{1\\-3\\-5} ,\, 
\\
	\text{or, }
\brak{\vec{B}-\vec{C}}^{\top}\brak{\vec{C}-\vec{A}} = 0
\end{align}

\item Show that the points A, B and C with position vectors, $3\hat{i}-4\hat{j}-4\hat{k}, 2\hat{i}-\hat{j}+\hat{k}$ and $\hat{i}-3\hat{j}-5\hat{k}$, respectively, form the vertices of a right angled
triangle.
\\
\solution
		    \begin{align}
         \vec{B} - \vec{A} = \myvec{-1\\3\\5},\, 
         \vec{C} - \vec{B} = \myvec{-1\\-2\\-6},\,
         \vec{C} - \vec{A} = \myvec{-2\\1\\-1},
        \label{eq:chapters/12/10/2/17/dir-vec}
	\\
	    \implies 
	    \brak{\vec{B} - \vec{A}}^\top
	    \brak{\vec{C} - \vec{A}} = 0
    \end{align}
Hence, $\triangle ABC$ is right angled at $\vec{A}$. 

\item Let $\vec{a}=\hat{i}+4\hat{j}+2\hat{k}, \vec{b}=3\hat{i}-2\hat{j}+7\hat{k}$ and $\vec{c}=2\hat{i}-\hat{j}+4\hat{k}$. Find a vector $\vec{d}$ which is perpendicular to both $\vec{a}$ and $\vec{b}$, and $\vec{c}\cdot \vec{d}$=15.\\
	\solution
		From the given information, 
\begin{align}
\vec{a}^{\top}\vec{d} &= 0\\
\vec{b}^{\top}\vec{d} &= 0\\
\vec{c}^{\top}\vec{d} &= 15
\end{align}
yielding
\begin{align}
\myvec{\vec{a}^{\top} \\\vec{b}^{\top}\\\vec{c}^{\top}}\vec{d} &= \myvec{0\\0\\15}\\
\implies \myvec{1&4&2 \\3&-2&7 \\2&-1&4}\vec{d} &= \myvec{0\\0\\15}
\label{eq:chapters/12/10/5/12/1}
\end{align}
%
Forming the augmented matrix, 
\begin{align}
	\myvec{1&4&2&\vrule&0\\ 3&-2&7&\vrule&0 \\ 2&-1&4&\vrule&15} 
	\xleftrightarrow[R_3\leftarrow R_3-2R_1]{R_2\leftarrow R_2-3R_1}
	\myvec{1&4&2&\vrule&0\\ 0&-14&1&\vrule&0 \\ 0&-9&0&\vrule&15}
\nonumber	\\
	\xleftrightarrow[]{R_3\leftarrow R_3-\frac{9}{14}R_2}
	\myvec{1&4&2&\vrule&0\\ 0&-14&1&\vrule&0 \\ 0&0&-\frac{9}{14}&\vrule&15}
	\label{eq:chapters/12/10/5/12/2}
\end{align}
yielding
%
\begin{align}
	\vec{d} &= \myvec{\frac{160}{3}\\[1ex]-\frac{5}{3}\\[1ex]-\frac{70}{3}}
\end{align}
upon back substitution.


\item Prove that $(\vec{a}+\vec{b})\cdot(\vec{a}+\vec{b})=|{\vec{a}}|^2+|{\vec{b}}|^2$, if and only if $\vec{a}, \vec{b}$ are perpendicular, given $\vec{a}\neq\vec{0}, \vec{b}\neq\vec{0}$.\\
	\solution
			\begin{align}
\because 		\brak{\vec{a}+\vec{b}}^{\top}\brak{\vec{a}+\vec{b}} 
		= \norm{\vec{a}}^2+\norm{\vec{b}}^2,
		\\
		 \norm{\vec{a}}^2+\norm{\vec{b}}^2+2\vec{a}^{\top}\vec{b}
		= \norm{\vec{a}}^2+\norm{\vec{b}}^2
		\\
		\implies 
		\vec{a}^{\top}\vec{b} = 0 
	\end{align}


\item $ABCD$ is a rectangle formed by the points $\vec{A}(–1, –1), \vec{B}(– 1, 4), \vec{C}(5, 4)$  and  $\vec{D}(5, – 1)$. $\vec{P}, \vec{Q}, \vec{R}$ and $\vec{S}$ are the mid-points of $AB, BC, CD$ and $DA$ respectively. Is the quadrilateral $PQRS$ a square? a rectangle? or a rhombus? Justify your answer.
	\\
	\solution 
See Fig. \ref{fig:10/7/4/8Fig3}. From 
  \eqref{eq:10/7/4/8det2f}, $PQRS$ is a parallelogram.
\begin{align}
  %\label{eq:10/7/4/8det2f}
  \vec{P}  = 
 \frac{3}{2},\, 
 \vec{Q}  = \myvec{
 2 \\
 4 \\
 } ,\,
 \vec{R}  = \myvec{
 5 \\
 \frac{3}{2}
 }   
  ,\,
 \vec{S}  = \myvec{
 2\\
 -1 \\
 }   
 \\
	\implies 
 \brak{\vec{Q}-\vec{P}}^\top\brak{\vec{R}-\vec{Q}}  \neq 0
 \\
 \brak{\vec{R}-\vec{P}}^\top\brak{\vec{S}-\vec{Q}}  = 0
\end{align}
Therefore $PQRS$ is a rhombus.
\begin{figure}[H]
	\begin{center}
		\includegraphics[width=0.75\columnwidth]{chapters/10/7/4/8/figs/fig.pdf}
	\end{center}
\caption{}
\label{fig:10/7/4/8Fig3}
\end{figure}


\item Without using the Baudhayana theorem, show that the points $A(4,4), B(3,5)$ and $C(-1,-1)$ are the vertices of a right angled triangle.
\label{chapters/11/10/1/6}
		See \figref{fig:11/10/1/6}.
\begin{align}
	\vec{C}-\vec{A}=\myvec{
-5 \\
	-5},\,
	\vec{A}-\vec{B}&=\myvec{
1 \\
-1 
}
\\
	\implies \brak{\vec{C}-\vec{A}}^{\top}
	\brak{\vec{A}-\vec{B}}&=0
\end{align}
Thus, $AB \perp AC$.
	\begin{figure}[H]
		\centering
 \includegraphics[width=0.75\columnwidth]{chapters/11/10/1/6/figs/fig.pdf}
		\caption{}
		\label{fig:11/10/1/6}
  	\end{figure}

\item The line through the points $(h, 3)$ and $(4, 1)$ intersects the line $7x- 9y- 19= 0$ at a right angle. Find the value of $h$.
\label{chapters/11/10/3/10}
\\
\solution
The direction vectors of the given lines are 
\begin{align}
\myvec{4-h\\ -2}
,\,
\myvec{9\\ 7}
\\
\implies 
\myvec{9& 7}\myvec{4-h\\ -2}=0\\
\implies h=\frac{22}{9}
\end{align}
See  
		\figref{fig:chapters/11/10/3/10/Figure}.
\begin{figure}[H]
\centering
\includegraphics[width=0.75\columnwidth]{chapters/11/10/3/10/figs/fig.pdf}
\caption{}
		\label{fig:chapters/11/10/3/10/Figure}
\end{figure}

\item In the following cases, determine whether the given planes are parallel or perpendicular, and in case they are neither, find the angles between them.
\begin{enumerate}
\item $7x + 5y + 6z + 30 = 0$ and $3x – y – 10z + 4 = 0$
\item $2x + y + 3z – 2 = 0$ and $x – 2y + 5 = 0$
\item $2x – 2y + 4z + 5 = 0$ and $3x – 3y + 6z – 1 = 0$
\item $2x – y + 3z – 1 = 0$ and $2x – y + 3z + 3 = 0$
\item $4x + 8y + z – 8 = 0$ and $y + z – 4 = 0$
\end{enumerate}
    \solution
		    See \tabref{tab:12/11/3/13}.
\begin{table}[H]
    \centering
    \caption{}
    \label{tab:12/11/3/13}
    \begin{tabular}{|c|c|c|c|c|c|}
        \hline
        $\vec{n}_1$ & $\vec{n}_1$ & $\vec{n}_1^{\top}\vec{n}_2$ & $\norm{\vec{n}_1}$ & $\norm{\vec{n}_2}$ & Angle\\
        \hline
        $\myvec{7\\5\\6}$ & $\myvec{3\\-1\\-10}$ & $-44$ & $\sqrt{110}$ & $\sqrt{110}$ & $\cos^{-1}-\frac{2}{5}$ \\
        \hline
        $\myvec{2\\1\\3}$ & $\myvec{1\\-2\\0}$ & $0$ & & & perpendicular \\
        \hline
        $\myvec{2\\-2\\4}$ & $\myvec{3\\-3\\6}$ & $36$ & $\sqrt{24}$ & $\sqrt{54}$ & parallel \\
        \hline
        $\myvec{2\\-1\\3}$ & $\myvec{2\\-1\\3}$ & $14$ & $\sqrt{14}$ & $\sqrt{14}$ & parallel \\
        \hline
        $\myvec{4\\8\\1}$ & $\myvec{0\\1\\1}$ & $9$ & $9$ & $\sqrt{2}$ & $45\degree$ \\
        \hline
    \end{tabular}
\end{table}

\iffalse
\begin{table}[H]
    \centering
    \caption{}
    \label{}
    \begin{tabular}{|c|c|c|c|c|c|}
        \hline
	    $\vec{n}_1$ & $\vec{n}_1$ &  $\vec{n}_1^{\top}\vec{n}_2$& $\norm{\vec{n}_1}$ &$\norm{\vec{n}_2}$  &  Angle\\
        \hline
	    \myvec{7\\5\\6} & \myvec{3\\-1\\-10} & -44 & \sqrt{110} &\sqrt{110}  & \cos^{-1}-\frac{2}{5} \\
        \hline
\myvec{2\\1\\3}  & \myvec{1\\-2\\0}& 0 &  &  & perpendicular\\
        \hline
 \myvec{2\\-2\\4} & \myvec{3\\-3\\6} & 36  & \sqrt{24} & \sqrt{54} &  parallel \\
        \hline
 \myvec{2\\-1\\3} & \myvec{2\\-1\\3} & 14 & \sqrt{14} & \sqrt{14} & parallel \\
        \hline
 \myvec{4\\8\\1} & \myvec{0\\1\\1} & 9 & 9 & \sqrt{2} &  45\degree  \\
        \hline
    \end{tabular}
\end{table}
\fi

		\item 
 Show that the line joining the origin to the point $P(2, 1, 1)$ is perpendicular to the
line determined by the points $A(3, 5, – 1), B(4, 3, – 1)$.
\\
    \solution
				\begin{align}
			\brak{\vec{A}-\vec{B}}^\top\vec{P}=
			\myvec{-1&2&0}\myvec{2\\1\\1}=0 \qed
		\end{align}

	\item  If $l_1, m_1,n_1 \text{ and } l_2,m_2,n_2$ are the direction cosines of two mutually perpendicular lines, show that the direction cosines of the line perpendicular to both these are  $m_1n_2-m_2n_1,n_1l_2-n_2l_1,l_1m_2-l_2m_1$.
\\
    \solution
		\begin{align}
\vec{P} 
	=\myvec{
l_1&l_2&m_1n_2-m_2n_1\\
        m_1&m_2&n_1l_2-n_2l_1\\
        n_1&n_2&l_1m_2-l_2m_1
}
	\end{align}
	satisfies 
\eqref{eq:12/10/3/5/inner}.
	Hence, the three vectors are mutually perpendicular.

	\item If the lines $\frac{x-1}{-3} = \frac{y-2}{2k} = \frac{z-3}{2}$ and  $\frac{x-1}{3k} = \frac{y-1}{1} = \frac{z-6}{-5}$ are perpendicular, find the value of $k$.\\
    \solution
		From the given information,
\begin{align}
\vec{m}_1 = \myvec{-3\\ 2k\\ 2},\,  \vec{m}_2 =\myvec{3k\\ 1\\ -5} 
\\
	\implies \myvec{-3& 2k& 2}^{\top} \myvec{3k\\ 1\\ -5} =0
	\\
	\implies k = -\frac{10}{7}
\end{align}
\iffalse
See 
     \figref{fig:chapters/12/11/4/6/1}
\begin{figure}[H]
  \centering
   \includegraphics[width=0.75\columnwidth]{chapters/12/11/4/6/figs/line_1.png}
    \caption{lines represented for the given points and direction vector with k=$\frac{-10}{7}$}
     \label{fig:chapters/12/11/4/6/1}
     \end{figure}  
     \fi



\item If $\vec{a},\vec{b},\vec{c}$ are mutually perpendicular vectors of equal magnitudes, show that the vector $\vec{c}\cdot\vec{d}$=15 is equally inclined to $\vec{a},\vec{b}$ and $\vec{c}$.
    \item If $ \vec{A},\vec{B},\vec{C} $ are mutually perpendicular vectors of equal magnitudes,show that the  $ \vec{A}+\vec{B}+\vec{C} $ is equally inclined to $ \vec{A},\vec{B}  \text{ and }  \vec{C} $.
\item Check whether $(5,-2), (6,4)$ and $(7,-2)$ are the vertices of an isosceles triangle.
\item The perpendicular bisector of the line segment joining the points $\vec{A} (1, 5) \text{ and }
\vec{B} (4, 6)$ cuts the y-axis at
\begin{enumerate}
	\item$(0, 13)$ 
	\item $(0, –13)$
	\item$(0, 12) $
	\item$(13, 0)$
\end{enumerate}
\item The point which lies on the perpendicular bisector of the line segment joining the
	points $\vec{A} (–2, –5)\text { and } \vec{B} (2, 5) $ is
\begin{enumerate}
\item  	$(0, 0)$
\item  $(0, 2)$ 
\item  $(2, 0)$ 
\item  $(–2, 0)$
\end{enumerate}
\item The points $ (–4, 0), (4, 0), (0, 3) $ are the vertices of
	\begin{enumerate}
\item right triangle 
\item isosceles triangle
\item  equilateral triangle
\item  scalene triangle 
\end{enumerate}
\item The point $\vec{A}(2,7)$ lies on the perpendicular bisector of line segment joining the points $\vec{P}(6,5)\text{ and } \vec{Q}(0,-4)$.
\item The points $\vec{A}(-1,-2), \vec{B}(4,3), \vec{C}(2,5) \text{ and } \vec{D}(-3,0)$ in that order form a rectangle.
\item Name the type of triangle formed by the points $\vec{A}(-5,6),\vec{B}(-4,-2),\text{ and }\vec{C}(7,5)$.
\item What type of a quadrilateral do the points $\vec{A}(2,-2),\vec{B}(7,3),\vec{C}(11,-1),\text{ and }\vec{D}(6,-6)$ taken in that order, form?
\item Find the coordinates of the point $\vec{Q}$ on the $x$-axis which lies on the perpendicular bisector of the line segment joining the points $\vec{A}(-5,-2) \text{ and } \vec{B}(4,-2)$. Name the type of triangle formed by points $\vec{Q},\vec{A}\text{ and }\vec{B}$.
\item The points $\vec{A}(2,9),\vec{B}(a,5) \text{ and }\vec{C}(5,5)$ are the vertices of a triangle $\vec{ABC}$ right angled at $\vec{B}$. Find the values of a and hence the area of $\triangle \vec{ABC}$.
\item Find a vector of magnitude 6, which is perpendicular to both the vectors $2\hat{i}-\hat{j}$+$2\hat{k}\text{ and }4\hat{i}-\hat{j}+3\hat{k}$.
\item If A,B,C,D  are the points with position vectors $\hat{i}+\hat{j}-\hat{k}$, $2\hat{i}-\hat{j}+3\hat{k}$, $2\hat{i}-3\hat{k}$, $3\hat{i}$-$2\hat{j}$+$\hat{k}$, respectively, find the projection of $\overline{AB}$ $\text{ along }$ $\overline{CD}$.
\item Find the value of $\lambda$ such that the vectors $\vec{a}=2\hat{i}+\lambda\hat{j}+\hat{k}$ $\text{and}$ $\vec{b}=\hat{i}+2\hat{j}+3\hat{k}$ are orthogonal.
	\begin{enumerate}
\item 0
\item 1 
\item $\frac{3}{2}$
\item $-\frac{5}{2}$
	\end{enumerate}
\item Projection vector of $\vec{a}$ on $\vec{b}$ is
	\begin{enumerate}
\item $\left(\dfrac{\vec{a}\cdot\vec{b}}{\abs{\vec{b}}^2}\right)$
\item $\frac{\vec{a}\cdot\vec{b}}{\abs{\vec{b}}}$
\item $\frac{\vec{a}\cdot\vec{b}}{\abs{\vec{a}}}$
\item $\left(\dfrac{\vec{a}\cdot\vec{b}}{\abs{\vec{a}}^2}\right)$
\end{enumerate}
\item The vectors $\lambda\hat{i}+\lambda\hat{j}+2\hat{k}$, $\hat{i}+\lambda\hat{j}-\hat{k}$ $\text{ and }$ $2\hat{i}-\hat{j}+\lambda\hat{k}$ are coplanar if
	\begin{enumerate}
\item	$\lambda=-2$
\item $\lambda=0$
\item $\lambda=1$
\item	$\lambda=-1$
\end{enumerate}
\item The number of vectors of unit length perpendicular to the vectors $\vec{a}=2\hat{i}+\hat{j}+2\hat{k}$ $\text{ and }$ $\vec{b}=\hat{j}+\hat{k}$ is
	\begin{enumerate}
\item one
\item  two
\item three
\item infinite
\end{enumerate}
\item If $\vec{r}\cdot\vec{a}=0, \vec{r}\cdot\vec{b}=0$ and $\vec{r}\cdot\vec{c}=0$ for some non-zero vector $\vec{r}$, then the value of $\vec{a}\cdot(\vec{b}\times\vec{c})$ is \rule{1cm}{0.15mm}.
\item If $\abs{\vec{a}+\vec{b}}$ = $\abs{\vec{a}-\vec{b}}$, then the vectors $\vec{a}$ $\text {and}$ $\vec{b}$ are orthogonal.
\item Prove that the lines $x=py+q , z=ry+s \text{ and } x=p^{\prime}y+q^{\prime}, z=r^{\prime}y+s^{\prime} $ are perpendicular if $pp^{\prime}+rr^{\prime}+1=0$.
\item Find the equation of a plane which  bisects perpendicularly the line joining the points A$(2,3,4)$ and B$(4,5,8)$ at right angles.
\item $\overrightarrow{AB}=3\hat{i}-\hat{j}+\hat{k}$ and $\overrightarrow{CD}=-3\hat{i}+2\hat{j}+4\hat{k}$ are two vectors. The position vectors of the points A and C are $6\hat{i}+7\hat{j}+4\hat{k}$ and $-9\hat{j}+2\hat{k},$ respectively. Find the position vector of a point P on the line AB and a point Q on the line CD such that $\overrightarrow{PQ}$ is perpendicular to $\overrightarrow{AB}$ and $\overrightarrow{CD}$ both.
\item Show that the straight lines whose direction cosines are given by $2l+2m-n=0$ and $mn+nl+lm=0$ are at right angles.
\item If $l_1, m_1, n_1;l_2, m_2, n_2;l_3, m_3, n_3$ are the direction cosines of the three mutually perpendcular lines, prove that the line whose direction cosines are propotional to $l_1+l_2+l_3 , m_1+m_2,m_3, n_1+n_2+n_3$ make angles with them.
\item The intercepts made by the plane $2x-3y+5z+4=0$ on the co-ordinate axis are $\brak{-2,\dfrac{4}{3},-\dfrac{4}{5}}$.
\item The line $\overrightarrow{r}=2\hat{i}-3\hat{j}-\hat{k}+\lambda(\hat{i}-\hat{j}+2\hat{k})$ lies in the plane $\overrightarrow{r} \cdot (3\hat{i}+\hat{j}-\hat{k})+2=0$.
\item Line joining the points (3,-4) and (-2,6) is perpendicular to the line joining the points (-3,6) and (9,-18).
\end{enumerate}

\subsection{Vector Product}
\begin{enumerate}[label=\thesubsection.\arabic*,ref=\thesubsection.\theenumi]
		\item Find $\abs{\overrightarrow{a}\times\overrightarrow{b}},\text{ if }\overrightarrow{a}=\hat{i}-7\hat{j}+7\hat{k}\text{ and } \overrightarrow{b}=3\hat{i}-2\hat{j}+2\hat{k}$.
	\\
		\solution
		  From \eqref{eq:cross3d-submat},
\begin{align}
	\mydet{\vec{A}_{23}&\vec{B}_{23}}=\mydet{-7 & -2 \\ 7 & 2}=0\\
	\mydet{\vec{A}_{31}&\vec{B}_{31}}=\mydet{1 & 3 \\ 7 & 2}=-19\\
	\mydet{\vec{A}_{12}&\vec{B}_{12}}=\mydet{1 & 3 \\ -7 & -2}=19,
	\\
	\norm{\vec{a}\times\vec{b}}
	 = \norm{\myvec{ \mydet{\vec{A}_{23} & \vec{B}_{23}} \\[1ex] \mydet{\vec{A}_{31} & \vec{B}_{31}} \\[1ex] \mydet{\vec{A}_{12}  & \vec{B}_{12}}}}
=19\sqrt{2}
\end{align}
from 
  \eqref{eq:cross3d}.

\item Find $\lambda$ and $\mu$ if $(2\hat{i}+6\hat{j}+27\hat{k})\times(\hat{i}+\lambda \hat{j} + \mu \hat{k})=\overrightarrow{0}$.
	\\
		\solution
		From 
		 Formula \ref{prop:lin-dep-cross},
performing row reduction, 
\begin{align}
 \myvec{2&6&27 \\ 1& \lambda & \mu}
	\xleftrightarrow{R_{2}\leftarrow 2R_{2}-R_{1}}  	
 \myvec{2&6&27 \\ 0& 2\lambda -6 & 2\mu-27}
\end{align}
For the above matrix to have rank 1,
\begin{align}
	\mu=\frac{27}{2},
	\lambda=3.
\end{align}


\item Find the area of the triangle with vertices $A(1, 1, 2), B(2, 3, 5)$ and $C(1, 5, 5)$.
	\\
		\solution
		\begin{align}
\because \vec{B}-\vec{A} = \myvec{1\\2\\3}, 
\vec{C}-\vec{A} = \myvec{0\\4\\3},
\\
	\frac{1}{2} \norm{\myvec{1\\2\\3} \times \myvec{0\\4\\3}} 
	= 	\frac{1}{2}\norm{\myvec{-6\\3\\4}}
= \frac{\sqrt{61}}{2}
\end{align}
			using 
        \eqref{eq:11/10/1/1area-diag}, 
which is the the desired area.






\item Find the area of the parallelogram whose adjacent sides are determined by the vectors $\overrightarrow{a}=\hat{i}-\hat{j}+3\hat{k}$ and $\overrightarrow{b}=2\hat{i}-7\hat{j}+\hat{k}$.
	\\
		\solution
					From \eqref{eq:tri-area-cross},
			the desired area is obtained as
\begin{align}
	\norm{\myvec{1\\-1\\3} \times \myvec{2\\ -7 \\ 1}}
	=\norm{\myvec{20\\5\\-5}}
= 15\sqrt{2}
\end{align}


\item Find the area of a rhombus if its vertices are $A(3,0), B(4,5), C(-1,4)$  and  $D(-2,-1)$ taken in order. 
	\\
		\solution
	The area of the rhombus is
\begin{align}
                \norm{\myvec{\vec{A-D}}\times \myvec{\vec{B-A}}}=\mydet{5 & 1\\1 & 5} = 24
\end{align}
See 
\figref{fig:chapters/10/7/2/10/gFig1}.
\begin{figure}[!h]
 \begin{center}
  \includegraphics[width=\columnwidth]{chapters/10/7/2/10/figs/fig.pdf}
 \end{center}
\caption{}
\label{fig:chapters/10/7/2/10/gFig1}
\end{figure}

\item Let the vectors $\overrightarrow{a}$ and $\overrightarrow{b}$ be such that $|\overrightarrow{a}| = 3$ and $|\overrightarrow{b}| = \dfrac{\sqrt{2}}{3}$, then $\overrightarrow{a} \times \overrightarrow{b}$ is a unit vector, if the angle between $\overrightarrow{a}$ and $\overrightarrow{b}$ is
	\begin{multicols}{2}
	\begin{enumerate}
\item $\frac{\pi}{6}$
\item $\frac{\pi}{4}$
\item $\frac{\pi}{3}$
\item $\frac{\pi}{2}$
\end{enumerate}
	\end{multicols}
		\solution
		From the given information and 
	\eqref{eq:cross-sin}
%
\begin{align}
	\norm{\vec{a} \times \vec{b}} & = \norm{\vec{a}} \norm{\vec{b}} \sin \theta =1\\
\implies\sin \theta & = \frac{1} {\norm{\vec{a}} \norm{\vec{b}}}
 = \frac{1}{\sqrt{2}}\\
\implies\theta &= 
 \frac{\pi}{4} 
\end{align}

\item Area of a rectangle having vertices A, B, C and D with position vectors $ -\hat{i}+ \frac{1}{2} \hat{j}+4\hat{k}, \hat{i}+ \frac{1}{2} \hat{j}+4\hat{k}, \hat{i}-\frac{1}{2} \hat{j}+4\hat{k}$ and $-\hat{i}- \frac{1}{2} \hat{j}+4\hat{k}$, respectively is
	\\
		\solution
		Since
\begin{align}
\vec{A} - \vec{B} &= \myvec{-2\\0\\0}\\
\vec{C} -\vec{B} &= \myvec{0\\-1\\0}
\end{align}
area of the rectangle is
\begin{align}
 \norm{\brak{\vec{A} -\vec{B}} \times \brak{\vec{C}-\vec{D}}}
= 2
\end{align} 
\iffalse
See Fig. 
   \ref{fig:chapters/12/10/4/12Rect_ABCD}
\begin{figure}[H]
  \centering
   \includegraphics[width=0.75\columnwidth]{chapters/12/10/4/12/figs/Figure_1.png}
   \caption{}
   \label{fig:chapters/12/10/4/12Rect_ABCD}
\end{figure}
\fi





\item Find the area of the triangle whose vertices are 
\begin{enumerate}
\item $(2, 3), (–1, 0), (2, – 4)$
\item $(–5, –1), (3, –5), (5, 2)$ 
\end{enumerate}
		\label{10/7/3/1}
\solution
		    See \tabref{eq:10/7/3/1/area}.
\begin{table}[H]
    \centering
    \caption{}
    \label{eq:10/7/3/1/area}
    \begin{tabular}{|c|c|c|c|}
        \hline
	     & $\vec{A}-\vec{B}$  & $\vec{A}-\vec{C}$  & $\frac{1}{2}\|\brak{\vec{A}-\vec{B}} \times \brak{\vec{A}-\vec{C}}\|$ \\
        \hline
         a)& $\myvec{ 3 \\3 }$ & $\myvec{ 0 \\ 7 }$ & $\frac{21}{2}$ \\
        \hline
	    b)& $\myvec{
 -8 \\
 4 
 }$
         &$\myvec{
 -10 \\
 -3 
 }$
  &  $32$   \\
        \hline
    \end{tabular}
\end{table}


\item Find the area of the triangle formed by joining the mid-points of the sides of the triangle whose vertices are $A(0, –1), B(2, 1)$  and  $C(0, 3)$. Find the ratio of this area to the area of the given triangle.
	\\
\solution
		Using 
	  \eqref{eq:section_formula},
the mid point coordinates are given by
	\begin{align}
		\vec{P} = \frac{1}{2}\vec(\vec{A}+\vec{B})  = \myvec{1\\0}\\
		\vec{Q} = \frac{1}{2}\vec(\vec{B}+\vec{C}) = \myvec{1\\2}\\
		\vec{R} = \frac{1}{2}\vec(\vec{A}+\vec{C}) = \myvec{0\\1}
	\end{align}
	\begin{align}
\because		\vec{P}-\vec{Q} =  \myvec{
 0 \\
 -2 
 },\,
		\vec{Q}-\vec{R} =   \myvec{
 1 \\
 1 
 }
 \\
		ar(PQR)=\frac{1}{2}{\norm{\vec(\vec{P}-\vec{Q})\times\vec(\vec{Q}-\vec{R})}}
		=1
	\end{align}
	Similarly, 
	\begin{align}
		\vec{A}-\vec{B} = \myvec{
 -2 \\
 -2 
 }
 ,\,
		\vec{A}-\vec{C} =  \myvec{
 0 \\
 -4 
 }
 \\
 \implies
		ar(ABC)=\frac{1}{2}{\norm{\vec(\vec{A}-\vec{B})\times\vec(\vec{A}-\vec{C})}}
=4
\\
		\implies \frac{ar\brak{PQR}}{ar\brak{ABC}} = \frac{1}{4}
	\end{align}
	See 
\figref{fig:10/7/3/3Fig}
\begin{figure}[H]
	\begin{center} 
	    \includegraphics[width=0.75\columnwidth]{chapters/10/7/3/3/figs/fig.pdf}
	\end{center}
\caption{}
\label{fig:10/7/3/3Fig}
\end{figure}


\item Find the area of the quadrilateral whose vertices, taken in order, are $A(– 4, – 2), B(– 3, – 5), C(3, – 2)$  and $ D(2, 3)$.
	\\
\solution
		See 
\figref{fig:chapters/10/7/3/4/Fig1}
\begin{figure}[H]
 \begin{center}
  \includegraphics[width=0.75\columnwidth]{chapters/10/7/3/4/figs/fig.pdf}
 \end{center}
\caption{}
\label{fig:chapters/10/7/3/4/Fig1}
\end{figure}
\begin{align}
\because	\vec{A}- \vec{B} =\myvec{-1\\3},\,
	  \vec{A}- \vec{D} =\myvec{-6\\-5},
	  \\
	\vec{B}- \vec{C} =\myvec{-6\\-5},\,
	  \vec{B}- \vec{D} =\myvec{-3\\-8},
	  \\
	  ar(ABD)=\frac{1}{2} \norm{\brak{\vec{A}-\vec{B}}  \times 
   \brak{\vec{A}- \vec{D}}} 
	=	\frac{23}{2}
	\\
	  ar(BCD)=\frac{1}{2} \norm{\brak{\vec{B}-\vec{C}}  \times 
   \brak{\vec{B}- \vec{D}}} 
	=	\frac{33}{2}
	\\
\implies	ar(ABCD)=  ar(ABD) +  ar(BCD)
	= 28
\end{align}


\item Verify that a median of a triangle divides it into two triangles of equal areas for $\triangle ABC$ whose vertices are $\vec{A}(4, -6), \vec{B}(3, 2), \text{ and } \vec{C}(5, 2)$. 
		\label{10/7/3/5}
		\\
\solution
		\begin{align}
\vec{D}=\frac{\vec{B}+\vec{C}}{2}
=\myvec{4\\ 0},
\\
	\vec{A}- \vec{B} =\myvec{1\\ -4},\,
	  \vec{A}- \vec{D} =\myvec{0\\ -6}
	  \\
	  \implies
  ar(ABD)=\frac{1}{2} \norm{\brak{\vec{A}-\vec{B}}  \times 
   \brak{\vec{A}- \vec{D}}} 
	       =3	
	       \\
	\vec{A}- \vec{C} =\myvec{-1\\ -8},\,
	  \vec{A}- \vec{D} =\myvec{0\\ -6}
	  \\
	  \implies
  ar(ACD)=\frac{1}{2} \norm{\brak{\vec{A}-\vec{C}}  \times 
   \brak{\vec{A}- \vec{D}}} 
   \\
	= 3 =
ar(ABD)
\end{align}
See  
\figref{fig:10/7/3/5/}.
\begin{figure}[H]
\centering
\includegraphics[width=0.75\columnwidth]{chapters/10/7/3/5/figs/fig.pdf}
\caption{}
\label{fig:10/7/3/5/}
\end{figure} 

\item The two adjacent sides of a parallelogram are 
$\vec{a}= 2\hat{i}-4\hat{j}+5\hat{k}$  and  $\vec{b} =\hat{i}-2\hat{j}-3\hat{k}$.
Find the unit vector parallel to its diagonal. Also, find its area.\\
	\solution
		The diagonals of the parallelogram are given by
\begin{align}
 \vec{a} + \vec{b} = \myvec{3 \\-6\\2},\, 
 \vec{a} - \vec{b} = \myvec{1 \\-2\\8}
\end{align}
and the corresponding unit vectors are
\begin{align}
	\frac{\vec{a} + \vec{b}}{\norm{\vec{a} + \vec{b}}}  = \myvec{\frac{3}{\sqrt{45}}\\[1ex]-\frac{6}{\sqrt{45}}\\[1ex]\frac{2}{\sqrt{45}}},\, 
	\frac{\vec{a} - \vec{b}}{\norm{\vec{a} - \vec{b}}}  = \myvec{\frac{1}{\sqrt{69}}\\[1ex]-\frac{2}{\sqrt{69}}\\[1ex]\frac{8}{\sqrt{69}}}
\end{align}
%
The area of the parallelogram is given by
\begin{align}
	\norm{\vec{a}\times\vec{b}}  = \norm{\myvec{22 \\-11\\0}} = \sqrt{605}
\end{align}


\item The vertices of a $\triangle ABC$ are $\vec{A}(4,6), \vec{B}(1,5)$ and  $\vec{C}(7,2)$. A line is drawn to intersect sides $AB$ and $AC$ at $\vec{D}$ and $\vec{E}$ respectively, such that $\frac{AD}{AB} = \frac{AE}{AC} = \frac{1}{4}$. Calculate the area of $\triangle ADE$ and compare it with the area of the $\triangle ABC$.
\\
\solution
	See  
\figref{fig:chapters/10/7/4/6Fig1}.
\begin{figure}[H]
 \begin{center}
 \includegraphics[width=0.75\columnwidth]{chapters/10/7/4/6/figs/fig.pdf}
 \end{center}
\caption{}
\label{fig:chapters/10/7/4/6Fig1}
\end{figure}
	Using section formula
	  \eqref{eq:section_formula},
\begin{align}
\vec{D} =\frac{3\vec{A}+\vec{B}}{4}
	=\frac{1}{4}\myvec{13\\ 23}
	\\
\vec{E} =\frac{3\vec{A}+\vec{C}}{4}
	=\frac{1}{4}\myvec{19\\ 20}
	\\
	\vec{A}- \vec{D} 
	=\frac{1}{4}\myvec{3\\ 1},\,
	  \vec{A}- \vec{E}  
	=\frac{1}{4}\myvec{-3\\ 1}
	\\
	\vec{A}- \vec{B} =\myvec{3\\1},
	  \vec{B}-\vec{C} =\myvec{-6\\3}
\end{align}
yielding
\begin{align}
ar(ABD) =\frac{1}{2} \norm{\brak{\vec{A}-\vec{D}}  \times 
   \brak{\vec{A}- \vec{E}}} 
	=	\frac{15}{32}
	\\
	  ar(ABC) =\frac{1}{2} \norm{\brak{\vec{A}-\vec{B}}  \times 
   \brak{\vec{B}- \vec{C}}} 
	=	\frac{15}{2}
	\\
	\implies \frac{ar\brak{ADE}}{ar\brak{ABC}}=\frac{1}{16}
\end{align}

    \item Draw a quadrilateral in the Cartesian plane, whose vertices are 
    \begin{align}
        \vec{A} = \myvec{-4\\5},\, \vec{B} = \myvec{0\\7},\, 
        \vec{C} = \myvec{5\\-5},\, \vec{D} = \myvec{-4\\-2}.
    \end{align}
    Also, find its area.
\label{chapters/11/10/1/1}
   \\ 
    \solution 
See \figref{fig:11/10/1/1quad}.
    From 
        \eqref{eq:11/10/1/1area-diag},
    \begin{align}
ar\brak{ABCD}
	       = \frac{121}{2}
        \label{eq:11/10/1/1ans}
    \end{align}
    \begin{figure}[H]
        \centering
        \includegraphics[width=0.75\columnwidth]{chapters/11/10/1/1/figs/fig.pdf}
        \caption{Plot of quadrilateral $ABCD$}
        \label{fig:11/10/1/1quad}
    \end{figure}

\item Find the area of region bounded by the triangle whose
	vertices are $(1, 0), (2, 2)$ and $(3, 1)$. 
\item Find the area of region bounded by the triangle whose vertices
	are $(– 1, 0), (1, 3)$  and  $(3, 2)$. 
\item Find the area of the $\triangle ABC$, coordinates of whose vertices are $\vec{A}(2, 0), \vec{B}(4, 5)$ and $\vec{C}(6, 3)$.
\item The area of a triangle with vertices $\vec{A}(3, 0), \vec{B}(7, 0)$ and  $\vec{C}(8, 4)$ is
\begin{enumerate}
\item 14
\item 28
\item 8
\item 6
\end{enumerate}
\item Find the area of the triangle whose vertices are $(-8,4),(-6,6)$ and $(-3,9)$.
\item If $\vec{D}\brak{\frac{-1}{2},\frac{5}{2}},\vec{E}(7,3)$ and $\vec{F}\brak{\frac{7}{2},\frac{7}{2}}$ are the midpoints of sides of $\triangle ABC$, find the area of the $\triangle ABC$.
\item Find the sine of the angle between the vectors $\vec{a}=3\hat{i}+\hat{j}+2\hat{k}$ $\text{ and }$ $\vec{b}=2\hat{i}-2\hat{j}+4\hat{k}$.
\item Using vectors, find the area of $\triangle{ABC}$ with vertices A(1,2,3), B(2,-1,4) and C(4,5,-1).
\item Find the area of the parallelogram whose diagonals are $2\hat{i}-\hat{j}+\hat{k}$ and $\hat{i}+3\hat{j}-\hat{k}$.

\item The vector from origin to the points A and B are $\vec{a}$ = $2\hat{i}-3\hat{j}+2\hat{k}$ and  $\vec{b}$ = $2\hat{i}+3\hat{j}+\hat{k}$, respectively, then the area of $\triangle {OAB}$ is
	\begin{enumerate}
\item 340 
\item $\sqrt{25}$
\item $\sqrt{229}$
\item $\frac{1}{2}\sqrt{229}$
\end{enumerate}
\item If $\vec{a} = \hat{i}+\hat{j}+\hat{k}$ and $\vec{b} = \hat{j}-\hat{k}$, find a vector $\vec{c}$ such that $\vec{a}\times\vec{c} = \vec{b}$ and $\vec{a}\cdot \vec{c}$ = 3.
%
\item The area of the quadrilateral ABCD, where A$(0,4,1)$, B$(2,3,-1)$, C$(4,5,0)$ and D$(2,6,2)$, is equal to 
\begin{enumerate}
	\item 9 sq. units
	\item 18 sq. units 
	\item 27 sq. units 
	\item 81 sq. units
\end{enumerate}
\item Find the area of region bounded by the triangle whose vertices are $(-1, 1), (0, 5)$ and $(3, 2)$.
\item The value of $\hat{i}\cdot(\hat{j}\times\hat{k})+\hat{j}\cdot(\hat{i}\times\hat{k})+\hat{k}\cdot(\hat{i}\times\hat{j})$ is
\begin{enumerate}
\item 0
\item -1
\item 1
\item 3
\end{enumerate}
\item The value of $\hat{i}\cdot (\hat{j}\times\hat{k})+\hat{j}\cdot (\hat{i}\times\hat{k})+\hat{k}\cdot (\hat{i}\times\hat{j})$ is
\begin{enumerate}
\item 0
\item -1
\item 1
\item 3
\end{enumerate}
\item Find area of the triangle with vertices at the point given in each of the following:
\begin{enumerate}
\item $(1,0), (6,0), (4,3)$
\item $(2.7), (1,1), (10,8)$
\item $(-2,-3), (3,2), (-1,8)$
\end{enumerate}
\item If area of triangle is 35 square units with vertices $(2,6), (5,-4)$ and $(k,4)$. Then $k$ is:
\begin{enumerate}
\item $12$
\item $-2$
\item $-12,-2$
\item $12, -2$
\end{enumerate}
\item Find values of $k$ if area of triangle is 4 square units and vertices are
\begin{enumerate}
\item $(k,0), (4,0), (0,2)$
\item $(-2,0), (0,4), (0,k)$
\end{enumerate}
\item Find the area of the triangle whose vertices are $(1,-1), (-4,6)$ and $(-3,5)$.
\item Find the area of a triangle formed by the points $A(5,2), B(4,7)$ and $(7,-4)$.
\item Find the area of the triangle formed by the points $P(-1.5,3), Q(6,-2)$ and $R(-3,4)$.
\item If $A(-5,7), B(-4,-5), C(-1,-6)$ and $D(4,5)$ are the vertices of a quadrilateral, find the area of quadrilateral ABCD.
\item Find the area of the triangle whose vertices are $(3,8), (-4,2)$ and $(5,1)$.
\item Find $\abs{\overrightarrow{a} \times \overrightarrow{b}}$, if $\overrightarrow{a} = 2\hat{i} +\hat{j} +3\hat{k}$, and $\overrightarrow{b} = 3\hat{i} +5\hat{j} -2\hat{k}$.
\item Find the area of a triangle having the points $A(1,1,1), B(1,2,3)$ and $C(2,3,1)$ as its vertices.
\item Find the area of a parallelogram whose adjacent sides are given by the vectors $\overrightarrow{a}=3\hat{i} +\hat{j} +4\hat{k}$ and $\overrightarrow{b}=\hat{i} -\hat{j} +\hat{k}$.
\end{enumerate}


\subsection{Formulae}
\begin{enumerate}[label=\thesubsection.\arabic*,ref=\thesubsection.\theenumi]
		\item Find $\abs{\overrightarrow{a}\times\overrightarrow{b}},\text{ if }\overrightarrow{a}=\hat{i}-7\hat{j}+7\hat{k}\text{ and } \overrightarrow{b}=3\hat{i}-2\hat{j}+2\hat{k}$.
	\\
		\solution
		  From \eqref{eq:cross3d-submat},
\begin{align}
	\mydet{\vec{A}_{23}&\vec{B}_{23}}=\mydet{-7 & -2 \\ 7 & 2}=0\\
	\mydet{\vec{A}_{31}&\vec{B}_{31}}=\mydet{1 & 3 \\ 7 & 2}=-19\\
	\mydet{\vec{A}_{12}&\vec{B}_{12}}=\mydet{1 & 3 \\ -7 & -2}=19,
	\\
	\norm{\vec{a}\times\vec{b}}
	 = \norm{\myvec{ \mydet{\vec{A}_{23} & \vec{B}_{23}} \\[1ex] \mydet{\vec{A}_{31} & \vec{B}_{31}} \\[1ex] \mydet{\vec{A}_{12}  & \vec{B}_{12}}}}
=19\sqrt{2}
\end{align}
from 
  \eqref{eq:cross3d}.

\item Find $\lambda$ and $\mu$ if $(2\hat{i}+6\hat{j}+27\hat{k})\times(\hat{i}+\lambda \hat{j} + \mu \hat{k})=\overrightarrow{0}$.
	\\
		\solution
		From 
		 Formula \ref{prop:lin-dep-cross},
performing row reduction, 
\begin{align}
 \myvec{2&6&27 \\ 1& \lambda & \mu}
	\xleftrightarrow{R_{2}\leftarrow 2R_{2}-R_{1}}  	
 \myvec{2&6&27 \\ 0& 2\lambda -6 & 2\mu-27}
\end{align}
For the above matrix to have rank 1,
\begin{align}
	\mu=\frac{27}{2},
	\lambda=3.
\end{align}


\item Find the area of the triangle with vertices $A(1, 1, 2), B(2, 3, 5)$ and $C(1, 5, 5)$.
	\\
		\solution
		\begin{align}
\because \vec{B}-\vec{A} = \myvec{1\\2\\3}, 
\vec{C}-\vec{A} = \myvec{0\\4\\3},
\\
	\frac{1}{2} \norm{\myvec{1\\2\\3} \times \myvec{0\\4\\3}} 
	= 	\frac{1}{2}\norm{\myvec{-6\\3\\4}}
= \frac{\sqrt{61}}{2}
\end{align}
			using 
        \eqref{eq:11/10/1/1area-diag}, 
which is the the desired area.






\item Find the area of the parallelogram whose adjacent sides are determined by the vectors $\overrightarrow{a}=\hat{i}-\hat{j}+3\hat{k}$ and $\overrightarrow{b}=2\hat{i}-7\hat{j}+\hat{k}$.
	\\
		\solution
					From \eqref{eq:tri-area-cross},
			the desired area is obtained as
\begin{align}
	\norm{\myvec{1\\-1\\3} \times \myvec{2\\ -7 \\ 1}}
	=\norm{\myvec{20\\5\\-5}}
= 15\sqrt{2}
\end{align}


\item Find the area of a rhombus if its vertices are $A(3,0), B(4,5), C(-1,4)$  and  $D(-2,-1)$ taken in order. 
	\\
		\solution
	The area of the rhombus is
\begin{align}
                \norm{\myvec{\vec{A-D}}\times \myvec{\vec{B-A}}}=\mydet{5 & 1\\1 & 5} = 24
\end{align}
See 
\figref{fig:chapters/10/7/2/10/gFig1}.
\begin{figure}[!h]
 \begin{center}
  \includegraphics[width=\columnwidth]{chapters/10/7/2/10/figs/fig.pdf}
 \end{center}
\caption{}
\label{fig:chapters/10/7/2/10/gFig1}
\end{figure}

\item Let the vectors $\overrightarrow{a}$ and $\overrightarrow{b}$ be such that $|\overrightarrow{a}| = 3$ and $|\overrightarrow{b}| = \dfrac{\sqrt{2}}{3}$, then $\overrightarrow{a} \times \overrightarrow{b}$ is a unit vector, if the angle between $\overrightarrow{a}$ and $\overrightarrow{b}$ is
	\begin{multicols}{2}
	\begin{enumerate}
\item $\frac{\pi}{6}$
\item $\frac{\pi}{4}$
\item $\frac{\pi}{3}$
\item $\frac{\pi}{2}$
\end{enumerate}
	\end{multicols}
		\solution
		From the given information and 
	\eqref{eq:cross-sin}
%
\begin{align}
	\norm{\vec{a} \times \vec{b}} & = \norm{\vec{a}} \norm{\vec{b}} \sin \theta =1\\
\implies\sin \theta & = \frac{1} {\norm{\vec{a}} \norm{\vec{b}}}
 = \frac{1}{\sqrt{2}}\\
\implies\theta &= 
 \frac{\pi}{4} 
\end{align}

\item Area of a rectangle having vertices A, B, C and D with position vectors $ -\hat{i}+ \frac{1}{2} \hat{j}+4\hat{k}, \hat{i}+ \frac{1}{2} \hat{j}+4\hat{k}, \hat{i}-\frac{1}{2} \hat{j}+4\hat{k}$ and $-\hat{i}- \frac{1}{2} \hat{j}+4\hat{k}$, respectively is
	\\
		\solution
		Since
\begin{align}
\vec{A} - \vec{B} &= \myvec{-2\\0\\0}\\
\vec{C} -\vec{B} &= \myvec{0\\-1\\0}
\end{align}
area of the rectangle is
\begin{align}
 \norm{\brak{\vec{A} -\vec{B}} \times \brak{\vec{C}-\vec{D}}}
= 2
\end{align} 
\iffalse
See Fig. 
   \ref{fig:chapters/12/10/4/12Rect_ABCD}
\begin{figure}[H]
  \centering
   \includegraphics[width=0.75\columnwidth]{chapters/12/10/4/12/figs/Figure_1.png}
   \caption{}
   \label{fig:chapters/12/10/4/12Rect_ABCD}
\end{figure}
\fi





\item Find the area of the triangle whose vertices are 
\begin{enumerate}
\item $(2, 3), (–1, 0), (2, – 4)$
\item $(–5, –1), (3, –5), (5, 2)$ 
\end{enumerate}
		\label{10/7/3/1}
\solution
		    See \tabref{eq:10/7/3/1/area}.
\begin{table}[H]
    \centering
    \caption{}
    \label{eq:10/7/3/1/area}
    \begin{tabular}{|c|c|c|c|}
        \hline
	     & $\vec{A}-\vec{B}$  & $\vec{A}-\vec{C}$  & $\frac{1}{2}\|\brak{\vec{A}-\vec{B}} \times \brak{\vec{A}-\vec{C}}\|$ \\
        \hline
         a)& $\myvec{ 3 \\3 }$ & $\myvec{ 0 \\ 7 }$ & $\frac{21}{2}$ \\
        \hline
	    b)& $\myvec{
 -8 \\
 4 
 }$
         &$\myvec{
 -10 \\
 -3 
 }$
  &  $32$   \\
        \hline
    \end{tabular}
\end{table}


\item Find the area of the triangle formed by joining the mid-points of the sides of the triangle whose vertices are $A(0, –1), B(2, 1)$  and  $C(0, 3)$. Find the ratio of this area to the area of the given triangle.
	\\
\solution
		Using 
	  \eqref{eq:section_formula},
the mid point coordinates are given by
	\begin{align}
		\vec{P} = \frac{1}{2}\vec(\vec{A}+\vec{B})  = \myvec{1\\0}\\
		\vec{Q} = \frac{1}{2}\vec(\vec{B}+\vec{C}) = \myvec{1\\2}\\
		\vec{R} = \frac{1}{2}\vec(\vec{A}+\vec{C}) = \myvec{0\\1}
	\end{align}
	\begin{align}
\because		\vec{P}-\vec{Q} =  \myvec{
 0 \\
 -2 
 },\,
		\vec{Q}-\vec{R} =   \myvec{
 1 \\
 1 
 }
 \\
		ar(PQR)=\frac{1}{2}{\norm{\vec(\vec{P}-\vec{Q})\times\vec(\vec{Q}-\vec{R})}}
		=1
	\end{align}
	Similarly, 
	\begin{align}
		\vec{A}-\vec{B} = \myvec{
 -2 \\
 -2 
 }
 ,\,
		\vec{A}-\vec{C} =  \myvec{
 0 \\
 -4 
 }
 \\
 \implies
		ar(ABC)=\frac{1}{2}{\norm{\vec(\vec{A}-\vec{B})\times\vec(\vec{A}-\vec{C})}}
=4
\\
		\implies \frac{ar\brak{PQR}}{ar\brak{ABC}} = \frac{1}{4}
	\end{align}
	See 
\figref{fig:10/7/3/3Fig}
\begin{figure}[H]
	\begin{center} 
	    \includegraphics[width=0.75\columnwidth]{chapters/10/7/3/3/figs/fig.pdf}
	\end{center}
\caption{}
\label{fig:10/7/3/3Fig}
\end{figure}


\item Find the area of the quadrilateral whose vertices, taken in order, are $A(– 4, – 2), B(– 3, – 5), C(3, – 2)$  and $ D(2, 3)$.
	\\
\solution
		See 
\figref{fig:chapters/10/7/3/4/Fig1}
\begin{figure}[H]
 \begin{center}
  \includegraphics[width=0.75\columnwidth]{chapters/10/7/3/4/figs/fig.pdf}
 \end{center}
\caption{}
\label{fig:chapters/10/7/3/4/Fig1}
\end{figure}
\begin{align}
\because	\vec{A}- \vec{B} =\myvec{-1\\3},\,
	  \vec{A}- \vec{D} =\myvec{-6\\-5},
	  \\
	\vec{B}- \vec{C} =\myvec{-6\\-5},\,
	  \vec{B}- \vec{D} =\myvec{-3\\-8},
	  \\
	  ar(ABD)=\frac{1}{2} \norm{\brak{\vec{A}-\vec{B}}  \times 
   \brak{\vec{A}- \vec{D}}} 
	=	\frac{23}{2}
	\\
	  ar(BCD)=\frac{1}{2} \norm{\brak{\vec{B}-\vec{C}}  \times 
   \brak{\vec{B}- \vec{D}}} 
	=	\frac{33}{2}
	\\
\implies	ar(ABCD)=  ar(ABD) +  ar(BCD)
	= 28
\end{align}


\item Verify that a median of a triangle divides it into two triangles of equal areas for $\triangle ABC$ whose vertices are $\vec{A}(4, -6), \vec{B}(3, 2), \text{ and } \vec{C}(5, 2)$. 
		\label{10/7/3/5}
		\\
\solution
		\begin{align}
\vec{D}=\frac{\vec{B}+\vec{C}}{2}
=\myvec{4\\ 0},
\\
	\vec{A}- \vec{B} =\myvec{1\\ -4},\,
	  \vec{A}- \vec{D} =\myvec{0\\ -6}
	  \\
	  \implies
  ar(ABD)=\frac{1}{2} \norm{\brak{\vec{A}-\vec{B}}  \times 
   \brak{\vec{A}- \vec{D}}} 
	       =3	
	       \\
	\vec{A}- \vec{C} =\myvec{-1\\ -8},\,
	  \vec{A}- \vec{D} =\myvec{0\\ -6}
	  \\
	  \implies
  ar(ACD)=\frac{1}{2} \norm{\brak{\vec{A}-\vec{C}}  \times 
   \brak{\vec{A}- \vec{D}}} 
   \\
	= 3 =
ar(ABD)
\end{align}
See  
\figref{fig:10/7/3/5/}.
\begin{figure}[H]
\centering
\includegraphics[width=0.75\columnwidth]{chapters/10/7/3/5/figs/fig.pdf}
\caption{}
\label{fig:10/7/3/5/}
\end{figure} 

\item The two adjacent sides of a parallelogram are 
$\vec{a}= 2\hat{i}-4\hat{j}+5\hat{k}$  and  $\vec{b} =\hat{i}-2\hat{j}-3\hat{k}$.
Find the unit vector parallel to its diagonal. Also, find its area.\\
	\solution
		The diagonals of the parallelogram are given by
\begin{align}
 \vec{a} + \vec{b} = \myvec{3 \\-6\\2},\, 
 \vec{a} - \vec{b} = \myvec{1 \\-2\\8}
\end{align}
and the corresponding unit vectors are
\begin{align}
	\frac{\vec{a} + \vec{b}}{\norm{\vec{a} + \vec{b}}}  = \myvec{\frac{3}{\sqrt{45}}\\[1ex]-\frac{6}{\sqrt{45}}\\[1ex]\frac{2}{\sqrt{45}}},\, 
	\frac{\vec{a} - \vec{b}}{\norm{\vec{a} - \vec{b}}}  = \myvec{\frac{1}{\sqrt{69}}\\[1ex]-\frac{2}{\sqrt{69}}\\[1ex]\frac{8}{\sqrt{69}}}
\end{align}
%
The area of the parallelogram is given by
\begin{align}
	\norm{\vec{a}\times\vec{b}}  = \norm{\myvec{22 \\-11\\0}} = \sqrt{605}
\end{align}


\item The vertices of a $\triangle ABC$ are $\vec{A}(4,6), \vec{B}(1,5)$ and  $\vec{C}(7,2)$. A line is drawn to intersect sides $AB$ and $AC$ at $\vec{D}$ and $\vec{E}$ respectively, such that $\frac{AD}{AB} = \frac{AE}{AC} = \frac{1}{4}$. Calculate the area of $\triangle ADE$ and compare it with the area of the $\triangle ABC$.
\\
\solution
	See  
\figref{fig:chapters/10/7/4/6Fig1}.
\begin{figure}[H]
 \begin{center}
 \includegraphics[width=0.75\columnwidth]{chapters/10/7/4/6/figs/fig.pdf}
 \end{center}
\caption{}
\label{fig:chapters/10/7/4/6Fig1}
\end{figure}
	Using section formula
	  \eqref{eq:section_formula},
\begin{align}
\vec{D} =\frac{3\vec{A}+\vec{B}}{4}
	=\frac{1}{4}\myvec{13\\ 23}
	\\
\vec{E} =\frac{3\vec{A}+\vec{C}}{4}
	=\frac{1}{4}\myvec{19\\ 20}
	\\
	\vec{A}- \vec{D} 
	=\frac{1}{4}\myvec{3\\ 1},\,
	  \vec{A}- \vec{E}  
	=\frac{1}{4}\myvec{-3\\ 1}
	\\
	\vec{A}- \vec{B} =\myvec{3\\1},
	  \vec{B}-\vec{C} =\myvec{-6\\3}
\end{align}
yielding
\begin{align}
ar(ABD) =\frac{1}{2} \norm{\brak{\vec{A}-\vec{D}}  \times 
   \brak{\vec{A}- \vec{E}}} 
	=	\frac{15}{32}
	\\
	  ar(ABC) =\frac{1}{2} \norm{\brak{\vec{A}-\vec{B}}  \times 
   \brak{\vec{B}- \vec{C}}} 
	=	\frac{15}{2}
	\\
	\implies \frac{ar\brak{ADE}}{ar\brak{ABC}}=\frac{1}{16}
\end{align}

    \item Draw a quadrilateral in the Cartesian plane, whose vertices are 
    \begin{align}
        \vec{A} = \myvec{-4\\5},\, \vec{B} = \myvec{0\\7},\, 
        \vec{C} = \myvec{5\\-5},\, \vec{D} = \myvec{-4\\-2}.
    \end{align}
    Also, find its area.
\label{chapters/11/10/1/1}
   \\ 
    \solution 
See \figref{fig:11/10/1/1quad}.
    From 
        \eqref{eq:11/10/1/1area-diag},
    \begin{align}
ar\brak{ABCD}
	       = \frac{121}{2}
        \label{eq:11/10/1/1ans}
    \end{align}
    \begin{figure}[H]
        \centering
        \includegraphics[width=0.75\columnwidth]{chapters/11/10/1/1/figs/fig.pdf}
        \caption{Plot of quadrilateral $ABCD$}
        \label{fig:11/10/1/1quad}
    \end{figure}

\item Find the area of region bounded by the triangle whose
	vertices are $(1, 0), (2, 2)$ and $(3, 1)$. 
\item Find the area of region bounded by the triangle whose vertices
	are $(– 1, 0), (1, 3)$  and  $(3, 2)$. 
\item Find the area of the $\triangle ABC$, coordinates of whose vertices are $\vec{A}(2, 0), \vec{B}(4, 5)$ and $\vec{C}(6, 3)$.
\item The area of a triangle with vertices $\vec{A}(3, 0), \vec{B}(7, 0)$ and  $\vec{C}(8, 4)$ is
\begin{enumerate}
\item 14
\item 28
\item 8
\item 6
\end{enumerate}
\item Find the area of the triangle whose vertices are $(-8,4),(-6,6)$ and $(-3,9)$.
\item If $\vec{D}\brak{\frac{-1}{2},\frac{5}{2}},\vec{E}(7,3)$ and $\vec{F}\brak{\frac{7}{2},\frac{7}{2}}$ are the midpoints of sides of $\triangle ABC$, find the area of the $\triangle ABC$.
\item Find the sine of the angle between the vectors $\vec{a}=3\hat{i}+\hat{j}+2\hat{k}$ $\text{ and }$ $\vec{b}=2\hat{i}-2\hat{j}+4\hat{k}$.
\item Using vectors, find the area of $\triangle{ABC}$ with vertices A(1,2,3), B(2,-1,4) and C(4,5,-1).
\item Find the area of the parallelogram whose diagonals are $2\hat{i}-\hat{j}+\hat{k}$ and $\hat{i}+3\hat{j}-\hat{k}$.

\item The vector from origin to the points A and B are $\vec{a}$ = $2\hat{i}-3\hat{j}+2\hat{k}$ and  $\vec{b}$ = $2\hat{i}+3\hat{j}+\hat{k}$, respectively, then the area of $\triangle {OAB}$ is
	\begin{enumerate}
\item 340 
\item $\sqrt{25}$
\item $\sqrt{229}$
\item $\frac{1}{2}\sqrt{229}$
\end{enumerate}
\item If $\vec{a} = \hat{i}+\hat{j}+\hat{k}$ and $\vec{b} = \hat{j}-\hat{k}$, find a vector $\vec{c}$ such that $\vec{a}\times\vec{c} = \vec{b}$ and $\vec{a}\cdot \vec{c}$ = 3.
%
\item The area of the quadrilateral ABCD, where A$(0,4,1)$, B$(2,3,-1)$, C$(4,5,0)$ and D$(2,6,2)$, is equal to 
\begin{enumerate}
	\item 9 sq. units
	\item 18 sq. units 
	\item 27 sq. units 
	\item 81 sq. units
\end{enumerate}
\item Find the area of region bounded by the triangle whose vertices are $(-1, 1), (0, 5)$ and $(3, 2)$.
\item The value of $\hat{i}\cdot(\hat{j}\times\hat{k})+\hat{j}\cdot(\hat{i}\times\hat{k})+\hat{k}\cdot(\hat{i}\times\hat{j})$ is
\begin{enumerate}
\item 0
\item -1
\item 1
\item 3
\end{enumerate}
\item The value of $\hat{i}\cdot (\hat{j}\times\hat{k})+\hat{j}\cdot (\hat{i}\times\hat{k})+\hat{k}\cdot (\hat{i}\times\hat{j})$ is
\begin{enumerate}
\item 0
\item -1
\item 1
\item 3
\end{enumerate}
\item Find area of the triangle with vertices at the point given in each of the following:
\begin{enumerate}
\item $(1,0), (6,0), (4,3)$
\item $(2.7), (1,1), (10,8)$
\item $(-2,-3), (3,2), (-1,8)$
\end{enumerate}
\item If area of triangle is 35 square units with vertices $(2,6), (5,-4)$ and $(k,4)$. Then $k$ is:
\begin{enumerate}
\item $12$
\item $-2$
\item $-12,-2$
\item $12, -2$
\end{enumerate}
\item Find values of $k$ if area of triangle is 4 square units and vertices are
\begin{enumerate}
\item $(k,0), (4,0), (0,2)$
\item $(-2,0), (0,4), (0,k)$
\end{enumerate}
\item Find the area of the triangle whose vertices are $(1,-1), (-4,6)$ and $(-3,5)$.
\item Find the area of a triangle formed by the points $A(5,2), B(4,7)$ and $(7,-4)$.
\item Find the area of the triangle formed by the points $P(-1.5,3), Q(6,-2)$ and $R(-3,4)$.
\item If $A(-5,7), B(-4,-5), C(-1,-6)$ and $D(4,5)$ are the vertices of a quadrilateral, find the area of quadrilateral ABCD.
\item Find the area of the triangle whose vertices are $(3,8), (-4,2)$ and $(5,1)$.
\item Find $\abs{\overrightarrow{a} \times \overrightarrow{b}}$, if $\overrightarrow{a} = 2\hat{i} +\hat{j} +3\hat{k}$, and $\overrightarrow{b} = 3\hat{i} +5\hat{j} -2\hat{k}$.
\item Find the area of a triangle having the points $A(1,1,1), B(1,2,3)$ and $C(2,3,1)$ as its vertices.
\item Find the area of a parallelogram whose adjacent sides are given by the vectors $\overrightarrow{a}=3\hat{i} +\hat{j} +4\hat{k}$ and $\overrightarrow{b}=\hat{i} -\hat{j} +\hat{k}$.
\end{enumerate}


\subsection{Miscellaneous}
\begin{enumerate}[label=\thesubsection.\arabic*,ref=\thesubsection.\theenumi]
\item The two opposite vertices of a square are $\vec{A}(–1, 2)$  and $ \vec{C}(3, 2)$. Find the coordinates of the other two vertices.
\\
\solution
	\begin{align}
\vec{C} - \vec{A} = \myvec{
4\\
0
} \equiv 
\myvec{
1\\
0
},\,
\implies \phi= 0\degree
\end{align}
		where
$\phi$ is the angle made by $AC$ with the x-axis.
Also, the diagonal
\begin{align}
	d = \norm{\vec{C}-\vec{A}} = 4
\end{align}
\begin{enumerate}
	\item We start with  the square in \figref{fig:7/4/4/4Fig3},
 with vertices as columns of the matrix
\begin{align}
	\vec{y} = \frac{d}{\sqrt 2}\myvec{0 & 1 & 1 & 0 \\ 0 & 0 & 1 & 1}
\end{align}
	in \eqref{eq:conic_affine}.
\item The next square, obtained as 
\begin{align}
\vec{P}\vec{y},
\end{align}
which is a rotated version of 
\figref{fig:7/4/4/4Fig3},
is available in 
\figref{fig:7/4/4/4Fig2}.  The angle of rotation
\begin{align}
	\theta = \phi - \frac{\pi}{4}
\end{align}
\item The desired square  is obtained using
\eqref{eq:conic_affine} as
\begin{align}
	\vec{x}=\vec{P}\vec{y} + \myvec{\vec{A} & \vec{A} &\vec{A} &\vec{A}} = 
		\myvec{
-1  &1 & 3 & 1 \\
2 & 0 & 2 & 4
	}
\end{align}
and available in 
\figref{fig:7/4/4/4Fig1}. The 2nd and 4th columns in the above matrix are 
$\vec{B}$ and $\vec{C}$ respectively.
\end{enumerate}
\begin{figure}[H]
	\begin{center} 
	    \includegraphics[width=0.75\columnwidth]{chapters/10/7/4/4/figs/fig.pdf}
	\end{center}
\caption{}
\label{fig:7/4/4/4Fig3}
\end{figure}
\begin{figure}[H]
	\begin{center} 
	    \includegraphics[width=0.75\columnwidth]{chapters/10/7/4/4/figs/fig1.pdf}
	\end{center}
\caption{}
\label{fig:7/4/4/4Fig2}
\end{figure}
\begin{figure}[H]
	\begin{center} 
	    \includegraphics[width=0.75\columnwidth]{chapters/10/7/4/4/figs/fig2.pdf}
	\end{center}
\caption{}
\label{fig:7/4/4/4Fig1}
\end{figure}

\item The base of an equilateral triangle with side $2a$ lies along the y-axis such that the mid-point of the base is at the origin. Find the vertices of the triangle.
\label{chapters/11/10/1/2}
	\\
	\solution 
	\begin{figure}[H]
		\centering
 \includegraphics[width=0.75\columnwidth]{chapters/11/10/1/2/figs/fig.pdf}
		\caption{$a = 2$.}
		\label{fig:11/10/1/2}
  	\end{figure}
		See \figref{fig:11/10/1/2}.
	Let the base be $BC$.  From the given information, 
\begin{align}
	\vec{B} = a\vec{e}_2,
	\vec{C} = -a\vec{e}_2
\end{align}
Since $\vec{A}$ lies on the $x$-axis, 
\begin{align}
	\vec{A} = k\vec{e}_1
\end{align}
and 
\begin{align}
	\norm{\vec{A}-\vec{C}}^2 &= \brak{2a}^2
	\\
	\implies \norm{\vec{A}}^2+\norm{\vec{C}}^2 - 2 \vec{A}^{\top}\vec{C} &= 4a^2
	\\
	\implies k^2 +a^2 &= 4a^2
\end{align}
yielding
\begin{align}
 k = \pm a\sqrt{3}
\end{align}
Thus, 
\begin{align}
	\vec{A} = \pm \sqrt{3}a\vec{e}_1
\end{align}


\item Let $\vec{a}$ and $\vec{b}$ be two unit vectors and $\theta$ the angle between them. Then $\vec{a}+\vec{b}$ is a unit vector if
	\begin{enumerate}
			\itemsep2pt
		\item $\theta = \frac{\pi}{4}$
		\item $\theta = \frac{\pi}{3}$
		\item $\theta = \frac{\pi}{2}$
		\item $\theta = \frac{2\pi}{3}$
			\end{enumerate}
\solution
\begin{align}
	\because \norm{\vec{a}}=\norm{\vec{b}}  = 3 \norm{\vec{a}+\vec{b}}&=1, \label{eq:12/10/5/17/2} \\
    \norm{\vec{a}+\vec{b}}^2 & = 1^2 \\
    \implies \norm{\vec{a}}^2 + \norm{\vec{b}}^2 + 2\vec{a}^{\top}\vec{b} & = 1 \label{eq:12/10/5/17/3} \\
    \implies (\norm{\vec{a}}\norm{\vec{b}}\cos{\theta}) & = \frac{-1}{2} \label{eq:12/10/5/17/4} \\
	\implies \cos{\theta}  = \frac{-1}{2}, \text{or, }\theta&=\frac{2\pi}{3}
\end{align}

\item Show that the tangent of an angle between the lines 
\begin{align}
	\frac{x}{a}+\frac{y}{b}&=1 \text{ and }
	\\
	\frac{x}{a}-\frac{y}{b}&=1 
\end{align}
is $\frac{2ab}{a^2-b^2}$.
\item Find $\abs{\overrightarrow {x}}$, if for a unit vector $\overrightarrow {a}, (\overrightarrow {x}-\overrightarrow {a})\cdot (\overrightarrow {x}+\overrightarrow {a}$)=12.
	\\
\solution 
		From the given information,
\begin{align}
  \label{eq:12/10/3/9det2f}
  \brak{\vec{x}-\vec{a}}^\top\brak{\vec{x}+\vec{a}} &= 12 \\
  \implies \norm{\vec{x}}^{2} - \norm{\vec{a}}^{2} &= 12 \\
\implies   
	\norm{\vec{x}} &= \sqrt{13}
\end{align}

\item Find $\abs{\overrightarrow {a}}$ and $\abs{\overrightarrow {b}}$, if ($\overrightarrow {a}+\overrightarrow {b})\cdot (\overrightarrow {a}-\overrightarrow {b})=8$ and $\abs{\overrightarrow {a}}=8\abs{\overrightarrow {b}}$.
	\\
	\solution
		\begin{align}
\because \brak{\vec{a}+\vec{b}}^\top\brak{\vec{a}-\vec{b}}=8,
\norm{\vec{a}} &= 8\norm{\vec{b}},\\
\norm{\vec{a}}^2-\norm{\vec{b}}^2&=8\\
\implies\norm{8\vec{b}}^2-\norm{\vec{b}}^2&=8\\
\implies \norm{\vec{b}}&=\frac{2\sqrt{2}}{3\sqrt{7}}
\end{align}
Thus, 
\begin{align}
\norm{\vec{a}}&=8\norm{\vec{b}}
=\frac{16\sqrt{2}}{3\sqrt{7}}
\end{align}

\item Find the magnitude of two vectors $\overrightarrow {a}$ and $\overrightarrow {b}$, having the same magnitude and such that the angle between them is $60\degree$ and their scalar product is $\frac{1}{2}$.
	\\
	\solution
		Given 
\begin{align}
	\norm{\vec{a}}= \norm{\vec{b}}, {\cos\theta} = \frac{1}{2}, 
	\vec{a}^{\top}{\vec{b}} = \frac{1}{2},  \\
\implies 
	\frac{1}{2} = \frac{\frac{1}{2}}{\norm{\vec{a}}^2}
\implies \norm{\vec{a}}
= \norm{\vec{b}}=1
\end{align}
by using  the definition of the scalar product
	in \eqref{eq:angle-inner}.

\item Show that $\abs {\overrightarrow {a}}\overrightarrow {b}+\abs{\overrightarrow {b}}\overrightarrow {a}$ is perpendicular to $\abs{\overrightarrow {a}} \overrightarrow {b}-\abs{\overrightarrow {b}} \overrightarrow {a}$, for any two nonzero vectors $\overrightarrow {a}$ and $\overrightarrow {b}$.
	\\
	\solution
		\begin{align}
\norm{\vec{a}}\vec{b}+\norm{\vec{b}}\vec{a}
=
	\norm{\vec{a}}\norm{\vec{b}}\brak{\frac{\vec{b}}{\norm{\vec{b}}}+\frac{\vec{a}}{\norm{\vec{a}}}}
	\\
\norm{\vec{a}}\vec{b}-\norm{\vec{b}}\vec{a}
=
	\norm{\vec{a}}\norm{\vec{b}}\brak{\frac{\vec{b}}{\norm{\vec{b}}}-\frac{\vec{a}}{\norm{\vec{a}}}}
	\\
	\implies 
	\brak{\norm{\vec{a}}\vec{b}+\norm{\vec{b}}\vec{a}}^{\top} \brak{\norm{\vec{a}}\vec{b}-\norm{\vec{b}}\vec{a}} = 0
\end{align}
	from \eqref{eq:12/10/3/11/unit}.

\item If $\vec{a}$, $\vec{b}$, $\vec{c}$ are unit vectors such that $\vec{a}$+$\vec{b}$+$\vec{c}$=0, then the value of $\vec{a} \cdot \vec{b}+\vec{b} \cdot \vec{c}+\vec{c} \cdot \vec{a}$ is
	\begin{enumerate}
\item 1
\item 3
\item $\frac{-3}{2}$
\item None of these
\end{enumerate}
	\solution
		\begin{align}
	\norm{{\vec{a}}+{\vec{b}}+{\vec{c}}}^2=0
	\nonumber \\
	\implies{\norm{\vec{a}}}^2+{\norm{\vec{b}}}^2+{\norm{\vec{c}}}^2+2({{\vec{a}^\top}{\vec{b}}+{\vec{b}^\top}{\vec{c}}+{\vec{c}^\top}{\vec{a}}})=0
	\nonumber \\
	\implies3+2({{\vec{a}^\top}{\vec{b}}+{\vec{b}^\top}{\vec{c}}+{\vec{c}^\top}{\vec{a}}})=0\nonumber \\
	\implies{\vec{a}^\top}{\vec{b}}+{\vec{b}^\top}{\vec{c}}+{\vec{c}^\top}\vec{a}=-\frac{3}{2}
\end{align}

\item If either vector $\overrightarrow {a}=0$ or $\overrightarrow {b}=0$, then $\overrightarrow {a}.\overrightarrow {b}$=0. But the converse need not be true. Justify your answer with an example.
	\\
	\solution
		\begin{align}
	\vec{a}=\myvec{1\\1},\,
\vec{b}=\myvec{1\\-1}\\
\implies \vec{a} ^\top \vec{b} =  0 
\end{align}



\item Prove that $(\vec{a}+\vec{b})\cdot(\vec{a}+\vec{b})=|{\vec{a}}|^2+|{\vec{b}}|^2$, if and only if $\vec{a}, \vec{b}$ are perpendicular, given $\vec{a}\neq\vec{0}, \vec{b}\neq\vec{0}$.\\
	\solution
			\begin{align}
\because 		\brak{\vec{a}+\vec{b}}^{\top}\brak{\vec{a}+\vec{b}} 
		= \norm{\vec{a}}^2+\norm{\vec{b}}^2,
		\\
		 \norm{\vec{a}}^2+\norm{\vec{b}}^2+2\vec{a}^{\top}\vec{b}
		= \norm{\vec{a}}^2+\norm{\vec{b}}^2
		\\
		\implies 
		\vec{a}^{\top}\vec{b} = 0 
	\end{align}


	\item  If $l_1, m_1,n_1 \text{ and } l_2,m_2,n_2$ are the direction cosines of two mutually perpendicular lines, show that the direction cosines of the line perpendicular to both these are  $m_1n_2-m_2n_1,n_1l_2-n_2l_1,l_1m_2-l_2m_1$.
\\
    \solution
		\begin{align}
\vec{P} 
	=\myvec{
l_1&l_2&m_1n_2-m_2n_1\\
        m_1&m_2&n_1l_2-n_2l_1\\
        n_1&n_2&l_1m_2-l_2m_1
}
	\end{align}
	satisfies 
\eqref{eq:12/10/3/5/inner}.
	Hence, the three vectors are mutually perpendicular.

\item
Find the angle between the lines whose direction ratios are $a,b,c$ and $b-c,c-a,a-b$.
\\
\solution
    \begin{align}
\because \myvec{a&b&c}\myvec{b-c\\c-a\\a-b} = 0,
   \theta=\frac{\pi}{2}
    \end{align}

\item The value of the expression $\abs{\vec{a}\times\vec{b}}$+ $({\vec{a}\cdot\vec{b}})$ is \rule{1cm}{0.15mm}
\item If $\abs{\vec{a}\times\vec{b}}^2$ + $\abs{\vec{a}\cdot\vec{b}}^2$=144 $\text{and}$  $\abs{\vec{a}}$=4, then $\abs{\vec{b}}$ is equal to \rule{1cm}{0.15mm}.
\item If the directions cosines of a line are $(k,k,k)$ then
\begin{enumerate}
	\item $k>0$
	\item $0<k<1$
	\item $k=1$ 
	\item $k=\dfrac{1}{\sqrt{3}}$ or $-\dfrac{1}{\sqrt{3}}$
\end{enumerate}
\item  Find the position vector of a point A in space such that $\overrightarrow{OA}$ is inclined at $60 \degree$ to OX and at $45 \degree$ to OY and $\abs{\overrightarrow{OA}} =10$ units.
\item If $(-4,3)\text{ and }(4,3)$ are two vertices of an equilateral triangle. Find the coordinates of the third vertex, given that the origin lies in the interior of the triangle. 
\item $\vec{A} (6,1),\vec{B}(8,2) \text{ and } \vec{C}(9,4)$ are three vertices of a parallelogram ABCD. If $\vec{C}$ is the midpoint of DC find the area of $\triangle ADE$.
\item If the points  $\vec{A}(1,-2), \vec{B}(2,3) , \vec{C}(a,2)\text{ and }\vec{D} (-4-3)$ form parallelogram, find the value of $a$ and height of the parallelogram taking AB as base.
\item Ayush starts walking from his house to office. Instead of going to the office directly, he goes to a bank first, from there to his daughter school and then reaches the office what is the extra distance travelled by Ayush in reaching his office? If the house is situated at $(2,4)$, bank at $(5,8)$, school at $(13,14)$ and office at $(13,26)$ and coordinates are in km.
\item Find the angle between the lines whose direction cosines are given by the equations $l+m+n=0$, $l^2+m^2-n^2=0$.
\item If a variable line in two adjacent positions has directions cosines $l, m, n$ and $l+\delta l, m+\delta m, n+\delta n$, show that the small angle $\delta\theta$ between the two positions is given by 
\begin{align}
	\delta\theta^2=\delta l^2+\delta m^2+\delta n^2
\end{align}
\item The vector $\vec{a}+\vec{b}$ bisects the angle between the non-collinear vectors $\vec{a}$ $\text{ and }$ $\vec{b}$ if \rule{1cm}{0.15mm}.
\item If $\vec{a}$ $\text{ and }$ $\vec{b}$ are adjacent sides of a rhombus, then $\vec{a}\cdot \vec{b}$=0.
    \item If $ \vec{A},\vec{B},\vec{C} $ are mutually perpendicular vectors of equal magnitudes,show that the  $ \vec{A}+\vec{B}+\vec{C} $ is equally inclined to $ \vec{A},\vec{B}  \text{ and }  \vec{C} $.
\item Projection vector of $\vec{a}$ on $\vec{b}$ is
	\begin{enumerate}
\item $\left(\frac{\vec{a}\cdot\vec{b}}{\abs{\vec{b}}^2}\right)$
\item $\frac{\vec{a}\cdot\vec{b}}{\abs{\vec{b}}}$
\item $\frac{\vec{a}\cdot\vec{b}}{\abs{\vec{a}}}$
\item $\left(\frac{\vec{a}\cdot\vec{b}}{\abs{\vec{a}}^2}\right)$
\end{enumerate}
\item If $\vec{a}$ is  any non-zero vector, then $(\vec{a}\cdot \hat{i})\hat{i}$+$(\vec{a}\cdot \hat{j})\hat{j}$+$(\vec{a}\cdot \hat{k})$ $\hat{k}$ equals \rule{1cm}{0.15mm}.
\item If $\vec{a},\vec{b},\vec{c}$ are the three vectors such that $\vec{a}+\vec{b}+\vec{c}=0$ $\text{ and }$ $|\vec{a}|=2$, $|\vec{b}|$=3, $|\vec{c}|$=5, the value of $\vec{a} \cdot \vec{b}+\vec{b} \cdot \vec{c}+\vec{c} \cdot \vec{a}$ is
	\begin{enumerate}
\item 0
\item 1	
\item -19
\item 38
\end{enumerate}
\item If $\vec{r}\cdot\vec{a}=0, \vec{r}\cdot\vec{b}=0$ and $\vec{r}\cdot\vec{c}=0$ for some non-zero vector $\vec{r}$, then the value of $\vec{a}\cdot(\vec{b}\times\vec{c})$ is \rule{1cm}{0.15mm}.
\item If $\abs{\vec{a}+\vec{b}}$ = $\abs{\vec{a}-\vec{b}}$, then the vectors $\vec{a}$ $\text {and}$ $\vec{b}$ are orthogonal.
\item Prove that the lines $x=py+q , z=ry+s \text{ and } x=p^{\prime}y+q^{\prime}, z=r^{\prime}y+s^{\prime} $ are perpendicular if $pp^{\prime}+rr^{\prime}+1=0$.
\item Show that the straight lines whose direction cosines are given by $2l+2m-n=0$ and $mn+nl+lm=0$ are at right angles.
\item If $l_1, m_1, n_1;l_2, m_2, n_2;l_3, m_3, n_3$ are the direction cosines of the three mutually perpendcular lines, prove that the line whose direction cosines are propotional to $l_1+l_2+l_3 , m_1+m_2,m_3, n_1+n_2+n_3$ make angles with them.
\item Assuming that straight lines work as the plane mirror for a point, find the image of the point $(1,2)$ in the line $x-3y+4=0$.
\item Find $\abs{\overrightarrow{a}-\overrightarrow{b}}$, if two vectors $\overrightarrow{a}$ and $\overrightarrow{b}$ are such that $\abs{\overrightarrow{a}}=2, \abs{\overrightarrow{b}}=3$ and $\overrightarrow{a} \cdot \overrightarrow{b}=4$.
\item If $\overrightarrow{a}$ is a unit vector and $(\overrightarrow{x}-\overrightarrow{a}) \cdot (\overrightarrow{x}+\overrightarrow{a})=8$, then find $\abs{\overrightarrow{x}}$.
\item Let $\overrightarrow{a}, \overrightarrow{b},$ and $ \overrightarrow{c}$ are three vectors such that $\abs{\overrightarrow{a}}=3, \abs{\overrightarrow{b}}=4 \abs{\overrightarrow{c}}=5$ and each one of them being perpndicular to the sum of the other to, find $\abs{\overrightarrow{a}+\overrightarrow{b}+\overrightarrow{c}}$.
\item If with reference to the right handed system of mutually perpendicular unit vectors $\hat{i},\hat{j}$ and $\hat{k}, \overrightarrow{\alpha} = 3\hat{i} -\hat{j}, \overrightarrow{\beta}= 2\hat{i} +\hat{j} -3\hat{k}$, then express $\overrightarrow{\beta}$ in the form $\overrightarrow{\beta} = \overrightarrow{\beta_1} +\overrightarrow{\beta_2}$ where $\overrightarrow{\beta_1}$ is parallel to $\overrightarrow{\alpha}$ and $\overrightarrow{\beta_2}$ is perpendicular to $\overrightarrow{\alpha}$.
\item Three vectors $\overrightarrow{a}, \overrightarrow{b}$ and $\overrightarrow{c}$ satisfy the condition $\overrightarrow{a} +\overrightarrow{b} +\overrightarrow{c} =0$. Evaluate the quantity $\mu = \overrightarrow{a}\cdot \overrightarrow{b} +\overrightarrow{b} \cdot \overrightarrow{c} +\overrightarrow{c} \cdot \overrightarrow{a}$, if $\abs{\overrightarrow{a}}=3, \abs{\overrightarrow{b}}=4$ and $\abs{\overrightarrow{c}}=2$. 
\item A line makes angles $\alpha, \beta, \gamma$ and $\delta$  with the diagonals of a cube, prove that 
\begin{align}
\cos^2\alpha +\cos^2\beta +\cos^2\gamma +\cos^2\delta = \frac{4}{3}.
\end{align}
\end{enumerate}

\subsection{Formulae}
%\begin{enumerate}[label=\arabic*.,ref=\theenumi]
\begin{enumerate}[label=\thesubsection.\arabic*.,ref=\thesubsection.\theenumi]
	\item 
The affine transformation is given by 
\begin{align}
	\label{eq:conic_affine}
	\vec{x} = \vec{P}^{\top}\vec{y}+\vec{c}
\end{align}
where 
\begin{align}
\vec{P} =
\myvec{
\cos\theta & -\sin\theta \\
\sin\theta & \cos\theta 
}
\end{align}
is the rotation matrix and $\vec{c}$ is the translation vector.
\item Given vertices $\vec{A}, \vec{C}$ of a square, the other two vertices are given by
\begin{align}
\begin{split}
	\vec{B} = \norm{\vec{C}-\vec{A}}\cos \frac{\pi}{4}\vec{P}^{\top}\vec{e}_1+\vec{A}
	\\
	\vec{D} = \norm{\vec{C}-\vec{A}}\cos \frac{\pi}{4}\vec{P}^{\top}\vec{e}_2+\vec{A}
\end{split}
	\label{eq:affine-square-bd}
\end{align}
%-c}}{\norm{\vec{n}}}\vec{n}
	\\
		\solution Shifting $\vec{A}$ to the origin and rotating the square clockwise by an angle $\phi$ made by $CA$ with the $x$-axis,
	from \eqref{eq:conic_affine},
\begin{align}
\vec{A} = \vec{P}\vec{0}+\vec{c}
\\
\implies 
\vec{c} = \vec{A}
\\
	\theta =  \phi -\frac{\pi}{4} 
\end{align}
and we obtain a square with the other vertices as
\begin{align}
\begin{split}
	\vec{B}_1 = \norm{\vec{C}-\vec{A}}\cos \frac{\pi}{4}\vec{e}_1
	\\
	\vec{D}_1 = \norm{\vec{C}-\vec{A}}\cos \frac{\pi}{4}\vec{e}_2
\end{split}
	\label{eq:affine-bd}
\end{align}
	From \eqref{eq:conic_affine}
	and 
	\eqref{eq:affine-bd},
	we obtain \eqref{eq:affine-square-bd}.
\end{enumerate}

\newpage
\section{Constructions}
\subsection{Triangle}
\begin{enumerate}[label=\thesubsection.\arabic*,ref=\thesubsection.\theenumi]
    \item Draw a triangle $\triangle ABC$ with $BC = 6 \text{ cm}$, $AB = 5 \text{ cm}$, and $\angle ABC = 60\degree$.  \hfill (10, 2018)
\item Construct a triangle with sides $5cm$, $6cm$ and $7cm$. 
		\hfill (10, 2019)
\item Construct an equilateral $\triangle ABC$ with each side $5 cm$. 
		\hfill (10, 2019)
\item Construct a right triangle in which sides (other than the hypotenuse) are $8 cm$ and $6 cm$. 
		\hfill (10, 2019)

\item Construct a $\triangle ABC$ in which $CA = 6cm$ , $AB = 5cm$ and $BAC= 45\degree$. 
		\hfill (10, 2019)
\item Construct a triangle $ABC$ with side $BC = 6 cm$, $\angle B=45\degree, \angle A= 105\degree$. 
		\hfill (10, 2019)
\item Write the steps of construction for drawing a $\triangle ABC$ in which $BC=8$cm, $\angle B=45\degree$ and $\angle C= 30\degree$. 
		\hfill (10, 2018)
\item Construct a triangle $ABC$ with side BC = $7$ cm, $\angle{B}$=$45\degree$, $\angle{A}$=$105\degree$. 
		\hfill (10, 2017)
\item Draw an isosceles $\triangle ABC$ in which $BC=5.5 cm$ and altitude $AL=5.3 cm$. 
		\hfill (10, 2016)
     \item Construct a right triangle ABC with $AB$= $6$ cm, $BC$ = $8$ cm and $\angle$ B = 90$\degree$. Draw $BD$, the perpendicular from $\vec{B}$ on AC. Draw the circle through $\vec{B}$, $\vec{C}$ and $\vec{D}$ and construct the tangents from $\vec{A}$ to this circle
		\hfill (10, 2015)

     \item Construct a $\triangle$ ABC in which  $AB$ = $6$ cm, $\angle$ A = $30\degree $ and $\angle$ B = $60\degree$. 
		\hfill (10, 2015)
\item Construct a triangle $ABC$ in which $AB = 5$ cm, $BC = 6$ cm and $\angle ABC = 60\degree$. 
		\hfill (10, 2015)
\item Draw a triangle $ABC$ with $BC = 7 \text{ cm}$, $\angle B = 45 \degree$ and $\angle C = 60 \degree$. 
		\hfill (10, 2012)
%construction
\item Construct a right triangle in which the sides, (other than the hypotenuse) are of length $6\text{ cm}$ and $8\text{ cm}$. 
		\hfill (10, 2012)
\end{enumerate}

\subsection{Quadrilateral}
\begin{enumerate}[label=\thesection.\arabic*,ref=\thesection.\theenumi]
\item Draw a parallelogram ${ABCD}$ in which $BC=5 cm, AB=3 cm$ and $\angle{ABC}=60\degree$, divide it into triangles ${ACB}\text{ and }{ABD}$ by the diagonal $BD$. 
\item Construct a square of side $3 cm$.
\item Construct  a rectangle whose adjacent sides are of lengths $5 cm$ and $3.5 cm$.
\item Construct a rhombus whose side is of length $3.4 cm$ and one of its angles is $45\degree$.
\item Construct a rhombus whose diagonals are 4 cm and 6 cm in lengths.
\end{enumerate}

\subsection{Formulae}
\begin{enumerate}[label=\thesubsection.\arabic*.,ref=\thesubsection.\theenumi]
\item Construct a $\triangle ABC$ given $a, \angle B$ and $K = b+c$.
		\label{prob:9/11/2/1}
	\\
	\solution 
	Using the cosine formula in  $\triangle ABC$,
\begin{align}
	{b}^2&= {a}^2 + {c}^2 - 2ac\cos{B}
\\
\implies	(K-c)^2 &= {a}^2 + c^2- 2  a  c\cos{B}
\\
\implies
	c &=
	\frac{K^2-a^2}{2\brak{K- a  \cos{B}}}
		\label{eq:9/11/2/1}
\end{align}
The coordinates of $\triangle ABC$ can then be expressed as
\begin{align}
		\label{eq:9/11/2/1-final}
	\vec{A}=c\myvec{\cos B \\ \sin B},
	\vec{B} = \vec{0},
	\vec{C} =\myvec{a \\ 0}.
\end{align}
\item Construct a $\triangle ABC$ given $\angle B, \angle C$ and $K = a+b+c$.
	\\
	\solution
	\begin{align}
a+b+c &= K \\
b\cos C + c \cos B -a &=0 \\
b\sin C - c \sin B &=0
\end{align}
resulting in the matrix equation
\begin{align}
		\label{eq:9/11/2/4}
	\myvec{1 & 1 & 1 \\ -1 & \cos C & \cos B  \\ 0 &\sin C & -\sin B } \myvec{a \\ b \\ c} = K \myvec{1 \\ 0 \\ 0}
\end{align}
which can be solved to obtain all the sides.  $\triangle ABC$ can then be plotted using
\begin{align}
\vec{A} = \myvec{a \\ b},\,
\vec{B} = \vec{0},\, 
\vec{C} = \myvec{a \\ 0}
		\label{eq:9/11/2/4-final}
\end{align}
\end{enumerate}

\newpage
\section{Linear Forms}
\subsection{Equation }
Find the equation of line 
\begin{enumerate}[label=\thesubsection.\arabic*,ref=\thesubsection.\theenumi]
	\item passing through the point $\vec{P}(– 4, 3)$ with slope $\frac{1}{2}$.
\label{chapters/11/10/2/2}
\\
\solution
			From \eqref{eq:line-school-normal},
\begin{align}
\vec{n}\equiv \myvec{\frac{1}{2}\\ -1}
\implies \myvec{\frac{1}{2}&-1}{\vec{x}}&=-5
\end{align}
using \eqref{eq:geo-normal}.
See 
		\figref{fig:chapters/11/10/2/2/Figure}.
\begin{figure}[H]
\centering
\includegraphics[width=0.75\columnwidth]{chapters/11/10/2/2/figs/fig.pdf}
\caption{}
		\label{fig:chapters/11/10/2/2/Figure}
\end{figure}

	\item passing through $\myvec{0\\0}$ with slope $m$.\\
\label{chapters/11/10/2/3}
\solution
\begin{align}
\because			\vec{n} =\myvec{m \\ -1},
		\end{align}
		the desired equation is 
		\begin{align}
			\myvec{m & -1}\brak{\vec{x}-\myvec{0\\0}} &=0\\
\implies			\myvec{m & -1}\vec{x} &= 0
		\end{align}

    \item passing through 
    $\vec{A} = \myvec{2\\2\sqrt{3}}$ and inclined with the x-axis at an angle 
    of 75\textdegree.
\label{chapters/11/10/2/4}
\\
    \solution 
    \begin{align}
	    \vec{n} &= \myvec{-1\\2+\sqrt{3}}
        \label{eq:11/10/2/4normal-vec}
	\\
	    \implies
        \implies \vec{n}^\top\vec{x} = \vec{n}^\top\vec{A} &= 4\brak{\sqrt{3}+1} \\
        \implies \myvec{-1&2+\sqrt{3}}\vec{x} &=\myvec{-1&2+\sqrt{3}}\myvec{2\\2\sqrt{3}}  
	    \\
	    &= 4\brak{\sqrt{3}+1}
        \label{eq:11/10/2/4line}
    \end{align}
is the desired equation.  See \figref{fig:11/10/2/4line}.
    \begin{figure}[!ht]
        \centering
        \includegraphics[width=\columnwidth]{chapters/11/10/2/4/figs/line.png}
        \caption{}
        \label{fig:11/10/2/4line}
    \end{figure}

\item intersecting the x-axis at a distance of 3 units to the left of origin with slope of -2.
\label{chapters/11/10/2/5}
\\
\solution 
		From the given information,
\begin{align}		
	\vec{A}=\myvec{-3\\0},\,
\vec{n} = \myvec{2 \\1}.
\end{align}
The desired equation of the line is
\begin{align}
\implies	\myvec { 2 & 1 } \brak{ \vec{x} - \myvec{ -3 \\ 0}} &= 0  \\
	\text{or, }	\myvec{ 2 & 1} \vec{x}  &= -6
        \label{eq:chapters/11/10/2/5/1}
\end{align}
See \figref{fig:chapters/11/10/2/5/Fig1}.
\begin{figure}[H]
	\begin{center}
		\includegraphics[width=0.75\columnwidth]{chapters/11/10/2/5/figs/fig.pdf}
	\end{center}
\caption{}
\label{fig:chapters/11/10/2/5/Fig1}
\end{figure}


\item intersecting the y-axis at a distance of 2 units above the origin and making an
angle of $30\degree$ with positive direction of the x-axis.
\\
\solution 
\begin{align}
    \vec{n} =  \myvec{-\frac{1}{\sqrt{3}} \\ 1},
    \vec{A} = \myvec{0 \\ 2}.
\end{align}
Hence, 
the equation of the line is given by
\begin{align}
\myvec{-\frac{1}{\sqrt{3}}&1}\brak{ \vec{x} - \myvec{0 \\ 2}} &= 0  \\
    \text{or, }	\myvec{-\frac{1}{\sqrt{3}}&1} \vec{x}  &= 2
\end{align}
%
\iffalse
See
    \figref{fig:chapters/11/10/2/6/line}.
\begin{figure}[H]
    \centering
    \includegraphics[width=0.75\columnwidth]{chapters/11/10/2/6/figs/fig.pdf}
    \caption{}
    \label{fig:chapters/11/10/2/6/line}
\end{figure}
\fi


\item Find the equation of the line passing through the points $\vec{A}\myvec{-1\\1}$ and $\vec{B}\myvec{2\\-4}$.
\label{chapters/11/10/2/7}
\\
\solution 
\begin{align}
	\vec{m} &= \vec{A} - \vec{B}
= \myvec{-3\\5}
\implies
\vec{n} &= \myvec{5\\3}
\end{align}
Thus, the equation of line is
\begin{align}
 \myvec{ 5 & 3}\vec{x}  &= -2
\end{align}
See 
   \figref{fig:chapters/11/10/2/7/Line_AB}.
\begin{figure}[h!]
  \centering
   \includegraphics[width=\linewidth]{chapters/11/10/2/7/figs/Figure_1.png}
   \caption{}
   \label{fig:chapters/11/10/2/7/Line_AB}
\end{figure}





\item 
The vertices of triangle $PQR$ are $\vec{P}(2,1), \vec{Q}(-2,3), \vec{R}(4,5)$. Find the equation of the median through $\vec{R}$.
\label{chapters/11/10/2/9}
\\
\solution
	\begin{figure}[!ht]
		\centering
 \includegraphics[width=\columnwidth]{chapters/11/10/2/9/figs/line.png}
		\caption{}
		\label{fig:11/10/2/9}
  	\end{figure}
	See Fig. 
		\ref{fig:11/10/2/9}.
Using section formula, the mid point of $PQ$ is
\begin{align}
\vec{A} = \frac{\vec{P} +\vec{Q} }{2}
	= {\myvec{0\\2}}
\end{align} 
So, the direction vector of $AR$ is 
\begin{align}
	\vec{m} ={\vec{R} - \vec{A}}
= \myvec{4 \\ 3}
\\
	\implies \vec{n} = 
 \myvec{3 \\ -4}
\end{align}
which is the normal vector.  Thus,
the equation of the line is 
\begin{align}
	\myvec{3 & -4}\brak{\vec{x} - \vec{R}} = 0
	\\
	\implies 
	\myvec{3 & -4}\vec{x} =8 
\end{align}

\item Find the equation of line  drawn perpendicular to the line $\frac{x}{4}+\frac{y}{6}=1$ through the point where it meets the y-axis \\
\solution
				The given line
parameters are
\begin{align}
		\vec{n} = \myvec{3\\2},\, c=12 ,\,
	\vec{m} =\myvec{-2 \\ 3}.
\end{align}
and the point on the y-axis is
\begin{align}
	\vec{A} =\myvec{0\\6}.
\end{align}
Thus, the equation of the desired line is 
\begin{align}
	\vec{m}^\top\brak{\vec{x}-\vec{A}}&=0\label{eq:chapters/11/10/4/7/5}
	\\
\implies
			\myvec{-2 & 3}\vec{x} &=-18
		\end{align}
		\iffalse
		See 
  \figref{fig:chapters/11/10/4/7/Figure}.
\begin{figure}[H]
\includegraphics[width=0.75\columnwidth]{chapters/11/10/4/7/figs/fig.png}
\caption{}
  \label{fig:chapters/11/10/4/7/Figure}
\end{figure}
\fi

\item Find the equation of line whose perpendicular distance from the origin is 5 units and the angle made by the perpendicular with the positive $x$-axis is $30\degree$.
\label{chapters/11/10/2/8}
\\
\solution
			From 
\eqref{eq:chapters/11/10/2/8-final},
		Thus, the equation of lines are
\begin{align}
	\myvec{\frac{\sqrt{3}}{2}& \frac{1}{2}}\vec{x}=\pm5
\end{align}
\iffalse
See 
\figref{fig:chapters/11/10/2/8/Fig1}.
\begin{figure}[H]
\begin{center}
\includegraphics[width=0.75\columnwidth]{chapters/11/10/2/8/figs/fig.pdf}
\end{center}
\caption{}
\label{fig:chapters/11/10/2/8/Fig1}
\end{figure}
\fi

\item 
	Find the equation of the line passing through  (-3,5) and perpendicular to the line through the points (2,5) and (-3,6).
	\\
	\solution 
\label{chapters/11/10/2/10}
See 
		\figref{fig:11/10/2/10}.
	\begin{figure}[H]
		\centering
 \includegraphics[width=0.75\columnwidth]{chapters/11/10/2/10/figs/fig.pdf}
		\caption{}
		\label{fig:11/10/2/10}
  	\end{figure}
The normal vector is
\begin{align}
\vec{n} =\myvec{2 \\5} -  \myvec{-3 \\ 6} 
=\myvec{
    5\\
    -1
}
\end{align}
Thus, the equation of the line is 
\begin{align}
\myvec{
    5 &-1
	}\brak{\vec{x} - \myvec{-3 \\5}}
= 0
\\
\implies 
\myvec{
    5 &-1
	}\vec{x} 
= -20
\end{align}

\item 
A line perpendicular to the line segment joining the points $\vec{P}(1,0)$ and $\vec{Q}(2,3)$ divides it in the ratio $1:n$. Find the equation of the line.
	\\
	\solution 
\label{chapters/11/10/2/11}
The direction vector of 
$PQ$ is 
\begin{align}
     \vec{Q
 }-  \vec{P
 }
=
     \myvec{
  1\\
  3
 }
\end{align}
 Using section formula, 
 \begin{align}
	 \vec{R}=\frac{\vec{Q}+n\vec{P}}{1+n}
\end{align}
is the point of intersection.
The 
equation of the desired line  is
\begin{align}
	\vec{m}^{\top}\brak{\vec{x}-\vec{R}}=0
\\
\implies 
	   \myvec{
		   1 &  3}\vec{x}
	   &= \myvec{
  1\ 3}\myvec{
  \frac{2+n}{1+n}\\
  \frac{3}{1+n}} 
  \\
	=	  \frac{11+n}{1+n} 
\end{align}
\iffalse
See
		\figref{fig:11/10/2/11}.
	\begin{figure}[H]
		\centering
 \includegraphics[width=0.75\columnwidth]{chapters/11/10/2/11/figs/linefig.pdf}
		\caption{}
		\label{fig:11/10/2/11}
  	\end{figure}
	\fi

\item Find the equation of a line that cuts off equal intercepts on the coordinate axes and passes through the point $(2,3)$.  
	\\
\solution 
\label{chapters/11/10/2/12}
Let $(a,0)$  and  $(0,a)$ be the intercept points. 
\begin{align}
\vec{m} 
        &=   \myvec{
		a \\
		0 
		} - \myvec{
		   0 \\
		   a
		}  
        		  \equiv \myvec{
                           1 \\
			   -1 
		         } 
			 \\
			 \implies
\vec{n} &=  \myvec{
		     1 \\
		     1
	     } 
\end{align}
and 
the equation of the  line is
\begin{align}
	\myvec { 1 & 1 } \brak{ \vec{ x  - \myvec{ 2 \\
                                   3
			     }
		}}  &= 0  \\
\implies		\myvec{ 1 & 1} \vec{x}  &= 5 
        \label{eq:11/10/2/12/1}
\end{align}
See  \figref{fig:11/10/2/12/Fig1}.
\begin{figure}[H]
	\begin{center}
		\includegraphics[width=0.75\columnwidth]{chapters/11/10/2/12/figs/fig.pdf}
	\end{center}
\caption{}
\label{fig:11/10/2/12/Fig1}
\end{figure}


\item 
Find the equation of a line passing trough a point (2,2) and cutting off intercepts on the axes whose sum is 9.
	\\
	\solution 
\label{chapters/11/10/2/13}
Let  the intercept points be
\begin{align}
{\vec{P}}=\myvec{
  a\\
  0}
 , {\vec{Q}}=\myvec{
  0\\
  b}
  \text{ and }
   {\vec{R}}=\myvec{
  2\\
  2}
\end{align}
be the given point.  
Forming the collinearity matrix from 
		\eqref{prop:lin-dep-rank},
\begin{align}
	\myvec{ \vec{P}-\vec{Q} &\vec{P}-\vec{R}} 
	=
	 \myvec{
  a & a-2\\
  -b & -2
 }
\end{align}
which is singular if 
\begin{align}
 ab -2\brak{a+b} = 0
 \implies ab = 18
		\label{eq:11/10/2/13-a+b}
		\\
\because  a + b = 9.
\end{align}
$\therefore a,b$
are the roots of
\begin{align}
	x^2 -9x +18 = 0.
\end{align}
yielding
\begin{align}
	\myvec{a \\ b} = \myvec{6 \\ 3}, \myvec{3\\6}
\end{align}
Since 
\begin{align}
	\vec{m} = \myvec{a \\ -b},
	\vec{n} = \myvec{b \\ a} \equiv \myvec{1 \\ 2}, \myvec{2\\1}
\end{align}
Thus, the possible equations of the line are 
\begin{align}
\myvec{1 & 2}\vec{x} = 6
	\\
	\myvec{2&1}\vec{x} = 6
\end{align}
		See \figref{fig:11/10/2/13}.
	\begin{figure}[H]
		\centering
 \includegraphics[width=0.75\columnwidth]{chapters/11/10/2/13/figs/fig.pdf}
		\caption{}
		\label{fig:11/10/2/13}
  	\end{figure}

\item 
	Find the equation of the line through the point (0,2) making an angle $\frac{2\pi}{3}$ with the positive X-axis. Also find the equation of the line parallel to it and crossing the Y-axis at a distance of 2 units below the origin.
	\\
	\solution
\label{chapters/11/10/2/14}
The equation of the first line is 
\begin{align}
	\myvec{\sqrt{3} &1}\myvec{\vec{x}-\myvec{0\\2}}&=0
	\\
	\implies 
	\myvec{\sqrt{3}&1}
	\vec{x}&=2
\end{align}
The equation of the second line is 
\begin{align}
	\myvec{\sqrt{3} &1}\myvec{\vec{x}-\myvec{0\\-2}}&=0
	\\
	\implies 
	\myvec{\sqrt{3}&1}
\vec{x}=-2
\end{align}
See
		\figref{fig:11/10/2/14}.
	\begin{figure}[H]
		\centering
 \includegraphics[width=0.75\columnwidth]{chapters/11/10/2/14/figs/fig.pdf}
		\caption{}
		\label{fig:11/10/2/14}
  	\end{figure}

\item 
	The perpendicular from the origin to a line meets it at the point $(-2,9)$. Find the equation of the line.
\label{chapters/11/10/2/15}
	\\
	\solution
It is obvious that the normal vector to the line is 
\begin{align}
\vec{n} =\myvec{2 \\ -9} -\vec{0} 
=\myvec{2 \\ -9}
\end{align}
Hence, the equation of the line is 
\begin{align}
	\myvec{2 & -9}\brak{\vec{x} - \myvec{2 \\ -9}}&= 0
	\\
	\implies 
	\myvec{2 & -9}\vec{x} &= 85
\end{align}
See 
		\figref{fig:11/10/2/15}.
	\begin{figure}[H]
		\centering
 \includegraphics[width=0.75\columnwidth]{chapters/11/10/2/15/figs/fig.pdf}
		\caption{}
		\label{fig:11/10/2/15}
  	\end{figure}

\item 
$P(a,b)$ is the mid-point of the line segment between axes. Show that the equation of the line is $\frac{x}{a}+\frac{y}{b}=2$
\label{chapters/11/10/2/18}
\\
\solution
From \probref{chapters/11/10/2/13},
\begin{align}
	\vec{n} = \myvec{b \\ a}
\\	
\implies	\myvec{b & a} \brak{\vec{x}-\myvec{a\\b}} &= 0\\
	\text{or, }	\myvec{b & a}\vec{x} &= 2ab.
\end{align}
is the desired line equation.



\item Point $\vec{R}\brak{h, k}$ divides a line segment between the axes in the ratio 1: 2. Find the equation of the line.
\label{chapters/11/10/2/19}
	\\
	\solution 
Choosing the intercept points in \probref{chapters/11/10/2/13},
\begin{align}
\vec{R} &= \frac{2\vec{A} + \vec{B}}{3} 
\implies
\myvec{h\\k} = \frac{1}{3}\myvec{2a\\b} \\
	\text{or, }
\myvec{b\\a} 
	&= \vec{n}  \equiv \myvec{2k\\h}
\end{align}
%
Thus, the equation of the line is given by,
\begin{align}
\myvec{2k&h}\vec{x} = \myvec{2k&h}\myvec{h\\k}= 3hk
\end{align}





\item Find the equation of the line  parallel to the line 3x-4y+2=0 and passing through the point (-2,3).
\label{chapters/11/10/3/7}
\\
\solution 
\begin{align}
	\myvec{3&-4}\vec{x}=\myvec{3&-4}\myvec{-2\\3}
	=-18 
\end{align}
is the required equation of the line.

\item Find the equation of line perpendicular to the line $x-7y+5=0$ and having $x$ intercept $3$\\
\label{chapters/11/10/3/8}
\solution
The desired equation is
		\begin{align}
			\myvec{7 & 1}\brak{\vec{x}-\myvec{3\\0}} &=0\\
		\implies 	\myvec{7 & 1}\vec{x} &= 21
		\end{align}
		\iffalse
		See 
\figref{fig:chapters/11/10/3/8/Fig1}.
		\begin{figure}[H]
\begin{center}
\includegraphics[width=0.75\columnwidth]{chapters/11/10/3/8/figs/fig.pdf}
\end{center}
\caption{}
\label{fig:chapters/11/10/3/8/Fig1}
\end{figure}
\fi

\item Prove that the line through the point $(x_1,y_1)$ and parallel to the line $Ax+By+C=0$ is $A(x-x_1)+B(y-y_1)=0$.
\label{chapters/11/10/3/11}
\\
\solution
The equation of the desired line is
\begin{align}
	\myvec{A &B}\brak{\vec{x}-\myvec{x_1\\y_1}}&=0\\
	\implies 
	\myvec{A &B}\vec{x} &= Ax_1+By_1
\end{align}

	\item Find the equation of the line passing through the point $\brak{1,2,-4}$ and perpendicular to the two lines
\begin{align}
	\frac{x-8}{3}=\frac{y+19}{-16}=\frac{z-10}{7} \text{ and }\\ \frac{x-15}{3}=\frac{y-29}{8}=\frac{z-5}{-5} 
\end{align}
    \solution
		The direction vector of the desired line 
is given by 
\begin{align*}
	\myvec{3 & -16 & 7\\3 & 8 & -5}\vec{m} = 0
	\xleftrightarrow[]{R_2\leftarrow R_2-R_1}
 	\myvec{3 & -16 & 7\\0 & 24 & -12}
	\\
	\xleftrightarrow[]{R_1\leftarrow R_1+\frac{2}{3}R_2}
	\myvec{3 & 0 & -1\\0 & 24 & -12}
	\xleftrightarrow[]{R_2\leftarrow R_2/12}
	\myvec{3 & 0 & -1\\0 & 2 & -1}
\end{align*}
yielding
\begin{align}
	\vec{m} = \myvec{2\\3\\6}
\end{align}
Hence the vector equation of the line passing through $\brak{1,2,-4}$ is,
\begin{align}
	\vec{x} = \myvec{1\\2\\-4} + \kappa \myvec{2\\3\\6}
\end{align}



	\item  Find the vector equation of the line passing through $\myvec{1&2&3}^{\top}$ and parallel to the planes $\myvec{1&-1&2}\vec{x} = 5$ and $\myvec{3&1&1}\vec{x} = 6$.  
		\\
    \solution
		The direction vector of the line  is given by 
\begin{align*}
 \myvec{1&-1&2 \\ 3&1&1}\vec{m} = 0
 \xleftrightarrow[]{R_2\rightarrow -\frac{3}{4}{R_1} + \frac{1}{4}{R_2}} \myvec{1&-1&2 \\ 0&1&-\frac{5}{4}}\\
   \myvec{1&-1&2 \\ 0&1&-\frac{5}{4}} \xleftrightarrow[]{R_1\rightarrow {R_1} + {R_2}} \myvec{1&0&\frac{3}{4} \\ 0&1&-\frac{5}{4}}
   \\
 \implies \vec{m} =\myvec{-3\\5\\4}
\end{align*}
$\therefore$ the equation of the line is
\begin{align}
    \vec{x} = \myvec{1\\2\\3} + \lambda\myvec{-3\\5\\4} 
\end{align}


	\item
 Two lines passing through the point (2,3) intersect each other at an angle of $60\degree$. If slope of one line is 2, find the equation of the other line.
\label{chapters/11/10/3/12}
 \\
 \solution
		Using the scalar product
\begin{align}
  \cos{60\degree}=
\frac{1}{2}=\frac{\myvec{
        1&2
    }\myvec{
        1\\m
    }}{\sqrt{5}\sqrt{m^2+1}}\\
\implies 11m^2+16m-1=0\\
   or, m=\frac{-8\pm5\sqrt{3}}{11}    
\end{align}
So, the desired equation of the line is
\begin{align}
\myvec{
    \frac{-8\pm5\sqrt{3}}{11}&-1
}\vec{x} 
	&=
\myvec{
    \frac{-8\pm5\sqrt{3}}{11}&-1
}
\myvec{
    2\\3}
    \\
	&=\frac{-49\pm16\sqrt{3}}{11}
\end{align}
\iffalse
See 
    \figref{fig:11.10.3.12}.
\begin{figure}[H]
    \centering
    \includegraphics[width=0.75\columnwidth]{chapters/11/10/3/12/fig/asgnt1.png}
    \caption{}
    \label{fig:11.10.3.12}
\end{figure}
\fi


\item
Find the value of $p$ so that the three lines $3x+y-2=0,px+2y-3=0$ and $2x-y-3=0$ may intersect at one point.
\label{11.10.4.9}
\\
\solution
Performing row operations
on the matrix
\begin{align*}  
\myvec{
    3 &1&-2 \\
     p&2&-3\\
     2&-1&-3
}
\xrightarrow[R_3=3R_3-2R_1]{R_2=3R_2-pR_1}&\myvec{
    3&1&-2\\
     $0$&6-p&-9+2p\\
     0&-5&-5}\\
 \xrightarrow{R_3=R_3(6-p)+5R_2}&\myvec{
    3&1&-2\\
     0&6-p&-9+2p\\
     0&0&-75+15p}
     \\
  \implies 
    p=5
\end{align*}
    See \figref{fig:11.10.4.9}.
\begin{figure}[H]
    \centering
    \includegraphics[width=0.75\columnwidth]{chapters/11/10/4/9/fig/11.10.4.9.png}
    \caption{}
    \label{fig:11.10.4.9}
\end{figure}


 \item The perpendicular from the origin to the line $y=mx+c$ meets it at the point $(-1,2)$. Find the values of m and c.
 \label{11.10.3.15}
	 \\
 \solution
 From \probref{chapters/11/10/2/15},
\begin{align}
	\vec{n} = \myvec{-1\\2} \implies m = \frac{1}{2}
\end{align}
Also, from the given equation of the line and the given point, 
\begin{align}
	c = \myvec{-m & 1}\myvec{-1\\2} = 
\frac{5}{2}  
\end{align}
\iffalse
 See \figref{fig:pic}.
\begin{figure}[H]
 \centering
\includegraphics[width=0.75\columnwidth]{chapters/11/10/3/15/figs/graph.jpg}
 \caption{Graph}
 \label{fig:pic}
\end{figure}
\fi

\item Find the equation of the lines through the point (3, 2) which make an angle of $45\degree$  with the line $x – 2y = 3$.
\label{chapters/11/10/4/11}\\
\solution
Following the approach in \probref{chapters/11/10/3/12},
\begin{align}
\cos45\degree =
\frac{1}{\sqrt{2}} = \frac{\myvec{2 & 1} \myvec{1\\m}}{\norm{\myvec{2\\1}}\norm{\myvec{1\\m}}}
\\
\implies 
 3m^2 - 8m -3 = 0
 \\
\text{or, }
m= - \frac{1}{3}, 3
\end{align} 
Thus, the desired equations are 
\begin{align}
	\myvec{1&3}\cbrak{\vec{x}-\myvec{3\\2}}&=0\\
 \implies 	\myvec{1 & 3}\vec{x} &= 9
\end{align}
and 
\begin{align}
	\myvec{3&-1}\cbrak{\vec{x}-\myvec{3\\2}}&=0\\
		\implies 	\myvec{3 & -1}\vec{x} &= 7
\end{align}
See
\figref{fig:chapters/11/10/4/11/figs/strline.jpg}.
\begin{figure}[H]
\centering
\includegraphics[width=0.75\columnwidth]{chapters/11/10/4/11/figs/fig.pdf}
\caption{}
\label{fig:chapters/11/10/4/11/figs/strline.jpg}
\end{figure}

\item Consider the following population and year graph. Find the slope of the line AB and using it, find what will be the population in the year 2010.
\\
\begin{figure}[!ht]
\centering
\includegraphics[width = \columnwidth]{chapters/11/10/1/14/figs/fig.png}
\caption{}
\label{fig:chapters/11/10/1/14/1}
\end{figure}
\solution
The direction vector of the line in \figref{fig:chapters/11/10/1/14/1} is
\begin{align}
\vec{m} = \vec{B} - \vec{A}
= \myvec{2 \\ 1}
\\
\implies \vec{n}
= \myvec{1 \\ -2}
\end{align}
 The equation of the line is then given by 
\begin{align}
\vec{n}^{\top} (\vec{x} -\vec{A}) &= 0 \\
\implies 
\myvec{1& -2} \vec{x} &= 1801
\\
\implies  \myvec{1&-2} \myvec{2010\\y} &= 1801 \\
\implies y &= \frac{209}{2}
\end{align}





\item  Find the vector equation of the line which is parallel to the vector $3\hat{i}-2\hat{j}+6\hat{k}$ and which passes through the point $(1,-2,3)$.
\item Find the equations of the two lines through the origin which intersect the line $ \dfrac{x-3}{2}=\dfrac{y-3}{1}=\dfrac{z}{1}$ at angles of  $\dfrac{\pi}{3}$each.
\item Find the equations of the line passing through the point $(3,0,1)$ and parallel to the planes $x+2y=0$ and $3y-z=0.$
\item The vector equation of the line $\dfrac{x-5}{3}=\dfrac{y+4}{7}=\dfrac{z-6}{2}$ is \noindent\rule{2cm}{0.4pt}. 
\item The vector equation of the line through the points $(3,4,-7)$ and $(1,-1,6)$ is \noindent\rule{2cm}{0.4pt}.
\item the unit vector normal to the plane $x+2y+3z-6=0$ is $\dfrac{1}{\sqrt{14}}\hat{i} + \dfrac{2}{\sqrt{14}}\hat{j} + \dfrac{3}{\sqrt{14}}\hat{k}$.
\item The vector equation of the line $\dfrac{x-5}{3}=\dfrac{y+4}{7}=\dfrac{z-6}{2}$ is
$$\overrightarrow{r}=5\hat{i}-4\hat{j}+6\hat{k}+\lambda(3\hat{i}+7\hat{j}+2\hat{k}).$$
\item The equation of a line, which is parallel to $2\hat{i}+\hat{j}+3\hat{k}$ and which passes through the point $(5,-2,4)$ is $\dfrac{x-5}{2}=\dfrac{y+2}{-1}=\dfrac{z-4}{3}$.
\item  Point $\vec{P}(0,2)$ is the point of intersection of $y$-axis and perpendicular bisector of line segment joining the points $\vec{A}(-1,1) \text{ and } \vec{B}(3,3)$

\item Prove that the line through A$(0,-1,-1)$ and B$(4,5,1)$ intersects the line through C$(3,9,4)$ and D$(-4,4,4)$.
\item Show the lines
$$\frac{x-1}{2}=\frac{y-2}{3}=\frac{z-3}{4}$$
$$\text{ and } \frac{x-4}{5}=\frac{y-1}{2}=z  \text{ intersect }.$$
 Also, find their point of intersection.
\item The area of the region bounded by the curve $y = x + 1$ and the lines $x = 2\text{ and }x = 3$ is
\begin{enumerate}
\item $\frac{7}{2}$ sq units
\item $\frac{9}{2}$ sq units
\item $\frac{11}{2}$ sq units
\item $\frac{13}{2}$ sq units
\end{enumerate}   
\item The area of the region bounded by the curve $x = 2 + 3$ and the $y$ lines $y = 1\text{ and }y = - 1$ is
\begin{enumerate}
\item 4 sq units 
\item $\frac{3}{2}$ sq units
\item 6 sq units
\item 8 sq units
\end{enumerate}
\item Compute the area bounded by the line $x + 2y = 2$, $y - x = 1\text{ and }2x + y = 7$.
\item Find the area bounded by the lines $y = 4x + 5$, $y = 5 - x\text{ and }4y = x + 5$.
\end{enumerate}

\subsection{Parallel}
\begin{enumerate}[label=\thesubsection.\arabic*,ref=\thesubsection.\theenumi]
	\item  Find the vector equation of the line passing through $\myvec{1&2&3}^{\top}$ and parallel to the planes $\myvec{1&-1&2}\vec{x} = 5$ and $\myvec{3&1&1}\vec{x} = 6$.  
		\\
    \solution
		The direction vector of the line  is given by 
\begin{align*}
 \myvec{1&-1&2 \\ 3&1&1}\vec{m} = 0
 \xleftrightarrow[]{R_2\rightarrow -\frac{3}{4}{R_1} + \frac{1}{4}{R_2}} \myvec{1&-1&2 \\ 0&1&-\frac{5}{4}}\\
   \myvec{1&-1&2 \\ 0&1&-\frac{5}{4}} \xleftrightarrow[]{R_1\rightarrow {R_1} + {R_2}} \myvec{1&0&\frac{3}{4} \\ 0&1&-\frac{5}{4}}
   \\
 \implies \vec{m} =\myvec{-3\\5\\4}
\end{align*}
$\therefore$ the equation of the line is
\begin{align}
    \vec{x} = \myvec{1\\2\\3} + \lambda\myvec{-3\\5\\4} 
\end{align}


	\item Find the equation of the plane with an intercept 3 on the Y-axis and parallel to ZOX-Plane.\\
    \solution
		The normal vector to the ZOX plane is
\begin{align} 
\vec{n} = \myvec{0\\1\\0}.
\end{align}
Since, Y-axis has the intercept 3, the desired plane passes through the point
\begin{align}
\vec{P}=\myvec{0\\3\\0}.
\end{align}
Thus, the equation of the plane is given by,
\begin{align}
	\vec{n}^\top \brak{\vec{x}-\vec{P}} &= 0\\
	\implies \myvec{0&1&0} \vec{x}&= 3
\end{align}
See Fig. 
     \ref{fig:chapters/12/11/3/8/1}.
\begin{figure}[H]
  \centering
   \includegraphics[width=0.75\columnwidth]{chapters/12/11/3/8/figs/fig.pdf}
    \caption{}
     \label{fig:chapters/12/11/3/8/1}
     \end{figure}  

\item Prove that the line through the point $(x_1,y_1)$ and parallel to the line $Ax+By+C=0$ is $A(x-x_1)+B(y-y_1)=0$.
\label{chapters/11/10/3/11}
\\
\solution
The equation of the desired line is
\begin{align}
	\myvec{A &B}\brak{\vec{x}-\myvec{x_1\\y_1}}&=0\\
	\implies 
	\myvec{A &B}\vec{x} &= Ax_1+By_1
\end{align}

\item Find the equation of the line  parallel to the line 3x-4y+2=0 and passing through the point (-2,3).
\label{chapters/11/10/3/7}
\\
\solution 
\begin{align}
	\myvec{3&-4}\vec{x}=\myvec{3&-4}\myvec{-2\\3}
	=-18 
\end{align}
is the required equation of the line.

\item 
	Find the equation of the line through the point (0,2) making an angle $\frac{2\pi}{3}$ with the positive X-axis. Also find the equation of the line parallel to it and crossing the Y-axis at a distance of 2 units below the origin.
	\\
	\solution
\label{chapters/11/10/2/14}
The equation of the first line is 
\begin{align}
	\myvec{\sqrt{3} &1}\myvec{\vec{x}-\myvec{0\\2}}&=0
	\\
	\implies 
	\myvec{\sqrt{3}&1}
	\vec{x}&=2
\end{align}
The equation of the second line is 
\begin{align}
	\myvec{\sqrt{3} &1}\myvec{\vec{x}-\myvec{0\\-2}}&=0
	\\
	\implies 
	\myvec{\sqrt{3}&1}
\vec{x}=-2
\end{align}
See
		\figref{fig:11/10/2/14}.
	\begin{figure}[H]
		\centering
 \includegraphics[width=0.75\columnwidth]{chapters/11/10/2/14/figs/fig.pdf}
		\caption{}
		\label{fig:11/10/2/14}
  	\end{figure}

\item  Find the vector equation of the line which is parallel to the vector $3\hat{i}-2\hat{j}+6\hat{k}$ and which passes through the point $(1,-2,3)$.
\item Find the equations of the line passing through the point $(3,0,1)$ and parallel to the planes $x+2y=0$ and $3y-z=0.$
\item The equation of a line, which is parallel to $2\hat{i}+\hat{j}+3\hat{k}$ and which passes through the point $(5,-2,4)$ is $\dfrac{x-5}{2}=\dfrac{y+2}{-1}=\dfrac{z-4}{3}$.
\item The value of $\lambda$ for which the vectors $3\hat{i}-6\hat{j}+\hat{k}$ $\text{and}$,  $2\hat{i}-4\hat{j}$+$\lambda\hat{k}$ are parallel is
	\begin{enumerate}
			\setlength{\itemsep}{1ex}
\item $\frac{2}{3}$
\item $\frac{3}{2}$
\item $\frac{5}{2}$
\item $\frac{2}{5}$
	\end{enumerate}	
\end{enumerate}

\subsection{Perpendicular}
\begin{enumerate}[label=\thesubsection.\arabic*,ref=\thesubsection.\theenumi]
\item  Reduce the following equations into normal form. Find their perpendicular distances from the origin and angle between perpendicular and the positive $x$-axis.
\label{chapters/11/10/3/3}
\begin{enumerate}
	\item $x-\sqrt{3}y+8=0$ 
	\item $y-2=0$
	\item $x-y=4$
\end{enumerate}
\solution
  See \tabref{tab:11/10/3/3}.
			\eqref{eq:PQ-final} was used for computing the distance from the origin.
			\begin{table}[H]
  \centering
  \begin{tabular}{|c|c|c|c|c|}
    \hline
    & $\vec{n}$ & Angle & $c$& Distance \\
    \hline
    a) & $\myvec{1 \\ -\sqrt{3}}$ & $\tan^{-1}(-\sqrt{3}) = \frac{2\pi}{3}$ &-8 & 4 \\
    \hline
    b) & $\myvec{0 \\ 1}$ & $\tan^{-1}\infty = \frac{\pi}{2}$ &2 & 2 \\
    \hline
    c) & $\myvec{1 \\ -1}$ & $\tan^{-1}(-1) = \frac{3\pi}{4}$ &4 & $2\sqrt{2}$ \\
    \hline
  \end{tabular}
  \caption{}
  \label{tab:11/10/3/3}
\end{table}


 \item  In each of the following cases, determine the direction cosines of the normal to
the plane and the distance from the origin.
\begin{enumerate}
	\item $z=2$ 
	\item $x + y + z = 1$
	\item $2x + 3y – z = 5$
	\item $5y + 8 = 0$
\end{enumerate}
    \solution
		  See 
  \tabref{tab:12/11/3/1}.
			\eqref{eq:PQ-final} was used for computing the distance from the origin.
			\begin{table}[H]
  \centering
  \begin{tabular}{|c|c|c|c|}
    \hline
    & $\vec{n}$ & $c$ & Distance \\
    \hline
    a) &		\myvec{0\\0\\1}  &2  & 2 \\
    \hline
    b) & $\myvec{1\\1\\1}$ & 1 & $\frac{1}{\sqrt{3}}$ \\
    \hline
    c) & $\myvec{2\\3\\-1}$ & 5 & $\frac{5}{\sqrt{14}}$ \\
    \hline
    d) & $\myvec{0\\-5\\0}$ & 8 & $\frac{8}{5}$ \\
    \hline
  \end{tabular}
  \caption{}
  \label{tab:12/11/3/1}
\end{table}
 


\item Find the distance of the point $(-1,1)$ from the line $12\brak{x+6} = 5\brak{y-2}$. 
\label{chapters/11/10/3/4}
	\\
\solution 
\begin{align}
		\vec{n} = \myvec{
	  12 \\
	  -5 
	  } ,   c = -82 
	  \\
	  \implies 
	d 
	= \frac{\abs{  \myvec{12 & -5 }\myvec{-1 \\ 1}-\brak{-82} }}{\sqrt{12^2+\brak{-5}^2}} 	
	= 5
\end{align}
\iffalse
See \figref{fig:11/10/3/4/Fig1}.
\begin{figure}[H]
	\begin{center}
		\includegraphics[width=0.75\columnwidth]{chapters/11/10/3/4/figs/problem4.pdf}
	\end{center}
\caption{}
\label{fig:11/10/3/4/Fig1}
\end{figure}
\fi

\item Find the coordinates of the foot of the perpendicular from $(-1, 3)$ to the line $3x-4y-16=0$.  
\label{chapters/11/10/3/14}
\\
\solution
Substituting
\begin{align}
 \vec{P}=\myvec{
-1\\
3
},
\vec{n}=\myvec{
3\\
-4
}, c=16
\end{align}
in 
	\eqref{eq:11/10/3/4/foot_of_perpendicular},
the desired foot of the perpendicular is then given by 
\begin{align}
\myvec{4&3\\3&-4}\vec{Q}=\myvec{\myvec{4&3}\myvec{-1\\3}\\16}
=\myvec{5\\16}  
\\
\implies
  \myvec{
   4 &  3  & 5\\
   3 & -4  & 16} 
  \xleftrightarrow[]{R_2=R_2-\frac{3}{4}R_1}
  \myvec{
  4 & 3 & 5\\
  0 & \frac{-25}{4} & \frac{49}{4}} 
\\
  \xleftrightarrow{R_2=\frac{-4}{25}}
  \myvec{
  4 & 3 & 5\\
  0 & 1 & \frac{-49}{25}}
  \xleftrightarrow{R_1=\frac{1}{4}R_1}
  \myvec{
  1 & \frac{3}{4} & \frac{5}{4}\\[1ex]
  0 & 1 & \frac{-49}{25}}
\\
  \xleftrightarrow{R_1=R_1-\frac{3}{4}R_2}
  \myvec{
  1 & 0 & \frac{68}{25}\\[1ex]
  0 & 1 & \frac{-49}{25}}          
\implies \vec{Q}=\myvec{
\frac{68}{25}\\[1ex]
\frac{-49}{25}
}
\end{align}
See 
\figref{fig:chapters/11/10/3/14/Fig}.
\begin{figure}[H]
	\begin{center} 
	    \includegraphics[width=0.75\columnwidth]{chapters/11/10/3/14/figs/fig.pdf}
	\end{center}
\caption{}
\label{fig:chapters/11/10/3/14/Fig}
\end{figure}

\item  If ${p}$ and ${q}$ are the lengths of perpendiculars from the origin to the lines ${x}\cos\theta - {y}\sin\theta =  {k}\cos2\theta$ and ${x}\sec\theta + {y}\cosec\theta = {k}$, respectively, prove that ${p}^2 + 4{q}^2 = {k}^2$
\label{chapters/11/10/3/16}
\\
\solution
The line parameters are
\begin{align}
    \vec{n}_1 = \myvec{\cos\theta \\ -\sin\theta},  {c}_1 &= {k}\cos2\theta\\
    \vec{n}_2 = \myvec{\sin\theta \\ \cos\theta},  {c}_2 &= \frac{1}{2}{k}\sin2\theta
\end{align}
			From \eqref{eq:PQ-final},
\begin{align}
    {p} &= \frac{\abs{  \vec{n}_1^{\top}\vec{x}-{c}_1 }}{\norm{\vec{n}_1}} 
    = \abs{{k}\cos2\theta} \\
     {q} &= \frac{\abs{  \vec{n}_2^{\top}\vec{x}-{c}_2 }}{\norm{\vec{n}_2}} 
    = \abs{ \frac{1}{2}{k}\sin2\theta}
    \\
	\implies
	{p}^2 + 4{q}^2 & 
= {k}^2
\end{align}

\item In the triangle $ABC$ with vertices $\vec{A} \brak{2, 3}$, $\vec{B} \brak{4, –1}$ and $\vec{C} \brak{1, 2}$, find the equation and length of altitude from the vertex $\vec{A}$.
\label{chapters/11/10/3/17}
\\
\solution
\begin{enumerate}
\item The normal vector of the altitude from $\vec{A}$ is,
\begin{align}
\vec{m}_{BC}
= \myvec{1\\-1},
\because \vec{n}_{BC} &= \myvec{1\\1}.
\end{align}
The equation of the desired altitude  is given by
\begin{align}
\vec{m}_{BC}^{\top}\vec{x} &=\vec{m}_{BC}^{\top}\vec{A}\\
\implies \myvec{1&-1}\vec{x} &= -1
\end{align}
	\item
The equation of line $BC$ is given by,
\begin{align}
{\vec{n}^{\top}_{BC}}\vec{x} &= {\vec{n}^{\top}_{BC}}\vec{B}\\
\implies \myvec{1&1}\vec{x}  &= 3
\end{align}
			From \eqref{eq:PQ-final},
the length of the desired altitude is 
\begin{align}
d =  \sqrt{2}
\end{align}

\end{enumerate}
\iffalse
See 
\figref{fig:chapters/11/10/3/17/1}.
\begin{figure}[H]
\centering
\includegraphics[width=0.75\columnwidth]{chapters/11/10/3/17/figs/fig.png}
\caption{}
\label{fig:chapters/11/10/3/17/1}
\end{figure}
\fi

\item If $p$ is the length of perpendicular from origin to the line whose intercepts on the axes are $a$ and $b$, then show that 
\begin{align}
	\frac{1}{p^2} = \frac{1}{a^2}+ \frac{1}{b^2}
\label{eq:11/10/3/18}
\end{align}
\label{chapters/11/10/3/18}
\\
\solution
Let the 
intercept points be
\begin{align}
\myvec{a\\0},\myvec{0\\b},
\because	\vec{n} = \myvec{b\\a},
\end{align}
The line equation is,
\begin{align}
\myvec{b & a}\brak{\vec{x} - \myvec{a\\0}} &= 0\\
\implies	\myvec{ b & a}\vec{x} &= ab
\end{align}
			From \eqref{eq:PQ-final},
the perpendicular distance from the origin  to the line is
\begin{align}
	p  
	&= \frac{ab}{\sqrt{a^2+b^2}}
	\implies 
\eqref{eq:11/10/3/18}
\end{align}

\item Find the points on the x-axis, whose distances from the line $\frac{x}{3}+\frac{y}{4}=1$ are 4 units.
\label{chapters/11/10/3/5}
	\\
	\solution
Let the desired point be
\begin{align}
	\vec{P} = x\vec{e}_{1} = \myvec{x\\0}
\end{align}
From the distance formula, 
\begin{align}
	d &= \frac{\abs{\vec{n}^\top\vec{P}-c}}{\norm{\vec{n}}}
	  = \frac{\abs{x\vec{n}^\top\vec{e}_{1}-c}}{\norm{\vec{n}}}
	  \\
	  \implies 
		x &= \frac{\pm d\norm{\vec{n}}+c}{\vec{n}^\top\vec{e}_{1}}
\end{align}
Substituting
\begin{align}
		\vec{n} = \myvec{4\\3} , c = 12,
	d = 4,
	\\
	x = 8,
	 -2
\end{align}
See \figref{fig:11/10/3/5/Fig1}.	
\begin{figure}[H]
	\begin{center} 
	    \includegraphics[width=0.75\columnwidth]{chapters/11/10/3/5/figs/fig.pdf}
	\end{center}
\caption{}
\label{fig:11/10/3/5/Fig1}
\end{figure}



\item What are the points on the y-axis whose distance from the line $\frac{x}{3}+\frac{y}{4}=1$ is 4 units.
\\
\solution
		Following the approach in \probref{chapters/11/10/3/5},
\begin{align}
		y = \frac{\pm d\norm{\vec{n}}+c}{\vec{n}^\top\vec{e}_{2}}
= \frac{32}{3}, \frac{-8}{3}.
\end{align}
\iffalse
See 
		\figref{fig:chapters/11/10/4/4/Figure}.
\begin{figure}[H]
\centering
\includegraphics[width=0.75\columnwidth]{chapters/11/10/4/4/figs/fig.png}
\caption{}
		\label{fig:chapters/11/10/4/4/Figure}
\end{figure}
\fi

\item Find perpendicular distance from the origin to the line joining the points $(\cos\theta,\sin\theta)$ and $(\cos\phi,\sin\phi)$.
\\
\solution
		The equation of the line is
\begin{align}
\myvec{\sin\phi-\sin\theta&\cos\theta-\cos\phi}\vec{x}&=\sin\brak{\phi-\theta}
\label{eq:chapters/11/10/4/5/1}
\end{align}
and from 
			\eqref{eq:PQ-final},
the distance is
\begin{align}
d
=\frac{\sin\brak{\phi-\theta}}{2\sin\brak{\frac{\phi-\theta}{2}}} = \cos\brak{\frac{\phi-\theta}{2}}
\label{eq:chapters/11/10/4/5/2}
\end{align}

\item Find the distance between parallel lines
\label{chapters/11/10/3/6}
\begin{enumerate}
	\item $15x+8y-34=0$ and  $15x+8y+31=0$ \\
	\item  $l(x+y)+p=0$ and  $l(x+y)-r=0$
\end{enumerate}
	\solution
	From \eqref{eq:parallel_lines}, the desired values are available in
  \tabref{tab:11/10/3/6}.
\begin{table}[H]
  \centering
  \begin{tabular}{|c|c|c|c|c|}
    \hline
    & $\vec{n}$ & $c_1$ & $c_2$ & $d$ \\
    \hline
    a) & $\myvec{15 \\ 8}$ & 34 & -31 & $\frac{65}{17}$ \\
    \hline
    b) & $\myvec{1 \\ 1}$ & $\frac{-p}{l}$ & $\frac{r}{l}$ & $\frac{\lvert p-r \rvert}{l\sqrt{2}}$ \\
    \hline
  \end{tabular}
  \caption{}
  \label{tab:11/10/3/6}
\end{table}

\item Find the equation of line which is equidistant from parallel lines $9x+6y-7=0$ and $3x+2y+6=0$.
\\
\solution
		Given
\begin{align}
	c_1 = \frac{7}{3},\,
c_2 = -6.
\end{align}
	From \eqref{eq:parallel_lines},
we need to find $c$ such that,
\begin{align}
	\abs{c-c_1} = \abs{c-c_2} \implies c = \frac{c_1+c_2}{2}
	 = -\frac{11}{6}.
\end{align}
Hence, the desired equation is
\begin{align}
	\myvec{3 & 2}\vec{x} &= -\frac{11}{6}
\end{align}
	See \figref{fig:chapters/11/10/4/21/1}.
\begin{figure}[H]
	\centering
	\includegraphics[width=0.75\columnwidth]{chapters/11/10/4/21/figs/fig.pdf}
	\caption{}
	\label{fig:chapters/11/10/4/21/1}
\end{figure}

	\item Prove that the products of the lengths of the perpendiculars drawn from the points $\myvec{\sqrt{a^2-b^2}& 0}^{\top}$ and $\myvec{-\sqrt{a^2-b^2} &0}^{\top}$ to the line $\frac{x}{a} \cos{\theta} + \frac{y}{b}\sin{\theta} =1 $ is $ b^2 $.
\\
    \solution 
		The input parameters for 
			\eqref{eq:PQ-final}
			are
\begin{align}
	\vec{n}=\myvec{\frac{\cos{\theta}}{a}  \\ \frac{\sin{\theta}}{b}},\,
  c = 1,\,
	\vec{P} =\pm \myvec{\sqrt{a^2-b^2}\\0} 
\end{align} 
The product of the distances is
\begin{align}
	d_1d_2 &=\frac{\abs{ \brak{\vec{n}^{\top} \vec{P}}^2 -  c^2 } }{\norm{\vec{n}}}
	=\frac{\abs{ \frac{\cos^2{\theta}\brak{a^2-b^2}}{a^2}- 1 }}{\frac{\cos^2{\theta}}{a^2} +\frac{\sin^2{\theta}}{b^2} }\\ 
	&= \frac{\brak{b^2 \cos^2{\theta} + a^2 \sin^2{\theta}}a^2 b^2}{\brak{b^2 \cos^2{\theta} + a^2 \sin^2{\theta}}a^2}
	= b^2
\end{align}

	\item The distance of the point $\vec{P}(2, 3)$ from the x-axis is

\begin{enumerate}
\item 2
\item 3
\item 1
\item 5 
\end{enumerate}

\item Find the foot of perpendicular from the point $(2,3,-8)$ to the line  $\dfrac{4-x}{2}=\dfrac{y}{6}=\dfrac{1-z}{3}$.Also, find the perpendicular distance from the given point to the line.
\item Find the distance of a point $(2,4,-1)$ from the line $$\frac{x+5}{1}=\frac{y+3}{4}=\frac{z-6}{-9}$$
\item Find the length and the foot of perpendicular from the point $ \brak{1,\dfrac{3}{2} ,2 }$ to the plane $2x-2y+4z+5=0.$
\item Show that the points $(\hat{i}-\hat{j}+3\hat{k})$ and $3(\hat{i}+\hat{j}+\hat{k})$ are equidistant from the plane $\overrightarrow{r} \cdot (5\hat{i}+2\hat{j}-7\hat{k})+9=0$ and lies on opposite side of it.
\item The distance of the plane $\overrightarrow{r} \cdot \brak{ \dfrac{2}{7}\hat{i}+\dfrac{3}{7}\hat{j}-\dfrac{6}{7}\hat{k}}=1$ from the origin is 
\begin{enumerate}
	\item 1
	\item 7
	\item $\dfrac{1}{7}$
	\item None of these	
\end{enumerate}
\item If the foot of perpendicular drawn from the origin to a plane is $(5,-3,-2)$, then the equation of plane is $\overrightarrow{r} \cdot (5\hat{i}-3\hat{j}-2\hat{k})=38.$
\item Find the equation of line  drawn perpendicular to the line $\frac{x}{4}+\frac{y}{6}=1$ through the point where it meets the y-axis \\
\solution
				The given line
parameters are
\begin{align}
		\vec{n} = \myvec{3\\2},\, c=12 ,\,
	\vec{m} =\myvec{-2 \\ 3}.
\end{align}
and the point on the y-axis is
\begin{align}
	\vec{A} =\myvec{0\\6}.
\end{align}
Thus, the equation of the desired line is 
\begin{align}
	\vec{m}^\top\brak{\vec{x}-\vec{A}}&=0\label{eq:chapters/11/10/4/7/5}
	\\
\implies
			\myvec{-2 & 3}\vec{x} &=-18
		\end{align}
		\iffalse
		See 
  \figref{fig:chapters/11/10/4/7/Figure}.
\begin{figure}[H]
\includegraphics[width=0.75\columnwidth]{chapters/11/10/4/7/figs/fig.png}
\caption{}
  \label{fig:chapters/11/10/4/7/Figure}
\end{figure}
\fi

\item Find the equation of line whose perpendicular distance from the origin is 5 units and the angle made by the perpendicular with the positive $x$-axis is $30\degree$.
\label{chapters/11/10/2/8}
\\
\solution
			From 
\eqref{eq:chapters/11/10/2/8-final},
		Thus, the equation of lines are
\begin{align}
	\myvec{\frac{\sqrt{3}}{2}& \frac{1}{2}}\vec{x}=\pm5
\end{align}
\iffalse
See 
\figref{fig:chapters/11/10/2/8/Fig1}.
\begin{figure}[H]
\begin{center}
\includegraphics[width=0.75\columnwidth]{chapters/11/10/2/8/figs/fig.pdf}
\end{center}
\caption{}
\label{fig:chapters/11/10/2/8/Fig1}
\end{figure}
\fi

\item 
	Find the equation of the line passing through  (-3,5) and perpendicular to the line through the points (2,5) and (-3,6).
	\\
	\solution 
\label{chapters/11/10/2/10}
See 
		\figref{fig:11/10/2/10}.
	\begin{figure}[H]
		\centering
 \includegraphics[width=0.75\columnwidth]{chapters/11/10/2/10/figs/fig.pdf}
		\caption{}
		\label{fig:11/10/2/10}
  	\end{figure}
The normal vector is
\begin{align}
\vec{n} =\myvec{2 \\5} -  \myvec{-3 \\ 6} 
=\myvec{
    5\\
    -1
}
\end{align}
Thus, the equation of the line is 
\begin{align}
\myvec{
    5 &-1
	}\brak{\vec{x} - \myvec{-3 \\5}}
= 0
\\
\implies 
\myvec{
    5 &-1
	}\vec{x} 
= -20
\end{align}

\item 
	The perpendicular from the origin to a line meets it at the point $(-2,9)$. Find the equation of the line.
\label{chapters/11/10/2/15}
	\\
	\solution
It is obvious that the normal vector to the line is 
\begin{align}
\vec{n} =\myvec{2 \\ -9} -\vec{0} 
=\myvec{2 \\ -9}
\end{align}
Hence, the equation of the line is 
\begin{align}
	\myvec{2 & -9}\brak{\vec{x} - \myvec{2 \\ -9}}&= 0
	\\
	\implies 
	\myvec{2 & -9}\vec{x} &= 85
\end{align}
See 
		\figref{fig:11/10/2/15}.
	\begin{figure}[H]
		\centering
 \includegraphics[width=0.75\columnwidth]{chapters/11/10/2/15/figs/fig.pdf}
		\caption{}
		\label{fig:11/10/2/15}
  	\end{figure}

\item Find the equation of line perpendicular to the line $x-7y+5=0$ and having $x$ intercept $3$\\
\label{chapters/11/10/3/8}
\solution
The desired equation is
		\begin{align}
			\myvec{7 & 1}\brak{\vec{x}-\myvec{3\\0}} &=0\\
		\implies 	\myvec{7 & 1}\vec{x} &= 21
		\end{align}
		\iffalse
		See 
\figref{fig:chapters/11/10/3/8/Fig1}.
		\begin{figure}[H]
\begin{center}
\includegraphics[width=0.75\columnwidth]{chapters/11/10/3/8/figs/fig.pdf}
\end{center}
\caption{}
\label{fig:chapters/11/10/3/8/Fig1}
\end{figure}
\fi

	\item Find the equation of the line passing through the point $\brak{1,2,-4}$ and perpendicular to the two lines
\begin{align}
	\frac{x-8}{3}=\frac{y+19}{-16}=\frac{z-10}{7} \text{ and }\\ \frac{x-15}{3}=\frac{y-29}{8}=\frac{z-5}{-5} 
\end{align}
    \solution
		The direction vector of the desired line 
is given by 
\begin{align*}
	\myvec{3 & -16 & 7\\3 & 8 & -5}\vec{m} = 0
	\xleftrightarrow[]{R_2\leftarrow R_2-R_1}
 	\myvec{3 & -16 & 7\\0 & 24 & -12}
	\\
	\xleftrightarrow[]{R_1\leftarrow R_1+\frac{2}{3}R_2}
	\myvec{3 & 0 & -1\\0 & 24 & -12}
	\xleftrightarrow[]{R_2\leftarrow R_2/12}
	\myvec{3 & 0 & -1\\0 & 2 & -1}
\end{align*}
yielding
\begin{align}
	\vec{m} = \myvec{2\\3\\6}
\end{align}
Hence the vector equation of the line passing through $\brak{1,2,-4}$ is,
\begin{align}
	\vec{x} = \myvec{1\\2\\-4} + \kappa \myvec{2\\3\\6}
\end{align}



 \item The perpendicular from the origin to the line $y=mx+c$ meets it at the point $(-1,2)$. Find the values of m and c.
 \label{11.10.3.15}
	 \\
 \solution
 From \probref{chapters/11/10/2/15},
\begin{align}
	\vec{n} = \myvec{-1\\2} \implies m = \frac{1}{2}
\end{align}
Also, from the given equation of the line and the given point, 
\begin{align}
	c = \myvec{-m & 1}\myvec{-1\\2} = 
\frac{5}{2}  
\end{align}
\iffalse
 See \figref{fig:pic}.
\begin{figure}[H]
 \centering
\includegraphics[width=0.75\columnwidth]{chapters/11/10/3/15/figs/graph.jpg}
 \caption{Graph}
 \label{fig:pic}
\end{figure}
\fi

\item  Point $\vec{P}(0,2)$ is the point of intersection of $y$-axis and perpendicular bisector of line segment joining the points $\vec{A}(-1,1) \text{ and } \vec{B}(3,3)$
\item 
A line perpendicular to the line segment joining the points $\vec{P}(1,0)$ and $\vec{Q}(2,3)$ divides it in the ratio $1:n$. Find the equation of the line.
	\\
	\solution 
\label{chapters/11/10/2/11}
The direction vector of 
$PQ$ is 
\begin{align}
     \vec{Q
 }-  \vec{P
 }
=
     \myvec{
  1\\
  3
 }
\end{align}
 Using section formula, 
 \begin{align}
	 \vec{R}=\frac{\vec{Q}+n\vec{P}}{1+n}
\end{align}
is the point of intersection.
The 
equation of the desired line  is
\begin{align}
	\vec{m}^{\top}\brak{\vec{x}-\vec{R}}=0
\\
\implies 
	   \myvec{
		   1 &  3}\vec{x}
	   &= \myvec{
  1\ 3}\myvec{
  \frac{2+n}{1+n}\\
  \frac{3}{1+n}} 
  \\
	=	  \frac{11+n}{1+n} 
\end{align}
\iffalse
See
		\figref{fig:11/10/2/11}.
	\begin{figure}[H]
		\centering
 \includegraphics[width=0.75\columnwidth]{chapters/11/10/2/11/figs/linefig.pdf}
		\caption{}
		\label{fig:11/10/2/11}
  	\end{figure}
	\fi

\item Find the vector equation of a plane which is at a distance of 7 units from the origin and normal to the vector $3\hat{i}+5\hat{j}-6\hat{k}$.
	\\
    \solution
		From the given information, 
\begin{align} 
\vec{n}=\myvec{3\\5\\-6},\,
	d=\frac{\abs{c}}{\norm{\vec{n}}} = 7
	\\
	\implies
c =\pm7\sqrt{70}
\end{align}	  

\item Find the equation of a plane which is at a distance 3$\sqrt{3}$ units from origin and the normal to which is equally inclined to coordinate axis.
\item If the line drawn from the point $(-2,-1,-3)$ meets a plane at right angle at the point $(1,-3,3)$, find the equation of the plane.
\item O is the origin and A is $(a,b,c)$.Find the direction cosines of the line OA and the equation of plane through A at right angle at OA.
\item Two systems of rectangular axis have the same origin. If a plane cuts them at distances $a,b,c$ and $a^{\prime},b^{\prime},c^{\prime}$, respectively, from the origin, prove that $$\frac{1}{a^2}+\frac{1}{b^2}+\frac{1}{c^2}=\frac{1}{{a^{\prime}}^2}+\frac{1}{{b^{\prime}}^2}+\frac{1}{{c^{\prime}}^2}$$.
\item Find the equation of the plane through the points $(2,1,-1)$ and $(-1,3,4),$ and 
perpendicular to the plane $x-2y+4z=10.$
	\item Find the values of $\theta \text{ and } p$, if the equation $x\cos\theta+y\sin\theta=p$ is the normal form
of the line $\sqrt{3}x+y+2=0$.
\\
\solution
				\begin{align}
	\vec{n}=\myvec{\sqrt{3}\\1},
			c=-2
			\\
			\implies
			\theta=\tan^{-1}\brak{\frac{1}{\sqrt{3}}}
			=\frac{\pi}{6},
			p=\frac{\abs{c}}{\norm{\vec{n}}}=1
		\end{align}
		\iffalse
See \figref{fig:chapters/11/10/4/2/Fig1}.
\begin{figure}[H]
	\begin{center} 
	    \includegraphics[width=0.75\columnwidth]{chapters/11/10/4/2/figs/line.png}
	\end{center}
\caption{}
\label{fig:chapters/11/10/4/2/Fig1}
\end{figure}
\fi

\end{enumerate}

\subsection{Formulae}
\begin{enumerate}[label=\thesubsection.\arabic*.,ref=\thesubsection.\theenumi]
\item Let the perpendicular distance from the origin to a line be $p$ and the angle made by the perpendicular with the positive $x$-axis be $\theta$.
	Then 
\begin{align}
	p\myvec{\cos \theta \\ \sin \theta}
\end{align}
is a point on the line as well as the normal vector.
Hence, the equation of the line is 
\begin{align}
	p\myvec{\cos \theta & \sin \theta}
	\cbrak{\vec{x}-p\myvec{\cos \theta \\ \sin \theta}} &= 0
	\\
	\implies 
	\myvec{\cos \theta & \sin \theta}
	\vec{x} &= p
\label{eq:chapters/11/10/2/8-final}
\end{align}
\item Let $\vec{Q}$ be the foot of the perpendicular from $\vec{P}$
	to the line
\begin{align}
			\label{eq:geo-norm-app}
    \vec{n}^{\top}  \vec{x} = c
\end{align}
From
			\eqref{eq:geo-param}
\begin{align}
			\label{eq:geo-param-app}
	\vec{Q} = \vec{P} + k\vec{n}
	\\
	\implies PQ = \norm{\vec{Q} - \vec{P}}=\abs{k} \norm{\vec{n}}
			\label{eq:geo-param-app-PQ}
\end{align}
is the distance from $\vec{Q}$
to the line in 
			\eqref{eq:geo-norm-app}.
			From \eqref{eq:geo-param-app},
\begin{align}
	\vec{n}^{\top}  \vec{Q} = \vec{n}^{\top}  \vec{P} + k\norm{\vec{n}}^2
	\\
	\implies \abs{k} = 
	\frac{\abs{\vec{n}^{\top}\brak{\vec{Q} - \vec{P}}}}{\norm{\vec{n}}^2}
			\label{eq:geo-param-app-k}
			\\
	\implies PQ =\abs{k}  
		\norm{\vec{n}}	=
	\frac{\abs{\vec{n}^{\top}\vec{P} - c}}{\norm{\vec{n}}}
			\label{eq:PQ-final}
\end{align}
upon substituting from 
			\eqref{eq:geo-param-app-PQ}.
\item The foot of the perpendicular is given by
\begin{align}
	\label{eq:11/10/3/4/foot_of_perpendicular}
	\myvec{\vec{m} & \vec{n}}^\top\vec{Q} &= 
	   \myvec{
              \vec{m}^\top\vec{P}\\
	      c
	      }
\end{align}
\item The distance between the parallel lines 
\begin{align}
	\label{eq:parallel_lines}
	\begin{split}
		\vec{n}^{\top}\vec{x} &= c_1
		\\
		\vec{n}^{\top}\vec{x} &= c_2
	\end{split}
\end{align}
is given by 
\begin{align}
	\label{eq:dist_lines_2d}
	d = \frac{\abs{   c_1-c_2 }}{\norm{\vec{n}}}	
\end{align}
\end{enumerate}

\subsection{Angle}
\begin{enumerate}[label=\thesubsection.\arabic*,ref=\thesubsection.\theenumi]
\item Find the equations of the two lines through the origin which intersect the line $ \dfrac{x-3}{2}=\dfrac{y-3}{1}=\dfrac{z}{1}$ at angles of  $\dfrac{\pi}{3}$each.
	\item
 Two lines passing through the point (2,3) intersect each other at an angle of $60\degree$. If slope of one line is 2, find the equation of the other line.
\label{chapters/11/10/3/12}
 \\
 \solution
		Using the scalar product
\begin{align}
  \cos{60\degree}=
\frac{1}{2}=\frac{\myvec{
        1&2
    }\myvec{
        1\\m
    }}{\sqrt{5}\sqrt{m^2+1}}\\
\implies 11m^2+16m-1=0\\
   or, m=\frac{-8\pm5\sqrt{3}}{11}    
\end{align}
So, the desired equation of the line is
\begin{align}
\myvec{
    \frac{-8\pm5\sqrt{3}}{11}&-1
}\vec{x} 
	&=
\myvec{
    \frac{-8\pm5\sqrt{3}}{11}&-1
}
\myvec{
    2\\3}
    \\
	&=\frac{-49\pm16\sqrt{3}}{11}
\end{align}
\iffalse
See 
    \figref{fig:11.10.3.12}.
\begin{figure}[H]
    \centering
    \includegraphics[width=0.75\columnwidth]{chapters/11/10/3/12/fig/asgnt1.png}
    \caption{}
    \label{fig:11.10.3.12}
\end{figure}
\fi


\item Find the equation of the lines through the point (3, 2) which make an angle of $45\degree$  with the line $x – 2y = 3$.
\label{chapters/11/10/4/11}\\
\solution
Following the approach in \probref{chapters/11/10/3/12},
\begin{align}
\cos45\degree =
\frac{1}{\sqrt{2}} = \frac{\myvec{2 & 1} \myvec{1\\m}}{\norm{\myvec{2\\1}}\norm{\myvec{1\\m}}}
\\
\implies 
 3m^2 - 8m -3 = 0
 \\
\text{or, }
m= - \frac{1}{3}, 3
\end{align} 
Thus, the desired equations are 
\begin{align}
	\myvec{1&3}\cbrak{\vec{x}-\myvec{3\\2}}&=0\\
 \implies 	\myvec{1 & 3}\vec{x} &= 9
\end{align}
and 
\begin{align}
	\myvec{3&-1}\cbrak{\vec{x}-\myvec{3\\2}}&=0\\
		\implies 	\myvec{3 & -1}\vec{x} &= 7
\end{align}
See
\figref{fig:chapters/11/10/4/11/figs/strline.jpg}.
\begin{figure}[H]
\centering
\includegraphics[width=0.75\columnwidth]{chapters/11/10/4/11/figs/fig.pdf}
\caption{}
\label{fig:chapters/11/10/4/11/figs/strline.jpg}
\end{figure}

\end{enumerate}

\subsection{Intersection}
\begin{enumerate}[label=\thesubsection.\arabic*,ref=\thesubsection.\theenumi]
	\item  Find the equation of the plane through the intersection of the planes $3{x} – {y} + 2{z} – 4 = 0 \text{ and } {x} + {y} + {z} – 2 = 0$ and the point $\myvec{2\\2\\1}$.
		\label{prob:12/11/3/9/plane}
		\\
    \solution
		The normal vector to the ZOX plane is
\begin{align} 
\vec{n} = \myvec{0\\1\\0}.
\end{align}
Since, Y-axis has the intercept 3, the desired plane passes through the point
\begin{align}
\vec{P}=\myvec{0\\3\\0}.
\end{align}
Thus, the equation of the plane is given by,
\begin{align}
	\vec{n}^\top \brak{\vec{x}-\vec{P}} &= 0\\
	\implies \myvec{0&1&0} \vec{x}&= 3
\end{align}
See Fig. 
     \ref{fig:chapters/12/11/3/8/1}.
\begin{figure}[H]
  \centering
   \includegraphics[width=0.75\columnwidth]{chapters/12/11/3/8/figs/fig.pdf}
    \caption{}
     \label{fig:chapters/12/11/3/8/1}
     \end{figure}  

	\item Find the area of triangle formed by the lines $y-x=0, x+y=0, \text{ and } x-k=0$.
		\\
\solution
		The vertices of the triangle can be expressed using the equations
\begin{align}
	\myvec{1&1\\-1&1} \vec{A} &= \vec{0}
	\\
	\myvec{1&1\\1&0} \vec{B} &= \myvec{0\\k}
	\\
	\myvec{1&0\\-1&1} \vec{C} &= \myvec{k\\0}
\end{align}
from which
\begin{align}
\vec{A} = \myvec{0\\0},
	\vec{B}=\myvec{k\\-k},
	\vec{C}=\myvec{k\\k}
\end{align}
are trivially obtained.
Thus, 
\begin{align}
ar(ABC) &=\frac{1}{2}\norm{(\vec{A}-\vec{B})\times(\vec{A}-\vec{C})}\\
	&=\frac{1}{2}\norm{\myvec{-k\\k}\times\myvec{-k\\-k}}
=k^2
\end{align}

	\item  Find the equation of the line parallel to y-axis and drawn through the point of
intersection of the lines x – 7y + 5 = 0 and 3x + y = 0.
\\
\solution
				Following the approach in \probref{prob:12/11/3/9/plane},
		the desired equation is 
\begin{align}
\myvec{	1&-7}\vec{x} -5
+
	k\myvec{3&1} \vec{x} = 0
	\\
	\implies 
	\myvec{	1 + 3k&-7+k} 
	 \vec{x} =5 
	 \\
	 \implies 
	\myvec{	1 + 3k \\ -7+k}  = \alpha \myvec{1 \\ 0}
	\text{or, } k = 7, \alpha =  22.
\end{align}
The desired equation is then given by 
\begin{align}
	\myvec{1&0}\vec{x}=\frac{5}{22}
\end{align}
The intersection of the lines is obtained using the augemented matrix as
\begin{align}
	\augvec{2}{1}{
		1 &-7 &-5
		\\
		3 & 1 & 0
	}
	\xleftrightarrow[R_1 = 22R_1+7R_2]{R_2 = R_2 - 3R_1}
	\augvec{2}{1}{
		22 &0 &-5
		\\
		0 & 22 & 15
	}
	\\
	\implies \vec{x} = \frac{5}{22}\myvec{-1 \\ 3}
\end{align}
See  
\figref{fig:chapters/11/10/4/6/Fig3}.
\begin{figure}[H]
  \begin{center} 
      \includegraphics[width=0.75\columnwidth]{chapters/11/10/4/6/figs/fig.pdf}
  \end{center}
\caption{}
\label{fig:chapters/11/10/4/6/Fig3}
\end{figure}

    \item A person standing at the junction (crossing) of two straight paths 
    represented by the equations 
    \begin{align}
        \myvec{2&-3}\vec{x} = -4 
        \label{eq:chapters/11/10/4/24/L1}
    \end{align}
    and
    \begin{align}
        \myvec{3&4}\vec{x} = 5
        \label{eq:chapters/11/10/4/24/L2}
    \end{align} 
    wants to reach the path whose equation is 
    \begin{align}
        \myvec{6&-7}\vec{x} = -8
        \label{eq:chapters/11/10/4/24/L3}
    \end{align}
    Find equation of the path that he should follow.
\\
    \solution 
		The junction of \eqref{eq:chapters/11/10/4/24/L1}
    and \eqref{eq:chapters/11/10/4/24/L2} is obtained as
    \begin{align*}
	    \augvec{2}{1}{2&-3&-4\\3&4&5} \xleftrightarrow[]{R_2\rightarrow2R_2-3R_1} 
        \augvec{2}{1}{2&-3&-4\\0&17&22} \\
		      \xleftrightarrow[]{R_1\rightarrow17R_1+3R_2} \augvec{2}{1}{17&0&-1\\0&17&22} 
		      \implies
        \vec{A} = \frac{1}{17}\myvec{-1\\22}
    \end{align*}
    Clearly, the man should follow the path perpendicular to \eqref{eq:chapters/11/10/4/24/L3} from
    $\vec{A}$ to reach it in the shortest time. The normal vector 
    of \eqref{eq:chapters/11/10/4/24/L3} is 
    \begin{align}
         \myvec{6\\-7}
	 \implies
        \vec{n} = \myvec{7\\6}
        \label{eq:chapters/11/10/4/24/L4-norm}
    \end{align}
    and the equation of the desired line is
   \begin{align}
        \myvec{7&6}\vec{x} &= \frac{1}{17}\myvec{7&6}\myvec{-1\\22} = \frac{125}{17}
        \label{eq:chapters/11/10/4/24/L4}
    \end{align}
		See Fig. \ref{fig:chapters/11/10/4/24/crossing}.
		\begin{figure}[H]
        \centering
        \includegraphics[width=0.75\columnwidth]{chapters/11/10/4/24/figs/fig.pdf}
        \caption{}
        \label{fig:chapters/11/10/4/24/crossing}
    \end{figure}

	\item Find the equation of the line passing through the point of intersection of the lines $4x + 7y - 3 = 0$ and $2x - 3y + 1 = 0$ that has equal intercepts on the axes.\\
	\solution 
	  		From \probref{prob:12/11/3/9/plane},
the intersection of the lines is given by 
		\begin{align}
       \myvec{4 + 2k &7-3k}\vec{x}=3-k
       \label{eq:11/10.4/12/3}
       \\
       \implies \myvec{4 + 2k \\7-3k} = \alpha\myvec{1 \\ 1} 
		\end{align}
			from  \probref{chapters/11/10/2/12}, yielding,
		\begin{align}
	\augvec{2}{1}{
				1 & -2 & 4
				\\
				1 & 3 & 7
			}
			\xleftrightarrow[]{R_2 = R_2 -R_1}
	\augvec{2}{1}{
				1 & -2 & 4
				\\
				0 & 5 & 3 
			}
			\\
			\text{or, } k = \frac{3}{5}
       \label{eq:11/10.4/12/4}
   \end{align}
 Substituting the above  
in       \eqref{eq:11/10.4/12/3}, the desired equation is 
    \begin{align}
        \myvec{1&1}\vec{x}=\frac{6}{13}
    \end{align}
    See
    \figref{fig:enter-label}.
\begin{figure}[H]
    \centering
    \includegraphics[width=0.75\columnwidth]{chapters/11/10/4/12/figs/fig.pdf}
    \caption{}
    \label{fig:enter-label}
\end{figure}

\item
Find the value of $p$ so that the three lines $3x+y-2=0, px+2y-3=0$ and $2x-y-3=0$ may intersect at one point.
\label{11.10.4.9}
\\
\solution
Performing row operations
on the matrix
\begin{align*}  
\myvec{
    3 &1&-2 \\
     p&2&-3\\
     2&-1&-3
}
\xrightarrow[R_3=3R_3-2R_1]{R_2=3R_2-pR_1}&\myvec{
    3&1&-2\\
     $0$&6-p&-9+2p\\
     0&-5&-5}\\
 \xrightarrow{R_3=R_3(6-p)+5R_2}&\myvec{
    3&1&-2\\
     0&6-p&-9+2p\\
     0&0&-75+15p}
     \\
  \implies 
    p=5
\end{align*}
    See \figref{fig:11.10.4.9}.
\begin{figure}[H]
    \centering
    \includegraphics[width=0.75\columnwidth]{chapters/11/10/4/9/fig/11.10.4.9.png}
    \caption{}
    \label{fig:11.10.4.9}
\end{figure}


\item Show the lines
$$\frac{x-1}{2}=\frac{y-2}{3}=\frac{z-3}{4}$$
$$\text{ and } \frac{x-4}{5}=\frac{y-1}{2}=z  \text{ intersect }.$$
 Also, find their point of intersection.
\item The area of the region bounded by the curve $y = x + 1$ and the lines $x = 2\text{ and }x = 3$ is
\begin{enumerate}
\item $\frac{7}{2}$ sq units
\item $\frac{9}{2}$ sq units
\item $\frac{11}{2}$ sq units
\item $\frac{13}{2}$ sq units
\end{enumerate}   
\item The area of the region bounded by the curve $x = 2 + 3$ and the $y$ lines $y = 1\text{ and }y = - 1$ is
\begin{enumerate}
\item 4 sq units 
\item $\frac{3}{2}$ sq units
\item 6 sq units
\item 8 sq units
\end{enumerate}
\item Compute the area bounded by the line $x + 2y = 2$, $y - x = 1\text{ and }2x + y = 7$.
\item Find the area bounded by the lines $y = 4x + 5$, $y = 5 - x\text{ and }4y = x + 5$.
\item Find the equation of the plane which is perpendicular to the plane $5x+3y+6z+8=0$ and which contains the line of intersection of the planes $x+2y+3z-4=0$ and $2x+y-z+5=0.$
\item  Point $\vec{P}(0,2)$ is the point of intersection of $y$-axis and perpendicular bisector of line segment joining the points $\vec{A}(-1,1) \text{ and } \vec{B}(3,3)$.
\item Prove that the line through A$(0,-1,-1)$ and B$(4,5,1)$ intersects the line through C$(3,9,4)$ and D$(-4,4,4)$.
\item Find the equation of the plane through the intersection of the planes $\overrightarrow{r} \cdot (\hat{i}+3\hat{j}) - 6=0$ and $\overrightarrow{r} \cdot (3\hat{i}-\hat{j}-4\hat{k})=0,$ whose perpendicular distance from origin is unity.
\item Find the equation of the line passing through the point of intersection of $2x+y=5\text{ and }x+3y+8=0$ and parallel the line $3x+4y=7$.
\item Find the equations of the lines through the point of intersection of the line $x-y+1=0 \text{ and }2x-3y+5=0$ and whose distance from the point (3,2) is $\frac{7}{5}$.
\item Equations of diagonals of the square formed by the lines $x=0$, $y=0$, $x=1$ and $y=1$ are
\begin{enumerate}
\item $y=x$, $y+x=1$
\item $y=x$,$x+y=2$
\item $2y=x$, $y+x=\frac{1}{3}$
\item $y=2x$, $y+2x=1$
\end{enumerate}
\end{enumerate}

\subsection{Miscellaneous }
\begin{enumerate}[label=\thesubsection.\arabic*,ref=\thesubsection.\theenumi]


\item Find the values of $k$ for which the line 
\begin{align}
(k-3)x-(4-k^2)y+k^2-7k+6=0 \label{eq:chapters/11/10/4/1/1}
\end{align}
is
\begin{enumerate}
\item Parallel to the $x$-axis
\item Parallel to the $y$-axis
\item Passing through the origin
\end{enumerate}
    \solution 
		\begin{align}
\vec{n} = \myvec{k-3\\-4+k^2}, c  = -k^2+7k-6
\label{eq:chapters/11/10/4/1/6}
\end{align}
\begin{enumerate}
    \item 
\begin{align}
\myvec{k-3\\-4+k^2} =\alpha\myvec{0\\1}
\implies
k =3,
\\
\implies        \myvec{0 & 5}\vec{x} =6
\end{align}
upon substituting from 
\eqref{eq:chapters/11/10/4/1/6}.

\item In this case, 
\begin{align}
\myvec{k-3\\-4+k^2} =\beta\myvec{1\\0}
	\implies k =\pm2
	\\
	\implies
        \myvec{-1 & 0}\vec{x} =4, \quad  k =2\\
        \myvec{-5 & 0}\vec{x} =-24, \quad  k =-2
\end{align}
\item 
	In this case, 
\begin{align}
	-k^2+7k-6 = 0
	\implies k =1,  k=6
	\\
	\implies
        \myvec{-2 & -3}\vec{x} =0, \quad  k =1\\
       \myvec{3 & 32}\vec{x} =0, \quad  k =6
\end{align}
\end{enumerate}

	\item Find the  equations of the lines, which cutoff intercepts on the axes  whose sum and product are 1 and -6 respectively.
\\
\solution
		Let the intercepts be $a$ and  $b$. Then
\begin{align}
a+b=1,
ab=-6 \label{eq:11/10/4/32a}
\\
\implies  a = 3, b = -2
\end{align}
Thus, the possible 
intercepts are
\begin{align}
\myvec{3\\0}, \myvec{0\\-2},
\myvec{-2\\0}, \myvec{0\\3}
\end{align}
From
		\eqref{prop:lin-eq-unit-mat},
\begin{align}
	\myvec{3 & 0 \\ 0 &-2}\vec{n} = \myvec{1 \\ 1}
	\\
	\implies \vec{n} = \myvec{\frac{1}{3} \\ -\frac{1}{2}}
	\\
	\text{or, } \myvec{2 & -3}\vec{x} = 6
\end{align}
using		\eqref{prop:lin-eq-unit}.
Similarly, the other line can be obtained
as
\begin{align}
	\myvec { 3 & -2 }  \vec{x}  = -6        
\end{align}
\iffalse
See  
\figref{fig:11/10/4/3line segmenta}.
\begin{figure}[H]
\centering
\includegraphics[width=0.75\columnwidth]{chapters/11/10/4/3/figs/inter.png}
\caption{}
\label{fig:11/10/4/3line segmenta}
\end{figure}
\fi

\item A ray of light passing through the point $\vec{P} = \brak{1, 2}$ reflects on the x-axis at point $\vec{A}$ and the reflected ray passes through the point $\vec{Q} =\brak{5, 3}$. Find the coordinates of $\vec{A}$.
\\
    \solution 
			From \eqref{eq:11/10/4/22},
the reflection of $\vec{Q}$ is 
\begin{align}
\vec{R}  
= \myvec{5\\-3}
\end{align}
Letting
\begin{align}
\vec{A} = \myvec{x\\0},
\end{align}
since 
$\vec{P},
\vec{A},  
\vec{R}  
$
are collinear, 
		from \eqref{prop:lin-dep-rank},
\begin{align}
	\myvec{
		1 & 1 & 2 
		\\ 
		1 & 5 & -3 
		\\
		1 & x & 0 }
	\xleftrightarrow[R_3=R_3 - R_1]{R_2 = R_2 - R_1}
	\myvec{
		1 & 1 & 2 
		\\ 
		0 & 4 & -5 
		\\
		0 & x-1 & -2 }
	\\
	\xleftrightarrow[]{R_3 = 4R_3 - \brak{x-1}R_2}
	\myvec{
		1 & 1 & 2 
		\\ 
		0 & 4 & -5 
		\\
		0 & 0 & 5x-13 }
	\implies x = \frac{13}{5}
\end{align}
See  
\figref{fig:chapters/11/10/4/22/1}.
\begin{figure}[H]
\centering
\includegraphics[width=0.75\columnwidth]{chapters/11/10/4/22/figs/fig.pdf}
\caption{}
\label{fig:chapters/11/10/4/22/1}
\end{figure}




\item Prove that in any $\triangle{ABC}$, cos A=$\frac{b^2+c^2-a^2}{2bc}$, where a,b,c are the magnitudes of the sides opposite to the vertices A,B,C respectively.
\item Distance of the point $(\alpha, \beta, \gamma)$ from y-axis is
\begin{enumerate}
	\item $\beta$ 
	\item $\abs{\beta}$
	\item $\abs{\beta+\gamma}$
	\item $\sqrt{\alpha^2+\gamma^2}$
\end{enumerate}
\item The reflection of the point $(\alpha, \beta, \gamma )$ in the xy-plane is 
\begin{enumerate}
	\item $\alpha,\beta,0)$
	\item $(0,0,\gamma)$
	\item $(-\alpha,-\beta,\gamma)$
	\item $(\alpha,\beta,-\gamma)$
\end{enumerate}
\item The plane $ax+by=0$ is rotated about its line of intersection with the plane $z=0$ through an angle $\alpha.$ Prove that the equation of the plane in its new position is $ax+by \pm (\sqrt{a^2+b^2} \tan\alpha)z=0.$
\item The locus represented by $xy+yz=0$ is 
\begin{enumerate}
	\item A pair of perpendicular lines
	\item A pair of parallel lines
	\item A pair of parallel planes 
	\item A pair of perpendicular planes
\end{enumerate}
\item For what values of $a$ and $b$ the intercepts cut off on the coordinate axes by the line $ax+by+8=0$are equal in length but opposite in signs to those cut off by the line $2x-3y+=0$ on the axes.
\item If the equation of the base of an equilateral triangle is $x+y=2$ and the vertex is (2,-1), then find the length of the side of the triangle. 
[\textbf{Hint} : Find length of perpendicular ($p$) from (2,-1) to the line and use $p=l \sin 60degree$,where $l$ is the length of the triangle].
\item A variable line passes through a fixed point $\vec{P}$.The algebraic sum of the perpendiculars drawn from the points (2,0),(0,2) and (1,1) on the line is zero. Find the coordinates of the point $\vec{P}$.  
[\textbf{Hint} : let the slope of the line be $m$. Then the equation of the line passing through the fixed point $\vec{P}(x_1,y_1) y-y_1=m(x-x_1)$. Taking the algebraic sum of perpendicular distances equal to zero, we get $y-l=m(x-1)$. Thus $(x_1,y_1)$ is (1,1).]
\item A straight line moves so that the sum of the reciprocals of its intercepts made on axes is constant. Show that the line passes through a fixed point. [\textbf{Hint} : $\frac{x}{a}+\frac{y}{b}=1\text{ where} \frac{1}{a}+\frac{1}{b}=\text{ constant }=\frac{1}{k}$(say). This implies that $\frac{k}{a}+\frac{k}{b}=1$ line passes through the fixed point $(k,k)$.]
\item If the sum of the distances of a moving point in a plane from the axes is $l$, then finds the locus of the point. [\textbf{Hint} :Given that $\abs{x}+\abs{y}=1$, which  gives four  sides of a square.] 
\item $\vec{P}_1,\vec{P}_2$ are points on either of the two lines $y-\sqrt{3}\abs{x}=2$ at a distance of 5 units from their point of intersection. Find the coordinates of the root of perpendiculars drawn from $P_1, P_2$ on the bisector of the angle between the given lines.
[\textbf{Hint} : Lines are $y=\sqrt{3}x+2 \text{ and }y=-\sqrt{3}x+2$ according as $x\geq0$ or $x0. y$-xis is the bisector of the angles between the lines. $P_1, P_2$ are the points on these lines at a distance of 5 units from the point of intersection of these lines which have a point on $y$-axis as a common foot of perpendiculars from these points. The $y$-coordinate of the foot of the perpendicular is given by 2=5 $\cos{30\degree}$.]
\item If $p$ is the length of perpendicular from the origin on the lien $\frac{x}{a}+\frac{y}{b}=1$ and $a^2,p^2,b^2$ are in A.P, then show that $a^4+b^4=0$.
\item The point (4,1)undergoes the following two successive transformations :
\begin{enumerate}
\item Reflection about the line $y=x$
\item Translation through a distance 2 units along the positive $x$-axis 
\end{enumerate}
Then the final coordinates of the point are
\begin{enumerate}
\item (4,3)
\item (3,4)
\item (1,4)
\item $\frac{7}{2}$,$\frac{7}{2}$
\end{enumerate}
\item One vertex of the equilateral with centroid at the origin and one side as $x+y-2=0$ is
\begin{enumerate}
\item (-1,-1)
\item (2,2)
\item (-2-2)
\item (2,-2)
\end{enumerate}
[\textbf{Hint} : Let $ABC$ be the equilateral triangle with vertex $\vec{A}(h,k)\text{ and let }\vec{D}(\alpha,\beta)$ be the point on $BC$. Then $\frac{2\alpha+h}{3}=0=\frac{2\beta+k}{3}$. Also ${\alpha+\beta-2=0}\text{ and }\frac{k-0}{h-o}x(-1)=-1$] 
\item If $a,b,c$ are is A.P.,then the straight lines $ax+by+c=0$ will always pass through \rule{1cm}{0.15mm}.
\item The points (3,4) and (2,-6)are situated on the \rule{1cm}{0.15mm} of the line $3x-4y-8=0$.
\item A point moves so that square of its distance from the point (3,-2) is numerically equal to its distance from the line $5x-12y=3$. The equation of its locus is %\rule{1cm}{0.15mm}.
\item Locus of the mid-points of the portion of the line $x\sin\theta+y\cos\theta=p$ intercepted between the axes is \rule{1cm}{0.15mm}.
State whether the statements in Exercises 48 to 56 are true or false. Justify.
\item If the vertices of a triangle have integral coordinates, then the triangle can not be equilateral.
\item The vertex of on equilateral triangle is (intercepted equation of the opposite side is $x+y=2$.then the other two sides are $y-3=(2\pm\sqrt{3})(x-2)$.
\item The line $\frac{x}{a}+\frac{y}{b}=1$ moves in such a way that $\frac{1}{a^2}+\frac{1}{b^2}=\frac{1}{c^2}$, where $c$ is a constant.The locus of the foot of the perpendicular from the origin on the given line is $x^2+y^2=c^2$.
\end{enumerate}
Match the following
\begin{enumerate}[resume]
\item 
	\begin{table}[!hb]
	\iffalse
\begin{center}
	\resizebox{\columnwidth}{!}{
\begin{tabular}{cc}
$C_1$ &   $C_2$
\end{tabular}   
	}
\\
\fi
\centering
	\resizebox{\columnwidth}{!}{
\begin{matchtabular}
  The coordinates of the points P and Q on the line x + 5y = 13 which are at a distance of 2 units from the line 12x – 5y + 26 = 0 are & (3,1),(-7,11)\\
  The coordinates of the point on the line x + y = 4, which are at a unit distance from the line 4x + 3y – 10 = 0 are & $-\frac{1}{11},\frac{11}{3}$ , $\frac{4}{3},\frac{7}{3}$\\
  The coordinates of the point on the line joining A (–2, 5) and B (3, 1) such that AP = PQ = QB are & 1,$\frac{12}{5}$ , $-3,\frac{16}{5}$\\
\end{matchtabular}
		}
		\caption{}
		\label{tab:lin-misc-1}
	\end{table}
\item The value of the $\lambda$, if the lines\\$(2x+3y+4)+\lambda(6x-y+12)=0$ are
\begin{center}
\begin{tabular}{cccccc}
\textbf{Column $C_1$} & & & & &  \textbf{Column $C_2$}\\
\end{tabular}   
\end{center}
\begin{matchtabular}
parallel to $y$-axis is & $\lambda =-\frac{3}{4}$\\
perpendicular to $7x+y-4=0$ is & $\lambda=-\frac{1}{3}$\\
passes through (1,2) is & $\lambda=-\frac{17}{41}$\\
parallel to $x$ axis is & $\lambda=3$\\
\end{matchtabular}
\\
\item The equation of the line through the intersection of the lines $2x-3y=0$ and $4x-5y=2$ and
\begin{center}
\begin{tabular}{cccccc}
\textbf{Column $C_1$} & & & & &  \textbf{Column $C_2$}\\
\end{tabular}   
\end{center}

\begin{matchtabular}
through the point (2,1) is & $2x-y=4$\\
perpendicular to the line & $x+y-5=0$\\
parallel to the line $3x-4y+5=0$ is & $x-y-1=0$\\
equally inclined to the axes is & $3x-4y-1=0$\\
\end{matchtabular}
\end{enumerate}

\subsection{Formulae}
%\begin{enumerate}[label=\arabic*.,ref=\theenumi]
\begin{enumerate}[label=\thesubsection.\arabic*.,ref=\thesubsection.\theenumi]
	\item 
The reflection of point $\vec{Q}$ w.r.t a line is given by
\begin{align}
	\label{eq:11/10/4/22}
\vec{R} = \vec{Q} -\frac{2\brak{\vec{n}^{\top}\vec{Q}-c}}{\norm{\vec{n}}}\vec{n}
\end{align}
\end{enumerate}

\newpage
\section{Skew Lines}
\subsection{Least Squares}
\begin{enumerate}[label=\thesubsection.\arabic*,ref=\thesubsection.\theenumi]
\item Find the shortest distance between the lines
\begin{align}
	\frac{x+1}{7}&=\frac{y+1}{-6}=\frac{z+1}{1} \text{ and}
	\\
	\frac{x-3}{1}&=\frac{y-5}{-2}=\frac{z-7}{1}
\end{align}
    \solution
		 The given lines  can be written as
\begin{align}
\vec{x} &= \myvec{-1\\-1\\-1} + \kappa_1\myvec{7\\-6\\1}\\
\vec{x} &= \myvec{3\\5\\7} + \kappa_2\myvec{1\\-2\\1} \\
\vec{A} = \myvec{-1\\-1\\-1},\, \vec{B} &= \myvec{3\\5\\7}, \,\vec{m}_1 = \myvec{7\\-6\\1}, \, \vec{m}_2 = \myvec{1\\-2\\1}
\end{align}
%
We first check whether the given lines are skew. The lines 
\begin{align}
\vec{x} = \vec{x_1} + \kappa_1\vec{m_1},\, \vec{x} = \vec{x_2} + \kappa_2\vec{m_2} 
\label{eq:chapters/12/11/2/15/1}
\end{align}
intersect if
\begin{align}
\vec{M}{\kappa} &= \vec{x_2} - \vec{x_1}\\
\vec{M} &\triangleq \myvec{\vec{m_1} & \vec{m_2}} \\
\bm{\kappa} &\triangleq \myvec{\kappa_1\\-\kappa_2}\\
\end{align}
Here we have,
\begin{align}
\vec{M} = \myvec{7&1\\-6&-2\\1&1}\,
\vec{x_2} - \vec{x_1} = \myvec{4\\6\\8}
\end{align}
We check whether the equation \eqref{eq:chapters/12/11/2/15/2} has a solution
\begin{align}
\myvec{7&1\\-6&-2\\1&1}\bm{\kappa} = \myvec{4\\6\\8}
\label{eq:chapters/12/11/2/15/2}
\end{align}
the augmented matrix is given by,
\begin{align}
\myvec{7&1&\vrule&4\\-6&-2&\vrule&6\\1&1&\vrule&8}
\xleftrightarrow[R_3 \leftarrow R_3 - \frac{1}{7}R_1]{R_2 \leftarrow R_2 + \frac{6}{7}R_1}\\
\myvec{7&1&\vrule&4\\&&\vrule\\0&-\frac{8}{7}&\vrule&\frac{66}{7}\\&&\vrule\\0&\frac{6}{7}&\vrule&-\frac{52}{7}}
\xleftrightarrow{R_3 \leftarrow R_3 + \frac{3}{4}R_2}\\
\myvec{2&3&\vrule&1\\&&\vrule\\0&-\frac{7}{2}&\vrule&\frac{1}{2}\\&&\vrule\\0&0&\vrule&-\frac{5}{14}}
\end{align}
The rank of the matrix is 3. So the given lines are skew.
The closest points on two skew lines defined by \eqref{eq:chapters/12/11/2/15/1} are given by 
\begin{align}
\vec{M}^\top \vec{M}\bm{\kappa} &= \vec{M}^\top\brak{\vec{x_2}-\vec{x_1}}\\
\implies \myvec{7&-6&1\\1&-2&1} \myvec{7&1\\-6&-2\\1&1}\bm{\kappa} &= \myvec{7&-6&1\\1&-2&1} \myvec{4\\6\\8}\\
\implies \myvec{86&20\\20&6}\bm{\kappa} &= \myvec{0\\0}
\label{eq:chapters/12/11/2/15/3}
\end{align}
The augmented matrix of the above equation \eqref{eq:chapters/12/11/2/15/3} is given by,
\begin{align}
\myvec{86&20&\vrule&0\\20&6&\vrule&0}
\xleftrightarrow{R_2 \leftarrow R_2 - \frac{10}{43}R_1}
\myvec{86&20&\vrule&0 \\&&\vrule\\ 0&\frac{58}{43}&\vrule&0}
\xleftrightarrow[R_2 \leftarrow \frac{43}{58}R_2]{R_1 \leftarrow \frac{1}{86} \brak{R_1 - \frac{430}{29}R_2}}\\
\myvec{1&0&\vrule&0 \\&&\vrule\\ 0&1&\vrule&0}
\end{align}
yielding
\begin{align}
\myvec{\kappa_1\\-\kappa_2} &= \myvec{0\\0}
\end{align}
The closest points $\vec{A}$ on line $l_1$ and $\vec{B}$ on line $l_2$ are given by,
\begin{align}
\vec{A} &= \vec{x_1} + \kappa_1\vec{m_1}
= \myvec{-1\\-1\\-1}\\
\vec{B} &= \vec{x_2} + \kappa_2\vec{m_2}
= \myvec{3\\5\\7}
\end{align}
The minimum distance between the lines is given by
\begin{align}
\norm{\vec{B}-\vec{A}} &= \norm{\myvec{4\\6\\8}}
= 2\sqrt{29}
\end{align}
%
\begin{figure}[!ht]
\centering
\includegraphics[width=\columnwidth]{chapters/12/11/2/15/figs/Figure_1.png}
\caption{}
\label{fig:chapters/12/11/2/15/}
\end{figure}


    \item Find the shortest distance between the lines whose vector equations are
    \begin{align}
\begin{split}
	\vec{x} &= \myvec{1\\2\\3} + \kappa_1\myvec{1\\-3\\2}
	\\
	\vec{x} &= \myvec{4\\5\\6} + \kappa_2\myvec{2\\3\\1}
\end{split}
        \label{eq:chapters/12/11/2/16/L2/svd}
    \end{align}
    \solution
		    In this case,
    \begin{align}
	    \vec{B}-\vec{A} &=  \myvec{3\\3\\3} \\
	    \vec{M} &=  \myvec{1&2\\-3&3\\2&1}. 
                \label{eq:chapters/12/11/2/16/given}
    \end{align}
	    forming the matrix in \eqref{eq:chapters/12/11/2/16/lsq/rank},
    \begin{align*}
        \myvec{1&2&3\\-3&3&3\\2&1&3} \xleftrightarrow[]{R_2\leftarrow R_2+3R_1} \myvec{1&2&3\\0&9&12\\2&1&3} \\
                \xleftrightarrow[]{R_3\leftarrow R_3-2R_1} \myvec{1&2&3\\0&9&12\\0&-3&-3} %\\
                \xleftrightarrow[]{R_3\leftarrow 3R_3+R_2} \myvec{1&2&3\\0&9&12\\0&0&3}
    \end{align*}
    Clearly, the rank of this matrix is 3, and therefore, the lines are skew.
%
        From \eqref{eq:chapters/12/11/2/16/lsq/vec-eqn},
    \begin{align*}
        \augvec{2}{1}{14&-5&0\\-5&14&18} \xleftrightarrow[]{R_1\leftarrow R_1+R_2} \augvec{2}{1}{9&9&18\\-5&14&18} \\
                 \xleftrightarrow[]{R_1\leftarrow\frac{R_1}{9}} \augvec{2}{1}{1&1&2\\-5&14&18} 
                 \xleftrightarrow[]{R_2\leftarrow R_2+5R_1} \augvec{2}{1}{1&1&2\\0&19&28} \\
                 \xleftrightarrow[]{R_1\leftarrow19R_1-R_2} \augvec{2}{1}{19&0&10\\0&19&28} 
                 \xleftrightarrow[]{\substack{R_1\leftarrow\frac{R_1}{19}\\R_2\leftarrow\frac{R_2}{9}}}
                    \augvec{2}{1}{1&0&\frac{10}{19}\\0&1&\frac{28}{19}} 
    \end{align*}
    yielding
\begin{align}
                    \bm{\kappa} = \frac{1}{19}\myvec{10\\28}
\label{eq:chapters/12/11/2/16/L2/svd/kappa}
\end{align}
        Substituting the above in \eqref{eq:chapters/12/11/2/16/L2/svd},
    \begin{align}
        \vec{x}_1 = \frac{1}{19}\myvec{29\\8\\77},\, \vec{x}_2 = \frac{1}{19}\myvec{20\\11\\86}.
    \end{align}
    Thus, the required distance is
    \begin{align}
        \norm{\vec{x}_2-\vec{x}_1} = \frac{3}{\sqrt{19}}
    \end{align}
See \figref{fig:chapters/12/11/2/16/skew}.
    \begin{figure}[H]
        \centering
        \includegraphics[width=0.75\columnwidth]{chapters/12/11/2/16/lsq/figs/skew.png}
        \caption{}
        \label{fig:chapters/12/11/2/16/skew}
    \end{figure}

\item Find the shortest distance between the lines $l_1$ and $l_2$ whose vector equations are 
\begin{align}
	\overrightarrow{r} &= \hat{i}+\hat{j}+\kappa(2\hat{i}-\hat{j}+\hat{k}) \text{ and}
	\\
	\overrightarrow{r} &= 2\hat{i}+\hat{j}-\hat{k}+\mu(3\hat{i}-5\hat{j}+2\hat{k}).
\end{align}
    \solution
				The given lines can be written  in vector form  as
\begin{align}
\begin{split}
	\vec{x} &= \myvec{1\\1\\0} + \kappa_1\myvec{2\\-1\\1},
	\\
	\vec{x} &= \myvec{2\\1\\-1} + \kappa_2\myvec{3\\-5\\2}
\end{split}
\label{eq:chapters/12/11/2/e11/lines}
\\
\begin{split}
	\vec{M} &= \myvec{2&3\\-1&-5\\1&2},
\vec{B} - \vec{A} = \myvec{1\\0\\-1}
\end{split}
\label{eq:chapters/12/11/2/e11/params}
\end{align}
%
Substituting the above in \eqref{eq:chapters/12/11/2/16/lsq/rank},
\begin{align}
\myvec{2&3&1\\-1&-5&0\\1&2&-1}
\xleftrightarrow[R_3 \leftarrow R_3 - \frac{1}{2}R_1]{R_2 \leftarrow R_2 + \frac{1}{2}R_1}
	\myvec{2&3&1\\[1ex]0&-\frac{7}{2}&\frac{1}{2}\\[1ex]0&\frac{1}{2}&-\frac{3}{2}}\\
\xleftrightarrow{R_3 \leftarrow R_3 + 7R_2}
	\myvec{2&3&1\\[1ex]0&-\frac{7}{2}&\frac{1}{2}\\[1ex]0&0&-10}
\end{align}
The rank of the matrix is 3. So the given lines are skew.
        From \eqref{eq:chapters/12/11/2/16/lsq/vec-eqn},
\begin{align}
\myvec{2&-1&1\\3&-5&2} \myvec{2&3\\-1&-5\\1&2}\bm{\kappa} &= \myvec{2&-1&1\\3&-5&2} \myvec{1\\0\\-1}\\
\implies \myvec{6&13\\13&38}\bm{\kappa} &= \myvec{1\\1}
\label{eq:chapters/12/11/2/e11/3}
\end{align}
The augmented matrix of the above equation \eqref{eq:chapters/12/11/2/e11/3} is given by,
\begin{align}
\myvec{6&13&\vrule&1\\13&38&\vrule&1}
\xleftrightarrow{R_2 \leftarrow R_2 - \frac{13}{6}R_1}
\myvec{6&13&\vrule&1 \\ 0&\frac{59}{6}&\vrule&-\frac{7}{6}}\\
\xleftrightarrow{R_1 \leftarrow R_1 - \frac{78}{59}R_2}
\myvec{6&0&\vrule&\frac{150}{59} \\ 0&\frac{59}{6}&\vrule&-\frac{7}{6}}
\end{align}
yielding
\begin{align}
	\myvec{\kappa_1\\-\kappa_2} &= \myvec{\frac{25}{59}\\[1ex]-\frac{7}{59}}
	\label{eq:chapters/12/11/2/e11/}
\end{align}
Substituting in \eqref{eq:chapters/12/11/2/e11/lines},
\begin{align}
\vec{x}_1 
= \frac{1}{59}\myvec{109\\34\\25},\,
\vec{x}_2 
= \frac{1}{59}\myvec{139\\24\\-45}.
\end{align}
The minimum distance between the lines is given by,
\begin{align}
\norm{\vec{x}_2-\vec{x}_1} &= \norm{\frac{1}{59}\myvec{30\\-10\\-70}}
= \frac{10}{\sqrt{59}}
\end{align}
See Fig. 
	\ref{fig:chapters/12/11/2/e11/}.
\begin{figure}[H]
\centering
\includegraphics[width=0.75\columnwidth]{chapters/12/11/2/e11/lsq/figs/skew.png}
\caption{}
	\label{fig:chapters/12/11/2/e11/}
\end{figure}

		%\begin{enumerate}[label=\thesubsection.\arabic*.,ref=\thesubsection.\theenumi]
	\item Perform the eigendecompositions 
    \begin{align}
	    \vec{MM}^\top &= \vec{U}\vec{D}_1\vec{U}^\top \label{eq:chapters/12/11/2/16/svd/decomp-1} \\
	    \vec{M}^\top\vec{M} &= \vec{V}\vec{D}_2\vec{V}^\top \label{eq:chapters/12/11/2/16/svd/decomp-2}
    \end{align}
	\item    The following expression is known as {\em singular value decomposition}
    \begin{align}
        \vec{M} = 
	\vec{U}\vec{\Sigma}\vec{V}^\top
        \label{eq:chapters/12/11/2/16/svd/M-svd}
    \end{align}
    where $\vec{\Sigma}$ is diagonal with
    entries obtained as in 
        \eqref{eq:chapters/12/11/2/16/svd/svd-params}.
 Substituting in 
        \eqref{eq:chapters/12/11/2/16/lsq/vec-eqn},
	\begin{align}
\vec{V}\vec{\Sigma}\vec{U}^\top\vec{U}\vec{\Sigma}\vec{V}^\top\bm{\kappa} &= \vec{V}\vec{\Sigma}\vec{U}^\top\brak{\vec{B}-\vec{A}} \\
\implies \vec{V}\vec{\Sigma}^2\vec{V}^\top\bm{\kappa} &= \vec{V}\vec{\Sigma}\vec{U}^\top\brak{\vec{B}-\vec{A}} \\
\implies \bm{\kappa} &= \brak{\vec{V}\vec{\Sigma}^2\vec{V}^\top}^{-1}\vec{V}\vec{\Sigma}\vec{U}^\top\brak{\vec{B}-\vec{A}} \\
\implies \bm{\kappa} &= \vec{V}\vec{\Sigma}^{-2}\vec{V}^\top\vec{V}\vec{\Sigma}\vec{U}^\top\brak{\vec{B}-\vec{A}} \\
\implies \bm{\kappa} &= \vec{V}\vec{\Sigma}^{-1}\vec{U}^\top\brak{\vec{B}-\vec{A}}
\label{eq:chapters/12/11/2/16/svd/kappa-sol}
\end{align}
    where $\vec{\Sigma}^{-1}$ is obtained by inverting the nonzero elements of
    $\vec{\Sigma}$. 
		\item 
	    From \eqref{eq:chapters/12/11/2/16/lsq/x1x2}, 
\begin{align}
	\vec{x}_1-\vec{x}_2 &= 
	\vec{A}+ \kappa_1\vec{m_1}
	 -\vec{B}  - \kappa_2\vec{m_2} 
	 \\
	 &=
	\vec{A} 
	 -\vec{B}  + \vec{M} 
	\bm{\kappa}
\end{align}
which, upon substitution from 
        \eqref{eq:chapters/12/11/2/16/svd/M-svd}
	yields
\begin{align}
	\vec{x}_1-\vec{x}_2 &= 
	\vec{A} 
	 -\vec{B}  + 
\vec{U}\vec{\Sigma}\vec{V}^\top
\vec{V}\vec{\Sigma}^{-1}\vec{U}^\top\brak{\vec{B}-\vec{A}}
\\
	&=
	\brak{	\vec{A} 
	 -\vec{B}}  \brak{\vec{I}- 
\vec{U}\vec{\Sigma}
	\vec{\Sigma}^{-1}\vec{U}^\top}
\end{align}
			Thus, 
    \begin{align}
	    \norm{\vec{x}_1-\vec{x}_2} = 
	\norm{\brak{	\vec{A} 
	 -\vec{B}}  \brak{\vec{I}- 
\vec{U}\vec{\Sigma}
	    \vec{\Sigma}^{-1}\vec{U}^\top}}
        \label{eq:chapters/12/11/2/16/svd/min-sol}
    \end{align}
\end{enumerate}

\item Find the shortest distance between the lines given by 
\begin{align}
	\overrightarrow{r}&=(8+3\kappa\hat{i}-(9+16\kappa)\hat{j}+(10+7\kappa)\hat{k} \text{ and} 
	\\
	\overrightarrow{r}&=15\hat{i}+29\hat{j}+5\hat{k}+\mu(3\hat{i}+8\hat{j}-5\hat{k}).
\end{align}
\item Find the shortest distance between the lines
\begin{align}
	\overrightarrow{r}&=(\hat{i}+2\hat{j}+\hat{k})+\kappa(\hat{i}-\hat{j}+\hat{k}) \text{ and} 
	\\
	\overrightarrow{r}&=2\hat{i}-\hat{j}-\hat{k}+\mu(2\hat{i}+\hat{j}+2\hat{k})
\end{align}
\item Find the matrix $\vec{X}$ so that $\vec{X}\myvec{1&2&3\\ 4&5&6}$= $\myvec{-7&-8&-9\\ 2&4&6}$.
\item Find the shortest distance between the lines whose vector equations are 
\begin{align} 
\overrightarrow{r}=(1-t)\hat{i}+(t-2)\hat{j}+(3-2t)\hat{k} \text{ and }\\ \overrightarrow{r}=(s+1)\hat{i}+(2s-1)\hat{j}-(2s+1)\hat{k}
\end{align}
\item Find the shortest distance between the lines $\overrightarrow{r}=6\hat{i}+2\hat{j}+2\hat{k}+\lambda(\hat{i}-2\hat{j}+2\hat{k})$ and $\overrightarrow{r}=-4\hat{i}-\hat{k}+\mu(3\hat{i}-2\hat{j}-2\hat{k})$.
\end{enumerate}

\subsection{Formulae}
%\begin{enumerate}[label=\arabic*.,ref=\theenumi]
\begin{enumerate}[label=\thesubsection.\arabic*.,ref=\thesubsection.\theenumi]
	\item The lines
\begin{align}
\begin{split}
	L_1: \quad   \vec{x} &=\vec{A}+ \kappa_1\vec{m_1}
	\\
L_2: \quad        
	\vec{x} &= \vec{B}  + \kappa_2\vec{m_2} 
\end{split}
	    \label{eq:chapters/12/11/2/16/lsq/L1L2}
\end{align}
will intersect if 
\begin{align}
\vec{A}+ \kappa_1\vec{m_1}
= \vec{B}  + \kappa_2\vec{m_2} 
\\
\implies 
 \myvec{\vec{m_1} & \vec{m_2}}\myvec{\kappa_1\\-\kappa_2}
	 =\vec{B}-\vec{A}
 \\
	\implies \rank\myvec{\vec{M}  
	& \vec{B}-\vec{A}} = 2 
	    \label{eq:chapters/12/11/2/16/lsq/rank}
\end{align}
where
\begin{align}
	\vec{M} = 
	\myvec{\vec{m_1} & \vec{m_2}} 
\end{align}
\item If $L_1, L_2$, do not intersect, let 
\begin{align}
\begin{split}
	\vec{x}_1 &=\vec{A}+ \kappa_1\vec{m_1}
	\\
	\vec{x}_2 &= \vec{B}  + \kappa_2\vec{m_2} 
\end{split}
	    \label{eq:chapters/12/11/2/16/lsq/x1x2}
\end{align}
be points on 
$L_1, L_2$ respectively, that are closest to each other.
Then, 
	    from \eqref{eq:chapters/12/11/2/16/lsq/x1x2}
\begin{align}
\vec{x_1} - \vec{x_2} =
	 \vec{A}-\vec{B}+
 \myvec{\vec{m_1} & \vec{m_2}}\myvec{\kappa_1\\-\kappa_2}
	\label{eq:chapters/12/11/2/16/lsq/x-diff}
\end{align}
Also, 
    \begin{align}
	    \brak{\vec{x}_1 -\vec{x}_2}^\top\vec{m}_1
	    =
	    \brak{\vec{x}_1 -\vec{x}_2}^\top\vec{m}_2
	    =0
	    \\
	    \implies 
	    \brak{\vec{x}_1 -\vec{x}_2}^\top\myvec{\vec{m_1} & \vec{m_2}} = \vec{0}
	    \\
	    \text{or, }	    \vec{M}^\top\brak{\vec{x}_1 -\vec{x}_2} = \vec{0}
	    \\
	    \implies \vec{M}^\top
	    \brak{\vec{A}-\vec{B}}+
 \vec{M}^\top\vec{M}\myvec{\kappa_1\\-\kappa_2} = \vec{0}
	    \label{eq:chapters/12/11/2/16/lsq/m-orth}
    \end{align}
	    from 
	\eqref{eq:chapters/12/11/2/16/lsq/x-diff},
	yielding
    \begin{align}
	    \vec{M}^\top\vec{M}\myvec{\kappa_1\\-\kappa_2} = \vec{M}^\top\brak{\vec{B}-\vec{A}}
        \label{eq:chapters/12/11/2/16/lsq/vec-eqn}
    \end{align}
    This is known as the {\em least squares solution}.
	\item Perform the eigendecompositions 
    \begin{align}
	    \vec{MM}^\top &= \vec{U}\vec{D}_1\vec{U}^\top \label{eq:chapters/12/11/2/16/svd/decomp-1} \\
	    \vec{M}^\top\vec{M} &= \vec{V}\vec{D}_2\vec{V}^\top \label{eq:chapters/12/11/2/16/svd/decomp-2}
    \end{align}
	\item    The following expression is known as {\em singular value decomposition}
    \begin{align}
        \vec{M} = 
	\vec{U}\vec{\Sigma}\vec{V}^\top
        \label{eq:chapters/12/11/2/16/svd/M-svd}
    \end{align}
    where $\vec{\Sigma}$ is diagonal with
    entries obtained as in 
        \eqref{eq:chapters/12/11/2/16/svd/svd-params}.
 Substituting in 
        \eqref{eq:chapters/12/11/2/16/lsq/vec-eqn},
	\begin{align}
\vec{V}\vec{\Sigma}\vec{U}^\top\vec{U}\vec{\Sigma}\vec{V}^\top\bm{\kappa} &= \vec{V}\vec{\Sigma}\vec{U}^\top\brak{\vec{B}-\vec{A}} \\
\implies \vec{V}\vec{\Sigma}^2\vec{V}^\top\bm{\kappa} &= \vec{V}\vec{\Sigma}\vec{U}^\top\brak{\vec{B}-\vec{A}} \\
\implies \bm{\kappa} &= \brak{\vec{V}\vec{\Sigma}^2\vec{V}^\top}^{-1}\vec{V}\vec{\Sigma}\vec{U}^\top\brak{\vec{B}-\vec{A}} \\
\implies \bm{\kappa} &= \vec{V}\vec{\Sigma}^{-2}\vec{V}^\top\vec{V}\vec{\Sigma}\vec{U}^\top\brak{\vec{B}-\vec{A}} \\
\implies \bm{\kappa} &= \vec{V}\vec{\Sigma}^{-1}\vec{U}^\top\brak{\vec{B}-\vec{A}}
\label{eq:chapters/12/11/2/16/svd/kappa-sol}
\end{align}
    where $\vec{\Sigma}^{-1}$ is obtained by inverting the nonzero elements of
    $\vec{\Sigma}$. 
		\item 
	    From \eqref{eq:chapters/12/11/2/16/lsq/x1x2}, 
\begin{align}
	\vec{x}_1-\vec{x}_2 &= 
	\vec{A}+ \kappa_1\vec{m_1}
	 -\vec{B}  - \kappa_2\vec{m_2} 
	 \\
	 &=
	\vec{A} 
	 -\vec{B}  + \vec{M} 
	\bm{\kappa}
\end{align}
which, upon substitution from 
        \eqref{eq:chapters/12/11/2/16/svd/M-svd}
	yields
\begin{align}
	\vec{x}_1-\vec{x}_2 &= 
	\vec{A} 
	 -\vec{B}  + 
\vec{U}\vec{\Sigma}\vec{V}^\top
\vec{V}\vec{\Sigma}^{-1}\vec{U}^\top\brak{\vec{B}-\vec{A}}
\\
	&=
	\brak{	\vec{A} 
	 -\vec{B}}  \brak{\vec{I}- 
\vec{U}\vec{\Sigma}
	\vec{\Sigma}^{-1}\vec{U}^\top}
\end{align}
			Thus, 
    \begin{align}
	    \norm{\vec{x}_1-\vec{x}_2} = 
	\norm{\brak{	\vec{A} 
	 -\vec{B}}  \brak{\vec{I}- 
\vec{U}\vec{\Sigma}
	    \vec{\Sigma}^{-1}\vec{U}^\top}}
        \label{eq:chapters/12/11/2/16/svd/min-sol}
    \end{align}
\item Least squares solution
	\begin{lstlisting}
	codes/book/skew_least.py
\end{lstlisting}
\item Least squares using builtin SVD 
	\begin{lstlisting}
	codes/book/skew_builtin.py
\end{lstlisting}
\item Code linking eigenvalues and singular values
	\begin{lstlisting}
	codes/book/skew_svd.py
\end{lstlisting}
\end{enumerate}

\subsection{Singular Value Decomposition}
\begin{enumerate}[label=\thesubsection.\arabic*,ref=\thesubsection.\theenumi]
    \item Find the shortest distance between the lines whose vector equations are
    \begin{align}
        \vec{x} = \myvec{1\\2\\3} + \lambda_1\myvec{1\\-3\\2}
    \end{align}
    and
    \begin{align}
        \vec{x} = \myvec{4\\5\\6} + \lambda_2\myvec{2\\3\\1}
    \end{align}
    \solution
		    For this problem,
    \begin{align}
        \vec{x} = \vec{x_2} - \vec{x_1} = \myvec{3\\3\\3} \\
        \vec{M} = \myvec{\vec{m_1} & \vec{m_2}} = \myvec{1&2\\-3&3\\2&1} 
    \end{align}
    Thus,
    \begin{align}
        \vec{M}^\top\vec{M} = \myvec{1&-3&2\\2&3&1}\myvec{1&2\\-3&3\\2&1} = \myvec{14&-5\\-5&14} \\
        \vec{MM}^\top = \myvec{1&2\\-3&3\\2&1}\myvec{1&-3&2\\2&3&1} = \myvec{5&3&4\\3&18&-3\\4&-3&5}
    \end{align}
    We perform the eigendecompositions for each matrix and bring them into the form
    \begin{align}
        \vec{MM}^\top &= \vec{P_1D_1P_1}^\top \label{eq:chapters/12/11/2/16/svd/decomp-1} \\
        \vec{M}^\top\vec{M} &= \vec{P_2D_2P_2}^\top \label{eq:chapters/12/11/2/16/svd/decomp-2}
    \end{align}
    \begin{enumerate}
        \item For $\vec{MM}^\top$, the characteristic polynomial is
        \begin{align}
		\text{char}{\vec{MM}^\top} &= \mydet{x-5&-3&-4\\-3&x-18&3\\-4&3&x-5} \\
                                      &= x\brak{x-9}\brak{x-19}
                                      \label{eq:chapters/12/11/2/16/svd/char-2}
        \end{align}
        Thus, the eigenvalues are given by
        \begin{align}
            \lambda_1 = 19,\ \lambda_2 = 9,\ \lambda_3 = 0
        \end{align}
        For $\lambda_1$, the augmented matrix formed from the 
        eigenvalue-eigenvector equation is
        \begin{align}
            &\myvec{-14&3&4&0\\3&-1&-3&0\\4&-3&-14&0} \nonumber \\
            &\xleftrightarrow[]{R_1 \leftarrow \frac{R_1+R_3}{-10}} \myvec{1&0&1&0\\3&-1&-3&0\\4&-3&-14&0} \\
            &\xleftrightarrow[]{R_2 \leftarrow -R_2+3R_1} \myvec{1&0&1&0\\0&1&6&0\\4&-3&-14&0} \\
            &\xleftrightarrow[]{R_3 \leftarrow R_3-4R_1} \myvec{1&0&1&0\\0&-1&-6&0\\0&-3&-18&0} \\
            &\xleftrightarrow[]{R_3 \leftarrow R_3-3R_2} \myvec{1&0&1&0\\0&-1&-6&0\\0&0&0&0}
        \end{align}
        Hence, the normalized eigenvector is
        \begin{align}
            \vec{p_1} = \frac{1}{\sqrt{38}}\myvec{-1\\-6\\1}
        \end{align}
        For $\lambda_2$, the augmented matrix formed from the 
        eigenvalue-eigenvector equation is
        \begin{align}
            &\myvec{-4&3&4&0\\3&9&-3&0\\4&3&-4&0} \nonumber \\
            &\xleftrightarrow[]{R_3 \leftarrow R_1+R_3} \myvec{-4&3&4&0\\3&9&-3&0\\0&0&0&0} \\
            &\xleftrightarrow[]{R_2 \leftarrow \frac{4R_2+3R_1}{45}} \myvec{-4&3&4&0\\0&1&0&0\\0&0&0&0} \\
            &\xleftrightarrow[]{R_1 \leftarrow \frac{R_1-3R_2}{-4}} \myvec{1&0&-1&0\\0&1&0&0\\0&0&0&0}
        \end{align}
        Hence, the normalized eigenvector is
        \begin{align}
            \vec{p_2} = \frac{1}{\sqrt{2}}\myvec{1\\0\\1}
        \end{align}
        For $\lambda_3$, the augmented matrix formed from the 
        eigenvalue-eigenvector equation is
        \begin{align}
            &\myvec{5&3&4&0\\3&18&-3&0\\4&-3&5&0} \nonumber \\ 
            &\xleftrightarrow[]{R_1 \leftarrow \frac{R_1+R_3}{9}} \myvec{1&0&1&0\\3&18&-3&0\\4&-3&5&0} \\
            &\xleftrightarrow[]{R_2 \leftarrow R_2-3R_1} \myvec{1&0&1&0\\0&18&-6&0\\4&-3&5&0} \\
            &\xleftrightarrow[]{R_3 \leftarrow R_3-4R_1} \myvec{1&0&1&0\\0&18&-6&0\\0&-3&1&0} \\
            &\xleftrightarrow[]{R_2 \leftarrow \frac{R_2}{6}} \myvec{1&0&1&0\\0&3&-1&0\\0&-3&1&0} \\
            &\xleftrightarrow[]{R_3 \leftarrow R_3+R_2} \myvec{1&0&1&0\\0&3&-1&0\\0&0&0&0}
        \end{align}
        Hence, the normalized eigenvector is
        \begin{align}
            \vec{p_3} = \frac{1}{\sqrt{19}}\myvec{-3\\1\\3}
        \end{align}
        Using \eqref{eq:chapters/12/11/2/16/svd/decomp-1}, we see that
        \begin{align}
            \vec{P_1} = \myvec{-\frac{1}{\sqrt{38}}&\frac{1}{\sqrt{2}}&-\frac{3}{\sqrt{19}}\\-\frac{6}{\sqrt{38}}&0&\frac{1}{\sqrt{19}}\\\frac{1}{\sqrt{38}}&-\frac{1}{\sqrt{2}}&\frac{3}{\sqrt{19}}} \\
            \vec{D_1} = \myvec{19&0&0\\0&9&0\\0&0&0}
            \label{eq:chapters/12/11/2/16/svd/eig-params-1}
        \end{align}
        \item For $\vec{M}^\top\vec{M}$, the characteristic polynomial is
        \begin{align}
		\text{char}{\vec{M}^\top\vec{M}} &= \mydet{x-14&5\\5&x-14} \\
                                      &= \brak{x-9}\brak{x-19}
%                                      \label{eq:chapters/12/11/2/16/svd/char-1}
        \end{align}
        Thus, the eigenvalues are given by
        \begin{align}
            \lambda_1 = 19,\ \lambda_2 = 9
        \end{align}
        For $\lambda_1$, the augmented matrix formed from the 
        eigenvalue-eigenvector equation is
        \begin{align}
            \myvec{-5&-5&0\\-5&-5&0} &\xleftrightarrow[]{R_1 \leftarrow R_1-R_2} \myvec{0&0&0\\-5&-5&0}
        \end{align}
        Hence, the normalized eigenvector is
        \begin{align}
            \vec{p_1} = \frac{1}{\sqrt{2}}\myvec{1\\-1}
        \end{align}
        For $\lambda_2$, the augmented matrix formed from the 
        eigenvalue-eigenvector equation is
        \begin{align}
            \myvec{5&-5&0\\-5&5&0} &\xleftrightarrow[]{R_1 \leftarrow R_1+R_2} \myvec{0&0&0\\5&-5&0}
        \end{align}
        Hence, the normalized eigenvector is
        \begin{align}
            \vec{p_2} = \frac{1}{\sqrt{2}}\myvec{1\\1}
        \end{align}
        Thus, from \eqref{eq:chapters/12/11/2/16/svd/decomp-2},
        \begin{align}
            \vec{P_2} = \myvec{\frac{1}{\sqrt{2}}&-\frac{1}{\sqrt{2}}\\\frac{1}{\sqrt{2}}&\frac{1}{\sqrt{2}}},\ \vec{D_2} = \myvec{9&0\\0&19}
            \label{eq:chapters/12/11/2/16/svd/eig-params-2}
        \end{align}
    \end{enumerate}
    Therefore, from \eqref{eq:chapters/12/11/2/16/svd/M-svd},
    \begin{align}
        \vec{U} &= \vec{P_1} \\ 
        \vec{V} &= \vec{P_2} \\
        \vec{\Sigma} &= \myvec{\sqrt{19}&0\\0&3\\0&0}
        \label{eq:chapters/12/11/2/16/svd/svd-params}
    \end{align}
    and substituting into \eqref{eq:chapters/12/11/2/16/svd/lambda-sol}, we get
    \begin{align}
        \vec{\lambda} = \frac{1}{19}\myvec{10\\28}
    \end{align}
    which agrees with earlier solutions as well.
    \iffalse
    The Python code
    \texttt{codes/svd.py} plots 
    \fi
    See Fig. \ref{fig:chapters/12/11/2/16/svd/svd} depicting the situation.
    \begin{figure}[!ht]
        \centering
        \includegraphics[width=\columnwidth]{chapters/12/11/2/16/svd/figs/skew_svd.png}
        \caption{Finding the shortest distance between two lines using SVD.}
        \label{fig:chapters/12/11/2/16/svd/svd}
    \end{figure}

\item Find the shortest distance between the lines $l_1$ and $l_2$ whose vector equations are 
\begin{align}
	\overrightarrow{r} &= \hat{i}+\hat{j}+\lambda(2\hat{i}-\hat{j}+\hat{k}) \text{ and}
	\\
	\overrightarrow{r} &= 2\hat{i}+\hat{j}-\hat{k}+\mu(3\hat{i}-5\hat{j}+2\hat{k}).
\end{align}
    \solution
		\begin{enumerate}
\item 
To check whether the given lines are skew,
from \eqref{eq:chapters/12/11/2/e11/params}
and 
	    \eqref{eq:chapters/12/11/2/16/lsq/rank},

\begin{align*}
\myvec{2&3&&1\\-1&-5&&0\\1&2&&-1}
\xleftrightarrow[R_3 \leftarrow R_3 - \frac{1}{2}R_1]{R_2 \leftarrow R_2 + \frac{1}{2}R_1}
\myvec{2&3&&1\\[1ex]0&-\frac{7}{2}&&\frac{1}{2}\\[1ex]0&\frac{1}{2}&&-\frac{3}{2}}\\
\xleftrightarrow{R_3 \leftarrow R_3 + 7R_2}
\myvec{2&3&&1\\[1ex]0&-\frac{7}{2}&&\frac{1}{2}\\[1ex]0&0&&-10}
\end{align*}
The rank of the matrix is 3. So the given lines are skew.
\item 
\begin{align}
\vec{M}^\top\vec{M} &= \myvec{2&-1&1\\3&-5&2}\myvec{2&3\\-1&-5\\1&2} \\ 
&= \myvec{6&13\\13&38} \label{eq:chapters/12/11/2/311/svd/MtM}
\end{align}
\begin{align}
\vec{MM}^\top &= \myvec{2&3\\-1&-5\\1&2}\myvec{2&-1&1\\3&-5&2}\\
&= \myvec{13&-17&8\\-17&26&-11\\8&-11&5} \label{eq:chapters/12/11/2/311/svd/MMt}
\end{align}
The characteristic polynomial of the matrix $\vec{MM}^\top$ is given by,
\begin{align}
\text{char}\brak{\vec{MM}^\top} &= \mydet{13-\lambda&-17&8\\-17&26-\lambda&-11\\8&-11&5-\lambda} \\
&= -\lambda^3 + 44\lambda^2-59\lambda
%\label{eq:chapters/12/11/2/311/svd/char-1}
\end{align}
resulting in 
\begin{align}
    \vec{U} &= \myvec{\frac{12-\sqrt{17}}{\sqrt{5}\sqrt{68-6\sqrt{17}}} & \frac{12+\sqrt{17}}{\sqrt{5}\sqrt{68+6\sqrt{17}}} & -\frac{3}{\sqrt{59}}\\
    \frac{1-3\sqrt{17}}{\sqrt{5}\sqrt{68-6\sqrt{17}}}&\frac{1+3\sqrt{17}}{\sqrt{5}\sqrt{68+6\sqrt{17}}} & \frac{1}{\sqrt{59}}\\
\frac{\sqrt{5}}{\sqrt{68-6\sqrt{17}}}&\frac{\sqrt{5}}{\sqrt{68+6\sqrt{17}}} & \frac{7}{\sqrt{59}} }
    \label{eq:chapters/12/11/2/311/svd/eig-params-1(a)}
\end{align}
and 
\begin{align}
	\vec{D}_1 &= \myvec{22+5\sqrt{17}&0&0\\0&22-5\sqrt{17}&0\\0&0&0}
    \label{eq:chapters/12/11/2/311/svd/eig-params-1(b)}
\end{align}
For $\vec{M}^\top\vec{M}$, the characteristic polynomial is
\begin{align}
    \text{char}\brak{\vec{M}^\top\vec{M}} &= \mydet{6-\lambda&13\\13&38-\lambda} \\&= \lambda^2-44\lambda+59
    \label{eq:chapters/12/11/2/311/svd/char-1}
\end{align}
Thus, the eigenvalues are given by
\begin{align}
    \lambda_1 = 22+5\sqrt{17},\ \lambda_2 = 22-5\sqrt{17}
\end{align}
resulting in 
\begin{align}
    \vec{V} &= \myvec{\frac{-16-5\sqrt{17}}{\sqrt{850+160\sqrt{17}}}&\frac{13}{\sqrt{850-160\sqrt{17}}}\\\frac{13}{\sqrt{850+160\sqrt{17}}}&\frac{-16+5\sqrt{17}}{\sqrt{850-160\sqrt{17}}}}
     \label{eq:chapters/12/11/2/311/svd/eig-params-2(a)}\\ 
	\vec{D}_2 &= \myvec{22-5\sqrt{17}&0\\0&22+5\sqrt{17}}
    \label{eq:chapters/12/11/2/311/svd/eig-params-2(b)}
\end{align}
Therefore, 
\begin{align}
    \vec{\Sigma} &= \myvec{\sqrt{22+5\sqrt{17}}&0\\0&\sqrt{22-5\sqrt{17}}\\0&0}
    \label{eq:chapters/12/11/2/311/svd/svd-params}
\end{align}
and substituting into 
        \eqref{eq:chapters/12/11/2/16/svd/min-sol},
\begin{align}
	\bm{\lambda} =  \myvec{\frac{25}{59}\\[1ex]-\frac{7}{59}}
\end{align}
which agrees with 
	\eqref{eq:chapters/12/11/2/e11/}.
\end{enumerate} 

\item Find the shortest distance between the lines given by 
\begin{align}
	\overrightarrow{r}&=(8+3\lambda\hat{i}-(9+16\lambda)\hat{j}+(10+7\lambda)\hat{k} \text{ and} 
	\\
	\overrightarrow{r}&=15\hat{i}+29\hat{j}+5\hat{k}+\mu(3\hat{i}+8\hat{j}-5\hat{k}).
\end{align}
\item Find the shortest distance between the lines
\begin{align}
	\overrightarrow{r}&=(\hat{i}+2\hat{j}+\hat{k})+\lambda(\hat{i}-\hat{j}+\hat{k}) \text{ and} 
	\\
	\overrightarrow{r}&=2\hat{i}-\hat{j}-\hat{k}+\mu(2\hat{i}+\hat{j}+2\hat{k})
\end{align}
\item Find the shortest distance between the lines
\begin{align}
	\frac{x+1}{7}&=\frac{y+1}{-6}=\frac{z+1}{1} \text{ and}
	\\
	\frac{x-3}{1}&=\frac{y-5}{-2}=\frac{z-7}{1}
\end{align}
\end{enumerate}

\subsection{Formulae}
\begin{enumerate}[label=\thesubsection.\arabic*.,ref=\thesubsection.\theenumi]
	\item Perform the eigendecompositions 
    \begin{align}
	    \vec{MM}^\top &= \vec{U}\vec{D}_1\vec{U}^\top \label{eq:chapters/12/11/2/16/svd/decomp-1} \\
	    \vec{M}^\top\vec{M} &= \vec{V}\vec{D}_2\vec{V}^\top \label{eq:chapters/12/11/2/16/svd/decomp-2}
    \end{align}
	\item    The following expression is known as {\em singular value decomposition}
    \begin{align}
        \vec{M} = 
	\vec{U}\vec{\Sigma}\vec{V}^\top
        \label{eq:chapters/12/11/2/16/svd/M-svd}
    \end{align}
    where $\vec{\Sigma}$ is diagonal with
    entries obtained as in 
        \eqref{eq:chapters/12/11/2/16/svd/svd-params}.
 Substituting in 
        \eqref{eq:chapters/12/11/2/16/lsq/vec-eqn},
	\begin{align}
\vec{V}\vec{\Sigma}\vec{U}^\top\vec{U}\vec{\Sigma}\vec{V}^\top\bm{\kappa} &= \vec{V}\vec{\Sigma}\vec{U}^\top\brak{\vec{B}-\vec{A}} \\
\implies \vec{V}\vec{\Sigma}^2\vec{V}^\top\bm{\kappa} &= \vec{V}\vec{\Sigma}\vec{U}^\top\brak{\vec{B}-\vec{A}} \\
\implies \bm{\kappa} &= \brak{\vec{V}\vec{\Sigma}^2\vec{V}^\top}^{-1}\vec{V}\vec{\Sigma}\vec{U}^\top\brak{\vec{B}-\vec{A}} \\
\implies \bm{\kappa} &= \vec{V}\vec{\Sigma}^{-2}\vec{V}^\top\vec{V}\vec{\Sigma}\vec{U}^\top\brak{\vec{B}-\vec{A}} \\
\implies \bm{\kappa} &= \vec{V}\vec{\Sigma}^{-1}\vec{U}^\top\brak{\vec{B}-\vec{A}}
\label{eq:chapters/12/11/2/16/svd/kappa-sol}
\end{align}
    where $\vec{\Sigma}^{-1}$ is obtained by inverting the nonzero elements of
    $\vec{\Sigma}$. 
		\item 
	    From \eqref{eq:chapters/12/11/2/16/lsq/x1x2}, 
\begin{align}
	\vec{x}_1-\vec{x}_2 &= 
	\vec{A}+ \kappa_1\vec{m_1}
	 -\vec{B}  - \kappa_2\vec{m_2} 
	 \\
	 &=
	\vec{A} 
	 -\vec{B}  + \vec{M} 
	\bm{\kappa}
\end{align}
which, upon substitution from 
        \eqref{eq:chapters/12/11/2/16/svd/M-svd}
	yields
\begin{align}
	\vec{x}_1-\vec{x}_2 &= 
	\vec{A} 
	 -\vec{B}  + 
\vec{U}\vec{\Sigma}\vec{V}^\top
\vec{V}\vec{\Sigma}^{-1}\vec{U}^\top\brak{\vec{B}-\vec{A}}
\\
	&=
	\brak{	\vec{A} 
	 -\vec{B}}  \brak{\vec{I}- 
\vec{U}\vec{\Sigma}
	\vec{\Sigma}^{-1}\vec{U}^\top}
\end{align}
			Thus, 
    \begin{align}
	    \norm{\vec{x}_1-\vec{x}_2} = 
	\norm{\brak{	\vec{A} 
	 -\vec{B}}  \brak{\vec{I}- 
\vec{U}\vec{\Sigma}
	    \vec{\Sigma}^{-1}\vec{U}^\top}}
        \label{eq:chapters/12/11/2/16/svd/min-sol}
    \end{align}
\end{enumerate}

\newpage
\section{Circle}
\subsection{Equation}
\begin{enumerate}[label=\thesubsection.\arabic*, ref=\thesubsection.\theenumi]
\item Find the coordinates of a point $\vec{A}$, where $AB$ is the diameter of a circle whose centre is $ \vec{C}(2,-3)$  and  $\vec{B}$ is $(1,4)$.
	\\
		\solution
		\begin{align}
	\vec{C} = \frac{\vec{A+B}}{2} 
	\implies 	\vec{A} = 2\vec{C}-\vec{B} 
	 = \myvec{3\\-10\\}	
	\end{align}       
	See 
\figref{fig:chapters/10/7/2/7Fig}.
\begin{figure}[!h]
\begin{center}	
	\includegraphics[width=\columnwidth]{chapters/10/7/2/7/figs/Vector1.png}
\end{center}
\caption{}
\label{fig:chapters/10/7/2/7Fig}
\end{figure}
	

  \item Find the equation of the circle passing through the points $(4, 1)$ and $(6, 5)$ and whose centre is on the line $ 4x+y=16. $
\label{chapters/11/11/1/10}
\\
\solution
	Following 
\appref{prop:chapters/11/11/1/pts},
\begin{align}
	\label{eq:chapters/11/11/1/10circPoints}
	\vec{x}_{1} = \myvec{4\\1},\ \vec{x}_{2} = \myvec{6\\5},\
	\vec{n} = \myvec{4 \\ 1},\  c= -16.
\end{align}
Substituting  in
	\eqref{eq:chapters/11/11/1/mat},
\begin{align}
	\myvec{
	        8 &  2 & 1\\
	       12 & 10 & 1\\
-4 & -1 & 0}
	\myvec{\vec{u}\\f} = 
	\myvec{16 \\ -61 \\ -17}
\end{align}
The augmented matrix is expressed as
\begin{align}
	\myvec{
	        8 &  2 & 1 & \vrule & -17\\
	       12 & 10 & 1 & \vrule & -61\\
-4 & -1 & 0 & \vrule & 16}
\end{align}
Performing a sequence of row operations to transform into an Echelon form
\begin{align*}
	\xleftrightarrow[R_2\rightarrow R_2+3R_1]{{R_3\rightarrow R_3+2R_1}}
	\myvec{-4 & -1 & 0 & \vrule & 16\\
	        0 &  7 & 1 & \vrule & -13\\
	        0 &  0 & 1 & \vrule & 15}
	\xleftrightarrow[]{{R_2\rightarrow R_2-R_3}}
	\myvec{-4 & -1 & 0 & \vrule & 16\\
	        0 &  7 & 0 & \vrule & -28\\
	        0 &  0 & 1 & \vrule & 15}\\
	\xleftrightarrow[]{{R_2\rightarrow \frac{R_2}{7},R_1\rightarrow \frac{-R_1}{4}}}
	\myvec{ 1 & \frac{1}{4} & 0 & \vrule & -4\\
	        0 &  1 & 0 & \vrule & -4\\
	        0 &  0 & 1 & \vrule & 15}
	\label{eq:chapters/11/11/1/10solution}	
	\xleftrightarrow[]{{R_1\rightarrow R_1-\frac{1}{4}R_2}}
	\myvec{ 1 &  0 & 0 & \vrule & -3\\
	        0 &  1 & 0 & \vrule & -4\\
	        0 &  0 & 1 & \vrule & 15}
\end{align*}
So, from \eqref{eq:chapters/11/11/1/10solution}
\begin{align}
	\vec{u} = -\myvec{3\\4},\
	f = 15.
\end{align}
See \figref{fig:chapters/11/11/1/10Fig1}.
\begin{figure}[H]
	\begin{center} 
	    \includegraphics[width=0.75\columnwidth]{chapters/11/11/1/10/figs/fig.pdf}
	\end{center}
\caption{}
\label{fig:chapters/11/11/1/10Fig1}
\end{figure}






  \item Find the equation of the circle passing through the points $\vec{x}_1(2, 3)$ and $\vec{x}_2(-1, 1)$ and whose centre is on the line $x-3y-11=0$.
\label{chapters/11/11/1/11}
\\
\solution 
Substituting numerical values in 
	\eqref{eq:chapters/11/11/1/mat},
\begin{align}
	\label{eq:vertex_system}
	\myvec{4&6&1\\-2& 2&1\\-1& 3&0}\myvec{\vec{u}\\f} = \myvec{-13\\-2 \\11}
\end{align}
yielding
\begin{align}
	\vec{u}=\frac{1}{2}\myvec{-7 \\5},\
f=-14.
\end{align}
\iffalse
See 
		\figref{fig:11/11/1/11}.
	\begin{figure}[H]
		\centering
 \includegraphics[width=0.75\columnwidth]{chapters/11/11/1/11/figs/fig.pdf}
		\caption{}
		\label{fig:11/11/1/11}
  	\end{figure}
	\fi

  \item Find the equation of the circle with radius 5 whose centre lies on the $X$ axis and passes through the point $(2, 3)$.
\label{chapters/11/11/1/12}
\\
\solution 
See 
		\figref{fig:11/11/1/12}.
	\begin{figure}[H]
		\centering
 \includegraphics[width=0.75\columnwidth]{chapters/11/11/1/12/figs/fig.pdf}
		\caption{}
		\label{fig:11/11/1/12}
  	\end{figure}
From the given information, the following equations can be formulated
using 
	\eqref{eq:circ-eq}.
\begin{align}
		\label{eq:11/11/1/12/1}
	\norm{\vec{P}}^2 + 2 \vec{u}^{\top}\vec{P} + f &= 0
	\\
		\label{eq:11/11/1/12/2}
	\vec{u} &= k\vec{e}_1
	\\
		\label{eq:11/11/1/12/3}
	\norm{\vec{u}}^2 - f &= r^2
\end{align}
where 
\begin{align}
	\vec{P} = \myvec{2\\3} \text{ and } r = 5
\end{align}
From 
		\eqref{eq:11/11/1/12/1}
		and 
		\eqref{eq:11/11/1/12/3},
\begin{align}
	\norm{\vec{P}}^2 + 2 \vec{u}^{\top}\vec{P} + \norm{\vec{u}}^2 &= r^2
\end{align}
Substituting from 
		\eqref{eq:11/11/1/12/2} in the above, 
\begin{align}
	k^2  + 2k \vec{e}_1^{\top}\vec{P} + \norm{\vec{P}}^2- r^2 = 0
\end{align}
resulting in 
\begin{align}
	k =  - \vec{e}_1^{\top}\vec{P} \pm \sqrt{\brak{{ \vec{e}_1^{\top}\vec{P}  }}^2 + r^2 - \norm{\vec{P}}^2 } 
\end{align}
Substituting numerical values, 
\begin{align}
	k = 2, -6
\end{align}
resulting in circles with centre
\begin{align}
	-\vec{u} = \myvec{-2 \\ 0} \text{ or } \myvec{6 \\ 0}.
\end{align}
This is verified in Fig. 
		\eqref{fig:11/11/1/12}.

  \item Find the equation of a circle with centre $(2, 2)$ and passing through the point $(4, 5)$.
\label{chapters/11/11/1/14}
\\
\solution
From the given information
\begin{align}
	\vec{u} = -\myvec{2\\2}, \, \vec{A} &= \myvec{4\\5}\\
\implies	\norm{\vec{A}}^2+2\vec{u}^\top\vec{A}+f &= 0\\
\implies	f = -\norm{\vec{A}}^2 - 2\vec{u}^\top\vec{A}
	&= -5
\end{align}
Hence the equation of circle is 
\begin{align}
	\norm{\vec{x}}^2+2\myvec{-2&-2}\vec{x}-5 = 0 	
\end{align}
\iffalse
See 
\figref{fig:chapters/11/11/1/14/1}.
\begin{figure}[H]
\centering
\includegraphics[width=0.75\columnwidth]{chapters/11/11/1/14/figs/fig.png}
\caption{}
\label{fig:chapters/11/11/1/14/1}
\end{figure}
\fi







  \item Does the point $(-2.5, 3.5)$ lie inside,  outside or on the circle $x^{2}+y^{2}=25?$
\\
\solution
See 
\tabref{tab:chapters/11/11/1/15/}.
\begin{table}[H]
\begin{center}
\input{chapters/11/11/1/15/tables/table_1.tex}
\end{center}
\caption{}
\label{tab:chapters/11/11/1/15/}
\end{table}
The given circle equation can be expressed as
\begin{align}
	\norm{\vec{x}}^2= 25
\end{align}
Let,
\begin{align}
	\vec{P}=\myvec{-2.5\\3.5}
\end{align}
Since
\begin{align}
	\norm{\vec{P} - \vec{O}}^2 =
 18.5 < 25,
\end{align}
the point lies inside the given circle.
See 
    \figref{fig:chapters/11/11/1/15/}.
\begin{figure}[H]
  \centering
    \includegraphics[width=0.75\columnwidth]{chapters/11/11/1/15/figs/fig.pdf}
    \caption{}
    \label{fig:chapters/11/11/1/15/}
\end{figure}

\item Find the centre of a circle passing though the points $(6, -6),  (3, -7)$ and $(3, 3)$. \\ 
\label{chapters/10/7/4/3}
\solution 
Substituting numerical values in 
	\eqref{eq:chapters/11/11/1/mat-3},
\begin{align}
 \myvec{6 & -14 & 1 \\
	12 & -12 & 1 \\
	6 & 6 & 1
	} \myvec {\vec{u} \\
	           f 
		}  = \myvec{-58 \\ -72 \\ -18 }
\end{align}
yielding
\begin{align}
	\vec{u} = \myvec{-3 \\ 2} \\ 
	f = -12 
\end{align}
See \figref{fig:10/7/4/3Fig1}.
\begin{figure}[H]
	\begin{center}
		\includegraphics[width=0.75\columnwidth]{chapters/10/7/4/3/figs/fig.pdf}
	\end{center}
\caption{}
\label{fig:10/7/4/3Fig1}
\end{figure}

  \item Find the equation of the circle passing through $(0, 0)$ and making intercepts $a$ and $b$ on the coordinate axes.
\end{enumerate}
In each of the following exercises,  find the equation of the circle with the following parameters
\begin{enumerate}[label=\thesubsection.\arabic*, ref=\thesubsection.\theenumi, resume*]
 \item centre $(0, 2)$ and radius $2$
	 \\
		\solution
\label{chapters/11/11/1/1}
Substituting numerical values in 
	\eqref{eq:circ-cr},
\begin{align}
	\vec{u} = \myvec{0\\-2},
	f 
	  = 0
\end{align}
Thus, the equation of circle is obtained as
\begin{align}
	\norm{\vec{x}}^2 - 2\myvec{0 & 2}\vec{x} = 0
\end{align}
\iffalse
See \figref{fig:11/11/1/1/Fig1}.
\begin{figure}[H]
	\begin{center} 
	    \includegraphics[width=0.75\columnwidth]{chapters/11/11/1/1/figs/circ1}
	\end{center}
\caption{}
\label{fig:11/11/1/1/Fig1}
\end{figure}
\fi

%
  \item centre $(-2, 3)$ and radius 4
	 \\
		\solution
\label{chapters/11/11/1/2}
Given
\begin{align}
	\vec{u} = -\myvec{-2\\3},  r = 4.
\end{align}
Substituting in 
	\eqref{eq:circ-cr},
\begin{align}
	f = -3
\end{align}
The equation of the circle is then obtained as
\begin{align}
	\norm{\vec{x}}^2 + 2\myvec{2&-3}\vec{x} -3=0     		       
\end{align}	
\iffalse
See  
\figref{fig:chapters/11/11/1/2/Fig1}.
\begin{figure}[H]
	\begin{center} 
	    \includegraphics[width=0.75\columnwidth]{chapters/11/11/1/2/figs/circle.png}
	\end{center}
\caption{}
\label{fig:chapters/11/11/1/2/Fig1}
\end{figure}
\fi


  \item centre $\left(\frac{1}{2},  \frac{1}{4}\right)$ and radius $\frac{1}{12}$
\label{chapters/11/11/1/3}
	 \\
		\solution
Substituting numerical values
	in \eqref{eq:circ-cr},
\begin{align}
	f
	=\frac{11}{36}
\end{align}
	Thus, the equation of the circle is
\begin{align}
	\norm{\vec{x}}^2 + \myvec{-1 & -\frac{1}{2}}\vec{x}+\frac{11}{36}=0
\end{align}
\iffalse
See \figref{fig:chapters/11/11/1/3/Fig1}.
\begin{figure}[H]
\begin{center}
\includegraphics[width=0.75\columnwidth]{chapters/11/11/1/3/figs/fig.pdf}
\end{center}
\caption{}
\label{fig:chapters/11/11/1/3/Fig1}
\end{figure}
\fi

  \item centre $(1, 1)$ and radius $\sqrt{2}$
	 \\
		\solution
Substituting
\begin{align}
	 r = \sqrt{2},\
	\vec{u}
	 = \myvec{-1\\-1}
\end{align}
in 
	\eqref{eq:circ-cr},
\begin{align}
	f 
	  =0	
\end{align}
Thus, the equation of the circle is 
\begin{align}
	\norm{\vec{x}}^2 -2\myvec{1&1}\vec{x} = 0       		       
\end{align}	
\iffalse
See 
\figref{fig:chapters/11/11/1/4/Fig1}.
\begin{figure}[H]
	\begin{center} 
	  \includegraphics[width=0.75\columnwidth]{chapters/11/11/1/4/figs/circ.png}
	\end{center}
\caption{}
\label{fig:chapters/11/11/1/4/Fig1}
\end{figure}
\fi

\end{enumerate}
In each of the following exercises,   find the centre and radius of the circles.
\begin{enumerate}[label=\thesubsection.\arabic*, ref=\thesubsection.\theenumi, resume*]
\item  $x^2+y^2 +10x -6y -2=0$. 
	 \\
		\solution
\label{chapters/11/11/1/6}
The circle parameters are
\begin{align}
 \vec{u}=\myvec{5\\ -3},\,
 f&=-2\\
\implies \vec{c}=\myvec{-5 \\ 3},\,
	r=\sqrt{\norm{\vec{u}}^2-f}
&= 6
\end{align}
\iffalse
See Fig. 
\ref{fig:chapters/11/11/1/6/Fig1}.
\begin{figure}[H]
	\begin{center} 
	   \includegraphics[width=0.75\columnwidth]{chapters/11/11/1/6/figs/fig.pdf}
	\end{center}
\caption{}
\label{fig:chapters/11/11/1/6/Fig1}
\end{figure}
\fi

\item  $x^{2}+y^{2}-4 x-8 y-45=0$
	 \\
		\solution
\label{chapters/11/11/1/7}
The given circle can be expressed as
\begin{align}
    \label{eq:chapters/11/11/1/7/given} 
    \norm{\vec{x}}^2  + 2\myvec{-2 & -4}\vec{x} - 45 = 0
\end{align}
where
\begin{align}	
	\vec{u} &= \myvec{-2\\-4},\, f = -45 \\
	\implies \vec{c} &= \myvec{2\\4},\,
	r = \sqrt{65}.	
\end{align}
\iffalse
See Fig. 
    \ref{fig:chapters/11/11/1/7/cicle}.
\begin{figure}[H]
    \centering
    \includegraphics[width=0.75\columnwidth]{chapters/11/11/1/7/figs/circle.png}
    \caption{}
    \label{fig:chapters/11/11/1/7/cicle}
\end{figure}
\fi


\item  $x^{2}+y^{2}-8 x+10 y-12=0$ 
	 \\
		\solution
\label{chapters/11/11/1/8}
From the given informtion,
\begin{align}
 \vec{u}=\myvec{-4\\5},\,
 f&=-12\\
\implies \vec{c}&=\myvec{4 \\ -5},\\
	r=\sqrt{\norm{\vec{u}}^2-f}
&=\sqrt{53}
\end{align}
\iffalse
See Fig. 
\ref{fig:chapters/11/11/1/8/Fig1}.
\begin{figure}[H]
	\begin{center} 
	   \includegraphics[width=0.75\columnwidth]{chapters/11/11/1/8/figs/11.1.8.png}
	\end{center}
\caption{}
\label{fig:chapters/11/11/1/8/Fig1}
\end{figure}
\fi

\item  $2 x^{2}+2 y^{2}-x=0$
	 \\
		\solution
\label{chapters/11/11/1/9}
The given equation can be expressed as 
\begin{align}
\norm{\vec{x}}^2+2\myvec{\frac{-1}{4} & 0}\vec{x}&=0
\end{align}	
The centre of circle is then given by 
\begin{align}
	\vec{u} = -\vec{c} 
=\myvec{\frac{1}{4}\\0}
\end{align}
and the radius of circle is obtained as
\begin{align}
	r=\sqrt{\norm{\vec{u}}^2 -f}
=\frac{1}{4}
\end{align}
\iffalse
See 
  \figref{fig:chapters/11/11/1/9/Figure}.
\begin{figure}[H]
\includegraphics[width=0.75\columnwidth]{chapters/11/11/1/9/figs/fig.png}
\caption{}
  \label{fig:chapters/11/11/1/9/Figure}
\end{figure}
\fi

\item The area of the circle centred at (1, 2) and passing through (4, 6) is
\begin{enumerate}
\item 5$\pi$ 
\item 10$\pi$
 \item 25$\pi$ 
\item none of these
\end{enumerate}
\item Equation of the circle with centre on the $Y$ axis and passing through the orgin and the point (2, 3) is
\begin{enumerate}
\item $x^2+y^2+6x+6y+3=0$ 
\item $x^2+y^2-6x-6y-9=0$
\item $x^2+y^2-6x-6y+9=0$
\item none of these
\end{enumerate}
\item Equation of the circle with centre on the  $Y$ axis and passing through the origin and the point (2, 3) is  
\begin{enumerate}
\item $x^2+y^2+13y=0$
\item $3x^2+3y^2+13x+3=0$
\item $6x^2+6y^2-13x=0$
\item $x^2+y^2+13x+3=0$
\end{enumerate}
 \item Find the equation of a circle concentric with the circle $x^2+y^2-6x+12y+15=0$ and has double of its area.
 \item If one end of a diameter of the circle $x^2+y^2-4x-6y+11 =0$ is (3, 4),  then find the coordinate of the other end of the diameter.
 \item Find the equation of the circle having (1, -2) as its centre  and passing through the intersection of $3x+y=14,  2x+5y=18$.
\item If the lines $2x-3y=5$ and $3x-4y=7$ are the diameters of a circle of area 154 square units,  then obtain the equation of the circle.
\item Find the equation of the circle which passes through the points (2, 3) and (4, 5) and the centre lies on the straight line $y-4x+3=0$.
\item Find the equation of a circle passing through the point (7, 3) having radius 3 units and whose centre lies on the line $y=x-1$.
\item The centre of a circle is $(2a,  a-7)$. Find the values of $a$ if the circle passes through the point $(11,  -9)$ and has diameter $10\sqrt{2}$ units.
\item A circle has its centre at the origin and a point $\vec{P}(5,  0)$ lies on it. The point $\vec{Q}(6,  8)$ lies outside the circle.
 \item The point $\vec{P}(-2,  4)$ lies on circle of radius 6 and center $\vec{C}(3,  5)$.
\item A circle drawn with origin as the
centre passes through 
		$\brak{\frac{13}{2},  0}$. The
point which does not lie in the
interior of the circle is
\begin{enumerate}
\item $\brak{\frac{-3}{4},  1}$
\item $\brak{2,  \frac{7}{3}}$
\item $\brak{5,  \frac{-1}{2}}$
\item $\brak{-6,  \frac{-5}{2}}$
\end{enumerate}
\end{enumerate}
State whether the statements are True or False 
\begin{enumerate}[label=\thesubsection.\arabic*, ref=\thesubsection.\theenumi, resume*]
\item The line $x+3y=0$ is a diameter of the circle $x^2+y^2+6x+2y=0$.
\item The point (1, 2) lies inside the circle $x^2+y^2-2x+6y+1=0$.
\end{enumerate}

\subsection{Formulae}
\begin{enumerate}[label=\thesubsection.\arabic*.,ref=\thesubsection.\theenumi]
\item For a circle with centre $\vec{c}$ and radius r,
\begin{align}
	\vec{u} = -\vec{c}, f = \norm{\vec{u}}^2 - r^2
	\label{eq:circ-cr}
\end{align}
\item Given  points
\label{prop:chapters/11/11/1/pts},
	$\vec{x}_{1},\  \vec{x}_{2} $
on the circle and the diameter
\begin{align}
	\vec{n}^\top \vec{x} &= c,
	\label{eq:chapters/11/11/1/dia}
\end{align}
the centre is given by
\begin{align}
\myvec{
 2 \vec{x}_1 & 2 \vec{x}_2 & \vec{n}
 \\
 1 & 1 & 0
 }^\top 
	\myvec{\vec{u} \\ f}
	=
-\myvec{ 	\norm{\vec{x}_1}^2 
\\
 	\norm{\vec{x}_2}^2 	
	\\
	c     
	}
	\label{eq:chapters/11/11/1/mat}
                     \end{align}
\solution
	From 
	\eqref{eq:circ-eq},
\begin{align}
\begin{split}
	\norm{\vec{x}_1}^2 + 2 \vec{u}^{\top}\vec{x}_1 + f = 0
	\\
	\norm{\vec{x}_2}^2 + 2 \vec{u}^{\top}\vec{x}_2 + f = 0
\end{split}
	\label{eq:chapters/11/11/1/poits}
\end{align}
and 
	\eqref{eq:chapters/11/11/1/dia}
	can be expressed as
\begin{align}
	\vec{u}^\top \vec{n} &= -c
	\label{eq:chapters/11/11/1/dia-u}
\end{align}
Clubbing 
	\eqref{eq:chapters/11/11/1/poits}
	and 
	\eqref{eq:chapters/11/11/1/dia-u},
	we obtain 
	\eqref{eq:chapters/11/11/1/mat}.
\item Given  points
\label{prop:chapters/11/11/1/pts-3}
	$\vec{x}_{1},\  \vec{x}_{2},\ 
	\vec{x}_{3} $
on the circle, 
the parameters are given by
\begin{align}
\myvec{
 2 \vec{x}_1 & 2 \vec{x}_2 & 2\vec{x}_3
 \\
 1 & 1 & 1
 }^\top 
	\myvec{\vec{u} \\ f}
	=
-\myvec{ 	\norm{\vec{x}_1}^2 
\\
 	\norm{\vec{x}_2}^2 	
	\\
 	\norm{\vec{x}_3}^2 	
	}
	\label{eq:chapters/11/11/1/mat-3}
                     \end{align}
\item Code for circle
	\begin{lstlisting}
	codes/book/circ.py
\end{lstlisting}
		     \iffalse
\item Any point $\vec{x}$ on a circle can be expressed as
\begin{align}
\vec{x} = \vec{c} + r\myvec{\cos \theta \\ \sin \theta}.
	\label{eq:circ-polar}
\end{align}
\item The equation of the common chord of intersection of two  circles is given by 
\begin{align}
	2\brak{\vec{u}_1
	   -\vec{u}_2}^{\top}\vec{x} + f_1 - f_2 = 0
	\label{eq:circ-chord}
\end{align}
\item The line joining the centre of a circle to the mid point of any chord is perpendicular to the chord.
	\label{prop:circ-chord-perp}
	\begin{proof}
	Let $AB$ be any chord of a circle with centre $\vec{O}= \vec{0}$ and radius $r$.  Then, 
\begin{align}
	\norm{\vec{A}}^2 
	=\norm{\vec{B}}^2  &= r^2
	\\
	\implies 
	\norm{\vec{A}}^2 
	-\norm{\vec{B}}^2  &=\vec{0}
	\\
	\text{or, }\brak{\vec{A}-\vec{B}}^{\top}\brak{\vec{A}+\vec{B}} &= \vec{0}
\end{align}
which can be expressed as 
\begin{align}
	\brak{\vec{A}-\vec{B}}^{\top}\brak{\frac{\vec{A}+\vec{B}}{2}-\vec{O}} = \vec{0}
\end{align}
	\end{proof}
\item Let 
\begin{align}
\vec{A} =  \myvec{\cos \theta_1 \\ \sin \theta_1},
\vec{B} =  \myvec{\cos \theta_2 \\ \sin \theta_2},
\end{align}
 be points on  a unit circle with centre $\vec{O}$ at the origin.  Then
\begin{align}
	\label{eq:circ-ang-centre}
	\cos AOB = \vec{A}^{\top}\vec{B} 
\end{align}
\item Let 
\begin{align}
\vec{A} =  \myvec{\cos \theta_1 \\ \sin \theta_1},
\vec{B} =  \myvec{\cos \theta_2 \\ \sin \theta_2},
\vec{C} =  \myvec{\cos \theta \\ \sin \theta},
\end{align}
 be points on  a unit circle.  Then
  \begin{align}
	\label{eq:circ-angle-1}
	  \cos ACB&= \frac{\brak{\vec{C}-\vec{A}}^{\top} \brak{\vec{C}-\vec{B}}}{\norm{\vec{C}-\vec{A}}\norm{\vec{C}-\vec{B}}}
	  \\
	  &= \cos \brak{\frac{\theta_1 - \theta_2}{2}}
	\label{eq:circ-angle-2}
  \end{align}
		\begin{proof}
			Since
\begin{align}
	\brak{\vec{C}-\vec{A}}^{\top} \brak{\vec{C}-\vec{B}} &= 
	\norm{\vec{C}}^2 - \vec{C}^{\top}\brak{\vec{A}+\vec{B}} + \vec{A}^{\top}\vec{B}
	\\
	&= 1 - \cos\brak{\theta-\theta_1} - \cos\brak{\theta-\theta_2} + \cos\brak{\theta_1-\theta_2}
	\\
	&= 2 \cos^2 \brak{\frac{\theta_1 - \theta_2}{2}}
	- 2 \cos \brak{\frac{\theta_1 - \theta_2}{2}}
	\cos \brak{\theta -\frac{\theta_1 + \theta_2}{2}}
	\\
	&= 4 \cos \brak{\frac{\theta_1 - \theta_2}{2}}
	\sin \brak{\frac{\theta - \theta_1}{2}}
	\sin \brak{\frac{\theta - \theta_2}{2}},
\end{align}
and 
\begin{align}
	\norm{\vec{C}-\vec{A}}^2 &= \norm{\vec{C}}^2+\norm{\vec{A}}^2 - 2\vec{C}^{\top}\vec{A},
	\\
&= 4 
	\sin^2 \brak{\frac{\theta - \theta_1}{2}}, 
	\\
	\norm{\vec{C}-\vec{B}}^2 &= \norm{\vec{C}}^2+\norm{\vec{B}}^2 - 2\vec{C}^{\top}\vec{B},
	\\
&= 4 
	\sin^2 \brak{\frac{\theta - \theta_2}{2}}, 
\end{align}
	\eqref{eq:circ-angle-1}
	can be expressed as
\begin{align}
	\frac{	  \cos \brak{\frac{\theta_1 - \theta_2}{2}}
	\sin \brak{\frac{\theta - \theta_1}{2}}
	\sin \brak{\frac{\theta - \theta_2}{2}}
	}
	{
	\sin \brak{\frac{\theta - \theta_1}{2}}
	\sin \brak{\frac{\theta - \theta_1}{2}} 
} 
\end{align}
yielding 
	\eqref{eq:circ-angle-2}
		\end{proof}
	\item From 
	\eqref{eq:circ-ang-centre}
	and 
	\eqref{eq:circ-angle-2},
	\label{prop:circ-ang-centre-arc}
\begin{align}
	\label{eq:circ-ang-centre-arc}
	\angle AOB = 2\angle AOC
\end{align}
\item The circumcentre $\vec{O}$ of $\triangle ABC$ is given by the matrix equation
\begin{align}
	\myvec{\vec{A}-\vec{B} & \vec{B}-\vec{C}}^{\top} \vec{O} &= \frac{1}{2}\myvec{\norm{\vec{A}}^2 -\norm{\vec{B}}^2 \\
	{\norm{\vec{B}}^2 -\norm{\vec{C}}^2}}
	\label{eq:circumcirc-eq}
\end{align}
	\solution Since $\vec{A},\vec{B}, \vec{C}$ lie on the circle, from 
	\eqref{eq:circ-eq}
\begin{align}
	\norm{\vec{A}}^2 + 2 \vec{u}^{\top}\vec{A} + f = 0
	\norm{\vec{B}}^2 + 2 \vec{u}^{\top}\vec{B} + f = 0
	\norm{\vec{C}}^2 + 2 \vec{u}^{\top}\vec{C} + f = 0
\end{align}
Subtracting equations and simplifying,
\begin{align}
			\brak{\vec{A}-\vec{B}}^{\top} \vec{u} &= -\frac{\norm{\vec{A}}^2 -\norm{\vec{B}}^2}{2}\\
			\brak{\vec{B}-\vec{C}}^{\top} \vec{u} &= -\frac{\norm{\vec{B}}^2 -\norm{\vec{C}}^2}{2}
\end{align}
which can be stacked to obtain 
	\eqref{eq:circumcirc-eq}.
	\fi
\end{enumerate}

\subsection{Miscellaneous}
\begin{enumerate}[label=\thesubsection.\arabic*,ref=\thesubsection.\theenumi]
  \item Find the equation of the circle passing through $(0,0)$ and making intercepts $a$ and $b$ on the coordinate axes.
 \item Find the equation of the circle with 
centre $(-a,-b)$ and radius $\sqrt{a^{2}-b^{2}}$.
	 \\
		\solution
\label{chapters/11/11/1/5}
%	From \eqref{eq:circ-cr},
\begin{align}
	\vec{u} &= \myvec{a\\b},\,
	f  
	  =2b^2
\end{align}
Thus, the equation of circle is 
\begin{align}
	\norm{\vec{x}}^2 +2 \myvec{a&b}\vec{x}+2b^2 &= 0       		       
\end{align}	

\end{enumerate}

\newpage
\section{Conics}
\subsection{Equation}
In the each of the following exercises, find the coordinates of the focus, axis of the parabola, the equation of the directrix and the length of the latus rectum.
\begin{enumerate}[label=\thesubsection.\arabic*,ref=\thesubsection.\theenumi]
\item $y^2$=12x 
\label{chapters/11/11/2/1}
\\
\solution
%%See 
\tabref{tab:std-conic-params-sol}
and 
\figref{fig:11/11/2/1Fig1}.
\begin{figure}[!h]
	\begin{center}
		\includegraphics[width=\columnwidth]{chapters/11/11/2/1/figs/problem1.pdf}
	\end{center}
\caption{}
\label{fig:11/11/2/1Fig1}
\end{figure}

\item $x^2$=6y 
\\
\solution
%%See \tabref{tab:rot-conic-params-sol}
and
\figref{fig:chapters/11/11/2/2/Fig1}.
\begin{figure}[!h]
	\begin{center} 
	    \includegraphics[width=\columnwidth]{chapters/11/11/2/2/figs/parabola}
	\end{center}
\caption{}
\label{fig:chapters/11/11/2/2/Fig1}
\end{figure}





\item 
$y^2 = –8x$
\\
\solution
%%See \tabref{tab:std-conic-params-sol} and 
\figref{fig:chapters/11/11/2/3/1}.
\begin{figure}[H]
\centering
\includegraphics[width=0.75\columnwidth]{chapters/11/11/2/3/figs/fig.png}
\caption{Graph}
\label{fig:chapters/11/11/2/3/1}
\end{figure}

\item $y^2$=-8x

\item $x^2$=-16y
\\
\solution
%%See \tabref{tab:rot-conic-params-sol}
and 
\figref{fig:chapters/11/11/2/4/Fig1}.
\begin{figure}[H]
	\begin{center} 
	    \includegraphics[width=0.75\columnwidth]{chapters/11/11/2/4/figs/parabola}
	\end{center}
\caption{}
\label{fig:chapters/11/11/2/4/Fig1}
\end{figure}


\item $y^2$=10x  

\item $x^2$=-9y  
  \item $\frac{x^2}{36}+\frac{y^2}{16}=1$
\\
\solution
%See 
\tabref{tab:std-conic-params-sol}
and 
\figref{fig:chapters/11/11/3/1/Fig1}.
\begin{figure}[H]
	\begin{center}
		\includegraphics[width=0.75\columnwidth]{chapters/11/11/3/1/figs/problem1.pdf}
	\end{center}
\caption{}
\label{fig:chapters/11/11/3/1/Fig1}
\end{figure}

  \item $\frac{x^2}{4}+\frac{y^2}{25}=1$
\\
\solution
%From \tabref{tab:rot-conic-params-sol}, it can be seen that this is not a standard ellipse, since $\lambda_1 > \lambda_2$.  Hence $\vec{P}$ plays a role and we need to use the affine transformation
\begin{align}
\vec{x} = \vec{P}\vec{y}
\end{align}
So the value of $\lambda_1$ and $\lambda_2$ need to be interchanged for all calculations and 
in
					\eqref{eq:dx-ell-hyp},
					$\vec{e}_2$ becomes the normal vector.
See \figref{fig:chapters/11/11/3/2/Fig1}.
\begin{figure}[H]
	\begin{center} 
	    \includegraphics[width=0.75\columnwidth]{chapters/11/11/3/2/figs/ellipse}
	\end{center}
\caption{}
\label{fig:chapters/11/11/3/2/Fig1}
\end{figure}

  \item $\frac{x^2}{16}+\frac{y^2}{9}=1$
\\
\solution
%See \tabref{tab:std-conic-params-sol}
and
\figref{fig:chapters/11/11/3/3/Fig1}.
\begin{figure}[!h]
	\begin{center}
		\includegraphics[width=\columnwidth]{chapters/11/11/3/3/figs/conic.png}
	\end{center}
\caption{}
\label{fig:chapters/11/11/3/3/Fig1}
\end{figure}

  \item $\frac{x^2}{25}+\frac{y^2}{100}=1$
  \item $\frac{x^2}{49}+\frac{y^2}{36}=1$
  \item $\frac{x^2}{100}+\frac{y^2}{400}=1$
  \item $36x^2+4y^2=144$
  \item $16x^2+y^2=16$
  \item $4x^2+9y^2=36$
	\item Find the coordinates of the focii, the vertices, the eccentricity and the length of the latus rectum of a hyperbola whose equation is given by $\frac{x^2}{16}-\frac{y^2}{9} = 1$. \\ 
		\solution
		%See 
\tabref{tab:std-conic-params-sol}
and
\figref{fig:11/11/4/1Fig1}.
\begin{figure}[!h]
	\begin{center}
		\includegraphics[width=\columnwidth]{chapters/11/11/4/1/figs/problem1.pdf}
	\end{center}
\caption{}
\label{fig:11/11/4/1Fig1}
\end{figure}

	\item Find the coordinates of the focii, the vertices, the eccentricity and the length of the latus rectum of a hyperbola whose equation is given by $\frac{y^2}{9}-\frac{x^2}{27}=1$.
		\\
		\solution
		\\
		%See \tabref{tab:rot-conic-params-sol}
and 
\figref{fig:chapters/11/11/4/2/Fig1}.
\begin{figure}[!h]
	\begin{center} 
	    \includegraphics[width=\columnwidth]{chapters/11/11/4/2/figs/hyperbola}
	\end{center}
\caption{}
\label{fig:chapters/11/11/4/2/Fig1}
\end{figure}




	\item Find the coordinates of the foci and the vertices, the eccentricity and the length of the latus rectum of the hyperbolas, whose equation is given by $5{y^2}-9{x^2}=36$.
		\\
		\solution
		\\
		%See \tabref{tab:rot-conic-params-sol}
and 
\figref{fig:chapters/11/11/4/5/1}.
In
\tabref{tab:rot-conic-params-sol}, $\vec{P}$ shifts the negative eigenvalue 
to get the hyperbola in standard form.
\begin{figure}[H]
	\begin{center} 
	    \includegraphics[width=0.75\columnwidth]{chapters/11/11/4/5/figs/hyperbola.png}
	\end{center}
\caption{}
\label{fig:chapters/11/11/4/5/1}
\end{figure}


	\item Find the equation of the hyperbola whose foci is $\brak{0,\pm 8}$ and vertices $\brak{0,\pm 5}$.
\\
\solution
\end{enumerate}

Each of the Exercises, find the equation of the parabola, that satisfies the given conditions.

\begin{enumerate}[label=\thesubsection.\arabic*,ref=\thesubsection.\theenumi,resume*]
\item Focus(6,0); directrix x=-6 
\item Focus(0,-3); directrix y=3
\item Vertex(0,0); Focus(3,0)
\item Vertex(0,0); Focus(-2,0) 
\item Vertex(0,0) passing through(2,3) and axis is along x-axis
\item Vertex(0,0) passing through(5,2) symmetric with respect to y-axis
\item vertices $(\pm5,0),\text{foci} (\pm4,0)$
\item vertices $(\pm0,13),\text{foci} (0,\pm5)$
\item vertices $(\pm6,0),\text{foci} (\pm4,0)$
\item Ends of major axis $(\pm3,0),\text{ends of minor axis}(0,\pm2)$
\item  ends of major axis $(0,\pm \sqrt{5}),\text{ends of minor axis} (\pm1,0)$
\item $\text {length of major axis 26},\text{foci} (\pm5,0)$
\item $\text {length of minor axis 16},\text{foci} (0,\pm6)$
\item $\text {foci} (\pm3,0),a=4$
\item $\text {b=3,c=4},\text {centre at the origin} ;\text{foci on the x axis}$
\item $\text {centre at} (0,0),\text{major axis on the y-axis and passes through the points} \text {(3,2) and (1,6)}$.
\\
\solution
%%Since the major axis is along the $y$-axis,
\begin{align}
\vec{n} = \vec{e}_2
\end{align}
Thus,
\begin{align}
\vec{V} = \myvec{1&0\\0&1-e^2} \label{eq:chapters/11/11/3/19/5} 
\end{align}
Since
\begin{align}
\vec{c} = \vec{0}, \vec{u}=\vec{0}.
\label{eq:chapters/11/11/3/19/8}
\end{align}
    From \eqref{eq:conic_quad_form},
    \begin{align}
        \vec{P}^\top\vec{VP} + 2\vec{u}^\top\vec{P} + f &= 0 \label{eq:chapters/11/11/3/19/ep1} \\
        \vec{Q}^\top\vec{VQ} + 2\vec{u}^\top\vec{Q} + f &= 0 \label{eq:chapters/11/11/3/19/ep2}
    \end{align}
    yielding
\begin{align}
4e^2 - f = 13 \label{eq:chapters/11/11/3/19/10}
\\
36e^2 - f = 37 \label{eq:chapters/11/11/3/19/11}
\end{align}
which can be formulated as the matrix equation
\begin{align}
\myvec{4&-1\\36&-1}\myvec{e^2\\f} = \myvec{13\\37}
\label{eq:chapters/11/11/3/19/12}
\end{align}
The augmented matrix is given by,
\begin{align*}
\myvec{4&-1&\vline&13\\36&-1&\vline&37}
\xleftrightarrow[]{R_1\leftarrow-\frac{R_1}{8}} \myvec{4&0&\vline&3\\36&-1&\vline&37} 
\\
\xleftrightarrow[]{R_2\leftarrow R_2-9R_1}
\myvec{4&0&\vline&3\\0&-1&\vline&10} 
\xleftrightarrow[R_2\leftarrow -R_2]{R_1\leftarrow \frac{R_1}{4}}
\myvec{1&0&\vline&\frac{3}{4}\\0&1&\vline&-10}
\end{align*}
Thus,
\begin{align}
e^2 = \frac{3}{4},\ f = -10
\end{align}
and the equation of the conic is given by
\begin{align}
\vec{x}^\top\myvec{1&0\\0&\frac{1}{4}}\vec{x} - 10 = 0
\end{align}
See  
\figref{fig:chapters/11/11/3/19/1}.
\begin{figure}[H]
\centering
\includegraphics[width=0.75\columnwidth]{chapters/11/11/3/19/figs/fig1.png}
\caption{Graph}
\label{fig:chapters/11/11/3/19/1}
\end{figure}

\item $\text{major axis on the x-axis and passes through the points (4,3) and (6,2)}$
\\
\solution
%%In this case, 
    \begin{align}
        \vec{n} = \myvec{1\\0}
    \end{align}
    Thus,
    \begin{align}
        \vec{V} = \myvec{1-e^2&0\\0&1} \label{eq:chapters/11/11/3/20/V-val} \\
    \end{align}
Since
\begin{align}
\vec{c} = \vec{0}, \vec{u}=\vec{0}.
\label{eq:chapters/11/11/3/20/8}
    \end{align}
    From \eqref{eq:conic_quad_form},
    \begin{align}
        \vec{P}^\top\vec{VP} + 2\vec{u}^\top\vec{P} + f &= 0 \label{eq:chapters/11/11/3/20/ep1} \\
        \vec{Q}^\top\vec{VQ} + 2\vec{u}^\top\vec{Q} + f &= 0 \label{eq:chapters/11/11/3/20/ep2}
    \end{align}
    yielding
    \begin{align}
        16e^2 - f = 25 \label{eq:chapters/11/11/3/20/e1}
	\\
        36e^2 - f = 40 \label{eq:chapters/11/11/3/20/e2}
    \end{align}
which can be formulated as the matrix equation
    \begin{align}
        \myvec{16&-1\\36&-1}\myvec{e^2\\f} = \myvec{25\\40}
        \label{eq:chapters/11/11/3/20/mtx-eqn}
    \end{align}
    and can be solved using the augmented matrix.
    \begin{align*}
        \myvec{16&-1&25\\36&-1&40} \xleftrightarrow[]{R_1\leftarrow R_1-R_2} \myvec{-20&0&-15\\36&-1&40} \\
                 \xleftrightarrow[]{\substack{R_1\leftarrow\frac{R_1}{-5}\\R_2\leftarrow -R_2}} \myvec{4&0&3\\-36&1&-40} 
                 \xleftrightarrow[]{R_2\leftarrow R_2+9R_1}\myvec{4&0&3\\0&1&-13} \\
                 \xleftrightarrow[]{R_1\leftarrow\frac{R_1}{4}}\myvec{1&0&\frac{3}{4}\\0&1&-13}
    \end{align*}
    Thus,
    \begin{align}
        e^2 = \frac{3}{4},\ f = -13
    \end{align}
    and the equation of the conic is given by
    \begin{align}
        \vec{x}^\top\myvec{\frac{1}{4}&0\\0&1}\vec{x} - 13 = 0
    \end{align}
    See \figref{fig:chapters/11/11/3/20/ellipse}.
    \begin{figure}[H]
        \centering
        \includegraphics[width=0.75\columnwidth]{chapters/11/11/3/20/figs/ellipse.png}
        \caption{Locus of the required ellipse.}
        \label{fig:chapters/11/11/3/20/ellipse}
    \end{figure}

\item Find the equations of hyperbola having Vertices $\myvec{0\\\pm 3}$ and Foci $\myvec{0\\\pm5}$
	\\
\solution
		%Following the approach in the earlier problems, it is obvious that
	\begin{align}
		\vec{n} 
			= \vec{e}_2,
	\vec{c} =\vec{u}=\vec{0}.
\end{align}
Consequently,
%
\begin{align}
	\vec{V} &= \myvec{1 &0\\ 0 & 1-e^2}
	\\
	\vec{F} &= ce^2\vec{e}_2 \implies \norm{\vec{F}} = ce^2=5
\label{eq:chapters/11/11/4/9/F}
	\\
	f 
	  &= 25 - c^2 e^2
\label{eq:chapters/11/11/4/9/f}
\end{align}
%
Since the vertices are  on the conic,
\begin{align}
	\vec{v_1}^{\top}\vec{V}\vec{v_1} +2\vec{u}^{\top}\vec{v_1}+f &= 0\\
\implies 9\brak{1-e^2} + f &= 0\\
 \label{eq:chapters/11/11/4/9/1}
\end{align}
Solving \eqref{eq:chapters/11/11/4/9/1},
\eqref{eq:chapters/11/11/4/9/F}
and
\eqref{eq:chapters/11/11/4/9/f},
\begin{align}
	c = \frac{9}{5},\ 
	e = \frac{5}{3},
\end{align}
%
yielding
\begin{align}
	\vec{V} = \myvec{1&0\\0& -\frac{16}{9}} ,\
	\vec{u} = \myvec{0\\0},\
	f = 16.
\end{align}
%
Thus, the desired equation of the hyperbola is
\begin{align}
	\vec{x}^{\top} \myvec{1&0\\ 0 & -\frac{16}{9}} \vec{x} +16 =0
\end{align}
%
See
%
    \figref{fig:chapters/11/11/4/9/}.
\begin{figure}[H]
  \centering
    \includegraphics[width=0.75\columnwidth]{chapters/11/11/4/9/figs/Figure_1.png}
    \caption{Figure 1}
    \label{fig:chapters/11/11/4/9/}
\end{figure}
%




	%It is obvious that
	\begin{align}
		\vec{n} 
			= \vec{e}_2,
	\vec{c} =\vec{u}=\vec{0}.
\end{align}
Consequently,
%
\begin{align}
	\vec{V} &= \myvec{1 &0\\ 0 & 1-e^2}
	\\
	\vec{F} &= ce^2\vec{e}_2 \implies \norm{\vec{F}} = ce^2=8
\label{eq:chapters/11/11/4/8/F}
	\\
	f 
	  &= 64 - c^2 e^2
\label{eq:chapters/11/11/4/8/f}
\end{align}
%
Since the vertices are  on the conic,
\begin{align}
	\vec{v_1}^{\top}\vec{V}\vec{v_1} +2\vec{u}^{\top}\vec{v_1}+f &= 0\\
\implies 25\brak{1-e^2} + f &= 0\\
	 \label{eq:chapters/11/11/4/8/1}
\end{align}
Solving \eqref{eq:chapters/11/11/4/8/1},
\eqref{eq:chapters/11/11/4/8/F}
and
\eqref{eq:chapters/11/11/4/8/f},
\begin{align}
	c = \frac{9}{5},\ 
	e = \frac{5}{3},
\end{align}
%
yielding
\begin{align}
	\vec{V} = \myvec{1&0\\0& -\frac{16}{9}} ,\
	\vec{u} = \myvec{0\\0},\
	f = 16.
\end{align}
%
Thus, the desired equation of the hyperbola is
\begin{align}
	\vec{x}^{\top} \myvec{1&0\\ 0 & -\frac{16}{9}} \vec{x} +16 =0
\end{align}
%
\item We know the Focii is given as
\begin{align}
	\vec{F} &= \pm \frac{\brak{\frac{1}{e\sqrt{1-e^2}}}\brak{e^2}\sqrt{\frac{\lambda_1}{f_0}}}{\frac{\lambda_1}{f_0}}\vec{e}_2\\
	        &= \frac{\frac{e}{\sqrt{1-e^2}}}{\sqrt{\frac{\lambda_1}{f_0}}}\vec{e}_2
\end{align}
Substituting \eqref{eq:chapters/11/11/4/8/eq1} we get
\begin{align}
	\vec{F} &= 5e\vec{e}_2\\
	\myvec{0\\8} &= 5e\vec{e}_2\\
	\implies e &= \frac{8}{5}
\end{align}
\item Now we know the eccentricity is given as
\begin{align}
	e = \sqrt{1-\frac{\lambda_2}{\lambda_1}}\\
	\label{eq:chapters/11/11/4/8/eq2}
	\implies \frac{\lambda_2}{\lambda_1} = -\frac{39}{25}
\end{align}
\item Now we know from the standard equation
\begin{align}
	\label{eq:chapters/11/11/4/8/eq3}
	f = \norm{\vec{n}}^2 \norm{\vec{F}}^2 - c^2 e^2
\end{align}
Calculating $\vec{n} \text{ and } c$
\begin{align}
	\vec{n} &= \sqrt{\frac{\lambda_1}{f_0}}\vec{e}_2 = \frac{1}{5}\sqrt{\frac{\lambda_1}{\lambda_2}}\vec{e}_2\\
	        &= \frac{1}{\sqrt{-39}}\vec{e}_2\\
	c &= \frac{1}{e\sqrt{1-e^2}} = \frac{25}{8\sqrt{-39}}	
\end{align}
Now
\begin{align}
	\norm{\vec{n}}^2 &= -\frac{1}{39}\\
	\norm{\vec{F}}^2 &= 64
\end{align}
Substituting all the values in \eqref{eq:chapters/11/11/4/8/eq3} we get
\begin{align}
	f &= -\brak{\frac{1}{39}}\brak{64} + \brak{\frac{25}{8}}^2 \brak{\frac{1}{39}} \brak{\frac{64}{25}}\\
	  &= -1\\
	\label{eq:chapters/11/11/4/8/eq4}  
	f_0  &= -f = 1
\end{align}
substituting \eqref{eq:chapters/11/11/4/8/eq4} in \eqref{eq:chapters/11/11/4/8/eq1} we get
\begin{align}
	\label{eq:chapters/11/11/4/8/eq5}
	\lambda_2 = \frac{1}{25} 
\end{align}
Substituting \eqref{eq:chapters/11/11/4/8/eq5} in \eqref{eq:chapters/11/11/4/8/eq2} we get
\begin{align}
	\lambda_1 = -\frac{1}{39}
\end{align}
\end{enumerate}
Therefore the equation of the hyperbola is given as
\begin{align}
	g\brak{\vec{x}}=\vec{x}^\top \vec{V} \vec{x} + 2\vec{u}^\top \vec{x} + f = 0
\end{align}
where
\begin{align}
	\vec{V} &= \myvec{\lambda_1&0\\0&\lambda_2} = \myvec{-\frac{1}{39}&0\\0&\frac{1}{25}}\\
	\vec{u} &= \vec{0}\\
	f &= -1
\end{align}
See Fig. \ref{fig:chapters/11/11/4/8/Fig1}.
\begin{figure}[H]
	\begin{center} 
	    \includegraphics[width=0.75\columnwidth]{chapters/11/11/4/8/figs/hyperbola2}
	\end{center}
\caption{}
\label{fig:chapters/11/11/4/8/Fig1}
\end{figure}


\item Find the equation of the hyperbola that satisfies the conditions - Foci \brak{\pm 4, 0}, the latus rectum is of length 12.
\\
\solution
		%The given information is available in 
\tabref{tab:chapters/11/11/4/13/1}.
Since two foci are given, the conic cannot be a parabola.
\begin{enumerate}
\item The direction vector of $F_1F_2$ is the normal vector of the directrix.  Hence, 
\begin{align}
\vec{n} = \vec{F_1} - \vec{F_2}
	\equiv \vec{e}_1
\end{align}
Substituting in 
  \eqref{eq:conic_quad_form_v},
\eqref{eq:conic_quad_form_u}
and
\eqref{eq:conic_quad_form_f},
\begin{align}
	\vec{V} &= \myvec{1-e^2&0\\0&1} \label{eq:chapters/11/11/4/13/6} 
	\\
	\vec{u} &= ce^2\vec{e}_1-\vec{F}
\label{eq:chapters/11/11/4/13/6/u} 
	\\
	f&=16-c^2e^2
\label{eq:chapters/11/11/4/13/6/f} 
\end{align}
\item From
\eqref{eq:chapters/11/11/4/13/6},
\begin{align}
\lambda_1 &= 1-e^2,\
\lambda_2 = 1
\label{eq:chapters/11/11/4/13/12}
\end{align}
which upon substituting
in
			\eqref{eq:latus-ellipse}, along with the value of the latus rectum 
from \tabref{tab:chapters/11/11/4/13/1}
		\begin{align}
	6\brak{1-e^2} = \sqrt{\abs{f}}
\label{eq:chapters/11/11/4/13/12/f}
\end{align}
\item  The centre of the conic is given by
\begin{align}
\vec{c} = \frac{\vec{F_1} + \vec{F_2}}{2}
= \vec{0}
\label{eq:chapters/11/11/4/13/5}
\end{align}
From \eqref{eq:chapters/11/11/4/13/6}, it is obvious that  
$\vec{V}$ is invertible.  Hence,  
from \eqref{eq:chapters/11/11/4/13/5}
and 
\eqref{eq:conic_parmas_c_def},
\begin{align}
\vec{u} = \vec{0}
	\label{eq:chapters/11/11/4/13/7/u}
\end{align}
Substituting the above in \eqref{eq:chapters/11/11/4/13/6/u}, 
\begin{align}
\vec{F} = ce^2\vec{e}_1 
\implies 
	\norm{\vec{F}} = 4 = ce^2
	\label{eq:chapters/11/11/4/13/7}
\end{align}
\item 
	From 
      \eqref{eq:f0}, 
	\eqref{eq:chapters/11/11/4/13/7/u}
and
\eqref{eq:chapters/11/11/4/13/6/f},
		\begin{align}
	36\brak{1-e^2}^2 = 16-c^2e^2
\label{eq:chapters/11/11/4/13/12/ec}
\end{align}
From
	\eqref{eq:chapters/11/11/4/13/7}
	and
\eqref{eq:chapters/11/11/4/13/12/ec}
\begin{align}
\frac{4}{e\sqrt{e^2-1}} &= 6
\\
\implies 9e^2\brak{e^2-1} &= 4\\
\implies 9e^4-9e^2-4 &= 0
\\
	\text{or, }\brak{3e^2-4}
	\brak{12e^2+1} &=0
\label{eq:chapters/11/11/4/13/14}
\end{align}
yielding
\begin{align}
e = \frac{4}{3}
\end{align}
as the only viable solution.
\end{enumerate}
The equation of the conic is then obtained as
\begin{align}
\vec{x}^\top\myvec{-\frac{1}{3}&0\\0&1}\vec{x} +4 = 0
\end{align}
See \figref{fig:chapters/11/11/4/13/1}.
\begin{figure}[H]
\centering
\includegraphics[width=0.75\columnwidth]{chapters/11/11/4/13/figs/fig1.png}
\caption{}
\label{fig:chapters/11/11/4/13/1}
\end{figure}
\begin{table}[H]
\centering
\input{chapters/11/11/4/13/tables/table1.tex}
\caption{}
\label{tab:chapters/11/11/4/13/1}
\end{table}

    \item Find the equation of the hyperbola whose eccentricity is $e = \frac{4}{3}$
    and whose vertices are
    \begin{align}
        \vec{P_1} = \myvec{7\\0},\ \vec{P_2} = \myvec{-7\\0}
        \label{eq:chapters/11/11/4/14/vert}
    \end{align}
\\
\solution
		%    The major axis of a conic is the chord which passes through the vertices of the conic.
    The direction vector of the major axis in this case is
    \begin{align}
        \vec{P}_2-\vec{P}_1 \equiv \vec{e}_1 = \vec{n}
\label{eq:chapters/11/11/4/13/6/n} 
    \end{align}
    which is the normal vector for the directrix.
    Since $e > 1$, the conic is a hyperbola.
Substituting  
\eqref{eq:chapters/11/11/4/13/6/n} 
in
  \eqref{eq:conic_quad_form_v},
\eqref{eq:conic_quad_form_u}
and
\eqref{eq:conic_quad_form_f},
\begin{align}
	\vec{V} &= \myvec{1-e^2&0\\0&1} = \myvec{-\frac{7}{9}&0\\0&1} \label{eq:chapters/11/11/4/14/6} 
	\\
	\vec{u} &= ce^2\vec{e}_1-\vec{F}
\label{eq:chapters/11/11/4/14/6/u} 
	\\
	f&=16-c^2e^2
\label{eq:chapters/11/11/4/14/6/f} 
\end{align}
    Thus,
    \begin{align}
        \vec{V} = \myvec{1-e^2&0\\0&1} \label{eq:chapters/11/11/4/14/V-val} \\
	    \vec{u} = ce^2\vec{e}_1 - \vec{F} \label{eq:chapters/11/11/4/14/u-val} \\
        f = \norm{\vec{F}}^2 - c^2e^2 \label{eq:chapters/11/11/4/14/f-val}
    \end{align}
    The centre of the hyperbola is 
\begin{align}
	\vec{c} = \frac{\vec{P}_1+\vec{P}_2}{2} = \vec{0} = \vec{u}
\end{align}
from \eqref{eq:conic_parmas_c_def}.      Substituting $\vec{P}_1$ and $\vec{P}_2$ in 
    \eqref{eq:conic_quad_form},
    \begin{align}
        \vec{P}_1^\top\vec{VP}_1 + 2\vec{u}^\top\vec{P}_1 + f &= 0 \label{eq:chapters/11/11/4/14/ep1} \\
        \vec{P}_2^\top\vec{VP}_2 + 2\vec{u}^\top\vec{P}_2 + f &= 0 \label{eq:chapters/11/11/4/14/ep2}
	\\
	    \implies f = \vec{P}_1^\top\vec{VP}_1  = 49\brak{e^2-1}&=\frac{343}{9}
    \end{align}
    upon adding 
    \eqref{eq:chapters/11/11/4/14/ep2} and \eqref{eq:chapters/11/11/4/14/ep1}
    and simplifying.
    Therefore, the equation of the conic is
    \begin{align}
        \vec{x}^\top\myvec{-\frac{7}{9}&0\\0&1}\vec{x} + \frac{343}{9} = 0
    \end{align}
See \figref{fig:chapters/11/11/4/14/hyperbola}.
    \begin{figure}[H]
        \centering
        \includegraphics[width=0.75\columnwidth]{chapters/11/11/4/14/figs/hyperbola.png}
        \caption{}
        \label{fig:chapters/11/11/4/14/hyperbola}
    \end{figure}

\item If the focus of a parabola is (0,-3) and its directrix is y=3, then its equation is
\begin{enumerate}
\item $x^2=-12y$
\item $x^2=12y$
\item $y^2=-12x$
\item $y^2=12x$
\end{enumerate}
\item If the parabola $y^2=4ax$ passes through the point (3,2), then the length of its latus rectum is
\begin{enumerate}
\item 2$\pm$3
\item 4$\pm$4
\item 1$\pm$3
\item 4
\end{enumerate}
\item If the vertex of the parabola is the point (-3,0) and the directrix is the line x+5=0, then its equation is
\begin{enumerate}
\item $y^2=8(x+3)$
\item $x^2=8(y+3)$
\item $y^2=-8(x+3)$
\item $y^2=8(x+5)$
\end{enumerate}
 \item Find the coordinates of a point on the parabla $y^2=8x$ whose focal distance is 4.
 \item Find the length of the line-segment joining the vertex of the parabola $y^2=4ax$ and a point on the parabola where the line - segment makes an angle 0 to the x-axis.
\item If the points (0,4) and (0,2) are respectively the vertex and focus of a parabola. then find the equation of the parabola
\end{enumerate}
Find the equation of each of the following parabolas
\begin{enumerate}[label=\thesection.\arabic*,ref=\thesection.\theenumi,resume*]
\item  Directrix x=0. focus ot (6,0)
\item  vertex  ot (0,4), focus at (0,2)
\item  Focus at (-1,2), directrix x-2y+3=0
\end{enumerate}
Fill in the Blanks
\begin{enumerate}[label=\thesection.\arabic*,ref=\thesection.\theenumi,resume*]
\item The equation of the parabola having focus at (-1,-2) and the directrix x-2y+3=0 is \makebox[1cm]{\hrulefill}   
\item show that the set of all points such that the difference of their distances from (4,0) and (-4,0) is always equal to 2 represent a hyperbola .
\item If the distance between the foci of a hyperbola is 16 and its eccentricity is $\sqrt{2}$, then obtain the equation of the hyperbola.
\item Find the eccentricity of the hyperbola $9y^2-4x^2=36$.
\item Find the equation of the hyperbola with eccentricity $\frac{3}{2}$ and foci at $(\pm2,0)$.
\item The eccentricity of the hyperbola whose latus rectum is 8 and conjugate axis is equal to half of th distance between the foci is 
\begin{enumerate}
\item $4\pm3$
\item $\frac{4}{\sqrt{3}}$
\item $\frac{2}{\sqrt{3}}$
\item none of these
\end{enumerate}
\item The distance between the foci of a hyperbola is 16 and its eccentricity is $\le{2}$. lts equation is
\begin{enumerate}
\item $x^2-y^2=3^2$
\item $\frac{x^2}{4-}\frac{y^2}{9}=1$
\item $2x-3y^2=7$
 \item none of these
 \end{enumerate}
 \item Equation of the hyperbola with eccentricty 3$\pm$2 and foci at ($\pm2,0)$ is
\begin{enumerate} 
	\item $\frac{x^2}{4}-\frac{y^2}{5}=\frac{4}{9}$

	\item  $\frac{x^2}{9}-\frac{y^2}{9}=\frac{4}{9}$
	\item  $\frac{x^2}{4}-\frac{y^2}{9}=1$
\item  none of these.
\end{enumerate}
\end{enumerate}
Find the equation of the hyperbola with
\begin{enumerate}[label=\thesection.\arabic*,ref=\thesection.\theenumi,resume*]
	 \item  vertices $(\pm5,0)$, focic $(\pm 7,0)$
	 \item vertices $(0\pm7)$ ,e =$\frac{4}{3}$ 
	 \item  Foci (0,$\pm\sqrt{10})$. passing through (2,3)
\end{enumerate}
State whether the statements are True or False 
\begin{enumerate}[label=\thesection.\arabic*,ref=\thesection.\theenumi,resume*]
\item The locus of the point of intersecton of lines $\sqrt{3}x+y-4\sqrt{3}k=0$ and $\sqrt{3}=0\sqrt{3}kx+ky-4\sqrt{3}=0$ for different value of K is a hyperboia whose eccentricity is 2.
[Hint: Eliminate k between the given equations]
\end{enumerate}
Fill in the Blanks
\begin{enumerate}[label=\thesection.\arabic*,ref=\thesection.\theenumi,resume*]
\item The equation of the hyperbola with vertices ot $(0,\pm6)$ and eccntricity $\frac{5}{3}$ is and its foci are \makebox[1cm]{\hrulefill}
 \item If the latus rectum of an ellipse is equal to half of minor axis, then find its eccentricity.
 \item Given the ellipse with equation $9x^2+25y^2=225,$ find the eccentricity and focl.
 \item If the eccentricity of an ellipse is $\frac{5}{8}$ and  the distance between its foci is 10 then find latus rectum of the ellipse.
 \item Find the equation of ellipse whose eccentricity is $\frac{2}{3}$,latus rectum is 5 and the centre is(0,0).
 \item Find the distance between the directrices of the ellipse $\frac{x^2}{36}+\frac{y^2}{20}$
\item Find the equation of the set of all points the sum of whose distances  from the points (3,0) and (9,0) is 12.
\item Find the equation of the set of all pints whose distance from (0,4) are 2$\pm$3 of their distance from the line y=9.
\item The equation of the ellipse whose focus is (1,-1), the directrix the line x-y-3
=0 and eccentricity 1pm2 is
\begin{enumerate}
\item $7x^2+2xy+7y^2-10x+10y+7=0$
\item $7x^2+2xy+7y^2+7=0$
\item $7x^2+2xy+7y^2+10x-10y-7=0$ 
\item none
\end{enumerate}
\item The length of the latus rectum of the ellipes $3x^2+y^2=12$ is
\begin{enumerate}
\item 4
\item 3
\item 8
\item $4\sqrt{3}$
\end{enumerate}
\item lf e is the eccentricity of the ellipes $\frac{x^2}{a^2}+\frac{y^2}{b^2}=1(a<b)$,then
\begin{enumerate}
\item $b^2=a^2(1-e^2)$
\item $a^2=b^2(1-e^2)$
\item $a^2=b^2(e^-1)$
\item $b^2=a^2(e^2-1)$
\end{enumerate}
\end{enumerate}
State whether the statements are True or False 
\begin{enumerate}[label=\thesection.\arabic*,ref=\thesection.\theenumi,resume*]
\item If ${P}$ is a point on the ellipse $\frac{x^2}{16}+\frac{y^2}{25}=1$ whose foci  are s and s' then Ps +Ps'=8.
\end{enumerate}
Fill in the Blanks
\begin{enumerate}[label=\thesection.\arabic*,ref=\thesection.\theenumi,resume*]
\item An ellipse is described by using an endless string which is passed over two pins lf the oxes are 6cm and 4cm, the length of the string and distance between the pins are  \makebox[1cm]{\hrulefill}             
\item The equation of the ellipse having foci (0,1),(0,1) and minor axis of length is \makebox[1cm]{\hrulefill}                
\end{enumerate}

\subsection{Formulae}
\begin{enumerate}[label=\thesubsection.\arabic*.,ref=\thesubsection.\theenumi]
\item
The equation of  a conic with directrix $\vec{n}^{\top}\vec{x} = c$, eccentricity $e$ and focus $\vec{F}$ is given by 
\begin{align}
    \label{eq:conic_quad_form}
	\text{g}\brak{\vec{x}} = \vec{x}^{\top}\vec{V}\vec{x}+2\vec{u}^{\top}\vec{x}+f=0
    \end{align}
where     
\begin{align}
  \label{eq:conic_quad_form_v}
\vec{V} &=\norm{\vec{n}}^2\vec{I}-e^2\vec{n}\vec{n}^{\top}, 
\\
\label{eq:conic_quad_form_u}
\vec{u} &= ce^2\vec{n}-\norm{\vec{n}}^2\vec{F}, 
\\
\label{eq:conic_quad_form_f}
f &= \norm{\vec{n}}^2\norm{\vec{F}}^2-c^2e^2
    \end{align}
\item
	\label{prob:conic-params}
  The eccentricity, directrices and foci of \eqref{eq:conic_quad_form} are given by 
\begin{align}
  \label{eq:conic_quad_form_e} 
  e&= \sqrt{1-\frac{\lambda_1}{\lambda_2}}
\\
\label{eq:conic_quad_form_nc} 
	\begin{split}
  \vec{n}&= \sqrt{\lambda_2}\vec{p}_1,  
  \\
	c &= 
  \begin{cases}
    \frac{e\vec{u}^{\top}\vec{n} \pm \sqrt{e^2\brak{\vec{u}^{\top}\vec{n}}^2-\lambda_2\brak{e^2-1}\brak{\norm{\vec{u}}^2 - \lambda_2 f}}}{\lambda_2e\brak{e^2-1}} & e \ne 1
    \\
    \frac{\norm{\vec{u}}^2 - \lambda_2 f   }{2\vec{u}^{\top}\vec{n}} & e = 1
  \end{cases}
	\end{split}
  \\
  \label{eq:conic_quad_form_F} 
  \vec{F}  &= \frac{ce^2\vec{n}-\vec{u}}{\lambda_2}
\end{align}  
	\item For a symmetric matrix, from \eqref{eq:conic_parmas_eig_def}, we have the eigendecomposition
\begin{align}
	\vec{V}&=\vec{P}\vec{D}
\vec{P}^{\top}
\end{align}
where 
%
\begin{align}
	\vec{P} &= \myvec{\vec{p}_1 & \vec{p}_2}, \quad 
\vec{P}^{\top}\vec{P} = 
	\vec{I}
\label{eq:eig_matrix}
\\
      \label{eq:D}
	  \vec{D} &= \myvec{\lambda_1 & 0 \\ 0 & \lambda_2}
\end{align}
			\item Using the affine transformation in
	\eqref{eq:conic_affine},
	the conic in     \eqref{eq:conic_quad_form} can be expressed in standard form 
	%(centre/vertex at the origin, major axis - $x$ axis)
	as
  \begin{align}
    %\begin{aligned}
    \label{eq:conic_simp_temp_nonparab}
	    \vec{y}^{\top}\brak{\frac{\vec{D}}{f_0}}\vec{y} &= 1   &  \abs{\vec{V}} &\ne 0
    \\
	    \vec{y}^{\top}\vec{D}\vec{y} &=  -\eta\vec{e}_1^{\top}\vec{y}   & \abs{\vec{V}} &= 0
    \label{eq:conic_simp_temp_parab}
    %\end{aligned}
    \end{align}
    where
  \begin{align}
      %\begin{split}
      \label{eq:f0}
	  f_0 &=\vec{u}^{\top}\vec{V}^{-1}\vec{u} -f \ne 0
	  \\
      \label{eq:eta}
       \eta &=2\vec{u}^{\top}\vec{p}_1
       \\
       \vec{e}_1 &=\myvec{1 \\ 0}
      \end{align}
  \solution  See \appref{app:parab}.
	  
    \item\leavevmode
		\begin{enumerate}
			\item The directrices for the  standard conic are given by 
				\begin{align}
					\label{eq:dx-ell-hyp}
					\vec{e}_1^{\top}\vec{y} &=  
					\pm\sqrt{\abs{\frac{f_0\lambda_2}{\lambda_1\brak{\lambda_2-\lambda_1}}}} & e \ne 1
				%	\pm \frac{1}{e}\sqrt{\frac{\abs{f_0}}{\lambda_2\brak{1-e^2}}} & e \ne 1
					\\
					\vec{e}_1^{\top}\vec{y} &= \frac{\eta}{2\lambda_2} & e = 1
					\label{eq:dx-parab}
				\end{align}
    \item The foci of the standard conic are given by 
				\begin{align}
					\label{eq:F-ell-hyp-parab}
					\vec{F} 
=
					\begin{cases}
						\pm \sqrt{\abs{f_0\brak{\frac{1}{\lambda_2}-\frac{1}{\lambda_1}}}}\vec{e}_1 & e \ne 1
						%\pm e\sqrt{\frac{\abs{f_0}}{\lambda_2\brak{1-e^2}}}\vec{e}_1 & e \ne 1
						\\
						 -\frac{\eta}{4\lambda_2}\vec{e}_1 & e = 1
					\end{cases}
				\end{align}
	
		\end{enumerate}
	%	where, without loss of generality, $f_0 < 0$ for the hyperbola.
    
%
    \item The equation of the minor and major  axes for the ellipse/hyperbola are respectively given by 
  \begin{align}
\vec{p}_i^{\top}\brak{\vec{x}-\vec{c}} = 0, i = 1,2
	  \label{eq:major-minor-axis-quad}
  \end{align}
  The axis of symmetry for the parabola is also given by 
	  \eqref{eq:major-minor-axis-quad}.

	\item
			The center of the standard ellipse/hyperbola, defined to be the mid point of the line joining the foci, is the origin.
	
	\item
			The principal (major) axis of the standard ellipse/hyperbola, defined to be the line joining the two foci   is the $x$-axis.  
	
	\item
			The minor axis of the standard ellipse/hyperbola, defined to be the line orthogonal to the $x$-axis is the $y$-axis. 
	


	\item
			The axis of symmetry of the standard parabola, defined to be the line perpendicular to the directrix and passing through the focus,  is the $x$- axis.
	


	\item
 The point where the parabola intersects its axis of symmetry is called the vertex. For the standard parabola, the vertex is the origin.
	
	\item
	 The {\em focal length} of the standard parabola, , defined to be the distance between the vertex and the focus, measured along the axis of symmetry, is $\abs{\frac{\eta}{4 \lambda_2}}$
    \item For the standard hyperbola/ellipse, the length of the major axis is 
  \begin{align}
\label{eq:chord-len-major}
 2\sqrt{\abs{\frac{
f_0}
{\lambda_1}
	  }}
  \end{align}
  and the minor axis is 
  \begin{align}
\label{eq:chord-len-minor}
 2\sqrt{\abs{\frac{
f_0}
{\lambda_2}
	  }}
  \end{align}
  \solution 
		See \appref{app:major}.
\item
    The latus rectum of a conic section is the chord that passes through the focus and is perpendicular to the major axis.
	The length of the latus rectum for a conic is given by
		\begin{align}
			l =
			\begin{cases}
				2\frac{\sqrt{\abs{f_0\lambda_1}}}{\lambda_2} & e \ne 1
			\\
			\frac{\eta}{\lambda_2} & e = 1
			\end{cases}
			\label{eq:latus-ellipse}
		\end{align}
		\solution 
		See \appref{app:latus}.
\item Code for parabola 
	\begin{lstlisting}
	codes/book/parab.py
\end{lstlisting}
%	    $\mydet{\vec{V}} \ne 0$, the lengths of the semi-major and semi-minor axes of the conic in \eqref{eq:conic_quad_form} are given by 
%  \begin{align} 
%    \label{eq:ellipse_axes}
%  %  \begin{aligned}[t]
%    \sqrt{\frac{\vec{u}^{\top}\vec{V}^{-1}\vec{u} -f}{\lambda_1}}, 
%    \sqrt{\frac{\vec{u}^{\top}\vec{V}^{-1}\vec{u} -f}{\lambda_2}}. \quad \brak{\text{ellipse}}
%    \\
%%
%       \sqrt{\frac{\vec{u}^{\top}\vec{V}^{-1}\vec{u} -f}{\lambda_1}}, 
%       \sqrt{\frac{f-\vec{u}^{\top}\vec{V}^{-1}\vec{u}}{\lambda_2}}, \quad \brak{\text{hyperbola }}
%%
%  %\end{aligned}
%  \label{eq:hyper_axes}
%\end{align} 
%\solution For \begin{align} \abs{\vec{V}} > 0, \quad \text{or, } \lambda_1 > 0, \lambda_2 > 0 
%  \end{align} and \eqref{eq:conic_simp_temp_nonparab} becomes 
%  \begin{align} 
%	  \lambda_1y_1^2 +\lambda_2y_2^2 = 
%  \vec{u}^{\top}\vec{V}^{-1}\vec{u} -f 
%	  \label{eq:hyper-pair-cond}
%  \end{align} 
%  yielding        \eqref{eq:ellipse_axes}.  Similarly, \eqref{eq:hyper_axes} can be obtained for 
%  \begin{align} 
%    \label{eq:conic_hyper_cond}
%    \abs{\vec{V}} 
%    < 0, \quad \text{or, } \lambda_1 > 0, \lambda_2 < 0 \end{align}
\end{enumerate}

\subsection{Miscellaneous}
\begin{enumerate}[label=\thesubsection.\arabic*,ref=\thesubsection.\theenumi]
\item  The cable of a uniformly loaded suspension bridge hangs in the form of a parabola. The roadway which is horizontal and 100 m long is supported by vertical wires attached to the cable, the longest wire being 30 m and the shortest being 6 m. Find the length of a supporting wire attached to the roadway 18 m from the middle.
\label{chapters/11/11/5/3}
\\
\solution
The parameters are then listed in  
    \tabref{tab:chapters/11/11/5/3/points}.
\begin{table}[H]
	\centering
    \input{chapters/11/11/5/3/tables/table.tex}
    \caption{points}
    \label{tab:chapters/11/11/5/3/points}
\end{table}
For the conic,
\begin{align}
    \vec{V} = \myvec{1&0\\0&0}.
\end{align}
Points $\vec{O}, \vec{A}$, and $\vec{B}$ are on conic, so we have
\begin{align}
	\vec{O}^{\top}\vec{V}\vec{O} + 2\vec{u}^{\top}\vec{O} + f &= 0\\
	\vec{A}^{\top}\vec{V}\vec{A} + 2\vec{u}^{\top}\vec{A} + f &= 0\\
	\vec{B}^{\top}\vec{V}\vec{B} + 2\vec{u}^{\top}\vec{B} + f &= 0	 
\end{align}
which can be expressed as
\begin{align}
	2\vec{O}^{\top}\vec{u} + f &= - \vec{O}^{\top}\vec{V}\vec{O}\\
	2\vec{A}^{\top}\vec{u} + f &= - \vec{A}^{\top}\vec{V}\vec{A}\\
	2\vec{B}^{\top}\vec{u} + f &= - \vec{B}^{\top}\vec{V}\vec{B}	
\end{align}
leading to the matrix equation
\begin{align}
	\myvec{2\vec{O}^{\top} & 1\\ 2\vec{A}^{\top} & 1\\ 2\vec{B}^{\top} & 1}\myvec{\vec{u} \\ f} = -\myvec{\vec{O}^{\top}\vec{V}\vec{O}\\ \vec{A}^{\top}\vec{V}\vec{A}\\ \vec{B}^{\top}\vec{V}\vec{B}}
\end{align}
Substituting numerical values in the above equation,
\begin{align}
    \myvec{0&0&1\\ 100&48&1\\ -100&48&1}\myvec{\vec{u} \\ f} = -\myvec{0\\-2500\\-2500}\\
    \implies f = 0 \text{ and } \vec{u} = \myvec{0\\-\frac{625}{12}}
\end{align}
So, the equation of the parabola is
\begin{align}
    \label{eq:chapters/11/11/5/3/parab1}  \vec{x}^{\top}\myvec{1&0\\0&0}\vec{x} + 2\myvec{0&-\frac{625}{12}}\vec{x} = 0 
\end{align}
The desired point can be expressed as
\begin{align}
	\vec{D} = \myvec{18 \\ x_2}
\end{align}
Substituting this in the parabola equation,
\begin{align}
    18^2 - \frac{6}{625}\lambda_2 = 0
    \\
\implies \lambda_2 = \frac{1944}{625}
\end{align}
Thus, the length of a supporting wire attached to the roadway $18 m$ from the middle is 
\begin{align}
     \lambda_2 + d_2 = \frac{5694}{625} m   
\end{align}
See  
    \figref{fig:chapters/11/11/5/3/parabola}.
\begin{figure}[H]
    \centering
    \includegraphics[width=0.75\columnwidth]{chapters/11/11/5/3/figs/parabola.png}
    \caption{}
    \label{fig:chapters/11/11/5/3/parabola}
\end{figure}


    \item Find the area of the triangle formed by the lines joining the vertex 
    of the parabola 
    \begin{align}
        x^2 = 12y
        \label{eq:chapters/11/11/5/6/parabola}
    \end{align}
    to the ends of its latus rectum.
\label{chapters/11/11/5/6}
\\
\solution
		Rewriting \eqref{eq:chapters/11/11/5/6/parabola} in matrix form,
    \begin{align}
        \vec{x}^\top\myvec{1&0\\0&0}\vec{x} + 2\myvec{0&-6}\vec{x} = 0
        \label{eq:chapters/11/11/5/6/parabola-mtx}
    \end{align}
    The above parabola can be  expressed in standard form using 
\begin{align}
	\vec{x} = \vec{P}\vec{y} = \myvec{0 & 1 \\ 1 & 0}\vec{y}
        \label{eq:chapters/11/11/5/6/affine}
\end{align}
yielding
    \begin{align}
        \vec{y}^\top\myvec{0&0\\0&1}\vec{x} + 2\myvec{-6&0}\vec{x} = 0
        \label{eq:chapters/11/11/5/6/parabola-mty}
    \end{align}
    Hence, 
    from
\eqref{eq:conic_quad_form_nc}, 
    \begin{align}
	    \vec{n} = \vec{e}_1
        \label{eq:chapters/11/11/5/6/n}
	\\
	    c = -\frac{36}{2\times 6} = -3
    \end{align}
    Substituting in 
  \eqref{eq:conic_quad_form_F} 
  yields
    \begin{align}
        \vec{F} = 3\vec{e}_1
    \end{align}
    Thus, the equation of the latus rectum is
\begin{align}
        \vec{x} = \vec{F} + \kappa \vec{e}_2
        \label{eq:chapters/11/11/5/6/x-general}
    \end{align}
        Substituting in \eqref{eq:chapters/11/11/5/6/parabola-mty} and simplifying,
    \begin{align}
\kappa = \pm 6
        \label{eq:chapters/11/11/5/6/x-latus}
    \end{align}
    Thus, the ends of the latus rectum are
    \begin{align}
        \vec{y} = \myvec{3 \\ \pm 6}
    \end{align}
    The relevant parameters with respect to 
        \eqref{eq:chapters/11/11/5/6/parabola-mtx}
	can now be obtained using 
        \eqref{eq:chapters/11/11/5/6/affine}.
See \figref{fig:chapters/11/11/5/6/parabola}.
The area of the required triangle is
    \begin{align}
        \textrm{ar}\brak{\triangle OAB} = \frac{1}{2}\mydet{6&3\\-6&3} = 18 
    \end{align}
    \begin{figure}[H]
        \centering
        %\includegraphics[width=0.75\columnwidth]{chapters/11/11/5/6/figs/parabola.png}
        \includegraphics[width=0.75\columnwidth]{chapters/11/11/5/6/figs/fig1.pdf}
        \caption{}
        \label{fig:chapters/11/11/5/6/parabola}
    \end{figure}

\item A man running a racecourse notes that the sum of the distances from the two flag posts from him is always 10 m and the distance between the flag posts is 8 m. Find the equation of the posts traced by the man. 
\label{chapters/11/11/5/7}
 \item Find the coordinates of a point on the parabola $y^2=8x$ whose focal distance is 4.
\item Show that the set of all points such that the difference of their distances from (4,0) and (-4,0) is always equal to 2 represent a hyperbola.
\item If the distance between the foci of a hyperbola is 16 and its eccentricity is $\sqrt{2}$, then obtain the equation of the hyperbola.
\item The eccentricity of the hyperbola whose latus rectum is 8 and conjugate axis is equal to half of the distance between the foci is 
\begin{enumerate}
\item $\frac{4}{3}$
\item $\frac{4}{\sqrt{3}}$
\item $\frac{2}{\sqrt{3}}$
\item none of these
\end{enumerate}
\item The distance between the foci of a hyperbola is 16 and its eccentricity is $\le{2}$. lts equation is
\begin{enumerate}
\item $x^2-y^2=3^2$
\item $\frac{x^2}{4-}\frac{y^2}{9}=1$
\item $2x-3y^2=7$
 \item none of these
 \end{enumerate}
 \item If the latus rectum of an ellipse is equal to half of minor axis, then find its eccentricity.
 \item If the eccentricity of an ellipse is $\frac{5}{8}$ and  the distance between its foci is 10 then find latus rectum of the ellipse.
 \item Find the distance between the directrices of the ellipse $\frac{x^2}{36}+\frac{y^2}{20}$
\item Find the equation of the set of all points the sum of whose distances  from the points (3,0) and (9,0) is 12.
\item If ${P}$ is a point on the ellipse $\frac{x^2}{16}+\frac{y^2}{25}=1$ whose foci  are $s$ and $s'$ then $Ps +Ps'=8$.
\item An arch is in the form of a parabola with its axis vertical. The arch is 10m high and 5m wide at the base. How wide is it 2m from the vertex of the parabola?
\label{chapters/11/11/5/2}
\item An equilateral triangle is inscribed in the parabola $y^{2} = 4ax$,where one vertex is at the vertex of the parabola. Find the length of the side of the triangle.
\label{chapters/11/11/5/8}
\item An arch is in the form of a semi-ellipse. It is 8 m wide and 2 m high at the centre. Find the height of the arch at a point 1.5 m from one end.
\label{chapters/11/11/5/4}
\item A rod of length 12cm moves with its ends always touching the coordinate axes. Determine the equation of locus of a point  P on the rod, which is 3cm from the end in contact with $x-axis$.
\label{chapters/11/11/5/5}
\item  Find the roots of the following quadratic  equations by factorisation:
\begin{enumerate}
\item $x^2-3x-10=0$
\item $2x^2+x-6=0$
\item $ \sqrt 2x^2+7x+5 \sqrt 2=0$
\item $2x^2-x+\frac{1}{8}=0$
\item $100x^2-20x+1=0$
\end{enumerate}
\item Represent the following situations mathematically;
\begin{enumerate}
\item John and Jivanti together have $45$ marbles. Both of them lost $5$ marbles each, and the product of the number of marbles they have is $124$. We would like to find out how many marbles they had to start with.
\item A cottage industry produces a certain number of toys in a day. The cost of production of each toy (in rupees) was found to be $55$ minus the number of toys produced in a day. On a particular day, the total cost of production was \rupee~750. We would like to find out the number of toys produced on that day.
\end{enumerate}
\item Find two numbers whose sum is $27$ and product is $182$.
\item Find two consecutive  positive integers, sum of whose squares is $365$.
\item  The altitude of a right triangle is 7 cm less than its base. If the hypotenuse is 13 cm, find the other two sides.
\item A cottage industry produces a certain number of pottery articles in a day. It was observed on a particular day that the cost of production of each article (in rupees) was $3$ more than twice the number of articles produced on that day. If the total cost of production on that day was \rupee~90, find the number of articles produced and the cost of each article.
\end{enumerate}

\newpage
\section{Intersection of Conics}
\subsection{Chords}
\begin{enumerate}[label=\thesubsection.\arabic*,ref=\thesubsection.\theenumi]
\item 
Find the area between the curves $y=x$ and $y=x^2$.
\\
\solution
\label{chapters/12/8/3/2}
	\begin{figure}[H]
		\centering
 \includegraphics[width=0.75\columnwidth]{chapters/12/8/3/2/figs/figure.png}
		\caption{}
		\label{fig:12/8/3/2}
  	\end{figure}
The given curve  can be expressed as a conic with parameters
\begin{align}
	\vec{V} &= \myvec{1 & 0\\0 & 0}, \vec{u} = \myvec{0 \\-\frac{1}{2}}, f = 0
	\end{align}
The given line parameters are
\begin{align}
\vec{h} = \myvec{0 \\0}, \vec{m}=\myvec{1\\1}
\end{align}
Substituting the given parameters in 
\eqref{eq:tangent_roots},
\begin{align}
\vec{x}_1=\myvec{0\\0}, \vec{x}_2=\myvec{1\\1}.
\end{align}
From  
		\figref{fig:12/8/3/2},
the area bounded by the curve $y=x^2$ and line $y=x$ is given by
\begin{align}
	\int_{0}^{1} \brak{x 
	-\frac{x^2}{2}} \,dx = \frac{1}{6}
\end{align}

\item 
Find the area of the region bounded by the curve $y^2=x$ and the lines $x=1$ and $x=4$ and the axis in the first quadrant.
\label{chapters/12/8/1/1}
	\\
	\solution
	\begin{figure}[H]
		\centering
 \includegraphics[width=0.75\columnwidth]{chapters/12/8/1/1/figs/conics1.png}
		\caption{}
		\label{fig:12/8/1/1}
  	\end{figure}

The parameters of the conic are
\begin{align}
	\vec{V} = \myvec{0 & 0\\0 & 1},
	\vec{u} = -\frac{1}{2}\myvec{ 1\\0},
	f = 0
	%\\
\end{align}
\iffalse
The point of intersection of the lines $x=1$ and $x=4$ to the parabola is given by


The points of intersection of the line 
\begin{align}
	L: \quad \vec{x} = \vec{q} + \kappa \vec{m} \quad \kappa \in \mathbf{R}
\label{eq:conic_tangent}
\end{align}
with the conic section are given by
\begin{align}
\vec{x}_i = \vec{q} + \kappa_i \vec{m}
\label{eq:conic_tangent_pts}
\end{align}
%
where
{\tiny
\begin{multline}
\kappa_i = \frac{1}
{
\vec{m}^T\vec{V}\vec{m}
}
\lbrak{-\vec{m}^T\brak{\vec{V}\vec{q}+\vec{u}}}
\\
\pm
\rbrak{\sqrt{
\sbrak{
\vec{m}^T\brak{\vec{V}\vec{q}+\vec{u}}
}^2
-
\brak
{
\vec{q}^T\vec{V}\vec{q} + 2\vec{u}^T\vec{q} +f
}
\brak{\vec{m}^T\vec{V}\vec{m}}
}
}
\label{eq:tangent_roots}
\end{multline}
}
\fi
For the line $x-1=0$, the parameters are  
\begin{align}
	\vec{q}_2=\myvec{1\\0},
	\vec{m}_2=\myvec{0\\1}
\end{align}
Substituting from the above in 
\eqref{eq:tangent_roots},
\begin{align}
\kappa_i=1,-1
\end{align}
yilelding 
the points of intersection 
\begin{align}
	\vec{a}_0=\myvec{1\\1},
	\vec{a}_1=\myvec{1\\-1}
\end{align}
Similarly, 
for the line $x-4=0$ 
\begin{align}
\vec{q_1}=\myvec{4\\0},
\vec{m_1}=\myvec{0\\1}
\end{align}
yielding
\begin{align}
\kappa_i=2,-2
\end{align}
from which, the points of 
intersection are
\begin{align}
\vec{a_3}=\myvec{4\\2},
\vec{a_2}=\myvec{4\\-2}
\end{align}
		See \figref{fig:12/8/1/1}.
Thus, 
the area of the parabola in between the lines $x=1$ and $x=4$ is given by
\begin{align}
\int_{0}^{4} \ \sqrt{x} \,dx-\int_{0}^{1} \ \sqrt{x} \,dx
=14/3
\end{align}

\item 
Find the area of the region bounded by the curve $y^2=9x$ and the lines $x=2$ and $x=4$ and the axis in the first quadrant.
\\
\solution
\label{chapters/12/8/1/2}
	\begin{figure}[H]
		\centering
 \includegraphics[width=0.75\columnwidth]{chapters/12/8/1/2/figs/conics1.png}
		\caption{}
		\label{fig:12/8/1/2}
  	\end{figure}
The parameters of the conic are
\begin{align}
 \vec{V} = \myvec{0 & 0\\0 & 1},
	\vec{u} = \frac{9}{2}\myvec{1 \\0},
 f = 0.
\end{align}
The parameters of 
the line $x-2=0$ are
\begin{align}
\vec{q_2}=\myvec{2\\0},
\vec{m_2}=\myvec{0\\1}
\end{align}
Substituting in 
\eqref{eq:tangent_roots},
\begin{align}
\kappa_i=\pm 3\sqrt{2}
\end{align}
yielding
\begin{align}
\vec{a_0}=\myvec{2\\3\sqrt{2}},
\vec{a_1}=\myvec{2\\-3\sqrt{2}}.
\end{align}
Similarly, 
for the line $x-4=0$,
\begin{align}
\vec{q_1}=\myvec{4\\0},
\vec{m_1}=\myvec{0\\1}
\end{align}
yielding
\begin{align}
\kappa_i=\pm 6.
\end{align}
Thus, 
\begin{align}
\vec{a_3}=\myvec{4\\6},
\vec{a_2}=\myvec{4\\-6}
\end{align}
and 
		from \figref{fig:12/8/1/2},
the 
desired area of the parabola is
\begin{align}
\int_{0}^{4} \ 3\sqrt{x} \,dx-\int_{0}^{2} \ 3\sqrt{x} \,dx
=16-4\sqrt{2}
\end{align}

\item Find the area of the region bounded by ${x}^2
= 4{y}$, ${y} = 2$, ${y} = 4$ and the y-axis in the
first quadrant.
\label{chapters/12/8/1/3}
\item Find the area of the region bounded by the ellipse \(\frac{{x}^2}{16}\ + \frac{{y}^2}{9} = 1\)
\label{chapters/12/8/1/4}
\item Find the area of the region bounded by the ellipse \(\frac{{x}^2}{4}\ + \frac{{y}^2}{9} = 1\)
\label{chapters/12/8/1/5}
\item 
		  Find the area of the region in the first quadrant enclosed by the x-axis, line $x=\sqrt{3}y$ and circle $x^2+y^2=4$.
		  \\
		  \solution
\label{chapters/12/8/1/6}
	\begin{figure}[H]
		\centering
 \includegraphics[width=0.75\columnwidth]{chapters/12/8/1/6/figs/conics-fig.pdf} 
		\caption{}
		\label{fig:12/8/1/6}
  	\end{figure}
  From the given information, the parameters of the  circle and line are
                      \begin{align}
			      f= -4, \vec{u}=\vec{0}, \vec{V}=\vec{I}, \vec{m}=\myvec{1 \\ \sqrt{3}}, \vec{h} = \vec{0}
		\label{eq:12/8/1/6}
                    \end{align}                                                                              
Substituting		    the above parameters in  
\eqref{eq:tangent_roots},
	  \begin{align}                                                                               
		  \mu= \sqrt{3}
	  \end{align}
	  yielding  
the desired point of intersection as                                               
\begin{align}
	\vec{x} = \myvec{\sqrt{3} \\ 1}                               
\end{align}
Note that we have chosen only the point of intersection in the first quadrant as shown in 
		\figref{fig:12/8/1/6}.
From
		\eqref{eq:12/8/1/6},
		the angle between the given line and the x axis is
\begin{align}
	\theta=30\degree
\end{align} 
and
the area of the sector is 
\begin{align}
	{\frac{\theta}{360}}\pi r^2=
	\frac{\pi}{3}
\end{align}

\item 
	Find the area of the smaller part of the circle $x^2+y^2=a^2 $ cut off by the line $x=\frac{a}{\sqrt{2}}$.
	\\
	\solution
\label{chapters/12/8/1/7}
	\begin{figure}[H]
		\centering
 \includegraphics[width=0.75\columnwidth]{chapters/12/8/1/7/figs/conic.png}
		\caption{}
		\label{fig:12/8/1/7}
  	\end{figure}
The given circle can be expressed as a conic with parameters
\begin{align}
\vec{V}=
\myvec{
1 & 0\\
0 & 1
},
\vec{u}=0,
f=-a^2
\end{align} 
The given line 
parameters are
\begin{align} 
	\vec{h}=\myvec{\frac{a}{\sqrt{2}} \\ 0},  \vec{m}=\vec{e}_2.
\end{align}
Substituting the above in
\eqref{eq:tangent_roots},
\begin{align}
    \kappa =\pm\frac{a}{\sqrt{2}}
\end{align}
yielding the
points of intersection of the line with circle as
\begin{align}
    \vec{A}=\myvec{
\frac{a}{\sqrt{2}}\\
-\frac{a}{\sqrt{2}}
    },
    \vec{B}=\myvec{
\frac{a}{\sqrt{2}}\\
\frac{a}{\sqrt{2}}
    }
\end{align}
 From 
		\figref{fig:12/8/1/7},
the total area of the portion is given by
\begin{align}
	ar( APQ)&=2 ar (APR)
	\\
&=2\int_{0}^{\frac{a}{\sqrt{2}}}\sqrt{a^2-x^2}\,dx 
	\\
	&=\frac{a^2}{2}\brak{1+\frac{\pi}{2}}
\end{align}

\item 
The area between $x = y^2$ and $x = 4$ is divided into two equal parts by the line $x = a$, find the value of a.
\\
\solution
\label{chapters/12/8/1/8}
	\begin{figure}[H]
		\centering
 \includegraphics[width=0.75\columnwidth]{chapters/12/8/1/8/figs/conics1.png}
		\caption{}
		\label{fig:12/8/1/8}
  	\end{figure}
The given conic parameters are
\begin{align}
 \vec{V} = \myvec{0 & 0\\0 & 1},
	\vec{u} = -\frac{1}{2}\vec{e}_1
 f = 0
\end{align}
The parameters of the lines are
\begin{align}
\vec{q}_2=\myvec{a\\0},
\vec{m}_2=\vec{e}_2
\end{align}
Substituting the above values in 
\eqref{eq:tangent_roots},
\begin{align}
\mu_i=a,-a
\end{align}
yielding  the points of  intersection as
\begin{align}
\vec{a_0}=\myvec{a\\a},
\vec{a_1}=\myvec{a\\-a}
\end{align}
Similarly, for the line $x-4=0$, 
\begin{align}
\vec{q_1}=\myvec{4\\0},
\vec{m_1}=\vec{e}_2
\end{align}
yielding
\begin{align}
\mu_i=2,-2
\end{align}
and
\begin{align}
\vec{a}_3=\myvec{4\\2},
\vec{a}_2=\myvec{4\\-2}.
\end{align}
Area between parabola and the line $x=4$ is divided equally by the line $x=a$.  Thus, 
		from \figref{fig:12/8/1/8},
\begin{align}
	A_1&=\int_{0}^{a} \ \sqrt{x} \,dx
	\\
	A_2&=\int_{a}^{4} \ \sqrt{x} \,dx
	\\
	\text{ and }
	A_1&=A_2 \\
\implies 
	a&=4^\frac{2}{3}
\end{align}

\iffalse
\section*{\large Construction}

{
\setlength\extrarowheight{5pt}
\begin{tabular}{|l|c|}
    \hline 
    \textbf{Points} & \textbf{intersection points} \\ \hline
   a0 & $\myvec{
   a\\
   a
   } $ \\\hline
   a1 & $\myvec{
   a\\
   -a
   } $ \\\hline
    
   a3 & $\myvec{
   4\\
   2
   } $ \\\hline
   a2 & $\myvec{
   4\\
   -2
   } $ \\\hline
      
      \end{tabular}
}

\end{document}
\fi

\item 
	Find the area of the region bounded by the parabola $y=x^2$ and $y= \abs{x}$.
\label{chapters/12/8/1/9}
\item 
Find the area bounded by the curve $x^2=4y$ and the line $x=4y-2$.
\\
\solution 
\label{chapters/12/8/1/10}
	\begin{figure}[H]
		\centering
 \includegraphics[width=0.75\columnwidth]{chapters/12/8/1/10/figs/conic.png}
		\caption{}
		\label{fig:12/8/1/10}
  	\end{figure}
The given curve  can be expressed as a conic with parameters
\begin{align}
	\vec{V} &= \myvec{1 & 0\\0 & 0}, \vec{u} = \myvec{0 \\-2}, f = 0
	\end{align}
The parameters of the given line are
\begin{align}
\vec{q} = \myvec{-2 \\0} , \vec{m}=\myvec{4\\1}
\end{align}
The points of intersection can then be obtained from \eqref{eq:tangent_roots} 
\begin{align}
\therefore \vec{x}_1=\myvec{2\\1} , \vec{x}_2=\myvec{-1\\ \frac{1}{4}}
\end{align}
The desired area is then obtained 
from 		\figref{fig:12/8/1/10}
as
\begin{align}
	A&=\int_{x_2}^{x_1} [f(x)-g(x)] \,dx
	\\
	&=\int_{-1}^{2} \brak{\frac{x+2}{4}-\frac{x^2}{4}} \,dx
	\\
	& = \frac{9}{8} 
\end{align}

\item Find the area of the region bounded by the curve ${y}^2
= 4{x}$ and the line ${x} = 3$.
\label{chapters/12/8/1/11}
\item Area lying in the first quadrant and bounded by the circle ${x}^2 + {y}^2 = 4$ and the lines ${x} = 0$ and ${x} = 2$ is 
\label{chapters/12/8/1/12}
\begin{enumerate}[itemsep=+2mm]
\item $\pi$
\item $\dfrac{\pi}{2}$
\item $\dfrac{\pi}{3}$  
\item $\dfrac{\pi}{4}$
\end{enumerate}
\item Find the area of the region bounded by the curve $y^2 = 4x$, y-axis and the line $y = 3$. 
\label{chapters/12/8/1/13}
\\
\solution
In this case, 
\begin{align}
	\vec{V} &= \myvec{ 0 & 0 \\ 0 & 1} \\
	\vec{u} &= \myvec{-2 \\ 0} \\
	f &= 0
\end{align}
For the given line $y=3$, the parameters are
\begin{align}
	\vec{h} = \myvec{0 \\ 3} , \vec{m} = \myvec{1 \\ 0 }
\end{align}
The intersection of 
the line with the conic is obtained from \eqref{eq:tangent_roots} 
as
\begin{align}
	\kappa  = \frac{9}{4} 
\end{align}
The point of contact is given as
\begin{align}
	\vec{a}_0 = \myvec{\frac{9}{4}  \\[1pt] \\ 3}
\end{align}
From \figref{fig:chapters/12/8/1/13/Fig1},
the desired area of the region is obtaioned as
\begin{align}
	\int_{0}^{3} \ \frac{y^2}{4} \,dy &= \frac{1}{12}\sbrak{y^3}_{0}^{3} \\
	&= \frac{1}{12}\brak{27-0} \\
	&= \frac{9}{4} \text{ sq.units}
\end{align}
\begin{figure}[H]
	\begin{center}
		\includegraphics[width=0.75\columnwidth]{chapters/12/8/1/13/figs/problem13.pdf}
	\end{center}
\caption{}
\label{fig:chapters/12/8/1/13/Fig1}
\end{figure}

\item 
Find the area of the region bounded by the curve $x^2=4y$ and the lines $y=2$ and $y=4$ and the y-axis in the first quadrant.
\\
\solution
\label{chapters/12/8/3/3}
	\begin{figure}[H]
		\centering
 \includegraphics[width=0.75\columnwidth]{chapters/12/8/3/3/figs/conic.png}
		\caption{}
		\label{fig:12/8/3/3}
  	\end{figure}
The conic parameters are
\begin{align}
	\vec{V} = \myvec{1 & 0\\0 & 0},
	\vec{u} = \myvec{0\\-2},
	f = 0
	%\\
\end{align}
The vector parameters of 
$y-4=0$
are
\begin{align}
	\vec{h}_1=\myvec{0\\4},
	\vec{m}_1=\myvec{1\\0}
\end{align}
Substituting the above in \eqref{eq:tangent_roots},
\begin{align}
\kappa_i=4,-4
\end{align}
yielding
the points of intersection with the parabola as
\begin{align}
\vec{a}_0=\myvec{4\\4},
\vec{a}_1=\myvec{-4\\4}
\end{align}
Similarly, for 
the line $y-2=0$, the vector parameters are
\begin{align}
\vec{h}_2=\myvec{0\\2},
\vec{m}_2=\myvec{1\\0}
\end{align}
yielding 
\begin{align}
\kappa_i=2.8,-2.8
\end{align}
and the points of intersection
\begin{align}
\vec{a}_2=\myvec{2.8\\2},
\vec{a}_3=\myvec{-2.8\\2}
\end{align}
From 
		\figref{fig:12/8/3/3},
the area of the parabola between the lines $y=2$ and $y=4$ is given by
\begin{align}
\int_{0}^{4} \ 2\sqrt{y} \,dy-\int_{0}^{2} \ 2\sqrt{y} \,dy
=6.895 
\end{align}

\item 
	Find the area enclosed by the parabola $4y=3x^2 $ and the line $2y=3x+12$.\\
	\solution
\label{chapters/12/8/3/7}
	\begin{figure}[H]
		\centering
 \includegraphics[width=0.75\columnwidth]{chapters/12/8/3/7/figs/conic.png}
		\caption{}
		\label{fig:12/8/3/7}
  	\end{figure}
The parameters of the given conic are
\begin{align}
\vec{V}=\myvec{
3 & 0\\
0 & 0
},
\vec{u}=\myvec{0\\-2},
f=0.
\end{align} 
For the line, the parameters are
\begin{align}
\vec{h} = \myvec{
-2\\
3
},
\vec{m} = \myvec{2 \\ 3}
\end{align}
yielding
\begin{align}
    \kappa=-2.5,2.7
\end{align}
upon substitution in \eqref{eq:tangent_roots}
resulting in the points of intersection
\begin{align}
    \vec{A}=\myvec{
-2\\
3
    },
    \vec{B}=\myvec{
4\\
12
    }.
\end{align}
From 
		\figref{fig:12/8/3/7},
the desired area is 
\begin{align}
\int_{-2}^{4} \frac{3x+12}{2} \,dx
-\int_{-2}^{4}\frac{3x^2}{4} \,dx 
= 27 
\end{align}

\item 
	Find the area of the smaller region bounded by the ellipse $\frac{x^2}{9}+\frac{y^2}{4}=1$
and the line $\frac{x}{3}+\frac{y}{2}=1$.\\
\solution
\label{chapters/12/8/3/8}
	\begin{figure}[H]
		\centering
 \includegraphics[width=0.75\columnwidth]{chapters/12/8/3/8/figs/conic_fig.png}
		\caption{}
		\label{fig:12/8/3/8}
  	\end{figure}
The given ellipse can be expressed as conics with parameters
\begin{align}
\vec{V}=\myvec{
b^2 & 0\\
0 & a^2
},
\vec{u}=0,
f=-(a^2b^2).
\end{align} 
The line parameters are
\begin{align}
\vec{h} &= \myvec{
a\\
0
},
\vec{m} = \myvec{\frac{1}{b} \\ -\frac{1}{a}}.
\end{align}
Substituting the given parameters in \eqref{eq:tangent_roots},
\begin{align}
    \mu=0,-6
\end{align}
yielding the points of intersection
\begin{align}
    \vec{A}=\myvec{
a\\
0
    },
    \vec{B}=\myvec{
0\\
b
    }.
\end{align}
From 
		\figref{fig:12/8/3/8},
the desired area is
\begin{multline}
\int_{0}^{a}\frac{b}{a}\sqrt{a^2-x^2} \,dx 
-\int_{0}^{a} \frac{b}{a}(a-x) \,dx
\\
	= \frac{ab}{2}\brak{\frac{\pi}{2}-1}
	= 3\brak{\frac{\pi}{2}-1}
\end{multline}
upon substituting $a=3, b=2$.

\item 
Find the area of the region bounded by the curve $x^2=y$ and the lines $y=x+2$ and the $x$ axis.
\label{chapters/12/8/3/10}
\item 
Find   the area bounded by the curve $y=x|x|, x$-axis and the ordinates $x$=-1 and $x$=1.
\label{chapters/12/8/3/17}
\item 
	Find the area of the region bounded by the curves $y=x^2+2$, $y=x$, $x=0$ and $x=3. $
\label{chapters/12/8/2/3}
\item 
Find the smaller area enclosed by the circle $x^2 + y^2 = 4$ and the line $x + y = 2$. 
\\
\solution
\label{chapters/12/8/2/6}
	\begin{figure}[H]
		\centering
 \includegraphics[width=0.75\columnwidth]{chapters/12/8/2/6/figs/conic.png}
		\caption{}
		\label{fig:12/8/2/6}
  	\end{figure}
The given circle can be expressed as conics with parameters,
\begin{align}
\vec{V}=\myvec{
4 & 0\\
0 & 4
},
\vec{u}=0,
f=-16
\end{align}
The line parameters are
\begin{align}
\vec{h} &= \myvec{
2\\
0
}, 
\vec{m} = \myvec{\frac{1}{2} \\ -\frac{1}{2}}
\end{align}
Substituting the parameters in \eqref{eq:tangent_roots},
\begin{align}
\kappa =0,-4
\end{align}
yielding the points of intersection as
\begin{align}
    \vec{A}=\myvec{
0\\
2
    },
    \vec{B}=\myvec{
2\\
0
    }
\end{align}
From 
		\figref{fig:12/8/2/6},
the desired area is
\begin{align}
\int_{0}^{2}\sqrt{4-x^2} \,dx 
-\int_{0}^{2} (2-x) \,dx
=\pi - 2
\end{align}

\item Find the area of the region bounded by the curves $y^2 = 9x$, $y = 3x$.
\item Find the area of the region bounded by the parabola $y^2 = 2px$, $x^2 = 2py$.
\item Find the area of the region bounded by the curve $y = x^2\text{ and }y = x + 6\text{ and }x = 0$.
\item Find the area of the region bounded by the curve $y^2 = 4x$, $x^2 = 4y$.
\item Find the area of the region included between $y^2 = 9x\text{ and }y =x$
\item Find the area of the region enclosed by the parabola $x^2 = y$ and the line $y = x + 2$
\item Find the area of region bounded by the line $x = 2$ and the parabola $y^2 = 8x$
\item Sketch the region ${(x,0) : y = \sqrt{4 - x^2}}$ and $x$-axis. Find the area of the region using integration.
\item Calculate the area under the curve $y = 2\sqrt{x}$ included between the lines $x = 0\text{ and }x = 1$.
\item Using integration, find the area of the region bounded by the line $2y = 5x + 7$, $x$-axis and the lines $x = 2\text{ and }x =8$.
\item Draw a rough sketch of the curve $y = \sqrt{x - 1}$ in the interval $[1, 5]$. Find the area under the curve and between the lines $x = 1\text{ and }x = 5$.
\item Determine the area under the curve $y = \sqrt{a^2 - x^2}$ included between the lines $x = 0\text{ and }x = a$
\item Find the area of the region bounded by $y = \sqrt{x}\text{ and }y = x$.
\item Find the area enclosed by the curve $y = - x^2$ and the straight line $x + y + 2 = 0$.
\item Find the area bounded by the curve $y = \sqrt{x}$, $x = 2y + 3$ in the first quadrant and $x$-axis.
\item Draw a rough sketch of the region ${(x, y) : y^2 \lessgtr 6ax\text{ and }x^2 + y^2 \lessgtr 16a^2}$.
\item Draw a  rough sketch of the given curve $y =1 + \abs{x + 1}$, $x = -3$, $x = 3$, $y = 0$, and find the area of the region bounded by them, using integration.
\item The area of the region bounded by the curve $x^2 = 4y$ and the straight line $x = 4y - 2$ is
\begin{enumerate}
\item $\frac{3}{8}$ sq units 
\item $\frac{5}{8}$ sq units
\item $\frac{7}{8}$ sq units 
\item $\frac{9}{8}$ sq units
\end{enumerate}
\item The area of the region bounded by the curve $y = \sqrt{16 - x^2}$ and $x$-axis is 
\begin{enumerate}
\item 8 sq units 
\item ${20\pi}$ sq units
\item ${16\pi}$ sq units
\item ${256\pi}$ sq units
\end{enumerate}
\item Area of the region in the first quadrant enclosed by the $x$-axis, the line $y = x$ and the circle $x^2 + y^2 = 32$ is 
\begin{enumerate}
\item ${16\pi}$ sq units 
\item ${4\pi}$ sq units
\item ${32\pi}$ sq units
\item ${24\pi}$ sq units
\end{enumerate}
\item The area of the region bounded by parabola $y^2 = x$ and the straight line $2y = x$ is
\begin{enumerate}
\item $\frac{4}{3}$ sq units
\item 1 sq units
\item $\frac{2}{3}$ sq units 
\item $\frac{1}{3}$ sq units
\end{enumerate}
\item Find the equation of a circle whose centre is (3,1) and which cuts off a chord of length  6 units on the  line $2x-5y+18=0$.
\end{enumerate}

\subsection{Curves}
\begin{enumerate}[label=\thesubsection.\arabic*,ref=\thesubsection.\theenumi]

\item 
	Find the area of the circle $4x^2+4y^2=9$ which is interior to the parabola $x^2=4y$.\\
	\solution
\label{chapters/12/8/2/1}
	\begin{figure}[!h]
		\centering
 \includegraphics[width=\columnwidth]{chapters/12/8/2/1/figs/conic.jpg}
		\caption{}
		\label{fig:12/8/2/1}
  	\end{figure}
The given circle and parabola can be expressed as conics with parameters 
\begin{align}
	\vec{V}_1&=4\vec{I},
\vec{u_1}=\vec{0},
f_1=-9
\\
	\vec{V}_2&=\myvec{
1 & 0\\
0 & 0
},
\vec{u_2}= -\myvec{
0\\
2
},
f_2=0
\end{align} 
	  From \eqref{eq:pair-mat-sing-conic-det},
\begin{align}
\mydet{\mu+4 & 0 & 0\\ 
0 & 4 & -2\mu \\
0 & -2\mu & -9
} &= 0
\\
	\implies   \mu &= -4.
\end{align}
 Thus, the parameters for the pair of  straight lines can be expressed as 
 \begin{align}
	\vec{V} &= 
\vec{V}_1 + \mu\vec{V}_2
=\myvec{ 0 & 0 \\ 0 & 4},
\\
	\vec{u} &=
\vec{u}_1+\mu \vec{u}_2
	= \myvec{
0\\
8
    }
\\
	f&=-9,
	\\
	\implies \vec{D} &= \vec{V}, \vec{P} = \vec{I}
    \end{align}
On substituting
\begin{align}
\vec{q} &= \myvec{
0\\
0.5
} 
\end{align}
\begin{align}
\vec{m} = \myvec{2 \\ 0}
\end{align}
With the given Parabola,\\ 
\begin{align}
	\vec{V} &= \myvec{
1 & 0\\
0 & 0
    }
\end{align}
\begin{align}
	\vec{u} = -\myvec{2 \\0}
 \end{align}
 \begin{align}
  f = 0
 \end{align}
The value of $\kappa$ ,\\
\begin{align}
    \kappa = \sqrt{2},-\sqrt{2}
\end{align}
The points of intersection with Parabola along circle are \\
\begin{align}
    \vec{A}=\myvec{
\sqrt{2}\\
0.5
    }
\end{align}
\begin{align}
    \vec{B}=\myvec{
-\sqrt{2}\\
0.5
    }
\end{align}
 From the figure,
total area of portion is given by, 
\begin{align}
 A=  \int_{-\sqrt{2}}^{\sqrt{2}} g(x)-f(x) \,dx 
\end{align}
Where g(x) is area of circle and f(x) is the area of parabola around the points\\ 
\begin{align}
A= \int_{-\sqrt{2}}^{\sqrt{2}} \frac{\sqrt{9-4x^2}}{2}-\frac{x^2}{4} \,dx 
\end{align}
Area  
\begin{align}
    A= 3.0053609 \,m^2
\end{align}

\item Find the area bounded by the curves $\brak{x-1}^2 + y^2 = 1 \text{ and } x^2+y^2=1$.
\label{chapters/12/8/2/2}
\\
\solution
The conic parameters for the two circles can be expressed as
\begin{align}
\begin{split}
	\vec{V}_1 &= \myvec{1&0\\0&1},\
	\vec{u}_1 = \myvec{-1\\0},\
	f_1 = 0,
	\\
	\vec{V}_2 &= \myvec{1&0\\0&1},\
	\vec{u}_2 = \myvec{0\\0},\
	f_2 = -1.
\end{split}
	\label{eq:12/8/2/2/vuf}
\end{align}
On substituting from
	\eqref{eq:12/8/2/2/vuf}
	in
	  \eqref{eq:pair-mat-sing-conic-det}, we obtain
\begin{align}
	\mydet{1+\mu & 0 & -1 \\ 0 & 1+\mu & 0 \\ -1 & 0 & -\mu} = 0
\end{align}
yileding
\begin{align}
	\implies \mu = -1.
\end{align}
Substituting 
	\eqref{eq:12/8/2/2/vuf}
in 
	  \eqref{eq:pair-mat-sing-conic},
	  we obtain
\begin{align}
	\vec{x}^\top\myvec{0&0\\0&0}\vec{x} + 2\myvec{-1&0}\vec{x} + 1 &= 0\\
\implies	\myvec{-2&0}\vec{x} &= -1 
\label{eq:12/8/2/2/chord}
\end{align}
Therefore the intersection of the two circles is a line with parameters
\begin{align}
	\vec{m} = \myvec{0\\1},  \vec{h} = \myvec{\frac{1}{2}\\0}.
\end{align}
	The intersection parameters of the 
	chord in 
\eqref{eq:12/8/2/2/chord}
with the 
first circle in \eqref{eq:12/8/2/2/vuf} is obtained from 
\eqref{eq:tangent_roots}
	as
\begin{align}
	\kappa_i= \pm\frac{\sqrt{3}}{2}
\end{align}
Hence the point of intersection are
obtained from
	\eqref{eq:chord-pts}
	as
\begin{align}
	\vec{a}_0 = \myvec{\frac{1}{2}\\[1ex]\frac{\sqrt{3}}{2}}, \vec{a}_2 = \myvec{\frac{1}{2}\\[1ex]-\frac{\sqrt{3}}{2}}.
\end{align}
The desired area of region is given as
\begin{multline}
	2\brak{\int_{0}^{\frac{1}{2}} \sqrt{1-\brak{x-1}^2}dx + \int_{\frac{1}{2}}^{1} \sqrt{1-x^2}dx}\\
		=2\sbrak{\frac{1}{2}\brak{x-1}\sqrt{1-\brak{x-1}^2}+\frac{1}{2}\sin^{-1}\brak{x-1}}_{0}^{\frac{1}{2}}\\
		 +2\sbrak{\frac{x}{2}\sqrt{1-x^2}+\frac{1}{2}\sin^{-1}x}_{\frac{1}{2}}^{1}
		 \\
	= \frac{2\pi}{3}-\frac{\sqrt{3}}{2}
\end{multline}
See \figref{fig:chapters/12/8/2/2/Fig1}.
\begin{figure}[H]
	\begin{center} 
	    \includegraphics[width=0.75\columnwidth]{chapters/12/8/2/2/figs/inter1}
	\end{center}
\caption{}
\label{fig:chapters/12/8/2/2/Fig1}
\end{figure}



















\item 
\label{chapters/12/8/3/2}
	\begin{figure}[H]
		\centering
 \includegraphics[width=0.75\columnwidth]{chapters/12/8/3/2/figs/figure.png}
		\caption{}
		\label{fig:12/8/3/2}
  	\end{figure}
The given curve  can be expressed as a conic with parameters
\begin{align}
	\vec{V} &= \myvec{1 & 0\\0 & 0}, \vec{u} = \myvec{0 \\-\frac{1}{2}}, f = 0
	\end{align}
The given line parameters are
\begin{align}
\vec{h} = \myvec{0 \\0}, \vec{m}=\myvec{1\\1}
\end{align}
Substituting the given parameters in 
\eqref{eq:tangent_roots},
\begin{align}
\vec{x}_1=\myvec{0\\0}, \vec{x}_2=\myvec{1\\1}.
\end{align}
From  
		\figref{fig:12/8/3/2},
the area bounded by the curve $y=x^2$ and line $y=x$ is given by
\begin{align}
	\int_{0}^{1} \brak{x 
	-\frac{x^2}{2}} \,dx = \frac{1}{6}
\end{align}

\item 
\label{chapters/12/8/3/18}
	\begin{figure}[H]
		\centering
 \includegraphics[width=0.75\columnwidth]{chapters/12/8/1/7/figs/conic.png}
		\caption{}
		\label{fig:12/8/1/7}
  	\end{figure}
The given circle can be expressed as a conic with parameters
\begin{align}
\vec{V}=
\myvec{
1 & 0\\
0 & 1
},
\vec{u}=0,
f=-a^2
\end{align} 
The given line 
parameters are
\begin{align} 
	\vec{h}=\myvec{\frac{a}{\sqrt{2}} \\ 0},  \vec{m}=\vec{e}_2.
\end{align}
Substituting the above in
\eqref{eq:tangent_roots},
\begin{align}
    \kappa =\pm\frac{a}{\sqrt{2}}
\end{align}
yielding the
points of intersection of the line with circle as
\begin{align}
    \vec{A}=\myvec{
\frac{a}{\sqrt{2}}\\
-\frac{a}{\sqrt{2}}
    },
    \vec{B}=\myvec{
\frac{a}{\sqrt{2}}\\
\frac{a}{\sqrt{2}}
    }
\end{align}
 From 
		\figref{fig:12/8/1/7},
the total area of the portion is given by
\begin{align}
	ar( APQ)&=2 ar (APR)
	\\
&=2\int_{0}^{\frac{a}{\sqrt{2}}}\sqrt{a^2-x^2}\,dx 
	\\
	&=\frac{a^2}{2}\brak{1+\frac{\pi}{2}}
\end{align}

\item Find the area of the region bounded by the curve $y^2 = 2x\text{ and }x^2 + y^2 = 4x$.
\end{enumerate}

\subsection{Formulae}
\begin{enumerate}[label=\thesubsection.\arabic*,ref=\thesubsection.\theenumi]
		\item
  The points of intersection of the line 
\begin{align}
L: \quad \vec{x} = \vec{h} + \kappa \vec{m} \quad \kappa \in \mathbb{R}
\label{eq:conic_tangent}
\end{align}
with the conic section in \eqref{eq:conic_quad_form} are given by
\begin{align}
\vec{x}_i = \vec{h} + \kappa_i \vec{m}
	\label{eq:chord-pts}
\end{align}
%
where
\begin{multline}
\kappa_i = \frac{1}
{
\vec{m}^{\top}\vec{V}\vec{m}
}
\lbrak{-\vec{m}^{\top}\brak{\vec{V}\vec{h}+\vec{u}}}
%\\
\pm
%{\small
\rbrak{\sqrt{
\sbrak{
\vec{m}^{\top}\brak{\vec{V}\vec{h}+\vec{u}}
}^2
	-\text{g}
\brak
{\vec{h}
%\vec{h}^{\top}\vec{V}\vec{h} + 2\vec{u}^{\top}\vec{h} +f
}
\brak{\vec{m}^{\top}\vec{V}\vec{m}}
}
}
%}
\label{eq:tangent_roots}
\end{multline}
See 
	 \ref{prop:chord}
	 for proof.

  \item
\eqref{eq:conic_quad_form} represents a pair of straight lines if 
the matrix 
  \begin{align} 
	  \myvec{\vec{V} & \vec{u}\\ \vec{u}^{\top} & f}  
	  \label{eq:pair-mat-sing}
  \end{align} 
  is singular.
\item The intersection of two conics 
with parameters $\vec{V}_i, \vec{u}_i, f_i,\ i = 1,2$
	is defined
as
\begin{align}
	\vec{x}^{\top}\brak{\vec{V}_1 + \mu\vec{V}_2}\vec{x}+2 \brak{\vec{u}_1+\mu \vec{u}_2}^{\top} \vec{x} 
	+ \brak{f_1+\mu f_2}= 0
	  \label{eq:pair-mat-sing-conic}
    \end{align}
	  
	  
\item From \eqref{eq:pair-mat-sing}, \eqref{eq:pair-mat-sing-conic} represents a pair of straight lines if
\begin{align}
	  \label{eq:pair-mat-sing-conic-det}
\mydet{\vec{V}_1 + \mu\vec{V}_2 & \vec{u}_1+\mu \vec{u}_2\\ \brak{\vec{u}_1+\mu \vec{u}_2}^{\top} & f_1 + \mu f_2} &= 0
\end{align}
\end{enumerate}


\newpage
\section{Tangent And Normal}
\subsection{Circle}
\begin{enumerate}[label=\thesection.\arabic*,ref=\thesection.\theenumi]

\item Find the points on the curve $x^2+y^2-2x-3=0$ at which the tangents are parallel to the x-axis.
\label{chapters/12/6/3/19}
\\
\solution
	\begin{figure}[H]
		\centering
 \includegraphics[width=0.75\columnwidth]{chapters/12/6/3/8/figs/main.png}
		\caption{}
		\label{fig:12/6/3/8}
  	\end{figure}
The equation of the conic can be represented as
\begin{align}
\vec{x}^{\top}\myvec{1&0\\0&0}\vec{x}+2\myvec{-2&\frac{-1}{2}}\vec{x}+4=0
\end{align}
So,
\begin{align}
\vec{V}=\myvec{1&0\\0&0},
\vec{u}^{\top}=\myvec{-2&\frac{-1}{2}},
f=4
\end{align}
The direction vector of the line passing through (2,0) and (4,4) is 
\begin{align}
\vec{m}=\myvec{1\\2}
\implies
\vec{n}=\myvec{2\\-1}.
\end{align}
The eigenvector corresponding to the zero eigenvalue is 
\begin{align}
\vec{p}_1=\myvec{0\\1},
\end{align}
In
\eqref{eq:conic_tangent_q_eigen},
\begin{align}
	\kappa=\frac{\myvec{0&1}\myvec{-2\\ \frac{-1}{2}}}{\myvec{0&1}\myvec{2\\-1}}
	=\frac{1}{2}
\end{align}
Substituting  $\kappa$,
from 
\eqref{eq:conic_tangent_q_eigen},
\begin{align}
	\myvec{\sbrak{\myvec{-2\\\frac{-1}{2}}+\frac{1}{2}\myvec{2\\-1}}^{\top} \\ \myvec{1&0\\0&0}}\vec{q} &= \myvec{-4 \\ \frac{1}{2}\myvec{2\\-1}-\myvec{-2\\\frac{-1}{2}}}\\
	\implies
	\myvec{-1&-1 \\ 1&0 \\ 0&0}\vec{q}&=\myvec{-4 \\ 3 \\ 0}
\end{align}
yielding
\begin{align}
\myvec{-1&-1 \\ 1&0}\vec{q} = \myvec{-4\\3}
\end{align}
The augmented matrix is 
\begin{align*}
  \myvec{
                -1&-1&\vrule&-4\\
	        1&0&\vrule&3}
  \xleftrightarrow[]{R_1 \leftarrow R_1+ 2R_2}
     \myvec{
	         1&-1&\vrule&2\\
	         1&0&\vrule&3}
      \\
 \xleftrightarrow[]{R_2 \leftarrow R_2 - R_1}
     \myvec{
	         1&-1&\vrule&2\\
	         0&1&\vrule&1}
 \xleftrightarrow[]{R_1 \leftarrow R_1 + R_2}
     \myvec{
	         1&0&\vrule&3\\
	         0&1&\vrule&1}
      \\ \implies \vec{q}=\myvec{3\\1}
\end{align*}
which is the desired 
point of contact.
See Fig. 
		\ref{fig:12/6/3/8}.

\item Find the equation of a circle of redius 5 which is touching another circle $x^2+y^2-2x-4y-20=0$ at (5,5).
\item The equation of the circle having centre at (3,-4) and touching the line $5x+12y-12=0$ is \makebox[1cm]{\hrulefill}                     
 \item Find the equation of the circle which touches both the axes in first quadrant and whose radius is $a$.
 \item Find the equation of the circle which touches x-axis and whose centre is $(1,2)$
 \item lf the lines $3x-4y+4=0$ and $6x-8y-7=0$ are tangents to a circle, then find the radius of the circle.
 \item Find the equation of a circle which touches  both the axes and the line $3x-4y+8=0$ and lies in the third quadrant.
\item At what points on the curve $x^2+y^2-2x-4y+1=0$, the tangents are parallel to the y-axis?
\item The shortest distance from the point (2,7) to the circle $x^2+y^2- 14x-10y-151=0$ is equal to 5.
\item lf the line $lx+my=1$ is a tangent to the circle $x^2+y^2=a^2$, then the point $(1,m)$ lies an a circle.

\end{enumerate}

\subsection{Conic}
\begin{enumerate}[label=\thesubsection.\arabic*,ref=\thesubsection.\theenumi]
\item Find the slope of the tangent to the curve $y = \frac{x-1}{x-2}$, $x \neq 2$ at $x=10$.
	\\
\solution 
\label{chapters/12/6/3/2}
The given equation of the curve can be rearranged as
\begin{align}
	xy-x-2y+1 &= 0 \\
        \label{eq:chapters/12/6/3/2/Eq1}
	\implies \vec{x}^\top\myvec{0 & \frac{1}{2} \\ \frac{1}{2} & 0}\vec{x} + \myvec{-1 & -2}\vec{x}+1 &= 0 
\end{align}
Thus, 
\begin{align}
	\vec{V} &= \myvec{ 0 & \frac{1}{2} \\ \frac{1}{2} & 0} \\
	\vec{u} &= -\myvec{\frac{1}{2} \\ 1} \\
	f &= 1 
\end{align}
$\because q_1 = 10$, the point of contact can be obtained as
\begin{align}
	 \vec{q} =\myvec{q_1 \\ q_2} = \myvec{10 \\ \frac{9}{8}}
\end{align}
  From \eqref{eq:conic_tangent_mq},
 the normal vector of the tangent to \eqref{eq:chapters/12/6/3/2/Eq1} is
\begin{align}
	\vec{n} = \myvec{1 \\ 64}
	\implies
	\vec{m} = \myvec{1 \\ \frac{-1}{64}}
\end{align}
The eigenvector matrix 
\begin{align}
	\myvec{\vec{p}_1 & \vec{p}_2} = \frac{1}{\sqrt{2}}\myvec{1 & 1 \\ 1 & -1}
\end{align}
which implies that  the conic is a $45\degree$ rotated hyperbola.
See \figref{fig:chapters/12/6/3/2/Fig1}.
\begin{figure}[H]
	\begin{center}
		\includegraphics[width=0.75\columnwidth]{chapters/12/6/3/2/figs/problem2.pdf}
	\end{center}
\caption{}
\label{fig:chapters/12/6/3/2/Fig1}
\end{figure}

\item 
		Find a point on the curve \begin{align}y=(x-2)^2\end{align} at which a tangent is parallel to the chord joining the points (2,0) and (4,4).
			\\
			\solution 
\label{chapters/12/6/3/8}
	\begin{figure}[H]
		\centering
 \includegraphics[width=0.75\columnwidth]{chapters/12/6/3/8/figs/main.png}
		\caption{}
		\label{fig:12/6/3/8}
  	\end{figure}
The equation of the conic can be represented as
\begin{align}
\vec{x}^{\top}\myvec{1&0\\0&0}\vec{x}+2\myvec{-2&\frac{-1}{2}}\vec{x}+4=0
\end{align}
So,
\begin{align}
\vec{V}=\myvec{1&0\\0&0},
\vec{u}^{\top}=\myvec{-2&\frac{-1}{2}},
f=4
\end{align}
The direction vector of the line passing through (2,0) and (4,4) is 
\begin{align}
\vec{m}=\myvec{1\\2}
\implies
\vec{n}=\myvec{2\\-1}.
\end{align}
The eigenvector corresponding to the zero eigenvalue is 
\begin{align}
\vec{p}_1=\myvec{0\\1},
\end{align}
In
\eqref{eq:conic_tangent_q_eigen},
\begin{align}
	\kappa=\frac{\myvec{0&1}\myvec{-2\\ \frac{-1}{2}}}{\myvec{0&1}\myvec{2\\-1}}
	=\frac{1}{2}
\end{align}
Substituting  $\kappa$,
from 
\eqref{eq:conic_tangent_q_eigen},
\begin{align}
	\myvec{\sbrak{\myvec{-2\\\frac{-1}{2}}+\frac{1}{2}\myvec{2\\-1}}^{\top} \\ \myvec{1&0\\0&0}}\vec{q} &= \myvec{-4 \\ \frac{1}{2}\myvec{2\\-1}-\myvec{-2\\\frac{-1}{2}}}\\
	\implies
	\myvec{-1&-1 \\ 1&0 \\ 0&0}\vec{q}&=\myvec{-4 \\ 3 \\ 0}
\end{align}
yielding
\begin{align}
\myvec{-1&-1 \\ 1&0}\vec{q} = \myvec{-4\\3}
\end{align}
The augmented matrix is 
\begin{align*}
  \myvec{
                -1&-1&\vrule&-4\\
	        1&0&\vrule&3}
  \xleftrightarrow[]{R_1 \leftarrow R_1+ 2R_2}
     \myvec{
	         1&-1&\vrule&2\\
	         1&0&\vrule&3}
      \\
 \xleftrightarrow[]{R_2 \leftarrow R_2 - R_1}
     \myvec{
	         1&-1&\vrule&2\\
	         0&1&\vrule&1}
 \xleftrightarrow[]{R_1 \leftarrow R_1 + R_2}
     \myvec{
	         1&0&\vrule&3\\
	         0&1&\vrule&1}
      \\ \implies \vec{q}=\myvec{3\\1}
\end{align*}
which is the desired 
point of contact.
See Fig. 
		\ref{fig:12/6/3/8}.

\item 
Find the equation of all lines having slope  -1 that are tangents to the curve
\begin{align}
y = \frac{1}{x-1}, x \neq 1
\label{chapters/12/6/3/10}
\end{align}
	\\
	\solution 
	\begin{figure}[H]
		\centering
 \includegraphics[width=0.75\columnwidth]{chapters/12/6/3/10/figs/conic1.pdf}
		\caption{}
		\label{fig:12/6/3/10}
  	\end{figure}
From the given information, 
\begin{align}
	\vec{V}
	=\myvec{
		0 & \frac{1}{2}\\\frac{1}{2}& 0\\
	},
\vec{u} = \myvec{
0 \\-\frac{1}{2}\\
},  f = -1, m=-1
	\label{eq:matrix-10-13-param}
\end{align}
From the above, the  normal vector is
\begin{align}
\vec{n}=\myvec{
-m \\ 1
	} = \myvec{1 \\ 1}
	\label{eq:matrix-10-13-param-n}
\end{align}
From 
\eqref{eq:conic_tangent_qk},
	the point(s) of contact are given by
\begin{align}
	\vec{q}&=\vec{V}^{-1}(k_i\vec{n}-\vec{u}) 
	\text{ where},\\
	k_i&=\pm \sqrt{\frac{f_0}{\vec{n}^{\top}\vec{V}^{-1}\vec{n}}}\\
	f_0&=f+\vec{u}^{\top}\vec{V}^{-1}\vec{u}
\end{align}
Substituting from 
	\eqref{eq:matrix-10-13-param-n}
	and
	\eqref{eq:matrix-10-13-param}
	in the above,
\begin{align}
\vec{q}=\myvec{0 \\-1}, \myvec{2 \\ 1}.
\end{align}
From 
  \eqref{eq:conic_tangent_final},
the equations of tangents are given by
\begin{align}
(\vec{V}\vec{q}+\vec{u})^{\top}\vec{x}+\vec{u}^{\top}\vec{q}+f=0
\end{align}
yielding
\begin{align}
	\myvec{1 & 1}\vec{x}+1&=0\\
\myvec{1 & 1}\vec{x}-3&=0\\
\end{align}
See 
		\figref{fig:12/6/3/10}.

\item 
Find the equation of all lines having slope 2 which are tangents to the curve 
\begin{align}
y=\frac{1}{x-3}, x\neq{3} 
\end{align}
\solution 
\label{chapters/12/6/3/11}
	\begin{figure}[H]
		\centering
 \includegraphics[width=0.75\columnwidth]{chapters/12/6/3/11/figs/con_fig.png}
		\caption{}
		\label{fig:12/6/3/11}
  	\end{figure}
From the given information
\begin{align}
	\vec{V}
	&=\myvec{
		0 &\frac{1}{2}\\\frac{1}{2} & 0\\
	},
\vec{u} = \myvec{
0 \\-\frac{3}{2}
},  f = -1, m=2
	\label{12/6/3/11/eq1}
	\\
	\implies
	\vec{n}&=\myvec{
-m \\ 1 
} = \myvec{-2 \\ 1}  \\
\label{12/6/3/11/eq2}
\end{align}
Hence, the given curve is a hyperbola.
%\raggedright
Substituting  numerical values, we obtain the condition in 
	\eqref{prop:conic-p-contact-nonparab-cond},
which implies that the line with slope 2 is not a tangent.  This can be verified from  
		\figref{fig:12/6/3/11}.

\item 
 Find points on the curve $\frac{x^2}{9}+\frac{y^2}{16}=1$ at which the tangents are 
 \begin{enumerate}
	 \item parallel to x-axis\\  
	 \item parallel to y-axis
 \end{enumerate}
 \solution 
\label{chapters/12/6/3/13}
	\begin{figure}[H]
		\centering
 \includegraphics[width=0.75\columnwidth]{chapters/12/6/3/13/figs/conic_1.pdf}
		\caption{}
		\label{fig:12/6/3/13}
  	\end{figure}
The parameters of the given conic are
\begin{align}
	\lambda_1&=16,\lambda_2=9 \\ \vec{V} &= \myvec{	\lambda_1& 0 \\
			          0 & \lambda_2}  
		    , \vec{u} = \myvec{0 \\0}, f = -144
		\label{eq:12/6/3/13/params}
	\end{align}
\begin{enumerate}
	\item The 
normal vector  in this case is
\begin{align}
		\vec{n_1}=\myvec{0\\1}
\end{align}
which can be used along with the parameters in 
		\eqref{eq:12/6/3/13/params}
		to obtain 
\begin{equation}
\vec{q_1}=\myvec{0\\4},
\vec{q_2}=\myvec{0\\-4}
\end{equation}
using 
\eqref{eq:conic_tangent_qk}.
\item Simlarly, 
	choosing
\begin{align}
	\vec{n_2}&=\myvec{1\\0},
	\\
	\vec{q_3}&=\myvec{3\\0},
	\vec{q_4}=\myvec{-3\\0}
\end{align}
\end{enumerate}
		See \figref{fig:12/6/3/13}.

\item 
Find the equation of the tangent line to the curve
\begin{align}
y=x^2-2x+7
\end{align}
\begin{enumerate}
    \item parallel to the line $2x-y+9=0$.
    \item perpendicular to the line $5y-15x=13$.
\end{enumerate}
\solution
\label{chapters/12/6/3/15}
	\begin{figure}[H]
		\centering
 \includegraphics[width=0.75\columnwidth]{chapters/12/6/3/15/figs/conic.png}
		\caption{}
		\label{fig:12/6/3/15}
  	\end{figure}
The parameters of the given conic are
\begin{align}
\vec{V} =\myvec{
	1 & 0\\
	0 & 0
	},
    \vec{u}=-\myvec{
	1 \\
	\frac{1}{2}
	},
    f=7
    \end{align}
\begin{enumerate}
	\item In this case,  the normal vector
		\begin{align}\vec{n}_1 = \myvec{2 \\ -1}\end{align}
			Since 
$\vec{V}$ is not invertible,  
		%Then given the normal vector \begin{align}\vec{n}\end{align}\, 
	the point of contact is given by 
\eqref{eq:conic_tangent_q_eigen} resulting in 
		%the matrix equation
\begin{align}
\myvec{\myvec{-1 \\ -\frac{1}{2}} + \frac{1}{2}\myvec{2 \\ -1}^{\top} \vspace{0.3cm}\\ \myvec{
	1 & 0\\
	0 & 0\\
	}} \vec{q}_1 = \myvec{-7 \vspace{0.3cm}\\ \frac{1}{2}\myvec{2 \\ -1} - \myvec{-1 \\ -\frac{1}{2}}}
\end{align}
By solving the above equation, we can get the point of contact as
    \begin{align}
  \vec{q}_1=\myvec{
	2 \\
	7 \\
	}
\end{align}
The 
tangent equation is then obtained as
\begin{align}
      \vec{n}_1^{\top}(\vec{x}-\vec{q}_1) = 0
  \\
	\implies  \myvec{2 & -1}\vec{x}+3 = 0
\end{align}
\item 
In this case, 
		\begin{align}\vec{n}_2 = \myvec{1 \\ 3} \end{align}
resulting in 
\begin{align}
\myvec{\myvec{-1 \\ -\frac{1}{2}} + -\frac{1}{6}\myvec{1 \\ 3}^{\top} \vspace{0.3cm}\\ \myvec{
	1 & 0\\
	0 & 0\\
	}} \vec{q}_2 = \myvec{-7 \vspace{0.3cm}\\ -\frac{1}{6}\myvec{1 \\ 3} - \myvec{-1 \\ -\frac{1}{2}}}
\end{align}
    \begin{align}
	    \text{or, }  \vec{q}_2=\myvec{
	\frac{5}{6}
\\
	\frac{217}{36}
	}
\end{align}
The tangent equation is
\begin{align}
      \vec{n}_2^{\top}(\vec{x}-\vec{q}_2) = 0
      \\
	\text{or, }    \myvec{1 & 3}\vec{x}= \frac{227}{12}
\end{align}
\end{enumerate}
		See \figref{fig:12/6/3/15}.

\item 
Find the equation of the tangent to the curve 
\begin{align}
	y = \sqrt{3x-2}
\end{align}
which is parallel to the line
\begin{align}
	4x-2y+5 = 0
\end{align}
\solution 
\label{chapters/12/6/3/25}
The parameters for the given conic are
\begin{align}
	\label{12/6/3/25eq:V_matrix}
	\vec{V} &= \myvec{0 & 0\\0 & 1},
	\\
	\label{12/6/3/25eq:u_vector}
	\vec{u} &= \myvec{-3/2\\0},
	\\
	\label{12/6/3/25eq:f_value}
	f &= 2
	%\\
\end{align}
which represent a parabola. 
Following the approach in 
\probref{chapters/12/6/3/15},
   \begin{align}
     \vec{p_1} = \myvec{1\\0},\
     \vec{n} = \myvec{-2\\1},
    \end{align}
yielding the matrix equation
\begin{align}
	\label{12/6/3/25eq:vertex_system}
	\myvec{-3&0\\0& 0\\0& 1}\vec{q} = \myvec{-41/16\\0 \\3/4}\\
\end{align}
The augmented matrix for \eqref{12/6/3/25eq:vertex_system} can be expressed as
\begin{align*}
	%\label{12/6/3/25eq:vertex_solv1}
	%\myvec{-3&0&\vrule&2\\0&0&\vrule&0\\0&1&\vrule&0}\\ 	
	%\label{12/6/3/25eq:vertex_solv2}
	\xleftrightarrow[]{R_2 \leftrightarrow R_3}\myvec{-3&0&\vrule&-41/16\\0&1&\vrule&0\\0&0&\vrule&3/4}
	\xleftrightarrow[]{-\frac{R_1}{-3} \leftarrow R_2}\myvec{1&0&\vrule&41/48\\0&1&\vrule&0\\0&0&\vrule&3/4}\\
	\implies\vec{q} = \myvec{\frac{41}{48}\\\frac{3}{4}}
\end{align*}
The equation of tangent is then obtained as
\begin{align}
	\myvec{-2 & 1}\vec{x} +\frac{23}{24} = 0 
\end{align}
See  
		\figref{fig:12/6/3/25}.
	\begin{figure}[H]
		\centering
 \includegraphics[width=0.75\columnwidth]{chapters/12/6/3/25/figs/conic.pdf}
		\caption{}
		\label{fig:12/6/3/25}
  	\end{figure}

\item 
Find the point at which the line $y = x + 1$ is a tangent to the curve $y^2 = 4x$.
\\
\solution 
\label{chapters/12/6/3/27}
	\begin{figure}[H]
		\centering
 \includegraphics[width=0.75\columnwidth]{chapters/12/6/3/27/figs/conicfig.png}
		\caption{}
		\label{fig:12/6/3/27}
  	\end{figure}
The parameters of the conic are
\begin{align}
 \vec{V} = \myvec{0&0\\0&1},  \vec{u} = \myvec{-2&0}, f = 0 
\end{align}
Following the approach in Problem 
\ref{chapters/12/6/3/15},
since
\begin{align}
	\vec{n} = \myvec{1 \\ -1}
\end{align}
we obtain
\begin{align}
	\vec{q} = \myvec{1 \\ 2}
\end{align}
See 
		\figref{fig:12/6/3/27}.


    \item The point on the curve 
\label{chapters/12/6/5/27}
    \begin{align}
        x^2 = 2y
        \label{eq:chapters/12/6/5/27/curve}
    \end{align}
    which is nearest to the point 
    $\vec{P} = \myvec{0\\5}$ is
    \begin{enumerate}
        \item $\myvec{2\sqrt{2}\\4}$
        \item $\myvec{2\sqrt{2}\\0}$
        \item $\myvec{0\\0}$
        \item $\myvec{2\\2}$
    \end{enumerate}
    \solution 
		We rewrite the conic \eqref{eq:chapters/12/6/5/27/curve} in matrix form.
    \begin{align}
        \vec{x}^\top\myvec{1&0\\0&0}\vec{x} + 2\myvec{0&-1}\vec{x} = 0
        \label{eq:chapters/12/6/5/27/curve-mtx}
    \end{align}
    Comparing with the general equation of the conic,
    \begin{align}
        \vec{V}_0 = \myvec{1&0\\0&0},\
        \vec{u}_0 = \myvec{0\\-1},\ 
        f_0 = 0 
    \end{align}
    Therefore, the equation of the normal where $\vec{q}$ is the point of contact 
    and 
\begin{align}
    \vec{R} \triangleq \myvec{0&-1\\1&0} 
\end{align}
is
    \begin{align}
	    \brak{\vec{V}_0\vec{q}+\vec{u}_0}^\top\vec{R}\brak{\myvec{0\\5}-\vec{q}} &= 0
        \label{eq:chapters/12/6/5/27/normal}
    \end{align}
    Substituting appropriate values and simplifying, we get 
    \begin{align}
	    \vec{q}^\top\myvec{0&1\\0&0}\vec{q} + 2\myvec{-2 & 0}\vec{q}= 0
        \label{eq:chapters/12/6/5/27/q-eqn}
    \end{align}
    which can be expressed as 
    \begin{multline}
	    \frac{1}{2}\lcbrak{
		    \vec{q}^\top\myvec{0&1\\0&0}\vec{q} + 2\myvec{-2 & 0}\vec{q}}
	\\
	    +
	    \rcbrak{
		    \vec{q}^\top\myvec{0&1\\0&0}^\top\vec{q} + 2\myvec{-2 & 0}\vec{q}}= 0
    \end{multline}
    yielding
    \begin{align}
	    \vec{q}^\top\myvec{0&\frac{1}{2}\\\frac{1}{2}&0}\vec{q} + 2\myvec{-2 &0}\vec{q} = 0
        \label{eq:chapters/12/6/5/27/q-affine}
    \end{align}
        \eqref{eq:chapters/12/6/5/27/q-affine}
	also looks like a conic with parameters
    \begin{align}
        \vec{V} = \myvec{1&0\\0&0},\
        \vec{u} = \myvec{0\\-1},\ 
        f = 0 
    \end{align}
The eigenparameters of $\vec{V}$ are
    \begin{align}
        \label{eq:chapters/12/6/5/27/PD}
        \vec{P} = \myvec{1&1\\1&-1},\ \vec{D} = \myvec{1&0\\0&-1}
    \end{align}
    Applying the affine transformation
    \begin{align}
        \label{eq:chapters/12/6/5/27/affine}
        \vec{q} &= \vec{Py} + \vec{c} \\
        \vec{c} &= -\vec{V}^{-1}\vec{u} 
 =\myvec{0\\4} \\
        f_0 &= \vec{u}^\top\vec{V}^{-1}\vec{u} - f 
            =  0
    \end{align}
$\because \det{V} = -\frac{1}{4} \neq 0$, 
	  using \eqref{eq:pair-cond},
        \eqref{eq:chapters/12/6/5/27/q-affine}
 represents a pair of straight lines.
      From \eqref{eq:pair-conic},
        \eqref{eq:chapters/12/6/5/27/PD},
	\eqref{eq:incircle-disc-v}
	and 
	\eqref{eq:incircle-disc-v-lam},
    \begin{align}
%        y_1^2-y_2^2 &= 0 \\
%        \implies y_1 &= \pm y_2 \\
%        \implies \vec{y} &= \myvec{a\\\pm a},\ a \in \mathbb{R}
\vec{y} = \kappa\myvec{1\\\pm 1}.
        \label{eq:chapters/12/6/5/27/y-sol}
    \end{align}
    Hence, using 
        \eqref{eq:chapters/12/6/5/27/affine},
    \begin{align}
        \vec{q} 
                = \myvec{0\\4}+\kappa\myvec{1\\\pm 1},
%                &= \myvec{a\pm a\\a\mp a+4} \\
        \label{eq:chapters/12/6/5/27/x-case}
    \end{align}
which, upon substituting in 
        \eqref{eq:chapters/12/6/5/27/curve-mtx}
	and solving for $\kappa$ yields
\begin{align}
	\kappa = \pm \sqrt{2}, -2.
\end{align}
 Thus, the points of contact are
    \begin{align}
        \vec{q}  = \cbrak{\myvec{\pm 2\sqrt{2}\\4},\myvec{0\\0}}
        \label{eq:chapters/12/6/5/27/poc-ans}
    \end{align}
    The nearest point out of these three candidates for $\vec{q}$ is
    $\myvec{\pm2\sqrt{2}\\4}$. 
See \figref{fig:chapters/12/6/5/27/normal}.
\begin{figure}[H]
        \centering
        \includegraphics[width=0.75\columnwidth]{chapters/12/6/5/27/figs/normal.png}
        \caption{}
        \label{fig:chapters/12/6/5/27/normal}
    \end{figure}

\item 
Find the equation of the normal to curve $x^2 = 4y$ which passes through the point
(1, 2).
\\
\solution 
\label{chapters/12/6/6/4}
	\begin{figure}[H]
		\centering
 \includegraphics[width=0.75\columnwidth]{chapters/12/6/6/4/figs/conics.png}
		\caption{}
		\label{fig:12/6/6/4}
  	\end{figure}
The conic parameters are
\begin{align}
	\vec{V} = \myvec{1 & 0\\0 & 0},
	\vec{u} = \myvec{0\\-2},
	f &= 0
	%\\
\end{align}
Choosing the direction and normal vectors as
\begin{align}
	\vec{m} = \myvec{1 \\ m}, \
	\vec{n} = \myvec{-m \\ 1}, 
\end{align}
and substituting these values in
	\eqref{eq:point_of_tangency-m},
% \eqref{eq:normal_solution}, 
 we obtain
\begin{align}
m = 1
\end{align}
as the only real solution.  Thus, 
\begin{align}
%\vec{n} = \myvec{1 \\ 1},
	\vec{m} = \myvec{1 \\ 1}, 
%    \mu = -1
\end{align}
and 
	the equation of the normal is then obtained as
\begin{align}
	\vec{m}^{\top}\brak{\vec{x}-\vec{h}} &= 0
	\\
	\implies
\myvec{
1 & 1
}
		\vec{x}
	&=
\myvec{
1 & 1
}
	\myvec{1 \\2 }
	\\
	&= 3
\end{align}
		See \figref{fig:12/6/6/4}.

\item 
 The line $y=mx+1$ is a tangent to the curve $y^2 = 4x$, find the value of $m$. 
 \\
 \solution 
\label{chapters/12/6/6/21}
	\begin{figure}[H]
		\centering
 \includegraphics[width=0.75\columnwidth]{chapters/12/6/6/21/figs/im.pdf}
		\caption{}
		\label{fig:12/6/6/21}
  	\end{figure}
The parameters for the given conic are
\begin{align}
    \vec{V} = \myvec{0&0\\0&1}, \vec{u} = \myvec{-2\\0}, f = 0
		\label{eq:12/6/6/21/param}
\end{align}
The given tangent can be expressed in parametric form as
\begin{align}
		\label{eq:12/6/6/21/tangent}
\vec{x} = \vec{e}_2 + \mu\vec{m}
\end{align}
Substituting from 
		\eqref{eq:12/6/6/21/tangent}
		and
		\eqref{eq:12/6/6/21/param}
		in 
	  \eqref{eq:h-tangents-cond}
and solving, we obtain 
\begin{align}
	m = 1.
\end{align}
		See \figref{fig:12/6/6/21}.

\item 
\label{chapters/12/6/6/22}
Find the normal at the point (1,1) on the curve 
\begin{align}
2y+x^2=3
\end{align}
\solution
Use
  \eqref{eq:conic_tangent_mq}
  with 
\begin{align}
	\vec{m} = \myvec{1 \\m}
\end{align}

 \item If the line $y=\sqrt{3}x+K$ touches the parabola $x^2=16y,$ then find the value of $K$.
\item If the line $y=mx+1$ is tangent to the parabola $y^2=4x$ then find the value of $m$.
\item Find the condition that the curves $2x=y^2$ and $2xy=k$ intersect orthogonally.
\item Prove that the curves $xy=4$ and $x^2+y^2=8$ touch each other.
\item Find the angle of intersection of the curves $y=4-x^2$ and $y=x^2$.
\item Prove that the curves $y^2=4x$ and $x^2+y^2-6x+1=0$ touch each each other at the point (1,2).
\item Find the equation of the normal lines to the curve $3x^2-y^2=8$ which are parallel to the line $x+3y=4$.
 \item The equation of the normal to the curve $3x^2-y^2 =8$ which is parallel to the line $x+3y=8$ is
 \begin{enumerate}
 \item $3x-y=8$
 \item $3x+y+8=0$
 \item $x+3y+8=0$
 \item $x+3y=0$
 \end{enumerate}
\item The equation of the tangent to the curve $(1+y^2) =2-x$, where it crosses the x-axis is 
\begin{enumerate}
\item $x+5y=2$
\item $x-5y=2$
\item $5x-y=2$
\item $5x+y=2$
\end{enumerate}
\end{enumerate}
State whether the statements are True or False 
\begin{enumerate}[label=\thesection.\arabic*,ref=\thesection.\theenumi,resume*]
\item The line $lx+my+n=0$ will touch the parabola $y^2=4 ax$ if $ln =am^2$,
\item The line $2x+3y=12$ touches the ellipse $\frac{x^2}{9}+\frac{y^2}{4}=2$ at the point (3,2).
\end{enumerate}

\subsection{Formulae}
\begin{enumerate}[label=\thesubsection.\arabic*,ref=\thesubsection.\theenumi]
\item
  If $L$ in \eqref{eq:conic_tangent} touches \eqref{eq:conic_quad_form} at exactly one point $\vec{q}$, 
  \begin{align}
  \vec{m}^{\top}\brak{\vec{V}\vec{q}+\vec{u}} = 0
  \label{eq:conic_tangent_mq}
  \end{align}
\begin{proof}
  In this case, \eqref{eq:conic_intercept} has exactly one root.  Hence, 
  in \eqref{eq:tangent_roots}
  \begin{align}
  \sbrak{
  \vec{m}^{\top}\brak{\vec{V}\vec{q}+\vec{u}}
  }^2 -\brak{\vec{m}^{\top}\vec{V}\vec{m}}
	  \text{g}\brak
  {
  \vec{q}
%  \vec{q}^{\top}\vec{V}\vec{q} + 2\vec{u}^{\top}\vec{q} +f
  } = 0                                                                                             
  \label{eq:conic_tangent_disc}
  \end{align}                    
  $\because \vec{q}$ is the point of contact,
	%$\vec{q}$ satisfies \eqref{eq:conic_quad_form}
%  and 
  \begin{align}
	  \text{g}\brak{  \vec{q}} = 0
%  \vec{q}^{\top}\vec{V}\vec{q} + 2\vec{u}^{\top}\vec{q} +f = 0
  \label{eq:conic_tangent_qquad}
  \end{align}
  Substituting \eqref{eq:conic_tangent_qquad} in \eqref{eq:conic_tangent_disc} and simplifying, we obtain \eqref{eq:conic_tangent_mq}.
\end{proof}
\item For a circle, the points of contact are
	\begin{align}
	\vec{q}_{ij} &= \brak{\pm r \frac{\vec{n}_j}{\norm{\vec{n}_j}}-\vec{u}}, \quad i,j = 1,2
\label{eq:conic_tangent_qk-circ}
\end{align}
\begin{proof}
	From 
\eqref{eq:conic_tangent_qk},
and 
	\eqref{eq:circ-cr},
\begin{align}
\kappa_{ij} &= \pm 
\frac{r
}
{
	\norm{\vec{n}_j}
}
\end{align}
\end{proof}

\end{enumerate}

\subsection{Construction}
\begin{enumerate}[label=\thesubsection.\arabic*.,ref=\thesubsection.\theenumi]
\item Construct a $\triangle ABC$ given $a, \angle B$ and $K = b+c$.
		\label{prob:9/11/2/1}
	\\
	\solution 
	Using the cosine formula in  $\triangle ABC$,
\begin{align}
	{b}^2&= {a}^2 + {c}^2 - 2ac\cos{B}
\\
\implies	(K-c)^2 &= {a}^2 + c^2- 2  a  c\cos{B}
\\
\implies
	c &=
	\frac{K^2-a^2}{2\brak{K- a  \cos{B}}}
		\label{eq:9/11/2/1}
\end{align}
The coordinates of $\triangle ABC$ can then be expressed as
\begin{align}
		\label{eq:9/11/2/1-final}
	\vec{A}=c\myvec{\cos B \\ \sin B},
	\vec{B} = \vec{0},
	\vec{C} =\myvec{a \\ 0}.
\end{align}
\item Construct a $\triangle ABC$ given $\angle B, \angle C$ and $K = a+b+c$.
	\\
	\solution
	\begin{align}
a+b+c &= K \\
b\cos C + c \cos B -a &=0 \\
b\sin C - c \sin B &=0
\end{align}
resulting in the matrix equation
\begin{align}
		\label{eq:9/11/2/4}
	\myvec{1 & 1 & 1 \\ -1 & \cos C & \cos B  \\ 0 &\sin C & -\sin B } \myvec{a \\ b \\ c} = K \myvec{1 \\ 0 \\ 0}
\end{align}
which can be solved to obtain all the sides.  $\triangle ABC$ can then be plotted using
\begin{align}
\vec{A} = \myvec{a \\ b},\,
\vec{B} = \vec{0},\, 
\vec{C} = \myvec{a \\ 0}
		\label{eq:9/11/2/4-final}
\end{align}
\end{enumerate}

%
\appendices
\section{Triangle}
%\numberwithin{equation}{section}
Consider a triangle with vertices
		\begin{align}
			\label{eq:tri-pts}
			\vec{A} = \myvec{1 \\ -1},\,
			\vec{B} = \myvec{-4 \\ 6},\,
			\vec{C} = \myvec{-3 \\ -5}
		\end{align}
\subsection{Sides}
%\renewcommand{\theequation}{\theenumi}
\begin{enumerate}[label=\thesubsection.\arabic*.,ref=\thesubsection.\theenumi]
%\numberwithin{equation}{enumi}
\item The direction vector of $AB$ is defined as
		\begin{align}
			\vec{B}-
			\vec{A}
		\end{align}
Find the direction vectors of $AB, BC$ and $CA$.
\\
\solution 
\begin{enumerate} 
\item  The Direction vector of $AB$ is 
	\begin{align}  \vec{B} - \vec{A} 
		=\myvec{ -4\\ 6 } - \myvec{ 1\\ -1 }
 = \myvec{ -4 - 1\\ 6 - (-1) } = \myvec{ -5\\ 7 }
		\label{eq:app-geo-dir-vec-ab}
 \end{align}
\item The Direction vector of $BC$ is
	\begin{align} \vec{C} - \vec{B}=\myvec{ -3\\ -5} - \myvec{ -4\\ 6 }
 = \myvec{ -3 - (-4)\\ -5 - 6 } = \myvec{1\\ -11 }
		\label{eq:app-geo-dir-vec-bc}
  \end{align}
  \item  The Direction vector of $CA$  is
	  \begin{align}  \vec{A} - \vec{C} =\myvec{ 1\\ -1 }-\myvec{ -3\\ -5}
 = \myvec{ 1 - (-3)\\ -1 - (-5) } = \myvec{ 4\\ 4 }
		\label{eq:app-geo-dir-vec-ca}
  \end{align}
 \end{enumerate}
%	\input{solutions/1/1/1/prob_1.tex}
	\item The length of side $BC$ is 
		\label{prob:side-length}
		\begin{align}
			c = \norm{\vec{B}-\vec{A}} \triangleq \sqrt{\brak{\vec{B}-\vec{A}}^{\top}\brak{\vec{B}-\vec{A}}}
		\end{align}
		where
		\begin{align}
			\vec{A}^{\top}\triangleq\myvec{1 & -1}
		\end{align}
		Similarly, 
		\begin{align}
b = \norm{\vec{C}-\vec{B}},\,
a = \norm{\vec{A}-\vec{C}}
		\end{align}
		Find $a, b, c$.
\begin{enumerate}
	\item 
	From 	
		\eqref{eq:app-geo-dir-vec-ab},
\begin{align}
\vec{A}-\vec{B} &= \myvec{5\\-7}, \\
\implies 	c &= 	\norm{\vec{B}-\vec{A}} = \norm{\vec{A}-\vec{B}} 
	\\
	&= \sqrt{\myvec{5 & -7}\myvec{5\\-7}}
= \sqrt{\brak{5}^2 +\brak{7}^2}\\
	&=\sqrt{74}
		\label{eq:app-geo-norm-ab}
\end{align}
	\item Similarly, from 
		\eqref{eq:app-geo-dir-vec-bc},
\begin{align}
	a &= \norm{\vec{B}-\vec{C}} 
	= \sqrt{\myvec{-1 & 11}\myvec{-1\\11}}
\\
&= \sqrt{\brak{1}^2+\brak{11}^2}
	= \sqrt{122}
		\label{eq:app-geo-norm-bc}
\end{align}
and
		from 		\eqref{eq:app-geo-dir-vec-ca},
	\item 
		\begin{align}
			b &= \norm{\vec{A}-\vec{C}} = \sqrt{\myvec{4 & 4}\myvec{4\\4}}
\\
&= \sqrt{\brak{4}^2+\brak{4}^2}
	=\sqrt{32}
		\label{eq:app-geo-norm-ca}
\end{align}
\end{enumerate}
%  \\            
  %\documentclass[journal]{IEEEtran}
\usepackage{gvv-book}
\usepackage{gvv}
%\usepackage{styles/front}
%\usepackage{Wiley-AuthoringTemplate}
%\usepackage[sectionbib,authoryear]{natbib}% for name-date citation comment the below line
%\usepackage[sectionbib,numbers]{natbib}% for numbered citation comment the above line

%%********************************************************************%%
%%       How many levels of section head would you like numbered?     %%
%% 0= no section numbers, 1= section, 2= section, 3= subsection %%
\setcounter{secnumdepth}{3}
%%********************************************************************%%
%%**********************************************************************%%
%%     How many levels of section head would you like to appear in the  %%
%%				Table of Contents?			%%
%% 0= chapter, 1= section, 2= section, 3= subsection titles.	%%
\setcounter{tocdepth}{2}
%%**********************************************************************%%

%\includeonly{ch01}
\makeindex

\begin{document}
\bibliographystyle{IEEEtran}
\onecolumn


\title{
	\begin{flushleft}
	MATRICES \\ In Geometry
	\\
\rule{0.4\columnwidth}{0.4pt}
\end{flushleft}
}
\author{
\vspace{7cm}
	\begin{flushleft}
\includegraphics[width=0.2\columnwidth]{figs/logo.jpg}
\\
		{	\huge G. V. V. Sharma}
		\\
\vspace{1cm}
https://creativecommons.org/licenses/by-sa/3.0/
\\
and
\\
https://www.gnu.org/licenses/fdl-1.3.en.html
	\end{flushleft}
%\IEEEpubid{\makebox[\columnwidth]{978-1-7281-5966-1/20/\$31.00 ©2020 IEEE \hfill} \hspace{\columnsep}\makebox[\columnwidth]{ }}
}
\maketitle

\newpage


\tableofcontents

\newpage
\twocolumn

%\section{Triangle}
\section{Vectors}
Consider a triangle with vertices
		\begin{align}
			\label{eq:tri-pts}
			\vec{A} = \myvec{1 \\ -1},\,
			\vec{B} = \myvec{-4 \\ 6},\,
			\vec{C} = \myvec{-3 \\ -5}
		\end{align}
\subsection{Sides}
%\renewcommand{\theequation}{\theenumi}
\begin{enumerate}[label=\thesubsection.\arabic*.,ref=\thesubsection.\theenumi]
%\numberwithin{equation}{enumi}
\item The direction vector of $AB$ is defined as
		\begin{align}
			\vec{B}-
			\vec{A}
		\end{align}
Find the direction vectors of $AB, BC$ and $CA$.
\\
\solution 
\begin{enumerate} 
\item  The Direction vector of $AB$ is 
	\begin{align}  \vec{B} - \vec{A} 
		=\myvec{ -4\\ 6 } - \myvec{ 1\\ -1 }
 = \myvec{ -4 - 1\\ 6 - (-1) } = \myvec{ -5\\ 7 }
		\label{eq:app-geo-dir-vec-ab}
 \end{align}
\item The Direction vector of $BC$ is
	\begin{align} \vec{C} - \vec{B}=\myvec{ -3\\ -5} - \myvec{ -4\\ 6 }
 = \myvec{ -3 - (-4)\\ -5 - 6 } = \myvec{1\\ -11 }
		\label{eq:app-geo-dir-vec-bc}
  \end{align}
  \item  The Direction vector of $CA$  is
	  \begin{align}  \vec{A} - \vec{C} =\myvec{ 1\\ -1 }-\myvec{ -3\\ -5}
 = \myvec{ 1 - (-3)\\ -1 - (-5) } = \myvec{ 4\\ 4 }
		\label{eq:app-geo-dir-vec-ca}
  \end{align}
 \end{enumerate}
%	\input{solutions/1/1/1/prob_1.tex}
	\item The length of side $BC$ is 
		\label{prob:side-length}
		\begin{align}
			c = \norm{\vec{B}-\vec{A}} \triangleq \sqrt{\brak{\vec{B}-\vec{A}}^{\top}\brak{\vec{B}-\vec{A}}}
		\end{align}
		where
		\begin{align}
			\vec{A}^{\top}\triangleq\myvec{1 & -1}
		\end{align}
		Similarly, 
		\begin{align}
b = \norm{\vec{C}-\vec{B}},\,
a = \norm{\vec{A}-\vec{C}}
		\end{align}
		Find $a, b, c$.
\begin{enumerate}
	\item 
	From 	
		\eqref{eq:app-geo-dir-vec-ab},
\begin{align}
\vec{A}-\vec{B} &= \myvec{5\\-7}, \\
\implies 	c &= 	\norm{\vec{B}-\vec{A}} = \norm{\vec{A}-\vec{B}} 
	\\
	&= \sqrt{\myvec{5 & -7}\myvec{5\\-7}}
= \sqrt{\brak{5}^2 +\brak{7}^2}\\
	&=\sqrt{74}
		\label{eq:app-geo-norm-ab}
\end{align}
	\item Similarly, from 
		\eqref{eq:app-geo-dir-vec-bc},
\begin{align}
	a &= \norm{\vec{B}-\vec{C}} 
	= \sqrt{\myvec{-1 & 11}\myvec{-1\\11}}
\\
&= \sqrt{\brak{1}^2+\brak{11}^2}
	= \sqrt{122}
		\label{eq:app-geo-norm-bc}
\end{align}
and
		from 		\eqref{eq:app-geo-dir-vec-ca},
	\item 
		\begin{align}
			b &= \norm{\vec{A}-\vec{C}} = \sqrt{\myvec{4 & 4}\myvec{4\\4}}
\\
&= \sqrt{\brak{4}^2+\brak{4}^2}
	=\sqrt{32}
		\label{eq:app-geo-norm-ca}
\end{align}
\end{enumerate}
%  \\            
  %\input{solutions/1/1/2a/main.tex}
\item   Points $\vec{A}, \vec{B}, \vec{C}$ are defined to be collinear if 
		\begin{align}
			\label{eq:app-app-line-rank}
			\rank{\myvec{1 & 1 & 1 \\ \vec{A}& \vec{B}&\vec{C}}} = 2
		\end{align}
Are the given points in
			\eqref{eq:app-tri-pts}
collinear?
\\
\solution 
From 
			\eqref{eq:app-tri-pts},
\begin{align}
    \label{eq:app-1.1.3,2}
\myvec{
    1 & 1 & 1\\
    \vec{A} & \vec{B} & \vec{C} \\
    } 
    =
    %\label{eq:app-matthrowoperations}
    \myvec{
    1 & 1 & 1
    \\
    1 & -4 & -3
    \\
    -1 & 6 & -5
    }
     \xleftrightarrow[]{R_3 \leftarrow R_3+R_2}
    \myvec{
    1 & 1 & 1
    \\
    1 & -4 & -3
    \\
    0 & 2 & -8 
    }
    \\
     \xleftrightarrow[]{R_2\leftarrow R_1-R_2}
    \myvec{
    1 & 1 & 1
    \\
    0 & 5 & 4
    \\
    0 & 2 & -8 
    }
     \xleftrightarrow[]{R_3\leftarrow R_3-\frac{2}{5}R_2}
    \myvec{
    1 & 1 & 1
    \\
    0 & 5 & 4
    \\
    0 & 0 & \frac{-48}{5}
    }
\end{align}
There are no zero rows. So,
\begin{align}
    \text{rank}\myvec{
    1 & 1 & 1\\
    \vec{A} & \vec{B} & \vec{C} \\
    } &= 3 
\end{align}  
Hence,  the points $\vec{A},\vec{B},\vec{C}$ are not collinear. 
This is visible in 
\figref{fig1:Triangle}.
\begin{figure}[H]
\centering
\includegraphics[width=0.75\columnwidth]{figs/triangle/vector.pdf}
\caption{$\triangle ABC$}
\label{fig1:Triangle}
\end{figure}
% \\		\input{solutions/1/1/3/main.tex}
\item The parameteric form of the equation  of $AB$ is 
		\begin{align}
			\label{eq:app-geo-param}
			\vec{x}=\vec{A}+k\vec{m} \quad k \ne 0,
		\end{align}
		where
		\begin{align}
\vec{m}=\vec{B}-\vec{A}
		\end{align}
is the direction vector of $AB$.
Find the parameteric equations of $AB, BC$ and $CA$.
\\
\solution
From 
			\eqref{eq:app-geo-param} and
		\eqref{eq:app-geo-dir-vec-ab},
the parametric equation for $AB$ is given by
\begin{align}
AB: \vec{x} = &\myvec{1\\-1} + k \myvec{-5\\7}
\end{align}
Similarly, from 
		\eqref{eq:app-geo-dir-vec-bc} and
		\eqref{eq:app-geo-dir-vec-ca},
\begin{align}
BC: \vec{x} = &\myvec{-4\\6} + k \myvec{1\\-11}\\
CA: \vec{x} = &\myvec{-3\\-5} + k \myvec{4\\4}
\end{align}

%		\input{solutions/1/1/4/main.tex}
\item The normal form of the equation of $AB$  is 
		\begin{align}
			\label{eq:app-geo-normal}
			\vec{n}^{\top}\brak{	\vec{x}-\vec{A}} = 0
		\end{align}
		where 
		\begin{align}
			\vec{n}^{\top}\vec{m}&=\vec{n}^{\top}\brak{\vec{B}-\vec{A}} = 0
			\\
			\text{or, } \vec{n}&=\myvec{0 & 1 \\ -1 & 0} \vec{m}
			\label{eq:app-geo-norm-vec}
		\end{align}
Find the normal form of the equations of $AB, BC$ and $CA$.
\\
\solution
\begin{enumerate}
	\item
From
		\eqref{eq:app-geo-dir-vec-bc}, 
the direction vector of side $\vec{BC}$ is
\begin{align}
\vec{m}
	&=\myvec{1\\-11}
	\\
\implies \vec{n} &= \myvec{0 & 1\\
  -1 & 0}\myvec{1\\-11}
 = \myvec{-11\\-1}
		\label{eq:app-geo-norm-vec-bc}
\end{align}
from 
			\eqref{eq:app-geo-norm-vec}.
Hence, from 
			\eqref{eq:app-geo-normal},
the normal equation of side $BC$ is 
\begin{align}
	\vec{n}^{\top}\brak{	\vec{x}-\vec{B}} &= 0
			\\
\implies    \myvec{-11 & -1}\vec{x}&=\myvec{-11 & -1}\myvec{-4\\6}\\
    \implies
BC: \quad    \myvec{11 & 1}\vec{x}&=-38
\end{align}
\item Similarly, for $AB$,
from 
		\eqref{eq:app-geo-dir-vec-ab}, 
\begin{align}
	\vec{m} &= \myvec{-5\\7}
	\\
\implies        \vec{n} 
                &= \myvec{0&1\\-1&0}\myvec{-5\\7}
                = \myvec{7\\5}
		\label{eq:app-geo-norm-vec-ab}
\end{align}
and 
\begin{align}
	\vec{n}^{\top}\brak{	\vec{x}-\vec{A}} &= 0
	\\
	\implies
                AB: \quad  \vec{n}^{\top}\vec{x} &= \myvec{7&5}\myvec{1\\-1}\\    
       \implies\myvec{7&5}\vec{x} &= 2
\end{align}
\item For 
$CA$, 
from 
		\eqref{eq:app-geo-dir-vec-ca}, 
\begin{align}
\vec{m} &= \myvec{1 \\ 1}
\\
		\label{eq:app-geo-norm-vec-ca}
\implies \vec{n} 
&= \myvec{0&1 \\ -1&0}\myvec{1 \\ 1}
= \myvec{1 \\ -1}\\
\\
\implies	\vec{n}^{\top}\brak{	\vec{x}-\vec{C}} &= 0
\\
\implies \myvec{1&-1}{\vec{x}} &= \myvec{1&-1}\myvec{-3 \\ -5} 
= 2 
\end{align}
\end{enumerate}

%\input{solutions/1/1/5/assign1.tex}
\item The area of $\triangle ABC$ is defined as
		\begin{align}
			\label{eq:app-tri-area-cross}
			\frac{1}{2}\norm{{\brak{\vec{A}-\vec{B}}\times \brak{\vec{A}-\vec{C}}}}
		\end{align}
		where
		\begin{align}
			\vec{A}\times\vec{B} \triangleq \mydet{1 & -4 \\-1 & 6}
		\end{align}
		Find the area of $\triangle ABC$.\\
\solution
From
		\eqref{eq:app-geo-dir-vec-ab}
		and
		\eqref{eq:app-geo-dir-vec-ca},
\begin{align}
	\vec{A}-\vec{B}=\myvec{5\\-7},
	\vec{A}-\vec{C}&=\myvec{4\\4}\\
\implies (\vec{A}-\vec{B})\times(\vec{A}-\vec{C}) &=\mydet{5 & 4\\-7 & 4}\\
&=5\times 4-4\times (-7)\\&=48\\
\implies\frac{1}{2}\norm{(\vec{A}-\vec{B})\times(\vec{A}-\vec{C})}&=\frac{48}{2}=24
\end{align}
which is the desired area.

%  		\input{solutions/1/1/6/main.tex}
	\item Find the angles $A, B, C$ if 
%    \label{prop:angle2d}
  \begin{align}
    \label{eq:app-angle2d}
			\cos A \triangleq 
\frac{\brak{\vec{B}-\vec{A}}^{\top}{\vec{C}-\vec{A}}}{\norm{\vec{B}-\vec{A}}\norm{\vec{C}-\vec{A}}}
  \end{align}\\
  \solution
\begin{enumerate}
	\item From 
		\eqref{eq:app-geo-dir-vec-ab},
		\eqref{eq:app-geo-dir-vec-ca},
		\eqref{eq:app-geo-norm-ab}
		and
		\eqref{eq:app-geo-norm-ca}
\begin{align}
	(\vec{B}-\vec{A})^{\top}(\vec{C}-\vec{A})&=\myvec{-5&7}\myvec{-4\\-4}\\
	&=-8
	\\
	\implies
	\cos{A}&= \frac{-8}{\sqrt{74} \sqrt{32}}
	= \frac{-1}{\sqrt{37}}\\
	\implies A&=\cos^{-1}{\frac{-1}{\sqrt{37}}}
\end{align}
	\item From 
		\eqref{eq:app-geo-dir-vec-ab},
		\eqref{eq:app-geo-dir-vec-bc},
		\eqref{eq:app-geo-norm-ab}
		and
		\eqref{eq:app-geo-norm-bc}
\begin{align}
	(\vec{C}-\vec{B})^{\top}(\vec{A}-\vec{B})&=\myvec{1&-11}\myvec{5\\-7}\\
	&= 82
	\\
	\implies
	\cos{B}&= \frac{82}{\sqrt{74} \sqrt{122}}
	= \frac{41}{\sqrt{2257}}\\
	\implies B&=\cos^{-1}{\frac{41}{\sqrt{2257}}}
\end{align}
	\item From 
		\eqref{eq:app-geo-dir-vec-bc},
		\eqref{eq:app-geo-dir-vec-ca},
		\eqref{eq:app-geo-norm-bc}
		and
		\eqref{eq:app-geo-norm-ca}
\begin{align}
	(\vec{A}-\vec{C})^{\top}(\vec{B}-\vec{C})&=\myvec{4&4}\myvec{-1\\11}\\
	&=40
	\\
\implies	\cos{C}&= \frac{40}{\sqrt{32} \sqrt{122}}
	= \frac{5}{\sqrt{61}}\\
	\implies C&=\cos^{-1}{\frac{5}{\sqrt{61}}}
\end{align}

\end{enumerate}
%  	\input{solutions/1/1/7/main.tex}
All codes for this section are available at
\begin{lstlisting}
	codes/triangle/sides.py
\end{lstlisting}
\end{enumerate}

\subsection{Median}
\input{chapters/triangle/median}
\subsection{Altitude}
\input{chapters/triangle/altitude}
\subsection{Perpendicular Bisector}
\input{chapters/triangle/perp-bisect}
\subsection{Angle Bisector}
\input{chapters/triangle/angle-bisect}
\subsection{Eigenvalues and Eigenvectors}
\input{chapters/triangle/eigen}
\section{Matrices}
The mid point of $PB$ is
\begin{align}
\vec{M} =\frac{1}{2}(\vec{P}+\vec{B})
	= \myvec{4 \\ -2}  
\end{align}
which is equal to the direction vector of $OM$.
\begin{align}
\because \vec{M} \equiv
	 \myvec{1 \\ -\frac{1}{2}},
	m = -\frac{1}{2}
\end{align}
which is the desired slope.
See 
		\figref{fig:11/10/1/5}.
	\begin{figure}[!ht]
		\centering
 \includegraphics[width=\columnwidth]{chapters/11/10/1/5/figs/line.png}
		\caption{}
		\label{fig:11/10/1/5}
  	\end{figure}


%\section{Quadrilateral}
%\input{./chapters/exercises/quad_geo_exer}

\appendices
\section{Tangents to a Circle}
\numberwithin{equation}{section}
	\begin{figure}[H]
		\centering
 \includegraphics[width=0.75\columnwidth]{chapters/12/6/3/8/figs/main.png}
		\caption{}
		\label{fig:12/6/3/8}
  	\end{figure}
The equation of the conic can be represented as
\begin{align}
\vec{x}^{\top}\myvec{1&0\\0&0}\vec{x}+2\myvec{-2&\frac{-1}{2}}\vec{x}+4=0
\end{align}
So,
\begin{align}
\vec{V}=\myvec{1&0\\0&0},
\vec{u}^{\top}=\myvec{-2&\frac{-1}{2}},
f=4
\end{align}
The direction vector of the line passing through (2,0) and (4,4) is 
\begin{align}
\vec{m}=\myvec{1\\2}
\implies
\vec{n}=\myvec{2\\-1}.
\end{align}
The eigenvector corresponding to the zero eigenvalue is 
\begin{align}
\vec{p}_1=\myvec{0\\1},
\end{align}
In
\eqref{eq:conic_tangent_q_eigen},
\begin{align}
	\kappa=\frac{\myvec{0&1}\myvec{-2\\ \frac{-1}{2}}}{\myvec{0&1}\myvec{2\\-1}}
	=\frac{1}{2}
\end{align}
Substituting  $\kappa$,
from 
\eqref{eq:conic_tangent_q_eigen},
\begin{align}
	\myvec{\sbrak{\myvec{-2\\\frac{-1}{2}}+\frac{1}{2}\myvec{2\\-1}}^{\top} \\ \myvec{1&0\\0&0}}\vec{q} &= \myvec{-4 \\ \frac{1}{2}\myvec{2\\-1}-\myvec{-2\\\frac{-1}{2}}}\\
	\implies
	\myvec{-1&-1 \\ 1&0 \\ 0&0}\vec{q}&=\myvec{-4 \\ 3 \\ 0}
\end{align}
yielding
\begin{align}
\myvec{-1&-1 \\ 1&0}\vec{q} = \myvec{-4\\3}
\end{align}
The augmented matrix is 
\begin{align*}
  \myvec{
                -1&-1&\vrule&-4\\
	        1&0&\vrule&3}
  \xleftrightarrow[]{R_1 \leftarrow R_1+ 2R_2}
     \myvec{
	         1&-1&\vrule&2\\
	         1&0&\vrule&3}
      \\
 \xleftrightarrow[]{R_2 \leftarrow R_2 - R_1}
     \myvec{
	         1&-1&\vrule&2\\
	         0&1&\vrule&1}
 \xleftrightarrow[]{R_1 \leftarrow R_1 + R_2}
     \myvec{
	         1&0&\vrule&3\\
	         0&1&\vrule&1}
      \\ \implies \vec{q}=\myvec{3\\1}
\end{align*}
which is the desired 
point of contact.
See Fig. 
		\ref{fig:12/6/3/8}.



\iffalse
\latexprintindex
\fi

\end{document}


\item   Points $\vec{A}, \vec{B}, \vec{C}$ are defined to be collinear if 
		\begin{align}
			\label{eq:app-app-line-rank}
			\rank{\myvec{1 & 1 & 1 \\ \vec{A}& \vec{B}&\vec{C}}} = 2
		\end{align}
Are the given points in
			\eqref{eq:app-tri-pts}
collinear?
\\
\solution 
From 
			\eqref{eq:app-tri-pts},
\begin{align}
    \label{eq:app-1.1.3,2}
\myvec{
    1 & 1 & 1\\
    \vec{A} & \vec{B} & \vec{C} \\
    } 
    =
    %\label{eq:app-matthrowoperations}
    \myvec{
    1 & 1 & 1
    \\
    1 & -4 & -3
    \\
    -1 & 6 & -5
    }
     \xleftrightarrow[]{R_3 \leftarrow R_3+R_2}
    \myvec{
    1 & 1 & 1
    \\
    1 & -4 & -3
    \\
    0 & 2 & -8 
    }
    \\
     \xleftrightarrow[]{R_2\leftarrow R_1-R_2}
    \myvec{
    1 & 1 & 1
    \\
    0 & 5 & 4
    \\
    0 & 2 & -8 
    }
     \xleftrightarrow[]{R_3\leftarrow R_3-\frac{2}{5}R_2}
    \myvec{
    1 & 1 & 1
    \\
    0 & 5 & 4
    \\
    0 & 0 & \frac{-48}{5}
    }
\end{align}
There are no zero rows. So,
\begin{align}
    \text{rank}\myvec{
    1 & 1 & 1\\
    \vec{A} & \vec{B} & \vec{C} \\
    } &= 3 
\end{align}  
Hence,  the points $\vec{A},\vec{B},\vec{C}$ are not collinear. 
This is visible in 
\figref{fig1:Triangle}.
\begin{figure}[H]
\centering
\includegraphics[width=0.75\columnwidth]{figs/triangle/vector.pdf}
\caption{$\triangle ABC$}
\label{fig1:Triangle}
\end{figure}
% \\		\documentclass[journal]{IEEEtran}
\usepackage{gvv-book}
\usepackage{gvv}
%\usepackage{styles/front}
%\usepackage{Wiley-AuthoringTemplate}
%\usepackage[sectionbib,authoryear]{natbib}% for name-date citation comment the below line
%\usepackage[sectionbib,numbers]{natbib}% for numbered citation comment the above line

%%********************************************************************%%
%%       How many levels of section head would you like numbered?     %%
%% 0= no section numbers, 1= section, 2= section, 3= subsection %%
\setcounter{secnumdepth}{3}
%%********************************************************************%%
%%**********************************************************************%%
%%     How many levels of section head would you like to appear in the  %%
%%				Table of Contents?			%%
%% 0= chapter, 1= section, 2= section, 3= subsection titles.	%%
\setcounter{tocdepth}{2}
%%**********************************************************************%%

%\includeonly{ch01}
\makeindex

\begin{document}
\bibliographystyle{IEEEtran}
\onecolumn


\title{
	\begin{flushleft}
	MATRICES \\ In Geometry
	\\
\rule{0.4\columnwidth}{0.4pt}
\end{flushleft}
}
\author{
\vspace{7cm}
	\begin{flushleft}
\includegraphics[width=0.2\columnwidth]{figs/logo.jpg}
\\
		{	\huge G. V. V. Sharma}
		\\
\vspace{1cm}
https://creativecommons.org/licenses/by-sa/3.0/
\\
and
\\
https://www.gnu.org/licenses/fdl-1.3.en.html
	\end{flushleft}
%\IEEEpubid{\makebox[\columnwidth]{978-1-7281-5966-1/20/\$31.00 ©2020 IEEE \hfill} \hspace{\columnsep}\makebox[\columnwidth]{ }}
}
\maketitle

\newpage


\tableofcontents

\newpage
\twocolumn

%\section{Triangle}
\section{Vectors}
Consider a triangle with vertices
		\begin{align}
			\label{eq:tri-pts}
			\vec{A} = \myvec{1 \\ -1},\,
			\vec{B} = \myvec{-4 \\ 6},\,
			\vec{C} = \myvec{-3 \\ -5}
		\end{align}
\subsection{Sides}
%\renewcommand{\theequation}{\theenumi}
\begin{enumerate}[label=\thesubsection.\arabic*.,ref=\thesubsection.\theenumi]
%\numberwithin{equation}{enumi}
\item The direction vector of $AB$ is defined as
		\begin{align}
			\vec{B}-
			\vec{A}
		\end{align}
Find the direction vectors of $AB, BC$ and $CA$.
\\
\solution 
\begin{enumerate} 
\item  The Direction vector of $AB$ is 
	\begin{align}  \vec{B} - \vec{A} 
		=\myvec{ -4\\ 6 } - \myvec{ 1\\ -1 }
 = \myvec{ -4 - 1\\ 6 - (-1) } = \myvec{ -5\\ 7 }
		\label{eq:app-geo-dir-vec-ab}
 \end{align}
\item The Direction vector of $BC$ is
	\begin{align} \vec{C} - \vec{B}=\myvec{ -3\\ -5} - \myvec{ -4\\ 6 }
 = \myvec{ -3 - (-4)\\ -5 - 6 } = \myvec{1\\ -11 }
		\label{eq:app-geo-dir-vec-bc}
  \end{align}
  \item  The Direction vector of $CA$  is
	  \begin{align}  \vec{A} - \vec{C} =\myvec{ 1\\ -1 }-\myvec{ -3\\ -5}
 = \myvec{ 1 - (-3)\\ -1 - (-5) } = \myvec{ 4\\ 4 }
		\label{eq:app-geo-dir-vec-ca}
  \end{align}
 \end{enumerate}
%	\input{solutions/1/1/1/prob_1.tex}
	\item The length of side $BC$ is 
		\label{prob:side-length}
		\begin{align}
			c = \norm{\vec{B}-\vec{A}} \triangleq \sqrt{\brak{\vec{B}-\vec{A}}^{\top}\brak{\vec{B}-\vec{A}}}
		\end{align}
		where
		\begin{align}
			\vec{A}^{\top}\triangleq\myvec{1 & -1}
		\end{align}
		Similarly, 
		\begin{align}
b = \norm{\vec{C}-\vec{B}},\,
a = \norm{\vec{A}-\vec{C}}
		\end{align}
		Find $a, b, c$.
\begin{enumerate}
	\item 
	From 	
		\eqref{eq:app-geo-dir-vec-ab},
\begin{align}
\vec{A}-\vec{B} &= \myvec{5\\-7}, \\
\implies 	c &= 	\norm{\vec{B}-\vec{A}} = \norm{\vec{A}-\vec{B}} 
	\\
	&= \sqrt{\myvec{5 & -7}\myvec{5\\-7}}
= \sqrt{\brak{5}^2 +\brak{7}^2}\\
	&=\sqrt{74}
		\label{eq:app-geo-norm-ab}
\end{align}
	\item Similarly, from 
		\eqref{eq:app-geo-dir-vec-bc},
\begin{align}
	a &= \norm{\vec{B}-\vec{C}} 
	= \sqrt{\myvec{-1 & 11}\myvec{-1\\11}}
\\
&= \sqrt{\brak{1}^2+\brak{11}^2}
	= \sqrt{122}
		\label{eq:app-geo-norm-bc}
\end{align}
and
		from 		\eqref{eq:app-geo-dir-vec-ca},
	\item 
		\begin{align}
			b &= \norm{\vec{A}-\vec{C}} = \sqrt{\myvec{4 & 4}\myvec{4\\4}}
\\
&= \sqrt{\brak{4}^2+\brak{4}^2}
	=\sqrt{32}
		\label{eq:app-geo-norm-ca}
\end{align}
\end{enumerate}
%  \\            
  %\input{solutions/1/1/2a/main.tex}
\item   Points $\vec{A}, \vec{B}, \vec{C}$ are defined to be collinear if 
		\begin{align}
			\label{eq:app-app-line-rank}
			\rank{\myvec{1 & 1 & 1 \\ \vec{A}& \vec{B}&\vec{C}}} = 2
		\end{align}
Are the given points in
			\eqref{eq:app-tri-pts}
collinear?
\\
\solution 
From 
			\eqref{eq:app-tri-pts},
\begin{align}
    \label{eq:app-1.1.3,2}
\myvec{
    1 & 1 & 1\\
    \vec{A} & \vec{B} & \vec{C} \\
    } 
    =
    %\label{eq:app-matthrowoperations}
    \myvec{
    1 & 1 & 1
    \\
    1 & -4 & -3
    \\
    -1 & 6 & -5
    }
     \xleftrightarrow[]{R_3 \leftarrow R_3+R_2}
    \myvec{
    1 & 1 & 1
    \\
    1 & -4 & -3
    \\
    0 & 2 & -8 
    }
    \\
     \xleftrightarrow[]{R_2\leftarrow R_1-R_2}
    \myvec{
    1 & 1 & 1
    \\
    0 & 5 & 4
    \\
    0 & 2 & -8 
    }
     \xleftrightarrow[]{R_3\leftarrow R_3-\frac{2}{5}R_2}
    \myvec{
    1 & 1 & 1
    \\
    0 & 5 & 4
    \\
    0 & 0 & \frac{-48}{5}
    }
\end{align}
There are no zero rows. So,
\begin{align}
    \text{rank}\myvec{
    1 & 1 & 1\\
    \vec{A} & \vec{B} & \vec{C} \\
    } &= 3 
\end{align}  
Hence,  the points $\vec{A},\vec{B},\vec{C}$ are not collinear. 
This is visible in 
\figref{fig1:Triangle}.
\begin{figure}[H]
\centering
\includegraphics[width=0.75\columnwidth]{figs/triangle/vector.pdf}
\caption{$\triangle ABC$}
\label{fig1:Triangle}
\end{figure}
% \\		\input{solutions/1/1/3/main.tex}
\item The parameteric form of the equation  of $AB$ is 
		\begin{align}
			\label{eq:app-geo-param}
			\vec{x}=\vec{A}+k\vec{m} \quad k \ne 0,
		\end{align}
		where
		\begin{align}
\vec{m}=\vec{B}-\vec{A}
		\end{align}
is the direction vector of $AB$.
Find the parameteric equations of $AB, BC$ and $CA$.
\\
\solution
From 
			\eqref{eq:app-geo-param} and
		\eqref{eq:app-geo-dir-vec-ab},
the parametric equation for $AB$ is given by
\begin{align}
AB: \vec{x} = &\myvec{1\\-1} + k \myvec{-5\\7}
\end{align}
Similarly, from 
		\eqref{eq:app-geo-dir-vec-bc} and
		\eqref{eq:app-geo-dir-vec-ca},
\begin{align}
BC: \vec{x} = &\myvec{-4\\6} + k \myvec{1\\-11}\\
CA: \vec{x} = &\myvec{-3\\-5} + k \myvec{4\\4}
\end{align}

%		\input{solutions/1/1/4/main.tex}
\item The normal form of the equation of $AB$  is 
		\begin{align}
			\label{eq:app-geo-normal}
			\vec{n}^{\top}\brak{	\vec{x}-\vec{A}} = 0
		\end{align}
		where 
		\begin{align}
			\vec{n}^{\top}\vec{m}&=\vec{n}^{\top}\brak{\vec{B}-\vec{A}} = 0
			\\
			\text{or, } \vec{n}&=\myvec{0 & 1 \\ -1 & 0} \vec{m}
			\label{eq:app-geo-norm-vec}
		\end{align}
Find the normal form of the equations of $AB, BC$ and $CA$.
\\
\solution
\begin{enumerate}
	\item
From
		\eqref{eq:app-geo-dir-vec-bc}, 
the direction vector of side $\vec{BC}$ is
\begin{align}
\vec{m}
	&=\myvec{1\\-11}
	\\
\implies \vec{n} &= \myvec{0 & 1\\
  -1 & 0}\myvec{1\\-11}
 = \myvec{-11\\-1}
		\label{eq:app-geo-norm-vec-bc}
\end{align}
from 
			\eqref{eq:app-geo-norm-vec}.
Hence, from 
			\eqref{eq:app-geo-normal},
the normal equation of side $BC$ is 
\begin{align}
	\vec{n}^{\top}\brak{	\vec{x}-\vec{B}} &= 0
			\\
\implies    \myvec{-11 & -1}\vec{x}&=\myvec{-11 & -1}\myvec{-4\\6}\\
    \implies
BC: \quad    \myvec{11 & 1}\vec{x}&=-38
\end{align}
\item Similarly, for $AB$,
from 
		\eqref{eq:app-geo-dir-vec-ab}, 
\begin{align}
	\vec{m} &= \myvec{-5\\7}
	\\
\implies        \vec{n} 
                &= \myvec{0&1\\-1&0}\myvec{-5\\7}
                = \myvec{7\\5}
		\label{eq:app-geo-norm-vec-ab}
\end{align}
and 
\begin{align}
	\vec{n}^{\top}\brak{	\vec{x}-\vec{A}} &= 0
	\\
	\implies
                AB: \quad  \vec{n}^{\top}\vec{x} &= \myvec{7&5}\myvec{1\\-1}\\    
       \implies\myvec{7&5}\vec{x} &= 2
\end{align}
\item For 
$CA$, 
from 
		\eqref{eq:app-geo-dir-vec-ca}, 
\begin{align}
\vec{m} &= \myvec{1 \\ 1}
\\
		\label{eq:app-geo-norm-vec-ca}
\implies \vec{n} 
&= \myvec{0&1 \\ -1&0}\myvec{1 \\ 1}
= \myvec{1 \\ -1}\\
\\
\implies	\vec{n}^{\top}\brak{	\vec{x}-\vec{C}} &= 0
\\
\implies \myvec{1&-1}{\vec{x}} &= \myvec{1&-1}\myvec{-3 \\ -5} 
= 2 
\end{align}
\end{enumerate}

%\input{solutions/1/1/5/assign1.tex}
\item The area of $\triangle ABC$ is defined as
		\begin{align}
			\label{eq:app-tri-area-cross}
			\frac{1}{2}\norm{{\brak{\vec{A}-\vec{B}}\times \brak{\vec{A}-\vec{C}}}}
		\end{align}
		where
		\begin{align}
			\vec{A}\times\vec{B} \triangleq \mydet{1 & -4 \\-1 & 6}
		\end{align}
		Find the area of $\triangle ABC$.\\
\solution
From
		\eqref{eq:app-geo-dir-vec-ab}
		and
		\eqref{eq:app-geo-dir-vec-ca},
\begin{align}
	\vec{A}-\vec{B}=\myvec{5\\-7},
	\vec{A}-\vec{C}&=\myvec{4\\4}\\
\implies (\vec{A}-\vec{B})\times(\vec{A}-\vec{C}) &=\mydet{5 & 4\\-7 & 4}\\
&=5\times 4-4\times (-7)\\&=48\\
\implies\frac{1}{2}\norm{(\vec{A}-\vec{B})\times(\vec{A}-\vec{C})}&=\frac{48}{2}=24
\end{align}
which is the desired area.

%  		\input{solutions/1/1/6/main.tex}
	\item Find the angles $A, B, C$ if 
%    \label{prop:angle2d}
  \begin{align}
    \label{eq:app-angle2d}
			\cos A \triangleq 
\frac{\brak{\vec{B}-\vec{A}}^{\top}{\vec{C}-\vec{A}}}{\norm{\vec{B}-\vec{A}}\norm{\vec{C}-\vec{A}}}
  \end{align}\\
  \solution
\begin{enumerate}
	\item From 
		\eqref{eq:app-geo-dir-vec-ab},
		\eqref{eq:app-geo-dir-vec-ca},
		\eqref{eq:app-geo-norm-ab}
		and
		\eqref{eq:app-geo-norm-ca}
\begin{align}
	(\vec{B}-\vec{A})^{\top}(\vec{C}-\vec{A})&=\myvec{-5&7}\myvec{-4\\-4}\\
	&=-8
	\\
	\implies
	\cos{A}&= \frac{-8}{\sqrt{74} \sqrt{32}}
	= \frac{-1}{\sqrt{37}}\\
	\implies A&=\cos^{-1}{\frac{-1}{\sqrt{37}}}
\end{align}
	\item From 
		\eqref{eq:app-geo-dir-vec-ab},
		\eqref{eq:app-geo-dir-vec-bc},
		\eqref{eq:app-geo-norm-ab}
		and
		\eqref{eq:app-geo-norm-bc}
\begin{align}
	(\vec{C}-\vec{B})^{\top}(\vec{A}-\vec{B})&=\myvec{1&-11}\myvec{5\\-7}\\
	&= 82
	\\
	\implies
	\cos{B}&= \frac{82}{\sqrt{74} \sqrt{122}}
	= \frac{41}{\sqrt{2257}}\\
	\implies B&=\cos^{-1}{\frac{41}{\sqrt{2257}}}
\end{align}
	\item From 
		\eqref{eq:app-geo-dir-vec-bc},
		\eqref{eq:app-geo-dir-vec-ca},
		\eqref{eq:app-geo-norm-bc}
		and
		\eqref{eq:app-geo-norm-ca}
\begin{align}
	(\vec{A}-\vec{C})^{\top}(\vec{B}-\vec{C})&=\myvec{4&4}\myvec{-1\\11}\\
	&=40
	\\
\implies	\cos{C}&= \frac{40}{\sqrt{32} \sqrt{122}}
	= \frac{5}{\sqrt{61}}\\
	\implies C&=\cos^{-1}{\frac{5}{\sqrt{61}}}
\end{align}

\end{enumerate}
%  	\input{solutions/1/1/7/main.tex}
All codes for this section are available at
\begin{lstlisting}
	codes/triangle/sides.py
\end{lstlisting}
\end{enumerate}

\subsection{Median}
\input{chapters/triangle/median}
\subsection{Altitude}
\input{chapters/triangle/altitude}
\subsection{Perpendicular Bisector}
\input{chapters/triangle/perp-bisect}
\subsection{Angle Bisector}
\input{chapters/triangle/angle-bisect}
\subsection{Eigenvalues and Eigenvectors}
\input{chapters/triangle/eigen}
\section{Matrices}
The mid point of $PB$ is
\begin{align}
\vec{M} =\frac{1}{2}(\vec{P}+\vec{B})
	= \myvec{4 \\ -2}  
\end{align}
which is equal to the direction vector of $OM$.
\begin{align}
\because \vec{M} \equiv
	 \myvec{1 \\ -\frac{1}{2}},
	m = -\frac{1}{2}
\end{align}
which is the desired slope.
See 
		\figref{fig:11/10/1/5}.
	\begin{figure}[!ht]
		\centering
 \includegraphics[width=\columnwidth]{chapters/11/10/1/5/figs/line.png}
		\caption{}
		\label{fig:11/10/1/5}
  	\end{figure}


%\section{Quadrilateral}
%\input{./chapters/exercises/quad_geo_exer}

\appendices
\section{Tangents to a Circle}
\numberwithin{equation}{section}
	\begin{figure}[H]
		\centering
 \includegraphics[width=0.75\columnwidth]{chapters/12/6/3/8/figs/main.png}
		\caption{}
		\label{fig:12/6/3/8}
  	\end{figure}
The equation of the conic can be represented as
\begin{align}
\vec{x}^{\top}\myvec{1&0\\0&0}\vec{x}+2\myvec{-2&\frac{-1}{2}}\vec{x}+4=0
\end{align}
So,
\begin{align}
\vec{V}=\myvec{1&0\\0&0},
\vec{u}^{\top}=\myvec{-2&\frac{-1}{2}},
f=4
\end{align}
The direction vector of the line passing through (2,0) and (4,4) is 
\begin{align}
\vec{m}=\myvec{1\\2}
\implies
\vec{n}=\myvec{2\\-1}.
\end{align}
The eigenvector corresponding to the zero eigenvalue is 
\begin{align}
\vec{p}_1=\myvec{0\\1},
\end{align}
In
\eqref{eq:conic_tangent_q_eigen},
\begin{align}
	\kappa=\frac{\myvec{0&1}\myvec{-2\\ \frac{-1}{2}}}{\myvec{0&1}\myvec{2\\-1}}
	=\frac{1}{2}
\end{align}
Substituting  $\kappa$,
from 
\eqref{eq:conic_tangent_q_eigen},
\begin{align}
	\myvec{\sbrak{\myvec{-2\\\frac{-1}{2}}+\frac{1}{2}\myvec{2\\-1}}^{\top} \\ \myvec{1&0\\0&0}}\vec{q} &= \myvec{-4 \\ \frac{1}{2}\myvec{2\\-1}-\myvec{-2\\\frac{-1}{2}}}\\
	\implies
	\myvec{-1&-1 \\ 1&0 \\ 0&0}\vec{q}&=\myvec{-4 \\ 3 \\ 0}
\end{align}
yielding
\begin{align}
\myvec{-1&-1 \\ 1&0}\vec{q} = \myvec{-4\\3}
\end{align}
The augmented matrix is 
\begin{align*}
  \myvec{
                -1&-1&\vrule&-4\\
	        1&0&\vrule&3}
  \xleftrightarrow[]{R_1 \leftarrow R_1+ 2R_2}
     \myvec{
	         1&-1&\vrule&2\\
	         1&0&\vrule&3}
      \\
 \xleftrightarrow[]{R_2 \leftarrow R_2 - R_1}
     \myvec{
	         1&-1&\vrule&2\\
	         0&1&\vrule&1}
 \xleftrightarrow[]{R_1 \leftarrow R_1 + R_2}
     \myvec{
	         1&0&\vrule&3\\
	         0&1&\vrule&1}
      \\ \implies \vec{q}=\myvec{3\\1}
\end{align*}
which is the desired 
point of contact.
See Fig. 
		\ref{fig:12/6/3/8}.



\iffalse
\latexprintindex
\fi

\end{document}


\item The parameteric form of the equation  of $AB$ is 
		\begin{align}
			\label{eq:app-geo-param}
			\vec{x}=\vec{A}+k\vec{m} \quad k \ne 0,
		\end{align}
		where
		\begin{align}
\vec{m}=\vec{B}-\vec{A}
		\end{align}
is the direction vector of $AB$.
Find the parameteric equations of $AB, BC$ and $CA$.
\\
\solution
From 
			\eqref{eq:app-geo-param} and
		\eqref{eq:app-geo-dir-vec-ab},
the parametric equation for $AB$ is given by
\begin{align}
AB: \vec{x} = &\myvec{1\\-1} + k \myvec{-5\\7}
\end{align}
Similarly, from 
		\eqref{eq:app-geo-dir-vec-bc} and
		\eqref{eq:app-geo-dir-vec-ca},
\begin{align}
BC: \vec{x} = &\myvec{-4\\6} + k \myvec{1\\-11}\\
CA: \vec{x} = &\myvec{-3\\-5} + k \myvec{4\\4}
\end{align}

%		\documentclass[journal]{IEEEtran}
\usepackage{gvv-book}
\usepackage{gvv}
%\usepackage{styles/front}
%\usepackage{Wiley-AuthoringTemplate}
%\usepackage[sectionbib,authoryear]{natbib}% for name-date citation comment the below line
%\usepackage[sectionbib,numbers]{natbib}% for numbered citation comment the above line

%%********************************************************************%%
%%       How many levels of section head would you like numbered?     %%
%% 0= no section numbers, 1= section, 2= section, 3= subsection %%
\setcounter{secnumdepth}{3}
%%********************************************************************%%
%%**********************************************************************%%
%%     How many levels of section head would you like to appear in the  %%
%%				Table of Contents?			%%
%% 0= chapter, 1= section, 2= section, 3= subsection titles.	%%
\setcounter{tocdepth}{2}
%%**********************************************************************%%

%\includeonly{ch01}
\makeindex

\begin{document}
\bibliographystyle{IEEEtran}
\onecolumn


\title{
	\begin{flushleft}
	MATRICES \\ In Geometry
	\\
\rule{0.4\columnwidth}{0.4pt}
\end{flushleft}
}
\author{
\vspace{7cm}
	\begin{flushleft}
\includegraphics[width=0.2\columnwidth]{figs/logo.jpg}
\\
		{	\huge G. V. V. Sharma}
		\\
\vspace{1cm}
https://creativecommons.org/licenses/by-sa/3.0/
\\
and
\\
https://www.gnu.org/licenses/fdl-1.3.en.html
	\end{flushleft}
%\IEEEpubid{\makebox[\columnwidth]{978-1-7281-5966-1/20/\$31.00 ©2020 IEEE \hfill} \hspace{\columnsep}\makebox[\columnwidth]{ }}
}
\maketitle

\newpage


\tableofcontents

\newpage
\twocolumn

%\section{Triangle}
\section{Vectors}
Consider a triangle with vertices
		\begin{align}
			\label{eq:tri-pts}
			\vec{A} = \myvec{1 \\ -1},\,
			\vec{B} = \myvec{-4 \\ 6},\,
			\vec{C} = \myvec{-3 \\ -5}
		\end{align}
\subsection{Sides}
%\renewcommand{\theequation}{\theenumi}
\begin{enumerate}[label=\thesubsection.\arabic*.,ref=\thesubsection.\theenumi]
%\numberwithin{equation}{enumi}
\item The direction vector of $AB$ is defined as
		\begin{align}
			\vec{B}-
			\vec{A}
		\end{align}
Find the direction vectors of $AB, BC$ and $CA$.
\\
\solution 
\begin{enumerate} 
\item  The Direction vector of $AB$ is 
	\begin{align}  \vec{B} - \vec{A} 
		=\myvec{ -4\\ 6 } - \myvec{ 1\\ -1 }
 = \myvec{ -4 - 1\\ 6 - (-1) } = \myvec{ -5\\ 7 }
		\label{eq:app-geo-dir-vec-ab}
 \end{align}
\item The Direction vector of $BC$ is
	\begin{align} \vec{C} - \vec{B}=\myvec{ -3\\ -5} - \myvec{ -4\\ 6 }
 = \myvec{ -3 - (-4)\\ -5 - 6 } = \myvec{1\\ -11 }
		\label{eq:app-geo-dir-vec-bc}
  \end{align}
  \item  The Direction vector of $CA$  is
	  \begin{align}  \vec{A} - \vec{C} =\myvec{ 1\\ -1 }-\myvec{ -3\\ -5}
 = \myvec{ 1 - (-3)\\ -1 - (-5) } = \myvec{ 4\\ 4 }
		\label{eq:app-geo-dir-vec-ca}
  \end{align}
 \end{enumerate}
%	\input{solutions/1/1/1/prob_1.tex}
	\item The length of side $BC$ is 
		\label{prob:side-length}
		\begin{align}
			c = \norm{\vec{B}-\vec{A}} \triangleq \sqrt{\brak{\vec{B}-\vec{A}}^{\top}\brak{\vec{B}-\vec{A}}}
		\end{align}
		where
		\begin{align}
			\vec{A}^{\top}\triangleq\myvec{1 & -1}
		\end{align}
		Similarly, 
		\begin{align}
b = \norm{\vec{C}-\vec{B}},\,
a = \norm{\vec{A}-\vec{C}}
		\end{align}
		Find $a, b, c$.
\begin{enumerate}
	\item 
	From 	
		\eqref{eq:app-geo-dir-vec-ab},
\begin{align}
\vec{A}-\vec{B} &= \myvec{5\\-7}, \\
\implies 	c &= 	\norm{\vec{B}-\vec{A}} = \norm{\vec{A}-\vec{B}} 
	\\
	&= \sqrt{\myvec{5 & -7}\myvec{5\\-7}}
= \sqrt{\brak{5}^2 +\brak{7}^2}\\
	&=\sqrt{74}
		\label{eq:app-geo-norm-ab}
\end{align}
	\item Similarly, from 
		\eqref{eq:app-geo-dir-vec-bc},
\begin{align}
	a &= \norm{\vec{B}-\vec{C}} 
	= \sqrt{\myvec{-1 & 11}\myvec{-1\\11}}
\\
&= \sqrt{\brak{1}^2+\brak{11}^2}
	= \sqrt{122}
		\label{eq:app-geo-norm-bc}
\end{align}
and
		from 		\eqref{eq:app-geo-dir-vec-ca},
	\item 
		\begin{align}
			b &= \norm{\vec{A}-\vec{C}} = \sqrt{\myvec{4 & 4}\myvec{4\\4}}
\\
&= \sqrt{\brak{4}^2+\brak{4}^2}
	=\sqrt{32}
		\label{eq:app-geo-norm-ca}
\end{align}
\end{enumerate}
%  \\            
  %\input{solutions/1/1/2a/main.tex}
\item   Points $\vec{A}, \vec{B}, \vec{C}$ are defined to be collinear if 
		\begin{align}
			\label{eq:app-app-line-rank}
			\rank{\myvec{1 & 1 & 1 \\ \vec{A}& \vec{B}&\vec{C}}} = 2
		\end{align}
Are the given points in
			\eqref{eq:app-tri-pts}
collinear?
\\
\solution 
From 
			\eqref{eq:app-tri-pts},
\begin{align}
    \label{eq:app-1.1.3,2}
\myvec{
    1 & 1 & 1\\
    \vec{A} & \vec{B} & \vec{C} \\
    } 
    =
    %\label{eq:app-matthrowoperations}
    \myvec{
    1 & 1 & 1
    \\
    1 & -4 & -3
    \\
    -1 & 6 & -5
    }
     \xleftrightarrow[]{R_3 \leftarrow R_3+R_2}
    \myvec{
    1 & 1 & 1
    \\
    1 & -4 & -3
    \\
    0 & 2 & -8 
    }
    \\
     \xleftrightarrow[]{R_2\leftarrow R_1-R_2}
    \myvec{
    1 & 1 & 1
    \\
    0 & 5 & 4
    \\
    0 & 2 & -8 
    }
     \xleftrightarrow[]{R_3\leftarrow R_3-\frac{2}{5}R_2}
    \myvec{
    1 & 1 & 1
    \\
    0 & 5 & 4
    \\
    0 & 0 & \frac{-48}{5}
    }
\end{align}
There are no zero rows. So,
\begin{align}
    \text{rank}\myvec{
    1 & 1 & 1\\
    \vec{A} & \vec{B} & \vec{C} \\
    } &= 3 
\end{align}  
Hence,  the points $\vec{A},\vec{B},\vec{C}$ are not collinear. 
This is visible in 
\figref{fig1:Triangle}.
\begin{figure}[H]
\centering
\includegraphics[width=0.75\columnwidth]{figs/triangle/vector.pdf}
\caption{$\triangle ABC$}
\label{fig1:Triangle}
\end{figure}
% \\		\input{solutions/1/1/3/main.tex}
\item The parameteric form of the equation  of $AB$ is 
		\begin{align}
			\label{eq:app-geo-param}
			\vec{x}=\vec{A}+k\vec{m} \quad k \ne 0,
		\end{align}
		where
		\begin{align}
\vec{m}=\vec{B}-\vec{A}
		\end{align}
is the direction vector of $AB$.
Find the parameteric equations of $AB, BC$ and $CA$.
\\
\solution
From 
			\eqref{eq:app-geo-param} and
		\eqref{eq:app-geo-dir-vec-ab},
the parametric equation for $AB$ is given by
\begin{align}
AB: \vec{x} = &\myvec{1\\-1} + k \myvec{-5\\7}
\end{align}
Similarly, from 
		\eqref{eq:app-geo-dir-vec-bc} and
		\eqref{eq:app-geo-dir-vec-ca},
\begin{align}
BC: \vec{x} = &\myvec{-4\\6} + k \myvec{1\\-11}\\
CA: \vec{x} = &\myvec{-3\\-5} + k \myvec{4\\4}
\end{align}

%		\input{solutions/1/1/4/main.tex}
\item The normal form of the equation of $AB$  is 
		\begin{align}
			\label{eq:app-geo-normal}
			\vec{n}^{\top}\brak{	\vec{x}-\vec{A}} = 0
		\end{align}
		where 
		\begin{align}
			\vec{n}^{\top}\vec{m}&=\vec{n}^{\top}\brak{\vec{B}-\vec{A}} = 0
			\\
			\text{or, } \vec{n}&=\myvec{0 & 1 \\ -1 & 0} \vec{m}
			\label{eq:app-geo-norm-vec}
		\end{align}
Find the normal form of the equations of $AB, BC$ and $CA$.
\\
\solution
\begin{enumerate}
	\item
From
		\eqref{eq:app-geo-dir-vec-bc}, 
the direction vector of side $\vec{BC}$ is
\begin{align}
\vec{m}
	&=\myvec{1\\-11}
	\\
\implies \vec{n} &= \myvec{0 & 1\\
  -1 & 0}\myvec{1\\-11}
 = \myvec{-11\\-1}
		\label{eq:app-geo-norm-vec-bc}
\end{align}
from 
			\eqref{eq:app-geo-norm-vec}.
Hence, from 
			\eqref{eq:app-geo-normal},
the normal equation of side $BC$ is 
\begin{align}
	\vec{n}^{\top}\brak{	\vec{x}-\vec{B}} &= 0
			\\
\implies    \myvec{-11 & -1}\vec{x}&=\myvec{-11 & -1}\myvec{-4\\6}\\
    \implies
BC: \quad    \myvec{11 & 1}\vec{x}&=-38
\end{align}
\item Similarly, for $AB$,
from 
		\eqref{eq:app-geo-dir-vec-ab}, 
\begin{align}
	\vec{m} &= \myvec{-5\\7}
	\\
\implies        \vec{n} 
                &= \myvec{0&1\\-1&0}\myvec{-5\\7}
                = \myvec{7\\5}
		\label{eq:app-geo-norm-vec-ab}
\end{align}
and 
\begin{align}
	\vec{n}^{\top}\brak{	\vec{x}-\vec{A}} &= 0
	\\
	\implies
                AB: \quad  \vec{n}^{\top}\vec{x} &= \myvec{7&5}\myvec{1\\-1}\\    
       \implies\myvec{7&5}\vec{x} &= 2
\end{align}
\item For 
$CA$, 
from 
		\eqref{eq:app-geo-dir-vec-ca}, 
\begin{align}
\vec{m} &= \myvec{1 \\ 1}
\\
		\label{eq:app-geo-norm-vec-ca}
\implies \vec{n} 
&= \myvec{0&1 \\ -1&0}\myvec{1 \\ 1}
= \myvec{1 \\ -1}\\
\\
\implies	\vec{n}^{\top}\brak{	\vec{x}-\vec{C}} &= 0
\\
\implies \myvec{1&-1}{\vec{x}} &= \myvec{1&-1}\myvec{-3 \\ -5} 
= 2 
\end{align}
\end{enumerate}

%\input{solutions/1/1/5/assign1.tex}
\item The area of $\triangle ABC$ is defined as
		\begin{align}
			\label{eq:app-tri-area-cross}
			\frac{1}{2}\norm{{\brak{\vec{A}-\vec{B}}\times \brak{\vec{A}-\vec{C}}}}
		\end{align}
		where
		\begin{align}
			\vec{A}\times\vec{B} \triangleq \mydet{1 & -4 \\-1 & 6}
		\end{align}
		Find the area of $\triangle ABC$.\\
\solution
From
		\eqref{eq:app-geo-dir-vec-ab}
		and
		\eqref{eq:app-geo-dir-vec-ca},
\begin{align}
	\vec{A}-\vec{B}=\myvec{5\\-7},
	\vec{A}-\vec{C}&=\myvec{4\\4}\\
\implies (\vec{A}-\vec{B})\times(\vec{A}-\vec{C}) &=\mydet{5 & 4\\-7 & 4}\\
&=5\times 4-4\times (-7)\\&=48\\
\implies\frac{1}{2}\norm{(\vec{A}-\vec{B})\times(\vec{A}-\vec{C})}&=\frac{48}{2}=24
\end{align}
which is the desired area.

%  		\input{solutions/1/1/6/main.tex}
	\item Find the angles $A, B, C$ if 
%    \label{prop:angle2d}
  \begin{align}
    \label{eq:app-angle2d}
			\cos A \triangleq 
\frac{\brak{\vec{B}-\vec{A}}^{\top}{\vec{C}-\vec{A}}}{\norm{\vec{B}-\vec{A}}\norm{\vec{C}-\vec{A}}}
  \end{align}\\
  \solution
\begin{enumerate}
	\item From 
		\eqref{eq:app-geo-dir-vec-ab},
		\eqref{eq:app-geo-dir-vec-ca},
		\eqref{eq:app-geo-norm-ab}
		and
		\eqref{eq:app-geo-norm-ca}
\begin{align}
	(\vec{B}-\vec{A})^{\top}(\vec{C}-\vec{A})&=\myvec{-5&7}\myvec{-4\\-4}\\
	&=-8
	\\
	\implies
	\cos{A}&= \frac{-8}{\sqrt{74} \sqrt{32}}
	= \frac{-1}{\sqrt{37}}\\
	\implies A&=\cos^{-1}{\frac{-1}{\sqrt{37}}}
\end{align}
	\item From 
		\eqref{eq:app-geo-dir-vec-ab},
		\eqref{eq:app-geo-dir-vec-bc},
		\eqref{eq:app-geo-norm-ab}
		and
		\eqref{eq:app-geo-norm-bc}
\begin{align}
	(\vec{C}-\vec{B})^{\top}(\vec{A}-\vec{B})&=\myvec{1&-11}\myvec{5\\-7}\\
	&= 82
	\\
	\implies
	\cos{B}&= \frac{82}{\sqrt{74} \sqrt{122}}
	= \frac{41}{\sqrt{2257}}\\
	\implies B&=\cos^{-1}{\frac{41}{\sqrt{2257}}}
\end{align}
	\item From 
		\eqref{eq:app-geo-dir-vec-bc},
		\eqref{eq:app-geo-dir-vec-ca},
		\eqref{eq:app-geo-norm-bc}
		and
		\eqref{eq:app-geo-norm-ca}
\begin{align}
	(\vec{A}-\vec{C})^{\top}(\vec{B}-\vec{C})&=\myvec{4&4}\myvec{-1\\11}\\
	&=40
	\\
\implies	\cos{C}&= \frac{40}{\sqrt{32} \sqrt{122}}
	= \frac{5}{\sqrt{61}}\\
	\implies C&=\cos^{-1}{\frac{5}{\sqrt{61}}}
\end{align}

\end{enumerate}
%  	\input{solutions/1/1/7/main.tex}
All codes for this section are available at
\begin{lstlisting}
	codes/triangle/sides.py
\end{lstlisting}
\end{enumerate}

\subsection{Median}
\input{chapters/triangle/median}
\subsection{Altitude}
\input{chapters/triangle/altitude}
\subsection{Perpendicular Bisector}
\input{chapters/triangle/perp-bisect}
\subsection{Angle Bisector}
\input{chapters/triangle/angle-bisect}
\subsection{Eigenvalues and Eigenvectors}
\input{chapters/triangle/eigen}
\section{Matrices}
The mid point of $PB$ is
\begin{align}
\vec{M} =\frac{1}{2}(\vec{P}+\vec{B})
	= \myvec{4 \\ -2}  
\end{align}
which is equal to the direction vector of $OM$.
\begin{align}
\because \vec{M} \equiv
	 \myvec{1 \\ -\frac{1}{2}},
	m = -\frac{1}{2}
\end{align}
which is the desired slope.
See 
		\figref{fig:11/10/1/5}.
	\begin{figure}[!ht]
		\centering
 \includegraphics[width=\columnwidth]{chapters/11/10/1/5/figs/line.png}
		\caption{}
		\label{fig:11/10/1/5}
  	\end{figure}


%\section{Quadrilateral}
%\input{./chapters/exercises/quad_geo_exer}

\appendices
\section{Tangents to a Circle}
\numberwithin{equation}{section}
	\begin{figure}[H]
		\centering
 \includegraphics[width=0.75\columnwidth]{chapters/12/6/3/8/figs/main.png}
		\caption{}
		\label{fig:12/6/3/8}
  	\end{figure}
The equation of the conic can be represented as
\begin{align}
\vec{x}^{\top}\myvec{1&0\\0&0}\vec{x}+2\myvec{-2&\frac{-1}{2}}\vec{x}+4=0
\end{align}
So,
\begin{align}
\vec{V}=\myvec{1&0\\0&0},
\vec{u}^{\top}=\myvec{-2&\frac{-1}{2}},
f=4
\end{align}
The direction vector of the line passing through (2,0) and (4,4) is 
\begin{align}
\vec{m}=\myvec{1\\2}
\implies
\vec{n}=\myvec{2\\-1}.
\end{align}
The eigenvector corresponding to the zero eigenvalue is 
\begin{align}
\vec{p}_1=\myvec{0\\1},
\end{align}
In
\eqref{eq:conic_tangent_q_eigen},
\begin{align}
	\kappa=\frac{\myvec{0&1}\myvec{-2\\ \frac{-1}{2}}}{\myvec{0&1}\myvec{2\\-1}}
	=\frac{1}{2}
\end{align}
Substituting  $\kappa$,
from 
\eqref{eq:conic_tangent_q_eigen},
\begin{align}
	\myvec{\sbrak{\myvec{-2\\\frac{-1}{2}}+\frac{1}{2}\myvec{2\\-1}}^{\top} \\ \myvec{1&0\\0&0}}\vec{q} &= \myvec{-4 \\ \frac{1}{2}\myvec{2\\-1}-\myvec{-2\\\frac{-1}{2}}}\\
	\implies
	\myvec{-1&-1 \\ 1&0 \\ 0&0}\vec{q}&=\myvec{-4 \\ 3 \\ 0}
\end{align}
yielding
\begin{align}
\myvec{-1&-1 \\ 1&0}\vec{q} = \myvec{-4\\3}
\end{align}
The augmented matrix is 
\begin{align*}
  \myvec{
                -1&-1&\vrule&-4\\
	        1&0&\vrule&3}
  \xleftrightarrow[]{R_1 \leftarrow R_1+ 2R_2}
     \myvec{
	         1&-1&\vrule&2\\
	         1&0&\vrule&3}
      \\
 \xleftrightarrow[]{R_2 \leftarrow R_2 - R_1}
     \myvec{
	         1&-1&\vrule&2\\
	         0&1&\vrule&1}
 \xleftrightarrow[]{R_1 \leftarrow R_1 + R_2}
     \myvec{
	         1&0&\vrule&3\\
	         0&1&\vrule&1}
      \\ \implies \vec{q}=\myvec{3\\1}
\end{align*}
which is the desired 
point of contact.
See Fig. 
		\ref{fig:12/6/3/8}.



\iffalse
\latexprintindex
\fi

\end{document}


\item The normal form of the equation of $AB$  is 
		\begin{align}
			\label{eq:app-geo-normal}
			\vec{n}^{\top}\brak{	\vec{x}-\vec{A}} = 0
		\end{align}
		where 
		\begin{align}
			\vec{n}^{\top}\vec{m}&=\vec{n}^{\top}\brak{\vec{B}-\vec{A}} = 0
			\\
			\text{or, } \vec{n}&=\myvec{0 & 1 \\ -1 & 0} \vec{m}
			\label{eq:app-geo-norm-vec}
		\end{align}
Find the normal form of the equations of $AB, BC$ and $CA$.
\\
\solution
\begin{enumerate}
	\item
From
		\eqref{eq:app-geo-dir-vec-bc}, 
the direction vector of side $\vec{BC}$ is
\begin{align}
\vec{m}
	&=\myvec{1\\-11}
	\\
\implies \vec{n} &= \myvec{0 & 1\\
  -1 & 0}\myvec{1\\-11}
 = \myvec{-11\\-1}
		\label{eq:app-geo-norm-vec-bc}
\end{align}
from 
			\eqref{eq:app-geo-norm-vec}.
Hence, from 
			\eqref{eq:app-geo-normal},
the normal equation of side $BC$ is 
\begin{align}
	\vec{n}^{\top}\brak{	\vec{x}-\vec{B}} &= 0
			\\
\implies    \myvec{-11 & -1}\vec{x}&=\myvec{-11 & -1}\myvec{-4\\6}\\
    \implies
BC: \quad    \myvec{11 & 1}\vec{x}&=-38
\end{align}
\item Similarly, for $AB$,
from 
		\eqref{eq:app-geo-dir-vec-ab}, 
\begin{align}
	\vec{m} &= \myvec{-5\\7}
	\\
\implies        \vec{n} 
                &= \myvec{0&1\\-1&0}\myvec{-5\\7}
                = \myvec{7\\5}
		\label{eq:app-geo-norm-vec-ab}
\end{align}
and 
\begin{align}
	\vec{n}^{\top}\brak{	\vec{x}-\vec{A}} &= 0
	\\
	\implies
                AB: \quad  \vec{n}^{\top}\vec{x} &= \myvec{7&5}\myvec{1\\-1}\\    
       \implies\myvec{7&5}\vec{x} &= 2
\end{align}
\item For 
$CA$, 
from 
		\eqref{eq:app-geo-dir-vec-ca}, 
\begin{align}
\vec{m} &= \myvec{1 \\ 1}
\\
		\label{eq:app-geo-norm-vec-ca}
\implies \vec{n} 
&= \myvec{0&1 \\ -1&0}\myvec{1 \\ 1}
= \myvec{1 \\ -1}\\
\\
\implies	\vec{n}^{\top}\brak{	\vec{x}-\vec{C}} &= 0
\\
\implies \myvec{1&-1}{\vec{x}} &= \myvec{1&-1}\myvec{-3 \\ -5} 
= 2 
\end{align}
\end{enumerate}

%\input{solutions/1/1/5/assign1.tex}
\item The area of $\triangle ABC$ is defined as
		\begin{align}
			\label{eq:app-tri-area-cross}
			\frac{1}{2}\norm{{\brak{\vec{A}-\vec{B}}\times \brak{\vec{A}-\vec{C}}}}
		\end{align}
		where
		\begin{align}
			\vec{A}\times\vec{B} \triangleq \mydet{1 & -4 \\-1 & 6}
		\end{align}
		Find the area of $\triangle ABC$.\\
\solution
From
		\eqref{eq:app-geo-dir-vec-ab}
		and
		\eqref{eq:app-geo-dir-vec-ca},
\begin{align}
	\vec{A}-\vec{B}=\myvec{5\\-7},
	\vec{A}-\vec{C}&=\myvec{4\\4}\\
\implies (\vec{A}-\vec{B})\times(\vec{A}-\vec{C}) &=\mydet{5 & 4\\-7 & 4}\\
&=5\times 4-4\times (-7)\\&=48\\
\implies\frac{1}{2}\norm{(\vec{A}-\vec{B})\times(\vec{A}-\vec{C})}&=\frac{48}{2}=24
\end{align}
which is the desired area.

%  		\documentclass[journal]{IEEEtran}
\usepackage{gvv-book}
\usepackage{gvv}
%\usepackage{styles/front}
%\usepackage{Wiley-AuthoringTemplate}
%\usepackage[sectionbib,authoryear]{natbib}% for name-date citation comment the below line
%\usepackage[sectionbib,numbers]{natbib}% for numbered citation comment the above line

%%********************************************************************%%
%%       How many levels of section head would you like numbered?     %%
%% 0= no section numbers, 1= section, 2= section, 3= subsection %%
\setcounter{secnumdepth}{3}
%%********************************************************************%%
%%**********************************************************************%%
%%     How many levels of section head would you like to appear in the  %%
%%				Table of Contents?			%%
%% 0= chapter, 1= section, 2= section, 3= subsection titles.	%%
\setcounter{tocdepth}{2}
%%**********************************************************************%%

%\includeonly{ch01}
\makeindex

\begin{document}
\bibliographystyle{IEEEtran}
\onecolumn


\title{
	\begin{flushleft}
	MATRICES \\ In Geometry
	\\
\rule{0.4\columnwidth}{0.4pt}
\end{flushleft}
}
\author{
\vspace{7cm}
	\begin{flushleft}
\includegraphics[width=0.2\columnwidth]{figs/logo.jpg}
\\
		{	\huge G. V. V. Sharma}
		\\
\vspace{1cm}
https://creativecommons.org/licenses/by-sa/3.0/
\\
and
\\
https://www.gnu.org/licenses/fdl-1.3.en.html
	\end{flushleft}
%\IEEEpubid{\makebox[\columnwidth]{978-1-7281-5966-1/20/\$31.00 ©2020 IEEE \hfill} \hspace{\columnsep}\makebox[\columnwidth]{ }}
}
\maketitle

\newpage


\tableofcontents

\newpage
\twocolumn

%\section{Triangle}
\section{Vectors}
Consider a triangle with vertices
		\begin{align}
			\label{eq:tri-pts}
			\vec{A} = \myvec{1 \\ -1},\,
			\vec{B} = \myvec{-4 \\ 6},\,
			\vec{C} = \myvec{-3 \\ -5}
		\end{align}
\subsection{Sides}
%\renewcommand{\theequation}{\theenumi}
\begin{enumerate}[label=\thesubsection.\arabic*.,ref=\thesubsection.\theenumi]
%\numberwithin{equation}{enumi}
\item The direction vector of $AB$ is defined as
		\begin{align}
			\vec{B}-
			\vec{A}
		\end{align}
Find the direction vectors of $AB, BC$ and $CA$.
\\
\solution 
\begin{enumerate} 
\item  The Direction vector of $AB$ is 
	\begin{align}  \vec{B} - \vec{A} 
		=\myvec{ -4\\ 6 } - \myvec{ 1\\ -1 }
 = \myvec{ -4 - 1\\ 6 - (-1) } = \myvec{ -5\\ 7 }
		\label{eq:app-geo-dir-vec-ab}
 \end{align}
\item The Direction vector of $BC$ is
	\begin{align} \vec{C} - \vec{B}=\myvec{ -3\\ -5} - \myvec{ -4\\ 6 }
 = \myvec{ -3 - (-4)\\ -5 - 6 } = \myvec{1\\ -11 }
		\label{eq:app-geo-dir-vec-bc}
  \end{align}
  \item  The Direction vector of $CA$  is
	  \begin{align}  \vec{A} - \vec{C} =\myvec{ 1\\ -1 }-\myvec{ -3\\ -5}
 = \myvec{ 1 - (-3)\\ -1 - (-5) } = \myvec{ 4\\ 4 }
		\label{eq:app-geo-dir-vec-ca}
  \end{align}
 \end{enumerate}
%	\input{solutions/1/1/1/prob_1.tex}
	\item The length of side $BC$ is 
		\label{prob:side-length}
		\begin{align}
			c = \norm{\vec{B}-\vec{A}} \triangleq \sqrt{\brak{\vec{B}-\vec{A}}^{\top}\brak{\vec{B}-\vec{A}}}
		\end{align}
		where
		\begin{align}
			\vec{A}^{\top}\triangleq\myvec{1 & -1}
		\end{align}
		Similarly, 
		\begin{align}
b = \norm{\vec{C}-\vec{B}},\,
a = \norm{\vec{A}-\vec{C}}
		\end{align}
		Find $a, b, c$.
\begin{enumerate}
	\item 
	From 	
		\eqref{eq:app-geo-dir-vec-ab},
\begin{align}
\vec{A}-\vec{B} &= \myvec{5\\-7}, \\
\implies 	c &= 	\norm{\vec{B}-\vec{A}} = \norm{\vec{A}-\vec{B}} 
	\\
	&= \sqrt{\myvec{5 & -7}\myvec{5\\-7}}
= \sqrt{\brak{5}^2 +\brak{7}^2}\\
	&=\sqrt{74}
		\label{eq:app-geo-norm-ab}
\end{align}
	\item Similarly, from 
		\eqref{eq:app-geo-dir-vec-bc},
\begin{align}
	a &= \norm{\vec{B}-\vec{C}} 
	= \sqrt{\myvec{-1 & 11}\myvec{-1\\11}}
\\
&= \sqrt{\brak{1}^2+\brak{11}^2}
	= \sqrt{122}
		\label{eq:app-geo-norm-bc}
\end{align}
and
		from 		\eqref{eq:app-geo-dir-vec-ca},
	\item 
		\begin{align}
			b &= \norm{\vec{A}-\vec{C}} = \sqrt{\myvec{4 & 4}\myvec{4\\4}}
\\
&= \sqrt{\brak{4}^2+\brak{4}^2}
	=\sqrt{32}
		\label{eq:app-geo-norm-ca}
\end{align}
\end{enumerate}
%  \\            
  %\input{solutions/1/1/2a/main.tex}
\item   Points $\vec{A}, \vec{B}, \vec{C}$ are defined to be collinear if 
		\begin{align}
			\label{eq:app-app-line-rank}
			\rank{\myvec{1 & 1 & 1 \\ \vec{A}& \vec{B}&\vec{C}}} = 2
		\end{align}
Are the given points in
			\eqref{eq:app-tri-pts}
collinear?
\\
\solution 
From 
			\eqref{eq:app-tri-pts},
\begin{align}
    \label{eq:app-1.1.3,2}
\myvec{
    1 & 1 & 1\\
    \vec{A} & \vec{B} & \vec{C} \\
    } 
    =
    %\label{eq:app-matthrowoperations}
    \myvec{
    1 & 1 & 1
    \\
    1 & -4 & -3
    \\
    -1 & 6 & -5
    }
     \xleftrightarrow[]{R_3 \leftarrow R_3+R_2}
    \myvec{
    1 & 1 & 1
    \\
    1 & -4 & -3
    \\
    0 & 2 & -8 
    }
    \\
     \xleftrightarrow[]{R_2\leftarrow R_1-R_2}
    \myvec{
    1 & 1 & 1
    \\
    0 & 5 & 4
    \\
    0 & 2 & -8 
    }
     \xleftrightarrow[]{R_3\leftarrow R_3-\frac{2}{5}R_2}
    \myvec{
    1 & 1 & 1
    \\
    0 & 5 & 4
    \\
    0 & 0 & \frac{-48}{5}
    }
\end{align}
There are no zero rows. So,
\begin{align}
    \text{rank}\myvec{
    1 & 1 & 1\\
    \vec{A} & \vec{B} & \vec{C} \\
    } &= 3 
\end{align}  
Hence,  the points $\vec{A},\vec{B},\vec{C}$ are not collinear. 
This is visible in 
\figref{fig1:Triangle}.
\begin{figure}[H]
\centering
\includegraphics[width=0.75\columnwidth]{figs/triangle/vector.pdf}
\caption{$\triangle ABC$}
\label{fig1:Triangle}
\end{figure}
% \\		\input{solutions/1/1/3/main.tex}
\item The parameteric form of the equation  of $AB$ is 
		\begin{align}
			\label{eq:app-geo-param}
			\vec{x}=\vec{A}+k\vec{m} \quad k \ne 0,
		\end{align}
		where
		\begin{align}
\vec{m}=\vec{B}-\vec{A}
		\end{align}
is the direction vector of $AB$.
Find the parameteric equations of $AB, BC$ and $CA$.
\\
\solution
From 
			\eqref{eq:app-geo-param} and
		\eqref{eq:app-geo-dir-vec-ab},
the parametric equation for $AB$ is given by
\begin{align}
AB: \vec{x} = &\myvec{1\\-1} + k \myvec{-5\\7}
\end{align}
Similarly, from 
		\eqref{eq:app-geo-dir-vec-bc} and
		\eqref{eq:app-geo-dir-vec-ca},
\begin{align}
BC: \vec{x} = &\myvec{-4\\6} + k \myvec{1\\-11}\\
CA: \vec{x} = &\myvec{-3\\-5} + k \myvec{4\\4}
\end{align}

%		\input{solutions/1/1/4/main.tex}
\item The normal form of the equation of $AB$  is 
		\begin{align}
			\label{eq:app-geo-normal}
			\vec{n}^{\top}\brak{	\vec{x}-\vec{A}} = 0
		\end{align}
		where 
		\begin{align}
			\vec{n}^{\top}\vec{m}&=\vec{n}^{\top}\brak{\vec{B}-\vec{A}} = 0
			\\
			\text{or, } \vec{n}&=\myvec{0 & 1 \\ -1 & 0} \vec{m}
			\label{eq:app-geo-norm-vec}
		\end{align}
Find the normal form of the equations of $AB, BC$ and $CA$.
\\
\solution
\begin{enumerate}
	\item
From
		\eqref{eq:app-geo-dir-vec-bc}, 
the direction vector of side $\vec{BC}$ is
\begin{align}
\vec{m}
	&=\myvec{1\\-11}
	\\
\implies \vec{n} &= \myvec{0 & 1\\
  -1 & 0}\myvec{1\\-11}
 = \myvec{-11\\-1}
		\label{eq:app-geo-norm-vec-bc}
\end{align}
from 
			\eqref{eq:app-geo-norm-vec}.
Hence, from 
			\eqref{eq:app-geo-normal},
the normal equation of side $BC$ is 
\begin{align}
	\vec{n}^{\top}\brak{	\vec{x}-\vec{B}} &= 0
			\\
\implies    \myvec{-11 & -1}\vec{x}&=\myvec{-11 & -1}\myvec{-4\\6}\\
    \implies
BC: \quad    \myvec{11 & 1}\vec{x}&=-38
\end{align}
\item Similarly, for $AB$,
from 
		\eqref{eq:app-geo-dir-vec-ab}, 
\begin{align}
	\vec{m} &= \myvec{-5\\7}
	\\
\implies        \vec{n} 
                &= \myvec{0&1\\-1&0}\myvec{-5\\7}
                = \myvec{7\\5}
		\label{eq:app-geo-norm-vec-ab}
\end{align}
and 
\begin{align}
	\vec{n}^{\top}\brak{	\vec{x}-\vec{A}} &= 0
	\\
	\implies
                AB: \quad  \vec{n}^{\top}\vec{x} &= \myvec{7&5}\myvec{1\\-1}\\    
       \implies\myvec{7&5}\vec{x} &= 2
\end{align}
\item For 
$CA$, 
from 
		\eqref{eq:app-geo-dir-vec-ca}, 
\begin{align}
\vec{m} &= \myvec{1 \\ 1}
\\
		\label{eq:app-geo-norm-vec-ca}
\implies \vec{n} 
&= \myvec{0&1 \\ -1&0}\myvec{1 \\ 1}
= \myvec{1 \\ -1}\\
\\
\implies	\vec{n}^{\top}\brak{	\vec{x}-\vec{C}} &= 0
\\
\implies \myvec{1&-1}{\vec{x}} &= \myvec{1&-1}\myvec{-3 \\ -5} 
= 2 
\end{align}
\end{enumerate}

%\input{solutions/1/1/5/assign1.tex}
\item The area of $\triangle ABC$ is defined as
		\begin{align}
			\label{eq:app-tri-area-cross}
			\frac{1}{2}\norm{{\brak{\vec{A}-\vec{B}}\times \brak{\vec{A}-\vec{C}}}}
		\end{align}
		where
		\begin{align}
			\vec{A}\times\vec{B} \triangleq \mydet{1 & -4 \\-1 & 6}
		\end{align}
		Find the area of $\triangle ABC$.\\
\solution
From
		\eqref{eq:app-geo-dir-vec-ab}
		and
		\eqref{eq:app-geo-dir-vec-ca},
\begin{align}
	\vec{A}-\vec{B}=\myvec{5\\-7},
	\vec{A}-\vec{C}&=\myvec{4\\4}\\
\implies (\vec{A}-\vec{B})\times(\vec{A}-\vec{C}) &=\mydet{5 & 4\\-7 & 4}\\
&=5\times 4-4\times (-7)\\&=48\\
\implies\frac{1}{2}\norm{(\vec{A}-\vec{B})\times(\vec{A}-\vec{C})}&=\frac{48}{2}=24
\end{align}
which is the desired area.

%  		\input{solutions/1/1/6/main.tex}
	\item Find the angles $A, B, C$ if 
%    \label{prop:angle2d}
  \begin{align}
    \label{eq:app-angle2d}
			\cos A \triangleq 
\frac{\brak{\vec{B}-\vec{A}}^{\top}{\vec{C}-\vec{A}}}{\norm{\vec{B}-\vec{A}}\norm{\vec{C}-\vec{A}}}
  \end{align}\\
  \solution
\begin{enumerate}
	\item From 
		\eqref{eq:app-geo-dir-vec-ab},
		\eqref{eq:app-geo-dir-vec-ca},
		\eqref{eq:app-geo-norm-ab}
		and
		\eqref{eq:app-geo-norm-ca}
\begin{align}
	(\vec{B}-\vec{A})^{\top}(\vec{C}-\vec{A})&=\myvec{-5&7}\myvec{-4\\-4}\\
	&=-8
	\\
	\implies
	\cos{A}&= \frac{-8}{\sqrt{74} \sqrt{32}}
	= \frac{-1}{\sqrt{37}}\\
	\implies A&=\cos^{-1}{\frac{-1}{\sqrt{37}}}
\end{align}
	\item From 
		\eqref{eq:app-geo-dir-vec-ab},
		\eqref{eq:app-geo-dir-vec-bc},
		\eqref{eq:app-geo-norm-ab}
		and
		\eqref{eq:app-geo-norm-bc}
\begin{align}
	(\vec{C}-\vec{B})^{\top}(\vec{A}-\vec{B})&=\myvec{1&-11}\myvec{5\\-7}\\
	&= 82
	\\
	\implies
	\cos{B}&= \frac{82}{\sqrt{74} \sqrt{122}}
	= \frac{41}{\sqrt{2257}}\\
	\implies B&=\cos^{-1}{\frac{41}{\sqrt{2257}}}
\end{align}
	\item From 
		\eqref{eq:app-geo-dir-vec-bc},
		\eqref{eq:app-geo-dir-vec-ca},
		\eqref{eq:app-geo-norm-bc}
		and
		\eqref{eq:app-geo-norm-ca}
\begin{align}
	(\vec{A}-\vec{C})^{\top}(\vec{B}-\vec{C})&=\myvec{4&4}\myvec{-1\\11}\\
	&=40
	\\
\implies	\cos{C}&= \frac{40}{\sqrt{32} \sqrt{122}}
	= \frac{5}{\sqrt{61}}\\
	\implies C&=\cos^{-1}{\frac{5}{\sqrt{61}}}
\end{align}

\end{enumerate}
%  	\input{solutions/1/1/7/main.tex}
All codes for this section are available at
\begin{lstlisting}
	codes/triangle/sides.py
\end{lstlisting}
\end{enumerate}

\subsection{Median}
\input{chapters/triangle/median}
\subsection{Altitude}
\input{chapters/triangle/altitude}
\subsection{Perpendicular Bisector}
\input{chapters/triangle/perp-bisect}
\subsection{Angle Bisector}
\input{chapters/triangle/angle-bisect}
\subsection{Eigenvalues and Eigenvectors}
\input{chapters/triangle/eigen}
\section{Matrices}
The mid point of $PB$ is
\begin{align}
\vec{M} =\frac{1}{2}(\vec{P}+\vec{B})
	= \myvec{4 \\ -2}  
\end{align}
which is equal to the direction vector of $OM$.
\begin{align}
\because \vec{M} \equiv
	 \myvec{1 \\ -\frac{1}{2}},
	m = -\frac{1}{2}
\end{align}
which is the desired slope.
See 
		\figref{fig:11/10/1/5}.
	\begin{figure}[!ht]
		\centering
 \includegraphics[width=\columnwidth]{chapters/11/10/1/5/figs/line.png}
		\caption{}
		\label{fig:11/10/1/5}
  	\end{figure}


%\section{Quadrilateral}
%\input{./chapters/exercises/quad_geo_exer}

\appendices
\section{Tangents to a Circle}
\numberwithin{equation}{section}
	\begin{figure}[H]
		\centering
 \includegraphics[width=0.75\columnwidth]{chapters/12/6/3/8/figs/main.png}
		\caption{}
		\label{fig:12/6/3/8}
  	\end{figure}
The equation of the conic can be represented as
\begin{align}
\vec{x}^{\top}\myvec{1&0\\0&0}\vec{x}+2\myvec{-2&\frac{-1}{2}}\vec{x}+4=0
\end{align}
So,
\begin{align}
\vec{V}=\myvec{1&0\\0&0},
\vec{u}^{\top}=\myvec{-2&\frac{-1}{2}},
f=4
\end{align}
The direction vector of the line passing through (2,0) and (4,4) is 
\begin{align}
\vec{m}=\myvec{1\\2}
\implies
\vec{n}=\myvec{2\\-1}.
\end{align}
The eigenvector corresponding to the zero eigenvalue is 
\begin{align}
\vec{p}_1=\myvec{0\\1},
\end{align}
In
\eqref{eq:conic_tangent_q_eigen},
\begin{align}
	\kappa=\frac{\myvec{0&1}\myvec{-2\\ \frac{-1}{2}}}{\myvec{0&1}\myvec{2\\-1}}
	=\frac{1}{2}
\end{align}
Substituting  $\kappa$,
from 
\eqref{eq:conic_tangent_q_eigen},
\begin{align}
	\myvec{\sbrak{\myvec{-2\\\frac{-1}{2}}+\frac{1}{2}\myvec{2\\-1}}^{\top} \\ \myvec{1&0\\0&0}}\vec{q} &= \myvec{-4 \\ \frac{1}{2}\myvec{2\\-1}-\myvec{-2\\\frac{-1}{2}}}\\
	\implies
	\myvec{-1&-1 \\ 1&0 \\ 0&0}\vec{q}&=\myvec{-4 \\ 3 \\ 0}
\end{align}
yielding
\begin{align}
\myvec{-1&-1 \\ 1&0}\vec{q} = \myvec{-4\\3}
\end{align}
The augmented matrix is 
\begin{align*}
  \myvec{
                -1&-1&\vrule&-4\\
	        1&0&\vrule&3}
  \xleftrightarrow[]{R_1 \leftarrow R_1+ 2R_2}
     \myvec{
	         1&-1&\vrule&2\\
	         1&0&\vrule&3}
      \\
 \xleftrightarrow[]{R_2 \leftarrow R_2 - R_1}
     \myvec{
	         1&-1&\vrule&2\\
	         0&1&\vrule&1}
 \xleftrightarrow[]{R_1 \leftarrow R_1 + R_2}
     \myvec{
	         1&0&\vrule&3\\
	         0&1&\vrule&1}
      \\ \implies \vec{q}=\myvec{3\\1}
\end{align*}
which is the desired 
point of contact.
See Fig. 
		\ref{fig:12/6/3/8}.



\iffalse
\latexprintindex
\fi

\end{document}


	\item Find the angles $A, B, C$ if 
%    \label{prop:angle2d}
  \begin{align}
    \label{eq:app-angle2d}
			\cos A \triangleq 
\frac{\brak{\vec{B}-\vec{A}}^{\top}{\vec{C}-\vec{A}}}{\norm{\vec{B}-\vec{A}}\norm{\vec{C}-\vec{A}}}
  \end{align}\\
  \solution
\begin{enumerate}
	\item From 
		\eqref{eq:app-geo-dir-vec-ab},
		\eqref{eq:app-geo-dir-vec-ca},
		\eqref{eq:app-geo-norm-ab}
		and
		\eqref{eq:app-geo-norm-ca}
\begin{align}
	(\vec{B}-\vec{A})^{\top}(\vec{C}-\vec{A})&=\myvec{-5&7}\myvec{-4\\-4}\\
	&=-8
	\\
	\implies
	\cos{A}&= \frac{-8}{\sqrt{74} \sqrt{32}}
	= \frac{-1}{\sqrt{37}}\\
	\implies A&=\cos^{-1}{\frac{-1}{\sqrt{37}}}
\end{align}
	\item From 
		\eqref{eq:app-geo-dir-vec-ab},
		\eqref{eq:app-geo-dir-vec-bc},
		\eqref{eq:app-geo-norm-ab}
		and
		\eqref{eq:app-geo-norm-bc}
\begin{align}
	(\vec{C}-\vec{B})^{\top}(\vec{A}-\vec{B})&=\myvec{1&-11}\myvec{5\\-7}\\
	&= 82
	\\
	\implies
	\cos{B}&= \frac{82}{\sqrt{74} \sqrt{122}}
	= \frac{41}{\sqrt{2257}}\\
	\implies B&=\cos^{-1}{\frac{41}{\sqrt{2257}}}
\end{align}
	\item From 
		\eqref{eq:app-geo-dir-vec-bc},
		\eqref{eq:app-geo-dir-vec-ca},
		\eqref{eq:app-geo-norm-bc}
		and
		\eqref{eq:app-geo-norm-ca}
\begin{align}
	(\vec{A}-\vec{C})^{\top}(\vec{B}-\vec{C})&=\myvec{4&4}\myvec{-1\\11}\\
	&=40
	\\
\implies	\cos{C}&= \frac{40}{\sqrt{32} \sqrt{122}}
	= \frac{5}{\sqrt{61}}\\
	\implies C&=\cos^{-1}{\frac{5}{\sqrt{61}}}
\end{align}

\end{enumerate}
%  	\documentclass[journal]{IEEEtran}
\usepackage{gvv-book}
\usepackage{gvv}
%\usepackage{styles/front}
%\usepackage{Wiley-AuthoringTemplate}
%\usepackage[sectionbib,authoryear]{natbib}% for name-date citation comment the below line
%\usepackage[sectionbib,numbers]{natbib}% for numbered citation comment the above line

%%********************************************************************%%
%%       How many levels of section head would you like numbered?     %%
%% 0= no section numbers, 1= section, 2= section, 3= subsection %%
\setcounter{secnumdepth}{3}
%%********************************************************************%%
%%**********************************************************************%%
%%     How many levels of section head would you like to appear in the  %%
%%				Table of Contents?			%%
%% 0= chapter, 1= section, 2= section, 3= subsection titles.	%%
\setcounter{tocdepth}{2}
%%**********************************************************************%%

%\includeonly{ch01}
\makeindex

\begin{document}
\bibliographystyle{IEEEtran}
\onecolumn


\title{
	\begin{flushleft}
	MATRICES \\ In Geometry
	\\
\rule{0.4\columnwidth}{0.4pt}
\end{flushleft}
}
\author{
\vspace{7cm}
	\begin{flushleft}
\includegraphics[width=0.2\columnwidth]{figs/logo.jpg}
\\
		{	\huge G. V. V. Sharma}
		\\
\vspace{1cm}
https://creativecommons.org/licenses/by-sa/3.0/
\\
and
\\
https://www.gnu.org/licenses/fdl-1.3.en.html
	\end{flushleft}
%\IEEEpubid{\makebox[\columnwidth]{978-1-7281-5966-1/20/\$31.00 ©2020 IEEE \hfill} \hspace{\columnsep}\makebox[\columnwidth]{ }}
}
\maketitle

\newpage


\tableofcontents

\newpage
\twocolumn

%\section{Triangle}
\section{Vectors}
Consider a triangle with vertices
		\begin{align}
			\label{eq:tri-pts}
			\vec{A} = \myvec{1 \\ -1},\,
			\vec{B} = \myvec{-4 \\ 6},\,
			\vec{C} = \myvec{-3 \\ -5}
		\end{align}
\subsection{Sides}
%\renewcommand{\theequation}{\theenumi}
\begin{enumerate}[label=\thesubsection.\arabic*.,ref=\thesubsection.\theenumi]
%\numberwithin{equation}{enumi}
\item The direction vector of $AB$ is defined as
		\begin{align}
			\vec{B}-
			\vec{A}
		\end{align}
Find the direction vectors of $AB, BC$ and $CA$.
\\
\solution 
\begin{enumerate} 
\item  The Direction vector of $AB$ is 
	\begin{align}  \vec{B} - \vec{A} 
		=\myvec{ -4\\ 6 } - \myvec{ 1\\ -1 }
 = \myvec{ -4 - 1\\ 6 - (-1) } = \myvec{ -5\\ 7 }
		\label{eq:app-geo-dir-vec-ab}
 \end{align}
\item The Direction vector of $BC$ is
	\begin{align} \vec{C} - \vec{B}=\myvec{ -3\\ -5} - \myvec{ -4\\ 6 }
 = \myvec{ -3 - (-4)\\ -5 - 6 } = \myvec{1\\ -11 }
		\label{eq:app-geo-dir-vec-bc}
  \end{align}
  \item  The Direction vector of $CA$  is
	  \begin{align}  \vec{A} - \vec{C} =\myvec{ 1\\ -1 }-\myvec{ -3\\ -5}
 = \myvec{ 1 - (-3)\\ -1 - (-5) } = \myvec{ 4\\ 4 }
		\label{eq:app-geo-dir-vec-ca}
  \end{align}
 \end{enumerate}
%	\input{solutions/1/1/1/prob_1.tex}
	\item The length of side $BC$ is 
		\label{prob:side-length}
		\begin{align}
			c = \norm{\vec{B}-\vec{A}} \triangleq \sqrt{\brak{\vec{B}-\vec{A}}^{\top}\brak{\vec{B}-\vec{A}}}
		\end{align}
		where
		\begin{align}
			\vec{A}^{\top}\triangleq\myvec{1 & -1}
		\end{align}
		Similarly, 
		\begin{align}
b = \norm{\vec{C}-\vec{B}},\,
a = \norm{\vec{A}-\vec{C}}
		\end{align}
		Find $a, b, c$.
\begin{enumerate}
	\item 
	From 	
		\eqref{eq:app-geo-dir-vec-ab},
\begin{align}
\vec{A}-\vec{B} &= \myvec{5\\-7}, \\
\implies 	c &= 	\norm{\vec{B}-\vec{A}} = \norm{\vec{A}-\vec{B}} 
	\\
	&= \sqrt{\myvec{5 & -7}\myvec{5\\-7}}
= \sqrt{\brak{5}^2 +\brak{7}^2}\\
	&=\sqrt{74}
		\label{eq:app-geo-norm-ab}
\end{align}
	\item Similarly, from 
		\eqref{eq:app-geo-dir-vec-bc},
\begin{align}
	a &= \norm{\vec{B}-\vec{C}} 
	= \sqrt{\myvec{-1 & 11}\myvec{-1\\11}}
\\
&= \sqrt{\brak{1}^2+\brak{11}^2}
	= \sqrt{122}
		\label{eq:app-geo-norm-bc}
\end{align}
and
		from 		\eqref{eq:app-geo-dir-vec-ca},
	\item 
		\begin{align}
			b &= \norm{\vec{A}-\vec{C}} = \sqrt{\myvec{4 & 4}\myvec{4\\4}}
\\
&= \sqrt{\brak{4}^2+\brak{4}^2}
	=\sqrt{32}
		\label{eq:app-geo-norm-ca}
\end{align}
\end{enumerate}
%  \\            
  %\input{solutions/1/1/2a/main.tex}
\item   Points $\vec{A}, \vec{B}, \vec{C}$ are defined to be collinear if 
		\begin{align}
			\label{eq:app-app-line-rank}
			\rank{\myvec{1 & 1 & 1 \\ \vec{A}& \vec{B}&\vec{C}}} = 2
		\end{align}
Are the given points in
			\eqref{eq:app-tri-pts}
collinear?
\\
\solution 
From 
			\eqref{eq:app-tri-pts},
\begin{align}
    \label{eq:app-1.1.3,2}
\myvec{
    1 & 1 & 1\\
    \vec{A} & \vec{B} & \vec{C} \\
    } 
    =
    %\label{eq:app-matthrowoperations}
    \myvec{
    1 & 1 & 1
    \\
    1 & -4 & -3
    \\
    -1 & 6 & -5
    }
     \xleftrightarrow[]{R_3 \leftarrow R_3+R_2}
    \myvec{
    1 & 1 & 1
    \\
    1 & -4 & -3
    \\
    0 & 2 & -8 
    }
    \\
     \xleftrightarrow[]{R_2\leftarrow R_1-R_2}
    \myvec{
    1 & 1 & 1
    \\
    0 & 5 & 4
    \\
    0 & 2 & -8 
    }
     \xleftrightarrow[]{R_3\leftarrow R_3-\frac{2}{5}R_2}
    \myvec{
    1 & 1 & 1
    \\
    0 & 5 & 4
    \\
    0 & 0 & \frac{-48}{5}
    }
\end{align}
There are no zero rows. So,
\begin{align}
    \text{rank}\myvec{
    1 & 1 & 1\\
    \vec{A} & \vec{B} & \vec{C} \\
    } &= 3 
\end{align}  
Hence,  the points $\vec{A},\vec{B},\vec{C}$ are not collinear. 
This is visible in 
\figref{fig1:Triangle}.
\begin{figure}[H]
\centering
\includegraphics[width=0.75\columnwidth]{figs/triangle/vector.pdf}
\caption{$\triangle ABC$}
\label{fig1:Triangle}
\end{figure}
% \\		\input{solutions/1/1/3/main.tex}
\item The parameteric form of the equation  of $AB$ is 
		\begin{align}
			\label{eq:app-geo-param}
			\vec{x}=\vec{A}+k\vec{m} \quad k \ne 0,
		\end{align}
		where
		\begin{align}
\vec{m}=\vec{B}-\vec{A}
		\end{align}
is the direction vector of $AB$.
Find the parameteric equations of $AB, BC$ and $CA$.
\\
\solution
From 
			\eqref{eq:app-geo-param} and
		\eqref{eq:app-geo-dir-vec-ab},
the parametric equation for $AB$ is given by
\begin{align}
AB: \vec{x} = &\myvec{1\\-1} + k \myvec{-5\\7}
\end{align}
Similarly, from 
		\eqref{eq:app-geo-dir-vec-bc} and
		\eqref{eq:app-geo-dir-vec-ca},
\begin{align}
BC: \vec{x} = &\myvec{-4\\6} + k \myvec{1\\-11}\\
CA: \vec{x} = &\myvec{-3\\-5} + k \myvec{4\\4}
\end{align}

%		\input{solutions/1/1/4/main.tex}
\item The normal form of the equation of $AB$  is 
		\begin{align}
			\label{eq:app-geo-normal}
			\vec{n}^{\top}\brak{	\vec{x}-\vec{A}} = 0
		\end{align}
		where 
		\begin{align}
			\vec{n}^{\top}\vec{m}&=\vec{n}^{\top}\brak{\vec{B}-\vec{A}} = 0
			\\
			\text{or, } \vec{n}&=\myvec{0 & 1 \\ -1 & 0} \vec{m}
			\label{eq:app-geo-norm-vec}
		\end{align}
Find the normal form of the equations of $AB, BC$ and $CA$.
\\
\solution
\begin{enumerate}
	\item
From
		\eqref{eq:app-geo-dir-vec-bc}, 
the direction vector of side $\vec{BC}$ is
\begin{align}
\vec{m}
	&=\myvec{1\\-11}
	\\
\implies \vec{n} &= \myvec{0 & 1\\
  -1 & 0}\myvec{1\\-11}
 = \myvec{-11\\-1}
		\label{eq:app-geo-norm-vec-bc}
\end{align}
from 
			\eqref{eq:app-geo-norm-vec}.
Hence, from 
			\eqref{eq:app-geo-normal},
the normal equation of side $BC$ is 
\begin{align}
	\vec{n}^{\top}\brak{	\vec{x}-\vec{B}} &= 0
			\\
\implies    \myvec{-11 & -1}\vec{x}&=\myvec{-11 & -1}\myvec{-4\\6}\\
    \implies
BC: \quad    \myvec{11 & 1}\vec{x}&=-38
\end{align}
\item Similarly, for $AB$,
from 
		\eqref{eq:app-geo-dir-vec-ab}, 
\begin{align}
	\vec{m} &= \myvec{-5\\7}
	\\
\implies        \vec{n} 
                &= \myvec{0&1\\-1&0}\myvec{-5\\7}
                = \myvec{7\\5}
		\label{eq:app-geo-norm-vec-ab}
\end{align}
and 
\begin{align}
	\vec{n}^{\top}\brak{	\vec{x}-\vec{A}} &= 0
	\\
	\implies
                AB: \quad  \vec{n}^{\top}\vec{x} &= \myvec{7&5}\myvec{1\\-1}\\    
       \implies\myvec{7&5}\vec{x} &= 2
\end{align}
\item For 
$CA$, 
from 
		\eqref{eq:app-geo-dir-vec-ca}, 
\begin{align}
\vec{m} &= \myvec{1 \\ 1}
\\
		\label{eq:app-geo-norm-vec-ca}
\implies \vec{n} 
&= \myvec{0&1 \\ -1&0}\myvec{1 \\ 1}
= \myvec{1 \\ -1}\\
\\
\implies	\vec{n}^{\top}\brak{	\vec{x}-\vec{C}} &= 0
\\
\implies \myvec{1&-1}{\vec{x}} &= \myvec{1&-1}\myvec{-3 \\ -5} 
= 2 
\end{align}
\end{enumerate}

%\input{solutions/1/1/5/assign1.tex}
\item The area of $\triangle ABC$ is defined as
		\begin{align}
			\label{eq:app-tri-area-cross}
			\frac{1}{2}\norm{{\brak{\vec{A}-\vec{B}}\times \brak{\vec{A}-\vec{C}}}}
		\end{align}
		where
		\begin{align}
			\vec{A}\times\vec{B} \triangleq \mydet{1 & -4 \\-1 & 6}
		\end{align}
		Find the area of $\triangle ABC$.\\
\solution
From
		\eqref{eq:app-geo-dir-vec-ab}
		and
		\eqref{eq:app-geo-dir-vec-ca},
\begin{align}
	\vec{A}-\vec{B}=\myvec{5\\-7},
	\vec{A}-\vec{C}&=\myvec{4\\4}\\
\implies (\vec{A}-\vec{B})\times(\vec{A}-\vec{C}) &=\mydet{5 & 4\\-7 & 4}\\
&=5\times 4-4\times (-7)\\&=48\\
\implies\frac{1}{2}\norm{(\vec{A}-\vec{B})\times(\vec{A}-\vec{C})}&=\frac{48}{2}=24
\end{align}
which is the desired area.

%  		\input{solutions/1/1/6/main.tex}
	\item Find the angles $A, B, C$ if 
%    \label{prop:angle2d}
  \begin{align}
    \label{eq:app-angle2d}
			\cos A \triangleq 
\frac{\brak{\vec{B}-\vec{A}}^{\top}{\vec{C}-\vec{A}}}{\norm{\vec{B}-\vec{A}}\norm{\vec{C}-\vec{A}}}
  \end{align}\\
  \solution
\begin{enumerate}
	\item From 
		\eqref{eq:app-geo-dir-vec-ab},
		\eqref{eq:app-geo-dir-vec-ca},
		\eqref{eq:app-geo-norm-ab}
		and
		\eqref{eq:app-geo-norm-ca}
\begin{align}
	(\vec{B}-\vec{A})^{\top}(\vec{C}-\vec{A})&=\myvec{-5&7}\myvec{-4\\-4}\\
	&=-8
	\\
	\implies
	\cos{A}&= \frac{-8}{\sqrt{74} \sqrt{32}}
	= \frac{-1}{\sqrt{37}}\\
	\implies A&=\cos^{-1}{\frac{-1}{\sqrt{37}}}
\end{align}
	\item From 
		\eqref{eq:app-geo-dir-vec-ab},
		\eqref{eq:app-geo-dir-vec-bc},
		\eqref{eq:app-geo-norm-ab}
		and
		\eqref{eq:app-geo-norm-bc}
\begin{align}
	(\vec{C}-\vec{B})^{\top}(\vec{A}-\vec{B})&=\myvec{1&-11}\myvec{5\\-7}\\
	&= 82
	\\
	\implies
	\cos{B}&= \frac{82}{\sqrt{74} \sqrt{122}}
	= \frac{41}{\sqrt{2257}}\\
	\implies B&=\cos^{-1}{\frac{41}{\sqrt{2257}}}
\end{align}
	\item From 
		\eqref{eq:app-geo-dir-vec-bc},
		\eqref{eq:app-geo-dir-vec-ca},
		\eqref{eq:app-geo-norm-bc}
		and
		\eqref{eq:app-geo-norm-ca}
\begin{align}
	(\vec{A}-\vec{C})^{\top}(\vec{B}-\vec{C})&=\myvec{4&4}\myvec{-1\\11}\\
	&=40
	\\
\implies	\cos{C}&= \frac{40}{\sqrt{32} \sqrt{122}}
	= \frac{5}{\sqrt{61}}\\
	\implies C&=\cos^{-1}{\frac{5}{\sqrt{61}}}
\end{align}

\end{enumerate}
%  	\input{solutions/1/1/7/main.tex}
All codes for this section are available at
\begin{lstlisting}
	codes/triangle/sides.py
\end{lstlisting}
\end{enumerate}

\subsection{Median}
\input{chapters/triangle/median}
\subsection{Altitude}
\input{chapters/triangle/altitude}
\subsection{Perpendicular Bisector}
\input{chapters/triangle/perp-bisect}
\subsection{Angle Bisector}
\input{chapters/triangle/angle-bisect}
\subsection{Eigenvalues and Eigenvectors}
\input{chapters/triangle/eigen}
\section{Matrices}
The mid point of $PB$ is
\begin{align}
\vec{M} =\frac{1}{2}(\vec{P}+\vec{B})
	= \myvec{4 \\ -2}  
\end{align}
which is equal to the direction vector of $OM$.
\begin{align}
\because \vec{M} \equiv
	 \myvec{1 \\ -\frac{1}{2}},
	m = -\frac{1}{2}
\end{align}
which is the desired slope.
See 
		\figref{fig:11/10/1/5}.
	\begin{figure}[!ht]
		\centering
 \includegraphics[width=\columnwidth]{chapters/11/10/1/5/figs/line.png}
		\caption{}
		\label{fig:11/10/1/5}
  	\end{figure}


%\section{Quadrilateral}
%\input{./chapters/exercises/quad_geo_exer}

\appendices
\section{Tangents to a Circle}
\numberwithin{equation}{section}
	\begin{figure}[H]
		\centering
 \includegraphics[width=0.75\columnwidth]{chapters/12/6/3/8/figs/main.png}
		\caption{}
		\label{fig:12/6/3/8}
  	\end{figure}
The equation of the conic can be represented as
\begin{align}
\vec{x}^{\top}\myvec{1&0\\0&0}\vec{x}+2\myvec{-2&\frac{-1}{2}}\vec{x}+4=0
\end{align}
So,
\begin{align}
\vec{V}=\myvec{1&0\\0&0},
\vec{u}^{\top}=\myvec{-2&\frac{-1}{2}},
f=4
\end{align}
The direction vector of the line passing through (2,0) and (4,4) is 
\begin{align}
\vec{m}=\myvec{1\\2}
\implies
\vec{n}=\myvec{2\\-1}.
\end{align}
The eigenvector corresponding to the zero eigenvalue is 
\begin{align}
\vec{p}_1=\myvec{0\\1},
\end{align}
In
\eqref{eq:conic_tangent_q_eigen},
\begin{align}
	\kappa=\frac{\myvec{0&1}\myvec{-2\\ \frac{-1}{2}}}{\myvec{0&1}\myvec{2\\-1}}
	=\frac{1}{2}
\end{align}
Substituting  $\kappa$,
from 
\eqref{eq:conic_tangent_q_eigen},
\begin{align}
	\myvec{\sbrak{\myvec{-2\\\frac{-1}{2}}+\frac{1}{2}\myvec{2\\-1}}^{\top} \\ \myvec{1&0\\0&0}}\vec{q} &= \myvec{-4 \\ \frac{1}{2}\myvec{2\\-1}-\myvec{-2\\\frac{-1}{2}}}\\
	\implies
	\myvec{-1&-1 \\ 1&0 \\ 0&0}\vec{q}&=\myvec{-4 \\ 3 \\ 0}
\end{align}
yielding
\begin{align}
\myvec{-1&-1 \\ 1&0}\vec{q} = \myvec{-4\\3}
\end{align}
The augmented matrix is 
\begin{align*}
  \myvec{
                -1&-1&\vrule&-4\\
	        1&0&\vrule&3}
  \xleftrightarrow[]{R_1 \leftarrow R_1+ 2R_2}
     \myvec{
	         1&-1&\vrule&2\\
	         1&0&\vrule&3}
      \\
 \xleftrightarrow[]{R_2 \leftarrow R_2 - R_1}
     \myvec{
	         1&-1&\vrule&2\\
	         0&1&\vrule&1}
 \xleftrightarrow[]{R_1 \leftarrow R_1 + R_2}
     \myvec{
	         1&0&\vrule&3\\
	         0&1&\vrule&1}
      \\ \implies \vec{q}=\myvec{3\\1}
\end{align*}
which is the desired 
point of contact.
See Fig. 
		\ref{fig:12/6/3/8}.



\iffalse
\latexprintindex
\fi

\end{document}


All codes for this section are available at
\begin{lstlisting}
	codes/triangle/sides.py
\end{lstlisting}
\end{enumerate}

\subsection{Formulae}
\begin{enumerate}[label=\thesection.\arabic*.,ref=\thesection.\theenumi]
\numberwithin{equation}{enumi}
\item The equation of a line is given by 
\begin{align}
			\label{eq:app-line-school}
	y &= mx + c
	\\
	\implies \myvec{x \\ y} &= \myvec{x \\ 
	 mx + c} =\myvec{0 \\ c} + x\myvec{1 \\ m}
\end{align}
			yielding \eqref{eq:geo-param}.
\item 			\eqref{eq:app-line-school} can also be expressed as
\begin{align}
	y - mx &= c 
	\\
	\implies \myvec{-m & 1}\myvec{x \\ y} &= c
\end{align}
			yielding \eqref{eq:geo-normal}.
		\item The direction vector is 
\begin{align}
			\label{eq:app-line-school-dir}
\vec{m} = \myvec{1 \\ m}
\end{align}
and the normal vector is
\begin{align}
\vec{n}=\myvec{-m \\ 1}
			\label{eq:app-line-school-normal}
\end{align}
  \item From \eqref{eq:geo-param}, 
	  if $\vec{A},\vec{D}$ and $\vec{C}$ are on the same line,
		\label{prop:app-lin-dep}
\begin{align}
			\vec{D}=\vec{A}+q\vec{m} 
			\\ 
			\vec{C}=\vec{D}+p\vec{m} \\
			\label{eq:app-collinear} 
			\implies 	p\brak{\vec{D}-\vec{A}} 
			+ q\brak{\vec{D}-\vec{C}} = 0, \quad p, q \ne 0 \\ 
			\implies \vec{D} = \frac{p\vec{A}+q\vec{C}}{p+q} 
			\end{align} 
			yielding \eqref{eq:section_formula} upon substituting \begin{align} k = \frac{p}{q}. \end{align} 
			$\brak{\vec{D}-\vec{A}}, \brak{\vec{D}-\vec{C}}$ 
		are then said to be {\em linearly dependent}.
	\item If $\vec{A}, \vec{B}, \vec{C}$ are collinear,  from \eqref{eq:geo-normal}, \begin{align}
	 \vec{n}^{\top}\vec{A} &=  c 
	 \\
	 \vec{n}^{\top}\vec{B} &=  c 
	 \\
	 \vec{n}^{\top}\vec{C} &=  c 
\end{align}
which can be expressed as 
\begin{align}
		\label{prop:app-lin-eq}
	\myvec{ \vec{A} & \vec{B} & \vec{C}}^{\top}\vec{n} = c\myvec{1 \\ 1 \\ 1}
	\\
	\equiv \myvec{ \vec{A} & \vec{B} & \vec{C}}^{\top}\vec{n} = \myvec{1 \\ 1 \\ 1}
		\label{prop:app-lin-eq-unit-mat},
	\\
	\implies 
	\myvec{ 1 & 1 &1 \\ \vec{A} & \vec{B} & \vec{C}}^{\top}\myvec{\vec{n} \\ -1} &= \vec{0}
		\label{prop:app-lin-dep-rank}
\end{align}
yielding
		\begin{align}
			\label{eq:app-line-rank-2}
			\rank{\myvec{1 & 1 & 1 \\ \vec{A}& \vec{B}&\vec{C}}} = 2
		\end{align}
			  Rank is defined to be the number of linearly indpendent rows or columns of a matrix.
		\item
The equation of a line can also be expressed as
\begin{align}
	 \vec{n}^{\top}\vec{x} &=   1
		\label{prop:app-lin-eq-unit}
\end{align}
	  \end{enumerate}

\subsection{Median}
\input{chapters/triangle/median}
\subsection{Altitude}
\input{chapters/triangle/altitude}
\subsection{Perpendicular Bisector}
\input{chapters/triangle/perp-bisect}
\subsection{Angle Bisector}
\input{chapters/triangle/angle-bisect}
\subsection{Eigenvalues and Eigenvectors}
\input{chapters/triangle/eigen}
\subsection{Formulae}
	\begin{figure}[H]
		\centering
 \includegraphics[width=0.75\columnwidth]{chapters/12/6/3/8/figs/main.png}
		\caption{}
		\label{fig:12/6/3/8}
  	\end{figure}
The equation of the conic can be represented as
\begin{align}
\vec{x}^{\top}\myvec{1&0\\0&0}\vec{x}+2\myvec{-2&\frac{-1}{2}}\vec{x}+4=0
\end{align}
So,
\begin{align}
\vec{V}=\myvec{1&0\\0&0},
\vec{u}^{\top}=\myvec{-2&\frac{-1}{2}},
f=4
\end{align}
The direction vector of the line passing through (2,0) and (4,4) is 
\begin{align}
\vec{m}=\myvec{1\\2}
\implies
\vec{n}=\myvec{2\\-1}.
\end{align}
The eigenvector corresponding to the zero eigenvalue is 
\begin{align}
\vec{p}_1=\myvec{0\\1},
\end{align}
In
\eqref{eq:conic_tangent_q_eigen},
\begin{align}
	\kappa=\frac{\myvec{0&1}\myvec{-2\\ \frac{-1}{2}}}{\myvec{0&1}\myvec{2\\-1}}
	=\frac{1}{2}
\end{align}
Substituting  $\kappa$,
from 
\eqref{eq:conic_tangent_q_eigen},
\begin{align}
	\myvec{\sbrak{\myvec{-2\\\frac{-1}{2}}+\frac{1}{2}\myvec{2\\-1}}^{\top} \\ \myvec{1&0\\0&0}}\vec{q} &= \myvec{-4 \\ \frac{1}{2}\myvec{2\\-1}-\myvec{-2\\\frac{-1}{2}}}\\
	\implies
	\myvec{-1&-1 \\ 1&0 \\ 0&0}\vec{q}&=\myvec{-4 \\ 3 \\ 0}
\end{align}
yielding
\begin{align}
\myvec{-1&-1 \\ 1&0}\vec{q} = \myvec{-4\\3}
\end{align}
The augmented matrix is 
\begin{align*}
  \myvec{
                -1&-1&\vrule&-4\\
	        1&0&\vrule&3}
  \xleftrightarrow[]{R_1 \leftarrow R_1+ 2R_2}
     \myvec{
	         1&-1&\vrule&2\\
	         1&0&\vrule&3}
      \\
 \xleftrightarrow[]{R_2 \leftarrow R_2 - R_1}
     \myvec{
	         1&-1&\vrule&2\\
	         0&1&\vrule&1}
 \xleftrightarrow[]{R_1 \leftarrow R_1 + R_2}
     \myvec{
	         1&0&\vrule&3\\
	         0&1&\vrule&1}
      \\ \implies \vec{q}=\myvec{3\\1}
\end{align*}
which is the desired 
point of contact.
See Fig. 
		\ref{fig:12/6/3/8}.

\subsection{Matrices}
The mid point of $PB$ is
\begin{align}
\vec{M} =\frac{1}{2}(\vec{P}+\vec{B})
	= \myvec{4 \\ -2}  
\end{align}
which is equal to the direction vector of $OM$.
\begin{align}
\because \vec{M} \equiv
	 \myvec{1 \\ -\frac{1}{2}},
	m = -\frac{1}{2}
\end{align}
which is the desired slope.
See 
		\figref{fig:11/10/1/5}.
	\begin{figure}[!ht]
		\centering
 \includegraphics[width=\columnwidth]{chapters/11/10/1/5/figs/line.png}
		\caption{}
		\label{fig:11/10/1/5}
  	\end{figure}

\newpage
\section{Conic Section}
\subsection{Equation}
\begin{enumerate}[label=\thesubsection.\arabic*.,ref=\thesubsection.\theenumi]
\item
  Let $\vec{q}$ be a point such that the ratio of its distance from a fixed point $\vec{F}$ and the distance ($d$) from a fixed line 
	\begin{align}
L: \vec{n}^{\top}\vec{x}=c 
	\end{align}
		is constant, given by 
\label{conics/30/def}
\begin{align}
\frac{\norm{\vec{q}-\vec{F}}}{d} = e    
\end{align}
The locus of $\vec{q}$ is known as a conic section. The line $L$ is known as the directrix and the point $\vec{F}$ is the focus. $e$ is defined to be 
the eccentricity of the conic.  
\begin{enumerate}
    \item For $e = 1$, the conic is a parabola
    \item For $e < 1$, the conic is an ellipse
    \item For $e > 1$, the conic is a hyperbola
\end{enumerate}

\item
The equation of  a conic with directrix $\vec{n}^{\top}\vec{x} = c$, eccentricity $e$ and focus $\vec{F}$ is given by 
\begin{align}
    \label{eq:app-conic_quad_form}
	\text{g}\brak{\vec{x}} = \vec{x}^{\top}\vec{V}\vec{x}+2\vec{u}^{\top}\vec{x}+f=0
    \end{align}
where     
\begin{align}
  \label{eq:app-conic_quad_form_v}
\vec{V} &=\norm{\vec{n}}^2\vec{I}-e^2\vec{n}\vec{n}^{\top}, 
\\
\label{eq:app-conic_quad_form_u}
\vec{u} &= ce^2\vec{n}-\norm{\vec{n}}^2\vec{F}, 
\\
\label{eq:app-conic_quad_form_f}
f &= \norm{\vec{n}}^2\norm{\vec{F}}^2-c^2e^2
    \end{align}
    \solution
  Using Definition \ref{conics/30/def} and 
			\eqref{eq:PQ-final},
for any point $\vec{x}$ on the conic,
\begin{align}
	\norm{\vec{x}-\vec{F}}^2&=e^2 \frac{\brak{{\vec{n}^{\top}\vec{x} - c}}^2}{\norm{\vec{n}}^2}\label{conics/30/eq:1} \\
	\implies \norm{\vec{n}}^2\brak{\vec{x}-\vec{F}}^{\top}\brak{\vec{x}-\vec{F}}&=e^2\brak{\vec{n}^{\top}\vec{x} - c}^2
\\
\implies \norm{\vec{n}}^2\brak{\vec{x}^{\top}\vec{x}-2\vec{F}^{\top}\vec{x}+\norm{\vec{F}}^2}
	&=e^2\brak{c^2+\brak{\vec{n}^{\top}\vec{x} }^2-2c\vec{n}^{\top}\vec{x}} \\
	&=e^2\brak{c^2+\brak{\vec{x}^{\top}\vec{n}\vec{n}^{\top}\vec{x} }-2c\vec{n}^{\top}\vec{x}}
\end{align}
%
which can be expressed as \eqref{eq:app-conic_quad_form} after simplification.
\item
  The eccentricity, directrices and foci of \eqref{eq:app-conic_quad_form} are given by 
\begin{align}
  \label{eq:app-conic_quad_form_e} 
  e&= \sqrt{1-\frac{\lambda_1}{\lambda_2}}
\\
\label{eq:app-conic_quad_form_nc} 
	\begin{split}
  \vec{n}&= \sqrt{\lambda_2}\vec{p}_1,  
  \\
	c &= 
  \begin{cases}
    \frac{e\vec{u}^{\top}\vec{n} \pm \sqrt{e^2\brak{\vec{u}^{\top}\vec{n}}^2-\lambda_2\brak{e^2-1}\brak{\norm{\vec{u}}^2 - \lambda_2 f}}}{\lambda_2e\brak{e^2-1}} & e \ne 1
    \\
    \frac{\norm{\vec{u}}^2 - \lambda_2 f   }{2\vec{u}^{\top}\vec{n}} & e = 1
  \end{cases}
	\end{split}
  \\
  \label{eq:app-conic_quad_form_F} 
  \vec{F}  &= \frac{ce^2\vec{n}-\vec{u}}{\lambda_2}
\end{align}  
	\label{app:conic-parameters}
	\solution
	From \eqref{eq:app-conic_quad_form_v}, using the fact that $\vec{V}$ is symmetric with $\vec{V} = \vec{V}^{\top}$,
  \begin{align}
	  \vec{V}^{\top} \vec{V}&=\brak{\norm{\vec{n}}^2\vec{I}-e^2\vec{n}\vec{n}^{\top}}^{\top}
	  \brak{\norm{\vec{n}}^2\vec{I}-e^2\vec{n}\vec{n}^{\top}}
    \\
	  \implies \vec{V}^{2} &= \norm{\vec{n}}^4\vec{I}+e^4\vec{n}\vec{n}^{\top}\vec{n}\vec{n}^{\top}
	  -2e^2\norm{\vec{n}}^2\vec{n}\vec{n}^{\top}
    \\
	  &= \norm{\vec{n}}^4\vec{I} + e^4\norm{\vec{n}}^2\vec{n}\vec{n}^{\top}
	%  \\
	  - 2e^2\norm{\vec{n}}^2\vec{n}\vec{n}^{\top}
    \\
	  &= \norm{\vec{n}}^4\vec{I} + e^2\brak{e^2 - 2}\norm{\vec{n}}^2\vec{n}\vec{n}^{\top}
    \\
	  &= \norm{\vec{n}}^4\vec{I} + \brak{e^2 - 2}\norm{\vec{n}}^2\brak{\norm{\vec{n}}^2\vec{I}- \vec{V}}
    \end{align}
%    
which can be expressed as
\begin{align}
  \vec{V}^{2} + \brak{e^2 - 2}\norm{\vec{n}}^2\vec{V} - \brak{e^2 - 1}\norm{\vec{n}}^4\vec{I}=0
  \label{eq:app-conic_quad_form_e_cayley}
\end{align}
	Using the Cayley-Hamilton theorem,
	\eqref{eq:app-conic_quad_form_e_cayley} results in the characteristic equation, 
\begin{align}
  \lambda^{2} - \brak{2-e^2}\norm{\vec{n}}^2\lambda + \brak{1-e^2 }\norm{\vec{n}}^4=0
\end{align}
which can be expressed as
\begin{align}
\brak{\frac{\lambda}{\norm{\vec{n}}^2}}^2 - \brak{2-e^2 }\brak{\frac{\lambda}{\norm{\vec{n}}^2}} 
	+ \brak{1-e^2 } = 0
	\\
	\implies \frac{\lambda}{\norm{\vec{n}}^2} = 1-e^2, 1
  \\
	\text{or, }\lambda_2 = \norm{\vec{n}}^2, \lambda_1 = \brak{1-e^2}\lambda_2 
  \label{eq:app-conic_quad_form_lam_cayley}
\end{align}
From   \eqref{eq:app-conic_quad_form_lam_cayley}, the eccentricity of \eqref{eq:app-conic_quad_form} is given by 
\eqref{eq:app-conic_quad_form_e}.   
%
Multiplying both sides of    \eqref{eq:app-conic_quad_form_v} by $\vec{n}$,
\begin{align}
\vec{V} \vec{n}&=\norm{\vec{n}}^2\vec{n}-e^2\vec{n}\vec{n}^{\top}\vec{n} 
\\
&=\norm{\vec{n}}^2\brak{1-e^2}\vec{n} 
 \\
  &=\lambda_1 \vec{n} 
	\\
  \label{eq:eigevecn}
\end{align}  
from \eqref{eq:app-conic_quad_form_lam_cayley}.
Thus,  $\lambda_1$ is the corresponding eigenvalue for $\vec{n}$.  From       \eqref{eq:eigevecP} and \eqref{eq:eigevecn}, this implies that 
\begin{align}  
	\vec{p}_1 &= \frac{\vec{n}}{\norm{\vec{n}}} 
	\\
	\text{or, }
   \vec{n}&= \norm{\vec{n}}\vec{p}_1  = \sqrt{\lambda_2}\vec{p}_1 
\end{align}  
from   \eqref{eq:app-conic_quad_form_lam_cayley} .
From \eqref{eq:app-conic_quad_form_u} and \eqref{eq:app-conic_quad_form_lam_cayley},
\begin{align}
\vec{F}  &= \frac{ce^2\vec{n}-\vec{u}}{\lambda_2}
 \\
 \implies \norm{\vec{F}}^2  &= \frac{\brak{ce^2\vec{n}-\vec{u}}^{\top}\brak{ce^2\vec{n}-\vec{u}}}{\lambda_2^2}
 \\
 \implies \lambda_2^2\norm{\vec{F}}^2  &= c^2e^4\lambda_2-2ce^2\vec{u}^{\top}\vec{n}+\norm{\vec{u}}^2
 \label{eq:app-conic_quad_form_u_temp}
    \end{align}
    Also, \eqref{eq:app-conic_quad_form_f} can be expressed as
    \begin{align}
    \lambda_2\norm{\vec{F}}^2 &= f+c^2e^2
    \label{eq:app-conic_quad_form_f_temp}
\end{align}
From  \eqref{eq:app-conic_quad_form_u_temp} and     \eqref{eq:app-conic_quad_form_f_temp},
\begin{align}
c^2e^4\lambda_2-2ce^2\vec{u}^{\top}\vec{n}+\norm{\vec{u}}^2 = \lambda_2\brak{f+c^2e^2}
\end{align}
\begin{align}
\implies \lambda_2e^2\brak{e^2-1}c^2-2ce^2\vec{u}^{\top}\vec{n}
	+\norm{\vec{u}}^2 - \lambda_2 f = 0
\end{align}
yielding
  \eqref{eq:app-conic_quad_form_F}. 
\item
\eqref{eq:app-conic_quad_form} represents 
	\begin{enumerate}
		\item a parabola for $\mydet{\vec{V}} = 0 $,
		\item ellipse for $\mydet{\vec{V}} > 0 $ and 
		\item hyperbola for $\mydet{\vec{V}} < 0 $.
	\end{enumerate}
\solution
  From \eqref{eq:app-conic_quad_form_e},
\begin{align}
  \frac{\lambda_1}{\lambda_2} = 1 - e^2
\end{align}
Also, 
\begin{align}
	\mydet{\vec{V}} =   \lambda_1\lambda_2 
\end{align}
	yielding \tabref{table:det}.
\begin{table}[H]
\centering
\resizebox{\columnwidth}{!}{%
\input{tables/det.tex}
	}
	\caption{}
\label{table:det}
\end{table}
			\item Using the affine transformation in
					\label{app:std-prm-P}
	\eqref{eq:conic_affine},
	the conic in     \eqref{eq:app-conic_quad_form} can be expressed in standard form 
	%(centre/vertex at the origin, major axis - $x$ axis)
	as
  \begin{align}
    %\begin{aligned}
    \label{eq:app-conic_simp_temp_nonparab}
	    \vec{y}^{\top}\brak{\frac{\vec{D}}{f_0}}\vec{y} &= 1   &  \abs{\vec{V}} &\ne 0
    \\
	    \vec{y}^{\top}\vec{D}\vec{y} &=  -\eta\vec{e}_1^{\top}\vec{y}   & \abs{\vec{V}} &= 0
    \label{eq:app-conic_simp_temp_parab}
    %\end{aligned}
    \end{align}
    where
  \begin{align}
      %\begin{split}
      \label{eq:app-f0}
	  f_0 &=\vec{u}^{\top}\vec{V}^{-1}\vec{u} -f \ne 0
	  \\
      \label{eq:app-eta}
       \eta &=2\vec{u}^{\top}\vec{p}_1
       \\
       \vec{e}_1 &=\myvec{1 \\ 0}
      \end{align}
      \solution
  \label{app:parab}
	Using 
\eqref{eq:conic_affine},
\eqref{eq:app-conic_quad_form} can be expressed as
\begin{align}
\brak{\vec{P}\vec{y}+\vec{c}}^{\top}\vec{V}\brak{\vec{P}\vec{y}+\vec{c}}+2\vec{u}^{\top}\brak{\vec{P}\vec{y}+\vec{c}}+ f
	= 0, 
\end{align}
yielding 
\begin{align}
\vec{y}^{\top}\vec{P}^{\top}\vec{V}\vec{P}\vec{y}+2\brak{\vec{V}\vec{c}+\vec{u}}^{\top}\vec{P}\vec{y}
+  \vec{c}^{\top}\vec{V}\vec{c} + 2\vec{u}^{\top}\vec{c} + f= 0
\label{eq:conic_simp_one}
\end{align}
%
From \eqref{eq:conic_simp_one} and \eqref{eq:conic_parmas_eig_def},
\begin{align}
\vec{y}^{\top}\vec{D}\vec{y}+2\brak{\vec{V}\vec{c}+\vec{u}}^{\top}\vec{P}\vec{y}
+  \vec{c}^{\top}\brak{\vec{V}\vec{c} + \vec{u}}+ \vec{u}^{\top}\vec{c} + f= 0
\label{eq:conic_simp}
\end{align}
When $\vec{V}^{-1}$ exists, choosing
\begin{align}
%\begin{split}
\vec{V}\vec{c}+\vec{u} &= \vec{0}, \quad \text{or}, \vec{c} = -\vec{V}^{-1}\vec{u},
\label{eq:conic_parmas_c_def}
\end{align}
%
%%From \eqref{eq:conic_parmas_k_def} and 
%%
and substituting \eqref{eq:conic_parmas_c_def}
in \eqref{eq:conic_simp}
yields \eqref{eq:app-conic_simp_temp_nonparab}. 
  %See Appendix \ref{app:parab}.
When $\abs{\vec{V}} = 0, \lambda_1 = 0$ and 
\begin{align}
\vec{V}\vec{p}_1 = 0, 
\vec{V}\vec{p}_2 = \lambda_2\vec{p}_2.
\label{eq:conic_parab_eig_prop} 
\end{align}
Substituting \eqref{eq:eig_matrix}
in \eqref{eq:conic_simp},
\begin{align*}
	\vec{y}^{\top}\vec{D}\vec{y}+2\brak{\vec{c}^{\top}\vec{V}+\vec{u}^{\top}}\myvec{\vec{p}_1 & \vec{p}_2}\vec{y}
%	\\
	+  \vec{c}^{\top}\brak{\vec{V}\vec{c} + \vec{u}}+ \vec{u}^{\top}\vec{c} + f= 0
\\
\implies \vec{y}^{\top}\vec{D}\vec{y}
+2\myvec{\brak{\vec{c}^{\top}\vec{V}+\vec{u}^{\top}}\vec{p}_1  \brak{\vec{c}^{\top}\vec{V}+\vec{u}^{\top}}\vec{p}_2}\vec{y}
%\\
	+  \vec{c}^{\top}\brak{\vec{V}\vec{c} + \vec{u}}+ \vec{u}^{\top}\vec{c} + f= 0
\\
\implies \vec{y}^{\top}\vec{D}\vec{y}
+2\myvec{\vec{u}^{\top}\vec{p}_1 & \brak{\lambda_2\vec{c}^{\top}+\vec{u}^{\top}}\vec{p}_2}\vec{y}
%\\
	+  \vec{c}^{\top}\brak{\vec{V}\vec{c} + \vec{u}}+ \vec{u}^{\top}\vec{c} + f= 0
\end{align*}
upon substituting from 
 \eqref{eq:conic_parab_eig_prop}, yielding
\begin{multline}
\lambda_2y_2^2+2\brak{\vec{u}^{\top}\vec{p}_1}y_1+  2y_2\brak{\lambda_2\vec{c}+\vec{u}}^{\top}\vec{p}_2
%\\
	+  \vec{c}^{\top}\brak{\vec{V}\vec{c} + \vec{u}}+ \vec{u}^{\top}\vec{c} + f= 0
\label{eq:conic_parab_foc_len_temp} 
\end{multline}
%which is the equation of a parabola. 
Thus, \eqref{eq:conic_parab_foc_len_temp} 
can be expressed as \eqref{eq:app-conic_simp_temp_parab} by choosing
\begin{align}
%\label{eq:app-eta}
\eta = 2\vec{u}^{\top}\vec{p}_1
\end{align}
%Choosing 
%\begin{align}
%\vec{u} + \lambda_2\vec{c} = 0,
%\vec{c}^{\top}\brak{\vec{V}\vec{c} + \vec{u}}+ \vec{u}^{\top}\vec{c} + f = 0,
%\end{align}
% the above equation becomes
%\begin{align}
%y_2^2= -\frac{2\vec{u}^{\top}\vec{p}_1}{ \lambda_2} \brak{y_1
%+  \frac{\vec{u}^{\top}\vec{V}\vec{u} - 2\lambda_2\vec{u}^{\top}\vec{u} + f\lambda_2^2}{2\vec{u}^{\top}\vec{p}_1\lambda_2^2}}
%\\
%or \eta = 2\vec{u}^{\top}\vec{p}_1
%%\label{eq:conic_simp_parab_new}
%\end{align}
and $\vec{c}$ in \eqref{eq:conic_simp} such that
\begin{align}
\label{eq:conic_parab_one}
2\vec{P}^{\top}\brak{\vec{V}\vec{c}+\vec{u}} &= \eta\myvec{1\\0}
\\
\vec{c}^{\top}\brak{\vec{V}\vec{c} + \vec{u}}+ \vec{u}^{\top}\vec{c} + f&= 0
\label{eq:conic_parab_two}
\end{align}
%we obtain  \eqref{eq:app-conic_simp_temp_parab}.
	\item
		The center/vertex of a conic section are given by
  \begin{align}
    \label{eq:app-conic_nonparab_c}
	    \vec{c} &= - \vec{V}^{-1}\vec{u}  & \mydet{\vec{V}} \ne 0
    \\
	    \myvec{ \vec{u}^{\top}+\frac{\eta}{2}\vec{p}_1^{\top} \\ \vec{v}}\vec{c} &= \myvec{-f \\ \frac{\eta}{2}\vec{p}_1-\vec{u}}  
& \mydet{\vec{V}} = 0
    \label{eq:conic_parab_c}
    \end{align}	
\solution	
$\because
\vec{P}^{\top}\vec{P} = \vec{I}$,
multiplying \eqref{eq:conic_parab_one} by $\vec{P}$ yields
\begin{align}
\label{eq:conic_parab_one_eig}
	\brak{\vec{V}\vec{c}+\vec{u}} &= \frac{\eta}{2}\vec{p}_1,
\end{align}
which, upon substituting in \eqref{eq:conic_parab_two}
results in 
\begin{align}
\frac{\eta}{2}\vec{c}^{\top}\vec{p}_1 + \vec{u}^{\top}\vec{c} + f&= 0
\label{eq:conic_parab_two_eig}
\end{align}
\eqref{eq:conic_parab_one_eig} and \eqref{eq:conic_parab_two_eig} can be clubbed together to obtain \eqref{eq:conic_parab_c}.
\item
			In 
			\eqref{eq:conic_affine}, substituting $\vec{y} = \vec{0}$, the center/vertex for the quadratic form is obtained as
    \begin{align}
	    \vec{x} = \vec{c}, 
    \end{align}
			where $\vec{c}$ is derived as 
    \eqref{eq:app-conic_nonparab_c}
    and 
    \eqref{eq:conic_parab_c}
in Appendix  \ref{app:parab}.
		\end{enumerate}
\subsection{Standard Conic}
\begin{enumerate}[label=\thesubsection.\arabic*.,ref=\thesubsection.\theenumi]
	  \item
		For the standard conic, 
				\begin{align}
					\label{eq:app-std-prm-P}
					\vec{P} &= \vec{I}
					\\
					\vec{u} &= 
				\begin{cases}
				0 & e \ne 1
       \\
				\frac{\eta}{2} \vec{e}_1 & e = 1
				\end{cases}
				\label{eq:std-prm-u}
				\\
				\lambda_1 &  
					\begin{cases}
						=0 & e = 1
						\\
						\ne 0 & e \ne 1
					\end{cases}
				\label{eq:std-prm-lam1}
				\end{align}
				where 
				\begin{align}
					\vec{I} = \myvec{\vec{e}_1 & \vec{e}_2}
				\end{align}
				is the identity matrix.
	\item
			\label{corr:center}
			The center of the standard ellipse/hyperbola, defined to be the mid point of the line joining the foci, is the origin.
	
	\item
		\label{corr:axis}
			The principal (major) axis of the standard ellipse/hyperbola, defined to be the line joining the two foci   is the $x$-axis.  
	
	\begin{proof}
		From 	\eqref{eq:app-F-ell-hyp-parab}, it is obvious that the line joining the foci passes through the origin.  Also, the direction vector of this line is $\vec{e}_1$.  Thus, the principal axis is the $x$-axis. 
	\end{proof}
	\item
		\label{corr:minor-axis}
			The minor axis of the standard ellipse/hyperbola, defined to be the line orthogonal to the $x$-axis is the $y$-axis. 
	


	\item
			The axis of symmetry of the standard parabola, defined to be the line perpendicular to the directrix and passing through the focus,  is the $x$- axis.
	
	\begin{proof}
	From \eqref{eq:n-parab} and 	
					\eqref{eq:app-F-ell-hyp-parab}, 
					the axis of the parabola  can be expressed 
     as 
		\begin{align}
			\vec{e}_2^{\top}\brak{\vec{y}  
			+\frac{\eta}{4\lambda_2}\vec{e}_1} &= 0
			\\
			\implies \vec{e}_2^{\top}\vec{y} &= 0
					\label{eq:axis-std-parab}, 
		\end{align}
		which is the equation of the $x$-axis.
	\end{proof}


	\item
			\label{corr:center-parab}
 The point where the parabola intersects its axis of symmetry is called the vertex. For the standard parabola, the vertex is the origin.
	
	\begin{proof}
					\eqref{eq:axis-std-parab} can be expressed as 
    \begin{align}
			\vec{y}= \alpha \vec{e}_1. 
					\label{eq:axis-std-parab-dir} 
    \end{align}
					Substituting \eqref{eq:axis-std-parab-dir} in 
    \eqref{eq:app-conic_simp_temp_parab}, 
    \begin{align}
	     \alpha^2 \vec{e}_1^{\top}\vec{D} \vec{e}_1 &=  -\eta\alpha \vec{e}_1^{\top} \vec{e}_1   
	     \\
	     \implies \alpha &=0, \text{ or, } \vec{y} = \vec{0}.
    %\end{aligned}
    \end{align}
	\end{proof}
    \item\leavevmode
		\begin{enumerate}
			\item The directrices for the  standard conic are given by 
				\begin{align}
					\label{eq:app-dx-ell-hyp}
					\vec{e}_1^{\top}\vec{y} &=  
					%\pm\sqrt{\abs{\frac{f_0\lambda_2}{\lambda_1\brak{\lambda_2-\lambda_1}}}} & e \ne 1
					\pm \frac{1}{e}\sqrt{\frac{\abs{f_0}}{\lambda_2\brak{1-e^2}}} & e \ne 1
					\\
					\vec{e}_1^{\top}\vec{y} &= \frac{\eta}{2\lambda_2} & e = 1
					\label{eq:app-dx-parab}
				\end{align}
    \item The foci of the standard ellipse and hyperbola are given by 
				\begin{align}
					\label{eq:app-F-ell-hyp-parab}
					\vec{F} 
=
					\begin{cases}
						\pm e\sqrt{\frac{\abs{f_0}}{\lambda_2\brak{1-e^2}}}\vec{e}_1 & e \ne 1
					%	\pm \sqrt{\abs{\frac{f_0}{\lambda_1}\brak{1 - \frac{\lambda_1}{\lambda_2}}}}\vec{e}_1 & e \ne 1
						\\
						 -\frac{\eta}{4\lambda_2}\vec{e}_1 & e = 1
					\end{cases}
				\end{align}
	
		\end{enumerate}
	%	where, without loss of generality, $f_0 < 0$ for the hyperbola.
    
	\begin{proof}%\leavevmode
  \label{app:foc-dir}
%  \input{appendix.tex}
		\begin{enumerate}
			\item For the standard hyperbola/ellipse in \eqref{eq:app-conic_simp_temp_nonparab}, from 
					\eqref{eq:app-std-prm-P},
\eqref{eq:app-conic_quad_form_nc}
and 
					\eqref{eq:std-prm-u},
				\begin{align}
\label{eq:n-ell-hyp}
					\vec{n} &= \sqrt{\frac{\lambda_2}{f_0}} \vec{e}_1 
					\\
					c &= 
					%\pm \frac{\sqrt{-\lambda_2\brak{e^2-1}\brak{\lambda_2 f_0}}}{\lambda_2e\brak{e^2-1}}
					\pm \frac{\sqrt{-\frac{\lambda_2}{f_0}\brak{e^2-1}\brak{\frac{\lambda_2}{ f_0}}}}{\frac{\lambda_2}{f_0}e\brak{e^2-1}}
					\\
					&=\pm \frac{1}{e\sqrt{1-e^2}}
%					\\
%					&=\pm\sqrt{\abs{\frac{f_0}{\brak{1 - \frac{\lambda_1}{\lambda_2}}\frac{\lambda_1}{\lambda_2}}}}
\label{eq:c-ell-hyp}
				\end{align}
				yielding 
					\eqref{eq:app-dx-ell-hyp} upon substituting from 
\eqref{eq:app-conic_quad_form_e} and simplifying.
For the standard parabola in \eqref{eq:app-conic_simp_temp_parab},  from 
					\eqref{eq:app-std-prm-P},
\eqref{eq:app-conic_quad_form_nc}
and 
					\eqref{eq:std-prm-u}, noting that $f = 0$,

				\begin{align}
\label{eq:n-parab}
					\vec{n} &= \sqrt{\lambda_2} \vec{e}_1 
					\\
					c &=
	\frac{\norm{\frac{\eta}{2} \vec{e}_1}^2   }{2\vec{\brak{\frac{\eta}{2}} \brak{\vec{e}_1}^{\top}\vec{n}}} 
\\
					\\
					&= \frac{\eta}{4\sqrt{\lambda_2}}
\label{eq:c-parab}
				\end{align}
				yielding 
					\eqref{eq:app-dx-parab}.

				\item 	For the standard ellipse/hyperbola, substituting from
\eqref{eq:c-ell-hyp},
\eqref{eq:n-ell-hyp},
\eqref{eq:std-prm-u}
and \eqref{eq:app-conic_quad_form_e}
in \eqref{eq:app-conic_quad_form_F},
				\begin{align}
					\vec{F} &= \pm \frac{\brak{\frac{1}{e\sqrt{1-e^2}}}\brak{e^2}\sqrt{\frac{\lambda_2}{f_0}}\vec{e}_1}{\frac{\lambda_2}{f_0}}
					%\pm\sqrt{\abs{\frac{f_0}{\brak{1 - \frac{\lambda_1}{\lambda_2}}\frac{\lambda_1}{\lambda_2}}}}
					%\brak{1 - \frac{\lambda_1}{\lambda_2}}\frac{\sqrt{\lambda_2}}{\lambda_2}\vec{e}_1
 			\end{align}
			yielding
					\eqref{eq:app-F-ell-hyp-parab}
					after simplification.
					For the standard parabola, substituting from 
\eqref{eq:c-parab},
\eqref{eq:n-parab},
\eqref{eq:std-prm-u}
and \eqref{eq:app-conic_quad_form_e}
in \eqref{eq:app-conic_quad_form_F},			
				\begin{align}
	\vec{F}  &= \frac{\brak{\frac{\eta}{4\sqrt{\lambda_2}}}\sqrt{\lambda_2}\vec{e}_1-\vec{\frac{\eta}{2} \vec{e}_1}}{\lambda_2}
\\
				\end{align}
				yielding 
					\eqref{eq:app-F-ell-hyp-parab} after simplification.
		\end{enumerate}
	\end{proof}
	\item
			\label{corr:foclen}
	 The {\em focal length} of the standard parabola, , defined to be the distance between the vertex and the focus, measured along the axis of symmetry, is $\abs{\frac{\eta}{4 \lambda_2}}$


		\end{enumerate}
\subsection{Conic Lines}
\begin{enumerate}[label=\thesubsection.\arabic*.,ref=\thesubsection.\theenumi]
 \item
	 \label{prop:chord}
  The points of intersection of the line 
\begin{align}
L: \quad \vec{x} = \vec{h} + \kappa \vec{m} \quad \kappa \in \mathbb{R}
\label{eq:app-conic_tangent}
\end{align}
with the conic section in \eqref{eq:app-conic_quad_form} are given by
\begin{align}
\vec{x}_i = \vec{h} + \kappa_i \vec{m}
	\label{eq:app-chord-pts}
\end{align}
%
where
\begin{multline}
\kappa_i = \frac{1}
{
\vec{m}^{\top}\vec{V}\vec{m}
}
\lbrak{-\vec{m}^{\top}\brak{\vec{V}\vec{h}+\vec{u}}}
%\\
\pm
%{\small
\rbrak{\sqrt{
\sbrak{
\vec{m}^{\top}\brak{\vec{V}\vec{h}+\vec{u}}
}^2
	-\text{g}
\brak
{\vec{h}
%\vec{h}^{\top}\vec{V}\vec{h} + 2\vec{u}^{\top}\vec{h} +f
}
\brak{\vec{m}^{\top}\vec{V}\vec{m}}
}
}
%}
\label{eq:app-tangent_roots}
\end{multline}
\solution
  Substituting \eqref{eq:app-conic_tangent}
in \eqref{eq:app-conic_quad_form}, 
\begin{align}
\brak{\vec{h} + \kappa \vec{m}}^{\top}\vec{V}\brak{\vec{h} + \kappa \vec{m}}  + 2 \vec{u}^{\top}\brak{\vec{h} + \kappa \vec{m}}+f &= 0
\\
\implies \kappa^2\vec{m}^{\top}\vec{V}\vec{m} + 2 \kappa\vec{m}^{\top}\brak{\vec{V}\vec{h}+\vec{u}} 
+ \vec{h}^{\top}\vec{V}\vec{h} + 2\vec{u}^{\top}\vec{h} +f &= 0
	\\
	\text{or, }
\kappa^2\vec{m}^{\top}\vec{V}\vec{m} + 2 \kappa\vec{m}^{\top}\brak{\vec{V}\vec{h}+\vec{u}} 
	+ \text{g}\brak{\vec{h}} &=0
	%^{\top}\vec{V}\vec{h} + 2\vec{u}^{\top}\vec{h} +f &= 0
\label{eq:conic_intercept}
\end{align}
for g defined in \eqref{eq:app-conic_quad_form}.
Solving the above quadratic in \eqref{eq:conic_intercept}
yields \eqref{eq:app-tangent_roots}.
	\item
		The length of the chord in 
\eqref{eq:app-conic_tangent}
is given by 
\begin{align}
 \frac{2\sqrt{
\sbrak{
\vec{m}^{\top}\brak{\vec{V}\vec{h}+\vec{u}}
}^2
-
\brak
{
\vec{h}^{\top}\vec{V}\vec{h} + 2\vec{u}^{\top}\vec{h} +f
}
\brak{\vec{m}^{\top}\vec{V}\vec{m}}
}
}
{
\vec{m}^{\top}\vec{V}\vec{m}
}\norm{\vec{m}}
\label{eq:chord-len}
  \end{align}
	
\begin{proof}
The distance between the points in 
	\eqref{eq:app-chord-pts}
is given by 
\begin{align}
	\norm{\vec{x}_1-\vec{x}_2} =  \abs{\kappa_1-\kappa_2} \norm{\vec{m}}
\label{eq:app-conic_tangent_pts_dist}
\end{align}
Substituing $\kappa_i$ from 
\eqref{eq:app-tangent_roots} in
\eqref{eq:app-conic_tangent_pts_dist}
yields
	\eqref{eq:chord-len}.
\end{proof}
	\item
 The affine transform for the conic section, preserves the norm.  This implies that the length of any chord of a conic
	is invariant to translation and/or rotation.
	
	\begin{proof}
	Let 
%From \eqref{eq:conic_affine}, 
\begin{align}
\vec{x}_i = \vec{P}\vec{y}_i+\vec{c} 
\label{eq:conic_affine_pts}
\end{align}
be any two points on the conic.  Then the distance between the points is given by 
\begin{align}
	\norm{\vec{x}_1-\vec{x}_2 } &= \norm{\vec{P}\brak{	\vec{y}_1 -\vec{y}_2 }}
\end{align}
which can be expressed as 
\begin{align}
	\norm{\vec{x}_1-\vec{x}_2 }^2 &= 		\brak{\vec{y}_1 -\vec{y}_2 }^{\top}\vec{P}^{\top}\vec{P}\brak{\vec{y}_1 -\vec{y}_2 }
	\\
	&= 		\norm{\vec{y}_1 -\vec{y}_2 }^2
\label{eq:conic_affine_norm_preserve}
\end{align}
since 
\begin{align}
	\vec{P}^{\top}\vec{P} = \vec{I}
\end{align}
	\end{proof}
    \item For the standard hyperbola/ellipse, the length of the major axis is 
  \begin{align}
\label{eq:app-chord-len-major}
 2\sqrt{\abs{\frac{
f_0}
{\lambda_1}
	  }}
  \end{align}
  and the minor axis is 
  \begin{align}
\label{eq:app-chord-len-minor}
 2\sqrt{\abs{\frac{
f_0}
{\lambda_2}
	  }}
  \end{align}
		\label{app:major}
%	See Appendix \ref{app:major}
		\solution
		Since the major axis passes through the origin, 
  \begin{align}
	  \vec{q} =			\vec{0} 
\end{align}  
Further, from Corollary  
		\eqref{corr:axis},
  \begin{align}
  \vec{m}&= \vec{e}_2,  
\end{align} and
from 
    \eqref{eq:app-conic_simp_temp_nonparab},
  \begin{align}
	  \vec{V} =     \frac{\vec{D} }{f_0}, 
	   \vec{u} = 0, 
	   f = -1
	    \label{eq:latus_rectum_ellipse_param}
\end{align}  
Substituting the above in
\eqref{eq:chord-len}, 
\begin{align}
 \frac{2\sqrt{
\vec{e}_1^{\top}\frac{\vec{D}}{f_0}\vec{e}_1
}
}
{
\vec{e}_1^{\top}\frac{\vec{D}}{f_0}\vec{e}_1
}\norm{\vec{e}_1}
  \end{align}
  yielding 
\eqref{eq:app-chord-len-major}.
Similarly, for the minor axis, the only different parameter is 
  \begin{align}
  \vec{m}&= \vec{e}_2,  
\end{align} 
Substituting the above in
\eqref{eq:chord-len}, 
\begin{align}
 \frac{2\sqrt{
\vec{e}_2^{\top}\frac{\vec{D}}{f_0}\vec{e}_2
}
}
{
\vec{e}_2^{\top}\frac{\vec{D}}{f_0}\vec{e}_2
}\norm{\vec{e}_2}
  \end{align}
  yielding 
\eqref{eq:app-chord-len-minor}.
    \item The equation of the minor and major  axes for the ellipse/hyperbola are respectively given by 
  \begin{align}
\vec{p}_i^{\top}\brak{\vec{x}-\vec{c}} = 0, i = 1,2
	  \label{eq:app-major-minor-axis-quad}
  \end{align}
  The axis of symmetry for the parabola is also given by 
	  \eqref{eq:app-major-minor-axis-quad}.

		\begin{proof}
From		\eqref{corr:axis}, the major/symmetry axis for the hyperbola/ellipse/parabola can be expressed using 
	\eqref{eq:conic_affine}
 as
  \begin{align}
	  \vec{e}_2^{\top}
		  \vec{P}^{\top}\brak{\vec{x}-\vec{c}} &= 0
		  \\
	  \implies 		  \brak{\vec{P}\vec{e}_2}^{\top}\brak{\vec{x}-\vec{c}} &= 0
  \end{align}
yielding	  \eqref{eq:app-major-minor-axis-quad}, and the proof for the minor axis is similar.
		\end{proof}
\item
    The latus rectum of a conic section is the chord that passes through the focus and is perpendicular to the major axis.
	The length of the latus rectum for a conic is given by
		\begin{align}
			l =
			\begin{cases}
				2\frac{\sqrt{\abs{f_0\lambda_1}}}{\lambda_2} & e \ne 1
			\\
			\frac{\eta}{\lambda_2} & e = 1
			\end{cases}
			\label{eq:app-latus-ellipse}
		\end{align}
%			See Appendix \ref{app:latus}.
		%\section{}
		\label{app:latus}
		\solution
			The latus rectum is perpendicular to the major axis for the standard conic.  Hence, from Corollary  
		\eqref{corr:axis},
  \begin{align}
  \vec{m}&= \vec{e}_2,  
\end{align}  
Since it passes through the focus, from 
					\eqref{eq:app-F-ell-hyp-parab}
  \begin{align}
	  \vec{q} =			\vec{F} 
=
					 \pm e\sqrt{\frac{f_0}{\lambda_2\brak{1-e^2}}} \vec{e }_1
%					 \frac{e}{\sqrt{f_0\lambda_2\brak{1-e^2}}}\vec{e }_1
\end{align}  
for the standard hyperbola/ellipse.  Also, 
from 
    \eqref{eq:app-conic_simp_temp_nonparab},
  \begin{align}
	  \vec{V} =     \frac{\vec{D} }{f_0}, 
	   \vec{u} = 0, 
	   f = -1
	    \label{eq:latus_rectum_ellipse_param-new}
\end{align}  
Substituting the above in
\eqref{eq:chord-len}, 
we obtain
%\eqref{eq:chord-len-sub-ell}.
%\begin{figure*}[!t]
\begin{align}
 \frac{2\sqrt{
\sbrak{
\vec{e}_2^{\top}\brak{\frac{\vec{D}}{f_0} e\sqrt{\frac{f_0}{\lambda_2\brak{1-e^2}}} \vec{e }_1}
}^2
-
\brak
{
 e\sqrt{\frac{f_0}{\lambda_2\brak{1-e^2}}} \vec{e }_1^{\top}\frac{\vec{D}}{f_0} e\sqrt{\frac{f_0}{\lambda_2\brak{1-e^2}}} \vec{e }_1 -1 
}
\brak{\vec{e}_2^{\top}\frac{\vec{D}}{f_0}\vec{e}_2}
}
}
{
\vec{e}_2^{\top}\frac{\vec{D}}{f_0}\vec{e}_2
}\norm{\vec{e}_2}
\label{eq:chord-len-sub-ell}
  \end{align}
%\end{figure*}
  Since 
  \begin{align}
\vec{e}_2^{\top}\vec{D}\vec{e}_1 = 0, 
%\vec{e}_2^{\top}\vec{e}_2 = 0,
\vec{e}_1^{\top}\vec{D}\vec{e}_1 = \lambda_1,
\vec{e}_1^{\top}\vec{e}_1 = 1,
	  \norm{\vec{e}_2} = 1,
\vec{e}_2^{\top}\vec{D}\vec{e}_2 = \lambda_2,
  \end{align}
\eqref{eq:chord-len-sub-ell} can be expressed as 
  \begin{align}
	&		\frac{2\sqrt{\brak{1-\frac{\lambda_1e^2}{{\lambda_2\brak{1-e^2}}}}\brak{\frac{\lambda_2}{f_0}}}}
{
	\frac{\lambda_2}{f_0}
	} 	
	\\
	&=		2\frac{\sqrt{
		f_0\lambda_1}}{\lambda_2}
 & \brak{ \because e^2 = 1-\frac{\lambda_1}{\lambda_2}}
		   \end{align}
For the standard parabola, the parameters in 
\eqref{eq:chord-len} are
\begin{align}  
	\vec{q} =\vec{F} =  -\frac{\eta}{4\lambda_2}\vec{e}_1, \vec{m} = \vec{e}_1, \vec{V} = \vec{D},
	\vec{u} = \frac{\eta}{2}\vec{e}_1^{\top}, f = 0
\end{align}  

Substituting the above in
\eqref{eq:chord-len}, 
%			from \eqref{eq:app-conic_simp_temp_nonparab},  
%					from \eqref{eq:app-F-ell-hyp-parab}
%and 						 \\
the length of the latus rectum  can be expressed as
\begin{align}
 \frac{2\sqrt{
\sbrak{
\vec{e}_2^{\top}\brak{\vec{D}\brak{-\frac{\eta}{4\lambda_2}\vec{e}_1}+\frac{\eta}{2}\vec{e}_1}
}^2
-
\brak
{
\brak{-\frac{\eta}{4\lambda_2}\vec{e}_1}^{\top}\vec{D}\brak{-\frac{\eta}{4\lambda_2}\vec{e}_1} + 2\frac{\eta}{2}\vec{e}_1^{\top}\brak{-\frac{\eta}{4\lambda_2}\vec{e}_1} 
}
\brak{\vec{e}_2^{\top}\vec{D}\vec{e}_2}
}
}
{
\vec{e}_2^{\top}\vec{D}\vec{e}_2
}\norm{\vec{e}_2}
\label{eq:chord-len-sub}
  \end{align}
%\eqref{eq:chord-len-sub}.
  Since 
  \begin{align}
\vec{e}_2^{\top}\vec{D}\vec{e}_1 = 0, 
\vec{e}_2^{\top}\vec{e}_2 = 0,
	  \vec{e}_1^{\top}\vec{D}\vec{e}_1 &= 0,\
	  \\
\vec{e}_1^{\top}\vec{e}_1 = 1,
	  \norm{\vec{e}_1} = 1,
	  \vec{e}_2^{\top}\vec{D}\vec{e}_2 &= \lambda_2,
  \end{align}
\eqref{eq:chord-len-sub} can be expressed as 
  \begin{align}
	  2 \frac{\sqrt{\frac{\eta^2}{4\lambda_2}\lambda_2}}{\lambda_2}
	  = \frac{\eta}{\lambda_2}
  \end{align}
%		See Appendix \ref{app:foc-dir}.
%

		\end{enumerate}
\subsection{Tangent and Normal}
\begin{enumerate}[label=\thesubsection.\arabic*.,ref=\thesubsection.\theenumi]
\item
  If $L$ in \eqref{eq:app-conic_tangent} touches \eqref{eq:app-conic_quad_form} at exactly one point $\vec{q}$, 
  \begin{align}
\label{eq:app-conic_tangent_mq}
  \vec{m}^{\top}\brak{\vec{V}\vec{q}+\vec{u}} = 0
  \end{align}
\begin{proof}
  In this case, \eqref{eq:conic_intercept} has exactly one root.  Hence, 
  in \eqref{eq:app-tangent_roots}
  \begin{align}
  \sbrak{
  \vec{m}^{\top}\brak{\vec{V}\vec{q}+\vec{u}}
  }^2 -\brak{\vec{m}^{\top}\vec{V}\vec{m}}
	  \text{g}\brak
  {
  \vec{q}
%  \vec{q}^{\top}\vec{V}\vec{q} + 2\vec{u}^{\top}\vec{q} +f
  } = 0                                                                                             
  \label{eq:app-conic_tangent_disc}
  \end{align}                    
  $\because \vec{q}$ is the point of contact,
	%$\vec{q}$ satisfies \eqref{eq:app-conic_quad_form}
%  and 
  \begin{align}
	  \text{g}\brak{  \vec{q}} = 0
%  \vec{q}^{\top}\vec{V}\vec{q} + 2\vec{u}^{\top}\vec{q} +f = 0
  \label{eq:app-conic_tangent_qquad}
  \end{align}
  Substituting \eqref{eq:app-conic_tangent_qquad} in \eqref{eq:app-conic_tangent_disc} and simplifying, we obtain \eqref{eq:app-conic_tangent_mq}.
\end{proof}
\item
  Given the point of contact $\vec{q}$, the equation of a tangent to \eqref{eq:app-conic_quad_form} is 
  \begin{align}
  \label{eq:app-conic_tangent_final}
  \brak{\vec{V}\vec{q}+\vec{u}}^{\top}\vec{x}+\vec{u}^{\top}\vec{q}+f = 0
  \end{align}
\begin{proof}
  The normal vector is obtained from \eqref{eq:app-conic_tangent_mq} 
  as
  %
  \begin{align}
  \label{eq:conic_normal_vec}
	  \kappa \vec{n} = \vec{V}\vec{q}+\vec{u}, \kappa \in \mathbb{R}
  \end{align}  
  From \eqref{eq:conic_normal_vec}, the equation of the tangent is\begin{align}
    \brak{\vec{V}\vec{q}+\vec{u}}^{\top}\brak{\vec{x}-\vec{q}} &=0
    \\
    \implies \brak{\vec{V}\vec{q}+\vec{u}}^{\top}\vec{x}-\vec{q}^{\top}\vec{V}\vec{q}-\vec{u}^{\top}\vec{q} &= 0
    \end{align}
    which, upon substituting from \eqref{eq:app-conic_tangent_qquad} and simplifying yields 
  \eqref{eq:app-conic_tangent_final}
%	\eqref{eq:app-conic_tangent}.
\end{proof}
\item
  Given the point of contact $\vec{q}$, the equation of the normal to \eqref{eq:app-conic_quad_form} is 
  \begin{align}
    \brak{\vec{V}\vec{q}+\vec{u}}^{\top}\vec{R}\brak{\vec{x}-\vec{q}} =0
  \end{align}
\begin{proof}
  The direction vector of the tangent is obtained from 
  \eqref{eq:conic_normal_vec} as
  as
  %
  \begin{align}
  \label{eq:app-conic_tangent_vec}
	  \vec{m} = \vec{R}\brak{\vec{V}\vec{q}+\vec{u}}, 
  \end{align}  
  where $\vec{R}$ is the rotation matrix.
  From \eqref{eq:app-conic_tangent_vec}, the equation of the normal is
  given by 
  \eqref{eq:conic_normal_final}
\end{proof}

\item Given the tangent 
\begin{align}
  \label{eq:app-conic_tangent_eq}
\vec{n}^{\top}\vec{x} = c,
\end{align}
the point of  contact to the conic in \eqref{eq:app-conic_quad_form} is given by 
\begin{align}
  \label{eq:app-conic_tangent_contact}
        \myvec{\vec{n}^{\top} \\ \vec{m}^{\top}\vec{V}} \vec{q} = \myvec{c\\ -\vec{m}^{\top}\vec{u}}
\end{align}
		\begin{proof}
			From
  \eqref{eq:app-conic_tangent_mq},
\begin{align}
	\vec{m}^{\top}(\vec{V}\vec{q}+\vec{u})&=0
	\\
	\implies        \vec{m}^{\top}\vec{V}\vec{q} &= -\vec{m}^{\top}\vec{u}
  \label{eq:app-conic_tangent_contact_eq}
\end{align}
Combining 
  \eqref{eq:app-conic_tangent_eq}
  and 
  \eqref{eq:app-conic_tangent_contact_eq}, 
  \eqref{eq:app-conic_tangent_contact} is obtained.

		\end{proof}
\item
  If $\vec{V}^{-1}$ exists, given the normal vector $\vec{n}$, the tangent points of contact to \eqref{eq:app-conic_quad_form} are given by
\begin{align}
  \begin{split}
\vec{q}_i &= \vec{V}^{-1}\brak{\kappa_i \vec{n}-\vec{u}}, i = 1,2
\\
\text{where }\kappa_i &= \pm \sqrt{
\frac{
f_0
%\vec{u}^{\top}\vec{V}^{-1}\vec{u}-f
}
{
\vec{n}^{\top}\vec{V}^{-1}\vec{n}
}
}
  \end{split}
\label{eq:app-conic_tangent_qk}
\end{align}
\begin{proof}
  From \eqref{eq:conic_normal_vec},
\begin{align}
\label{eq:conic_normal_vec_q}
 \vec{q} = \vec{V}^{-1}\brak{\kappa \vec{n}-\vec{u}}, \quad \kappa \in \mathbb{R}
\end{align}
Substituting \eqref{eq:conic_normal_vec_q}
in \eqref{eq:app-conic_tangent_qquad},
\begin{align}
\brak{\kappa \vec{n}-\vec{u}}^{\top}\vec{V}^{-1}\brak{\kappa \vec{n}-\vec{u}} 
%\\
+ 2\vec{u}^{\top}\vec{V}^{-1}\brak{\kappa \vec{n}-\vec{u}} +f &= 0
\\
\implies 
\kappa^2 \vec{n}^{\top}\vec{V}^{-1}\vec{n} - \vec{u}^{\top}\vec{V}^{-1}\vec{u} + f &=0
 \\
 \text{or, } \kappa = \pm \sqrt{\frac{
	 %\vec{u}^{\top}\vec{V}^{-1}\vec{u}-f
	f_0 
 }{\vec{n}^{\top}\vec{V}^{-1}\vec{n}}} &
	\label{eq:conic_normal_k}
\end{align}
%
%yileding 
Substituting \eqref{eq:conic_normal_k} in \eqref{eq:conic_normal_vec_q}
yields \eqref{eq:app-conic_tangent_qk}.
%
\end{proof}
\item For a conic/hyperbola, a line with normal vector $\vec{n}$ cannot be a tangent if 
\begin{align}
\frac{
\vec{u}^{\top}\vec{V}^{-1}\vec{u}-f
}
{
\vec{n}^{\top}\vec{V}^{-1}\vec{n}
} < 0
\end{align}

\item
	\label{eq:conic-p-contact-parab}
  If $\vec{V}$ is not invertible,  given the normal vector $\vec{n}$, the point of contact to \eqref{eq:app-conic_quad_form} is given by the matrix equation
\begin{align}
\label{eq:app-conic_tangent_q_eigen}
\myvec{
\vec{\brak{u+\kappa \vec{n}}}^{\top} \\ \vec{V}
}
\vec{q} &= 
\myvec{
-f
\\
\kappa\vec{n}-\vec{u}
}
\\
\text{where }  \kappa = \frac{\vec{p}_1^{\top}\vec{u}}{\vec{p}_1^{\top}\vec{n}}, \quad \vec{V}\vec{p}_1 &= 0
\label{eq:app-conic_tangent_qk_eigen}
\end{align}


\begin{proof}
  If $\vec{V}$ is non-invertible, it has a zero eigenvalue.  If the corresponding eigenvector is $\vec{p}_1$, then,
\begin{align}
\vec{V}\vec{p}_1 = 0
\label{eq:conic_zero_eigen}
\end{align}
From \eqref{eq:conic_normal_vec},
\begin{align}
\label{eq:conic_zero_eigen_normal}
\kappa \vec{n} &= \vec{V} \vec{q}+\vec{u}, \quad \kappa \in \mathbb{R}
\\
\implies \kappa \vec{p}_1^{\top}\vec{n} &= \vec{p}_1^{\top}\vec{V} \vec{q}+\vec{p}_1^{\top}\vec{u}
\\
\text{or, } \kappa \vec{p}_1^{\top}\vec{n} &= \vec{p}_1^{\top}\vec{u},  \quad \because \vec{p}_1^{\top} \vec{V} = 0, 
%\\
\quad 
\brak{\text{ from } \eqref{eq:conic_zero_eigen}}
%\label{eq:conic_normal_vec_q}
\end{align}
yielding $\kappa$ in \eqref{eq:app-conic_tangent_qk_eigen}. From \eqref{eq:conic_zero_eigen_normal},
\begin{align}
\kappa \vec{q}^{\top}\vec{n} &= \vec{q}^{\top}\vec{V} \vec{q}+\vec{q}^{\top}\vec{u}
\\
\implies \kappa \vec{q}^{\top}\vec{n} &= -f-\vec{q}^{\top}\vec{u} \quad \text{from } \eqref{eq:app-conic_tangent_qquad},
\\
\text{or, } \brak{\kappa \vec{n}+\vec{u}}^{\top}\vec{q} &= -f
\label{eq:conic_zero_eigen_normal_fq}
\end{align}
\eqref{eq:conic_zero_eigen_normal} can be expressed as
\begin{align}
\label{eq:conic_zero_eigen_normal_vq}
\vec{V} \vec{q} = \kappa \vec{n} - \vec{u}.
\end{align}
\eqref{eq:conic_zero_eigen_normal_fq} and \eqref{eq:conic_zero_eigen_normal_vq} clubbed together result in \eqref{eq:app-conic_tangent_q_eigen}.
\end{proof}
\item
	The asymptotes of the hyperbola in 
    \eqref{eq:app-conic_simp_temp_nonparab}, defined to be the lines that do not intersect the hyperbola, are given by 
    \begin{align} 
    \label{eq:app-pair-std}
    \myvec{\sqrt{\abs{\lambda_1}} & \pm \sqrt{\abs{\lambda_2}}}\vec{y} = 0
    \end{align} 
%  \begin{align}
%	  \myvec{\lambda_1 & \pm \lambda_2}\vec{y} = 0   
%  \end{align}
  
  \begin{proof}
	  From 
\eqref{eq:app-conic_simp_temp_nonparab},
it is obvious that 
the pair of lines represented by 
  \begin{align}
	    \vec{y}^{\top}\vec{D}\vec{y} = 0   
      \label{eq:pair-conic}
  \end{align}
  do not intersect the conic 
  \begin{align}
	    \vec{y}^{\top}\vec{D}\vec{y} =  f_0  
  \end{align}
  Thus, 
      \eqref{eq:pair-conic}
      represents the asysmptotes of the hyperbola in 
\eqref{eq:app-conic_simp_temp_nonparab} and can be expressed as 
  \begin{align} 
    \lambda_1y_1^2 +\lambda_2y_1^2 = 0, 
    \label{eq:quad_form_hyper}
    \end{align}
%    \eqref{eq:quad_form_hyper}
which can then be simplified  
using the steps in 
	\eqref{eq:incircle-disc-v}-
	\eqref{eq:incircle-disc-v-lam}
to obtain
    \eqref{eq:app-pair-std}.
  \end{proof}
  \item
\eqref{eq:app-conic_quad_form} represents a pair of straight lines if 
  \begin{align} 
	  \label{eq:pair-cond}
%	  \lambda_1y_1^2 +\lambda_2y_2^2 = 
  \vec{u}^{\top}\vec{V}^{-1}\vec{u} -f  = 0
  \end{align} 
  
  \item
%	  \label{them:pair-mat-sing}
\eqref{eq:app-conic_quad_form} represents a pair of straight lines if 
the matrix 
  \begin{align} 
	  \myvec{\vec{V} & \vec{u}\\ \vec{u}^{\top} & f}  
%	  \label{eq:pair-mat-sing}
  \end{align} 
  is singular.
  
  \begin{proof}
Let 
  \begin{align} 
	  \myvec{\vec{V} & \vec{u}\\ \vec{u}^{\top} & f}  \vec{x} =\vec{0}
  \end{align} 
  Expressing 
  \begin{align} 
	  \vec{x} =\myvec{\vec{y} \\ y_3}, 
  \end{align} 
  \begin{align} 
	  \myvec{\vec{V} & \vec{u}\\ \vec{u}^{\top} & f}   
	  \myvec{\vec{y} \\ y_3} &= \vec{0}
	  \\
	  \implies
	  \label{eq:pair-mat-sing-1}
	  \vec{V} \vec{y} + y_3\vec{u} &= \vec{0} \quad \text{and}
	  \\
	  \vec{u}^{\top}\vec{y} + fy_3 &=0
	  \label{eq:pair-mat-sing-2}
  \end{align} 
  From 
	  \eqref{eq:pair-mat-sing-1} we obtain,
  \begin{align} 
	  \vec{y}^{\top}  \vec{V} \vec{y} + y_3\vec{y}^{\top}\vec{u} &= \vec{0} 
	  \\
	  \implies 
	  \vec{y}^{\top}  \vec{V} \vec{y} + y_3\vec{u}^{\top}\vec{y} &= \vec{0} 
  \end{align} 
  yielding 
	  \eqref{eq:pair-cond} upon substituting from 
	  \eqref{eq:pair-mat-sing-2}.
  \end{proof}
  \item
	  Using the affine transformation, 
    \eqref{eq:app-pair-std}
 can be expressed as the lines 
%
\begin{align} 
\label{eq:quad_form_pair}
\myvec{\sqrt{\abs{\lambda_1}} & \pm \sqrt{\abs{\lambda_2}}}\vec{P}^{\top}\brak{\vec{x}-\vec{c}} = 0
\end{align} 
  
   \item
	   The angle between the asymptotes can be expressed as
\begin{align} 
\label{eq:app-quad_form_pair_ang}
\cos\theta=\frac{\abs{\lambda_1}-\abs{\lambda_2}}
{\abs{\lambda_1}+\abs{\lambda_2}}
\end{align} 
  
  \begin{proof}
The normal vectors of the lines in \eqref{eq:quad_form_pair} are 
  \begin{align} 
  \label{eq:quad_form_pair_normvecs}
  \begin{split}
  \vec{n}_1 &= \vec{P}\myvec{\sqrt{\abs{\lambda_1}} \\[2mm]  \sqrt{\abs{\lambda_2}}}
  \\
  \vec{n}_2 &= \vec{P}\myvec{\sqrt{\abs{\lambda_1}} \\[2mm] - \sqrt{\abs{\lambda_2}}}
  \end{split}
  \end{align} 
  The angle between the asymptotes is given by 
\begin{align} 
\label{eq:app-quad_form_pair_ang_exp}
\cos\theta=\frac{\vec{n_1}^{\top}\vec{n_2}}{\norm{\vec{n_1}}\norm{\vec{n_2}}}
\end{align} 
The orthogonal matrix $\vec{P}$ preserves the norm, i.e.
\begin{align} 
	\norm{\vec{n_1}} &= \norm{\vec{P}\myvec{\sqrt{\abs{\lambda_1}} \\[2mm]  \sqrt{\abs{\lambda_2}}}}
	=\norm{\myvec{\sqrt{\abs{\lambda_1}} \\[2mm]  \sqrt{\abs{\lambda_2}}}}
	\\
	&=\sqrt{\abs{\lambda_1}+\abs{\lambda_2}} = \norm{\vec{n_2}}
\end{align} 
It is easy to verify that 
\begin{align} 
\vec{n_1}^{\top}\vec{n_2} = \abs{\lambda_1}-\abs{\lambda_2}
\end{align} 
%
Thus, the angle between the asymptotes is obtained from \eqref{eq:app-quad_form_pair_ang_exp} as \eqref{eq:app-quad_form_pair_ang}.
  \end{proof}
\item For a circle, the points of contact are
	\begin{align}
	\vec{q}_{ij} &= \brak{\pm r \frac{\vec{n}_j}{\norm{\vec{n}_j}}-\vec{u}}, \quad i,j = 1,2
\end{align}
\begin{proof}
	From 
\eqref{eq:app-conic_tangent_qk},
and 
	\eqref{eq:circ-cr},
\begin{align}
\kappa_{ij} &= \pm 
\frac{r
}
{
	\norm{\vec{n}_j}
}
\end{align}
\end{proof}
\item A point $\vec{h}$ lies on a normal to the conic in \eqref{eq:app-conic_quad_form} 
	if
%\begin{multline}
\begin{equation}
	\label{eq:app-point_of_tangency-m}
	\brak{ {\vec{m}^\top(\vec{Vh}+\vec{u})}}^2\brak{\vec{n}^{\top}\vec{V}\vec{n}} 
%	\\
	- 2\brak{\vec{m}^\top\vec{V}\vec{n}} \brak{ {\vec{m}^\top(\vec{Vh}+\vec{u})}\vec{n}^{\top}\brak{\vec{V}\vec{h}+\vec{u}}} 
%	\\
+  \text{g}\brak{
  \vec{h}
	  }\brak{\vec{m}^\top\vec{V}\vec{n}}^2
%	\vec{h}^{\top}\vec{V}\vec{h} + 2\vec{u}^{\top}\vec{h} +f 
	= 0
\end{equation}
%\end{multline}
\begin{proof}
The point of contact for the normal passing through a point $\vec{h}$ is given by 
\begin{align}
	\label{eq:point_of_tangency}
	\vec{q} = \vec{h} + \mu\vec{n}
	%\vec{q} = \vec{h} + \mu\vec{R}\vec{m}
\end{align}
From 
  \eqref{eq:app-conic_tangent_mq},
	the tangent at $\vec{q}$ satisfies 
\begin{align}
	\label{eq:tangency_condition}
	\vec{m}^\top(\vec{Vq}+\vec{u}) = 0
\end{align}
Substituting \eqref{eq:point_of_tangency} in \eqref{eq:tangency_condition},
\begin{align}
	\label{eq:normal_simp_1}
	\vec{m}^\top(\vec{V}(\vec{h}+\mu\vec{n})+\vec{u}) = 0\\
	%\vec{m}^\top(\vec{V}(\vec{h}+\mu\vec{R}\vec{m})+\vec{u}) = 0\\
	\label{eq:normal_simp_2}
	\implies \mu\vec{m}^\top\vec{V}\vec{n} = -\vec{m}^\top(\vec{Vh}+\vec{u})
	%\implies \mu\vec{m}^\top\vec{V}\vec{R}\vec{m} = -\vec{m}^\top(\vec{Vh}+\vec{u})
\end{align}
yielding 
\begin{align}
	\label{eq:app-point_of_tangency-mu}
	\mu &=- \frac  {\vec{m}^\top(\vec{Vh}+\vec{u})}{\vec{m}^\top\vec{V}\vec{n}},
\end{align}
%	\eqref{eq:app-point_of_tangency-mu}.
	From 
\eqref{eq:conic_intercept},
\begin{align}
	\label{eq:normal_simp_2-quad}
\mu^2\vec{n}^{\top}\vec{V}\vec{n} + 2 \mu\vec{n}^{\top}\brak{\vec{V}\vec{h}+\vec{u}} 
+  \text{g}\brak{
  \vec{h}
	  }
%	\vec{h}^{\top}\vec{V}\vec{h} + 2\vec{u}^{\top}\vec{h} +f 
	&= 0
\end{align}
From 
	\eqref{eq:app-point_of_tangency-mu},
	\eqref{eq:normal_simp_2-quad} can be expressed
	as
\begin{multline}
%	\label{eq:normal_simp_2-quad}
\brak{- \frac{\vec{m}^\top(\vec{Vh}+\vec{u})}{\vec{m}^\top\vec{V}\vec{n}}}^2\vec{n}^{\top}\vec{V}\vec{n} 
	%\\
	+ 2 \brak{- \frac{\vec{m}^\top(\vec{Vh}+\vec{u})}{\vec{m}^\top\vec{V}\vec{n}}}\vec{n}^{\top}\brak{\vec{V}\vec{h}+\vec{u}} 
+  \text{g}\brak{
  \vec{h}
	  }
	= 0
\end{multline}
	yielding
	\eqref{eq:app-point_of_tangency-m}.
\end{proof}
\item  A point $\vec{h}$ lies on a tangent to the conic in \eqref{eq:app-conic_quad_form} if 
\begin{align}
	  \label{eq:app-h-tangents-cond}
  \vec{m}^{\top}  \sbrak{\brak{\vec{V}\vec{h}+\vec{u}}
	  \brak{\vec{V}\vec{h}+\vec{u}}^{\top}
   -\vec{V}
	  \text{g}\brak{
  \vec{h}
	  }
	  }\vec{m} 
	  &= 0                                                                                             
\end{align}
\begin{proof}
 From \eqref{eq:app-tangent_roots}
 and
  \eqref{eq:app-conic_tangent_disc}
  \begin{align}
  \sbrak{
  \vec{m}^{\top}\brak{\vec{V}\vec{h}+\vec{u}}
  }^2 -\brak{\vec{m}^{\top}\vec{V}\vec{m}}
	  \text{g}\brak{
  \vec{h}
	  }
	  &= 0                                                                                             
  \label{eq:app-conic_tangent_disc-h}
  \end{align}                    
  yielding
	  \eqref{eq:app-h-tangents-cond}.
\end{proof}
\item
	The normal vectors of the tangents 
to the conic in \eqref{eq:app-conic_quad_form} 
	from 
	a point $\vec{h}$ 
	are given by 
  \begin{align} 
  \label{eq:app-quad_form_pair_normvecs-sigma}
  \begin{split}
  \vec{n}_1 &= \vec{P}\myvec{\sqrt{\abs{\lambda_1}} \\[2mm]  \sqrt{\abs{\lambda_2}}}
  \\
  \vec{n}_2 &= \vec{P}\myvec{\sqrt{\abs{\lambda_1}} \\[2mm] - \sqrt{\abs{\lambda_2}}}
  \end{split}
  \end{align} 
  where $\lambda_i, \vec{P}$ are the eigenparameters of 
  \begin{align} 
		\bm{\Sigma} &= 
	   \brak{\vec{V}\vec{h}+\vec{u}}
	  \brak{\vec{V}\vec{h}+\vec{u}}^{\top}
   -
  \brak
  {
	  \text{g}\brak{
  \vec{h}
	  }
%  \vec{h}^{\top}\vec{V}\vec{h} + 2\vec{u}^{\top}\vec{h} +f
  }\vec{V}.
	  \label{eq:app-h-tangents-sigma}
  \end{align}                    
  \begin{proof}
	  From 
	  \eqref{eq:app-h-tangents-cond} we obtain
	  \eqref{eq:app-h-tangents-sigma}.  Consequently, from 
  \eqref{eq:quad_form_pair_normvecs}, 
  \eqref{eq:app-quad_form_pair_normvecs-sigma}
  can be obtained.
  \end{proof}

  \item
	  \label{them:pair-mat-sing}
\eqref{eq:app-conic_quad_form} represents a pair of straight lines if 
the matrix 
  \begin{align} 
	  \myvec{\vec{V} & \vec{u}\\ \vec{u}^{\top} & f}  
	  \label{eq:app-pair-mat-sing}
  \end{align} 
  is singular.
\item The intersection of two conics 
with parameters $\vec{V}_i, \vec{u}_i, f_i,\ i = 1,2$
	is defined
as
\begin{align}
	\vec{x}^{\top}\brak{\vec{V}_1 + \mu\vec{V}_2}\vec{x}+2 \brak{\vec{u}_1+\mu \vec{u}_2}^{\top} \vec{x} 
	+ \brak{f_1+\mu f_2}= 0
	  \label{eq:app-pair-mat-sing-conic}
    \end{align}
	  
	  
\item From \eqref{eq:app-pair-mat-sing}, \eqref{eq:app-pair-mat-sing-conic} represents a pair of straight lines if
\begin{align}
	  \label{eq:app-pair-mat-sing-conic-det}
\mydet{\vec{V}_1 + \mu\vec{V}_2 & \vec{u}_1+\mu \vec{u}_2\\ \brak{\vec{u}_1+\mu \vec{u}_2}^{\top} & f_1 + \mu f_2} &= 0
\end{align}
\end{enumerate}


\end{document}


